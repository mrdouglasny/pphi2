\documentclass[a4paper,11pt]{article}
%\usepackage{}
\usepackage[english]{babel}
\usepackage{amsmath,amssymb,graphicx,stmaryrd,mathrsfs,xcolor,latexsym,theorem}
%\usepackage[dvipsnames]{xcolor}
\usepackage{xcolor}
\usepackage[left= 2.5cm, bottom = 4cm, top = 3.5cm, right= 2.5cm]{geometry}
\usepackage[citecolor=blue,colorlinks=true]{hyperref}
\usepackage{authblk} % for affiliations
\usepackage{tikz}
%\usepackage{refcheck}
%\usepackage{showkeys}

\usepackage[T1]{fontenc} % for \k in Gawedzki

%%%%%%%%%% Start TeXmacs macros
\newcommand{\assign}{:=}
\newcommand{\backassign}{=:}
\newcommand{\cdummy}{\cdot}
\newcommand{\mathD}{\mathrm{D}}
\newcommand{\mathd}{\mathrm{d}}
\newcommand{\precprec}{\prec\!\!\!\prec}
\newcommand{\tmaffiliation}[1]{\\ #1}
\newcommand{\tmcolor}[2]{{\color{#1}{#2}}}
\newcommand{\tmemail}[1]{\\ \textit{Email:} \texttt{#1}}
\newcommand{\tmmathbf}[1]{\ensuremath{\boldsymbol{#1}}}
\newcommand{\tmop}[1]{\ensuremath{\operatorname{#1}}}
\newcommand{\tmrsup}[1]{\textsuperscript{#1}}
\newcommand{\tmtextit}[1]{{\itshape{#1}}}
\newenvironment{proof}{\noindent\textbf{Proof\ }}{\hspace*{\fill}$\Box$\medskip}
\newtheorem{theorem}{Theorem}[section]
\newtheorem{lemma}[theorem]{Lemma}
\newtheorem{proposition}[theorem]{Proposition}
\newtheorem{corollary}[theorem]{Corollary}
{\theorembodyfont{\rmfamily}\newtheorem{remark}[theorem]{Remark}}

\numberwithin{equation}{section}
\newcommand{\tmkeywords}{\textbf{Keywords:} }
%%%%%%%%%% End TeXmacs macros

\newcommand{\rmk}[1]{\textcolor{red}{#1}}


\iffalse
% old trees
\newcommand{\tthree}[1]{#1\!\tmrsup{\resizebox{.6em}{!}{\includegraphics{trees/3_tree.eps}}}}
\newcommand{\ttwor}[1]{#1\!\tmrsup{\resizebox{.6em}{!}{\includegraphics{trees/2r_tree_res.eps}}}}
\newcommand{\tthreer}[1]{#1\!\tmrsup{\!\resizebox{.6em}{!}{\includegraphics{trees/3r_tree_res.eps}}}}
\newcommand{\tthreeone}[1]{#1\!^{\!\resizebox{.6em}{!}{\includegraphics{trees/31_tree.eps}}}}
\newcommand{\ttwo}[1]{#1\!\tmrsup{\!\resizebox{.7em}{!}{\includegraphics{trees/2_tree.eps}}}}
\newcommand{\ttwothree}[1]{#1\!\tmrsup{\resizebox{.6em}{!}{\includegraphics{trees/23_tree.eps}}}}
\newcommand{\tthreethree}[1]{#1\!\tmrsup{\resizebox{.9em}{!}{\includegraphics{trees/33_tree.eps}}}}
\newcommand{\tthreetwo}[1]{#1\!\tmrsup{\resizebox{.9em}{!}{\includegraphics{trees/32_tree.eps}}}}
\newcommand{\tthreethreer}[1]{#1\!^{\resizebox{.9em}{!}{\includegraphics{trees/33_tree_res.eps}}}}
\newcommand{\tthreethreebr}[1]{#1\!\tmrsup{\resizebox{.9em}{!}{\includegraphics{trees/2a1_tree_res.eps}}}}
\newcommand{\ttwoone}[1]{#1\!^{\resizebox{.6em}{!}{\includegraphics{trees/21_tree.eps}}}}
\newcommand{\ttwothreer}[1]{#1\!^{\resizebox{.9em}{!}{\includegraphics{trees/23_tree_res.eps}}}}
\newcommand{\tthreetwor}[1]{#1\!^{\resizebox{.9em}{!}{\includegraphics{trees/32_tree_res.eps}}}}
\newcommand{\tone}[1]{#1\!\tmrsup{\resizebox{.4em}{!}{\includegraphics{trees/1_tree_mod.eps}}}}
\newcommand{\tzero}[1]{#1\!\tmrsup{{\varnothing}}}
\fi

% new trees
\newcommand{\tone}[1]{#1^{\resizebox{0.4em}{!}{% Created by Eps2pgf 0.7.0 (build on 2008-08-24) on Wed Apr 25 12:42:31 WEST 2018
\begin{tikzpicture}
\pgfpathmoveto{\pgfqpoint{0cm}{0cm}}
\pgfpathlineto{\pgfqpoint{0.776cm}{0cm}}
\pgfpathlineto{\pgfqpoint{0.776cm}{0.953cm}}
\pgfpathlineto{\pgfqpoint{0cm}{0.953cm}}
\pgfpathclose
\pgfusepath{clip}
\begin{pgfscope}
\begin{pgfscope}
\pgfpathmoveto{\pgfqpoint{0cm}{0cm}}
\pgfpathlineto{\pgfqpoint{0.776cm}{0cm}}
\pgfpathlineto{\pgfqpoint{0.776cm}{0.953cm}}
\pgfpathlineto{\pgfqpoint{0cm}{0.953cm}}
\pgfpathclose
\pgfusepath{clip}
\begin{pgfscope}
\definecolor{eps2pgf_color}{gray}{1}\pgfsetstrokecolor{eps2pgf_color}\pgfsetfillcolor{eps2pgf_color}
\pgfpathmoveto{\pgfqpoint{0.273cm}{0.8cm}}
\pgfpathcurveto{\pgfqpoint{0.273cm}{0.837cm}}{\pgfqpoint{0.259cm}{0.871cm}}{\pgfqpoint{0.233cm}{0.897cm}}
\pgfpathcurveto{\pgfqpoint{0.207cm}{0.923cm}}{\pgfqpoint{0.173cm}{0.937cm}}{\pgfqpoint{0.137cm}{0.937cm}}
\pgfpathcurveto{\pgfqpoint{0.1cm}{0.937cm}}{\pgfqpoint{0.066cm}{0.923cm}}{\pgfqpoint{0.04cm}{0.897cm}}
\pgfpathcurveto{\pgfqpoint{0.014cm}{0.871cm}}{\pgfqpoint{0cm}{0.837cm}}{\pgfqpoint{0cm}{0.8cm}}
\pgfpathcurveto{\pgfqpoint{0cm}{0.764cm}}{\pgfqpoint{0.014cm}{0.729cm}}{\pgfqpoint{0.04cm}{0.703cm}}
\pgfpathcurveto{\pgfqpoint{0.066cm}{0.678cm}}{\pgfqpoint{0.1cm}{0.663cm}}{\pgfqpoint{0.137cm}{0.663cm}}
\pgfpathcurveto{\pgfqpoint{0.173cm}{0.663cm}}{\pgfqpoint{0.207cm}{0.678cm}}{\pgfqpoint{0.233cm}{0.703cm}}
\pgfpathcurveto{\pgfqpoint{0.259cm}{0.729cm}}{\pgfqpoint{0.273cm}{0.764cm}}{\pgfqpoint{0.273cm}{0.8cm}}
\pgfusepath{fill}
\begin{pgfscope}
\pgfsetdash{}{0cm}
\pgfsetlinewidth{0.818mm}
\pgfsetroundcap
\pgfsetmiterlimit{7.0}
\definecolor{eps2pgf_color}{gray}{0}\pgfsetstrokecolor{eps2pgf_color}\pgfsetfillcolor{eps2pgf_color}
\pgfpathmoveto{\pgfqpoint{0.249cm}{0.064cm}}
\pgfpathlineto{\pgfqpoint{0.246cm}{0.811cm}}
\pgfusepath{stroke}
\end{pgfscope}
\definecolor{eps2pgf_color}{gray}{0}\pgfsetstrokecolor{eps2pgf_color}\pgfsetfillcolor{eps2pgf_color}
\pgfpathmoveto{\pgfqpoint{0.382cm}{0.792cm}}
\pgfpathcurveto{\pgfqpoint{0.382cm}{0.828cm}}{\pgfqpoint{0.368cm}{0.863cm}}{\pgfqpoint{0.342cm}{0.889cm}}
\pgfpathcurveto{\pgfqpoint{0.317cm}{0.914cm}}{\pgfqpoint{0.282cm}{0.929cm}}{\pgfqpoint{0.246cm}{0.929cm}}
\pgfpathcurveto{\pgfqpoint{0.21cm}{0.929cm}}{\pgfqpoint{0.175cm}{0.914cm}}{\pgfqpoint{0.149cm}{0.889cm}}
\pgfpathcurveto{\pgfqpoint{0.124cm}{0.863cm}}{\pgfqpoint{0.109cm}{0.828cm}}{\pgfqpoint{0.109cm}{0.792cm}}
\pgfpathcurveto{\pgfqpoint{0.109cm}{0.755cm}}{\pgfqpoint{0.124cm}{0.721cm}}{\pgfqpoint{0.149cm}{0.695cm}}
\pgfpathcurveto{\pgfqpoint{0.175cm}{0.669cm}}{\pgfqpoint{0.21cm}{0.655cm}}{\pgfqpoint{0.246cm}{0.655cm}}
\pgfpathcurveto{\pgfqpoint{0.282cm}{0.655cm}}{\pgfqpoint{0.317cm}{0.669cm}}{\pgfqpoint{0.342cm}{0.695cm}}
\pgfpathcurveto{\pgfqpoint{0.368cm}{0.721cm}}{\pgfqpoint{0.382cm}{0.755cm}}{\pgfqpoint{0.382cm}{0.792cm}}
\pgfusepath{fill}
\definecolor{eps2pgf_color}{gray}{1}\pgfsetstrokecolor{eps2pgf_color}\pgfsetfillcolor{eps2pgf_color}
\pgfpathmoveto{\pgfqpoint{0.774cm}{0.776cm}}
\pgfpathcurveto{\pgfqpoint{0.774cm}{0.812cm}}{\pgfqpoint{0.759cm}{0.847cm}}{\pgfqpoint{0.734cm}{0.873cm}}
\pgfpathcurveto{\pgfqpoint{0.708cm}{0.899cm}}{\pgfqpoint{0.673cm}{0.913cm}}{\pgfqpoint{0.637cm}{0.913cm}}
\pgfpathcurveto{\pgfqpoint{0.601cm}{0.913cm}}{\pgfqpoint{0.566cm}{0.899cm}}{\pgfqpoint{0.541cm}{0.873cm}}
\pgfpathcurveto{\pgfqpoint{0.515cm}{0.847cm}}{\pgfqpoint{0.501cm}{0.812cm}}{\pgfqpoint{0.501cm}{0.776cm}}
\pgfpathcurveto{\pgfqpoint{0.501cm}{0.74cm}}{\pgfqpoint{0.515cm}{0.705cm}}{\pgfqpoint{0.541cm}{0.679cm}}
\pgfpathcurveto{\pgfqpoint{0.566cm}{0.654cm}}{\pgfqpoint{0.601cm}{0.639cm}}{\pgfqpoint{0.637cm}{0.639cm}}
\pgfpathcurveto{\pgfqpoint{0.673cm}{0.639cm}}{\pgfqpoint{0.708cm}{0.654cm}}{\pgfqpoint{0.734cm}{0.679cm}}
\pgfpathcurveto{\pgfqpoint{0.759cm}{0.705cm}}{\pgfqpoint{0.774cm}{0.74cm}}{\pgfqpoint{0.774cm}{0.776cm}}
\pgfusepath{fill}
\end{pgfscope}
\end{pgfscope}
\end{pgfscope}
\end{tikzpicture}
}}}
\newcommand{\tzero}[1]{#1^{\varnothing}}
\newcommand{\tthree}[1]{#1^{\resizebox{0.6em}{!}{\input{trees-tikz/3_tree.tikz}}}}
\newcommand{\ttwor}[1]{#1\tmrsup{\resizebox{0.6em}{!}{\input{trees-tikz/2r_tree_res.tikz}}}}
\newcommand{\tthreer}[1]{#1\tmrsup{\!\resizebox{0.6em}{!}{% Created by Eps2pgf 0.7.0 (build on 2008-08-24) on Wed Apr 25 13:04:02 WEST 2018
\begin{tikzpicture}
\pgfpathmoveto{\pgfqpoint{0cm}{-0.035cm}}
\pgfpathlineto{\pgfqpoint{1.411cm}{-0.035cm}}
\pgfpathlineto{\pgfqpoint{1.411cm}{1.235cm}}
\pgfpathlineto{\pgfqpoint{0cm}{1.235cm}}
\pgfpathclose
\pgfusepath{clip}
\begin{pgfscope}
\begin{pgfscope}
\pgfpathmoveto{\pgfqpoint{0cm}{-0.035cm}}
\pgfpathlineto{\pgfqpoint{1.411cm}{-0.035cm}}
\pgfpathlineto{\pgfqpoint{1.411cm}{1.235cm}}
\pgfpathlineto{\pgfqpoint{0cm}{1.235cm}}
\pgfpathclose
\pgfusepath{clip}
\begin{pgfscope}
\definecolor{eps2pgf_color}{gray}{0}\pgfsetstrokecolor{eps2pgf_color}\pgfsetfillcolor{eps2pgf_color}
\pgfpathmoveto{\pgfqpoint{0.273cm}{1.065cm}}
\pgfpathcurveto{\pgfqpoint{0.273cm}{1.101cm}}{\pgfqpoint{0.259cm}{1.136cm}}{\pgfqpoint{0.233cm}{1.162cm}}
\pgfpathcurveto{\pgfqpoint{0.207cm}{1.187cm}}{\pgfqpoint{0.173cm}{1.202cm}}{\pgfqpoint{0.136cm}{1.202cm}}
\pgfpathcurveto{\pgfqpoint{0.1cm}{1.202cm}}{\pgfqpoint{0.066cm}{1.187cm}}{\pgfqpoint{0.04cm}{1.162cm}}
\pgfpathcurveto{\pgfqpoint{0.014cm}{1.136cm}}{\pgfqpoint{0cm}{1.101cm}}{\pgfqpoint{0cm}{1.065cm}}
\pgfpathcurveto{\pgfqpoint{0cm}{1.029cm}}{\pgfqpoint{0.014cm}{0.994cm}}{\pgfqpoint{0.04cm}{0.968cm}}
\pgfpathcurveto{\pgfqpoint{0.066cm}{0.942cm}}{\pgfqpoint{0.1cm}{0.928cm}}{\pgfqpoint{0.136cm}{0.928cm}}
\pgfpathcurveto{\pgfqpoint{0.173cm}{0.928cm}}{\pgfqpoint{0.207cm}{0.942cm}}{\pgfqpoint{0.233cm}{0.968cm}}
\pgfpathcurveto{\pgfqpoint{0.259cm}{0.994cm}}{\pgfqpoint{0.273cm}{1.029cm}}{\pgfqpoint{0.273cm}{1.065cm}}
\pgfusepath{fill}
\begin{pgfscope}
\pgfsetdash{}{0cm}
\pgfsetlinewidth{0.818mm}
\pgfsetroundcap
\pgfsetroundjoin
\pgfsetmiterlimit{7.0}
\pgfpathmoveto{\pgfqpoint{0.136cm}{1.065cm}}
\pgfpathlineto{\pgfqpoint{0.701cm}{0.315cm}}
\pgfpathlineto{\pgfqpoint{1.266cm}{1.065cm}}
\pgfusepath{stroke}
\end{pgfscope}
\pgfpathmoveto{\pgfqpoint{1.402cm}{1.065cm}}
\pgfpathcurveto{\pgfqpoint{1.402cm}{1.101cm}}{\pgfqpoint{1.388cm}{1.136cm}}{\pgfqpoint{1.362cm}{1.162cm}}
\pgfpathcurveto{\pgfqpoint{1.337cm}{1.187cm}}{\pgfqpoint{1.302cm}{1.202cm}}{\pgfqpoint{1.266cm}{1.202cm}}
\pgfpathcurveto{\pgfqpoint{1.23cm}{1.202cm}}{\pgfqpoint{1.195cm}{1.187cm}}{\pgfqpoint{1.169cm}{1.162cm}}
\pgfpathcurveto{\pgfqpoint{1.144cm}{1.136cm}}{\pgfqpoint{1.129cm}{1.101cm}}{\pgfqpoint{1.129cm}{1.065cm}}
\pgfpathcurveto{\pgfqpoint{1.129cm}{1.029cm}}{\pgfqpoint{1.144cm}{0.994cm}}{\pgfqpoint{1.169cm}{0.968cm}}
\pgfpathcurveto{\pgfqpoint{1.195cm}{0.942cm}}{\pgfqpoint{1.23cm}{0.928cm}}{\pgfqpoint{1.266cm}{0.928cm}}
\pgfpathcurveto{\pgfqpoint{1.302cm}{0.928cm}}{\pgfqpoint{1.337cm}{0.942cm}}{\pgfqpoint{1.362cm}{0.968cm}}
\pgfpathcurveto{\pgfqpoint{1.388cm}{0.994cm}}{\pgfqpoint{1.402cm}{1.029cm}}{\pgfqpoint{1.402cm}{1.065cm}}
\pgfusepath{fill}
\begin{pgfscope}
\pgfsetdash{}{0cm}
\pgfsetlinewidth{0.818mm}
\pgfsetmiterlimit{7.0}
\pgfpathmoveto{\pgfqpoint{0.701cm}{0.315cm}}
\pgfpathlineto{\pgfqpoint{0.699cm}{1.062cm}}
\pgfusepath{stroke}
\end{pgfscope}
\pgfpathmoveto{\pgfqpoint{0.835cm}{1.065cm}}
\pgfpathcurveto{\pgfqpoint{0.835cm}{1.101cm}}{\pgfqpoint{0.82cm}{1.136cm}}{\pgfqpoint{0.795cm}{1.162cm}}
\pgfpathcurveto{\pgfqpoint{0.769cm}{1.187cm}}{\pgfqpoint{0.734cm}{1.202cm}}{\pgfqpoint{0.698cm}{1.202cm}}
\pgfpathcurveto{\pgfqpoint{0.662cm}{1.202cm}}{\pgfqpoint{0.627cm}{1.187cm}}{\pgfqpoint{0.602cm}{1.162cm}}
\pgfpathcurveto{\pgfqpoint{0.576cm}{1.136cm}}{\pgfqpoint{0.562cm}{1.101cm}}{\pgfqpoint{0.562cm}{1.065cm}}
\pgfpathcurveto{\pgfqpoint{0.562cm}{1.029cm}}{\pgfqpoint{0.576cm}{0.994cm}}{\pgfqpoint{0.602cm}{0.968cm}}
\pgfpathcurveto{\pgfqpoint{0.627cm}{0.942cm}}{\pgfqpoint{0.662cm}{0.928cm}}{\pgfqpoint{0.698cm}{0.928cm}}
\pgfpathcurveto{\pgfqpoint{0.734cm}{0.928cm}}{\pgfqpoint{0.769cm}{0.942cm}}{\pgfqpoint{0.795cm}{0.968cm}}
\pgfpathcurveto{\pgfqpoint{0.82cm}{0.994cm}}{\pgfqpoint{0.835cm}{1.029cm}}{\pgfqpoint{0.835cm}{1.065cm}}
\pgfusepath{fill}
\begin{pgfscope}
\pgfsetdash{}{0cm}
\pgfsetlinewidth{0.818mm}
\pgfsetmiterlimit{4.0}
\pgfpathmoveto{\pgfqpoint{0.838cm}{0.178cm}}
\pgfpathcurveto{\pgfqpoint{0.838cm}{0.214cm}}{\pgfqpoint{0.823cm}{0.249cm}}{\pgfqpoint{0.798cm}{0.275cm}}
\pgfpathcurveto{\pgfqpoint{0.772cm}{0.3cm}}{\pgfqpoint{0.737cm}{0.315cm}}{\pgfqpoint{0.701cm}{0.315cm}}
\pgfpathcurveto{\pgfqpoint{0.665cm}{0.315cm}}{\pgfqpoint{0.63cm}{0.3cm}}{\pgfqpoint{0.605cm}{0.275cm}}
\pgfpathcurveto{\pgfqpoint{0.579cm}{0.249cm}}{\pgfqpoint{0.565cm}{0.214cm}}{\pgfqpoint{0.565cm}{0.178cm}}
\pgfpathcurveto{\pgfqpoint{0.565cm}{0.142cm}}{\pgfqpoint{0.579cm}{0.107cm}}{\pgfqpoint{0.605cm}{0.081cm}}
\pgfpathcurveto{\pgfqpoint{0.63cm}{0.055cm}}{\pgfqpoint{0.665cm}{0.041cm}}{\pgfqpoint{0.701cm}{0.041cm}}
\pgfpathcurveto{\pgfqpoint{0.737cm}{0.041cm}}{\pgfqpoint{0.772cm}{0.055cm}}{\pgfqpoint{0.798cm}{0.081cm}}
\pgfpathcurveto{\pgfqpoint{0.823cm}{0.107cm}}{\pgfqpoint{0.838cm}{0.142cm}}{\pgfqpoint{0.838cm}{0.178cm}}
\pgfusepath{stroke}
\end{pgfscope}
\end{pgfscope}
\end{pgfscope}
\end{pgfscope}
\end{tikzpicture}
}}}
\newcommand{\ttwo}[1]{#1^{\!\resizebox{0.6em}{!}{% Created by Eps2pgf 0.7.0 (build on 2008-08-24) on Wed Apr 25 13:01:48 WEST 2018
\begin{tikzpicture}
\pgfpathmoveto{\pgfqpoint{0cm}{-0.035cm}}
\pgfpathlineto{\pgfqpoint{1.376cm}{-0.035cm}}
\pgfpathlineto{\pgfqpoint{1.376cm}{0.917cm}}
\pgfpathlineto{\pgfqpoint{0cm}{0.917cm}}
\pgfpathclose
\pgfusepath{clip}
\begin{pgfscope}
\begin{pgfscope}
\pgfpathmoveto{\pgfqpoint{0cm}{-0.035cm}}
\pgfpathlineto{\pgfqpoint{1.376cm}{-0.035cm}}
\pgfpathlineto{\pgfqpoint{1.376cm}{0.917cm}}
\pgfpathlineto{\pgfqpoint{0cm}{0.917cm}}
\pgfpathclose
\pgfusepath{clip}
\begin{pgfscope}
\begin{pgfscope}
\pgfsetdash{}{0cm}
\pgfsetlinewidth{0.818mm}
\pgfsetroundcap
\pgfsetroundjoin
\pgfsetmiterlimit{7.0}
\definecolor{eps2pgf_color}{gray}{0}\pgfsetstrokecolor{eps2pgf_color}\pgfsetfillcolor{eps2pgf_color}
\pgfpathmoveto{\pgfqpoint{0.117cm}{0.791cm}}
\pgfpathlineto{\pgfqpoint{0.682cm}{0.041cm}}
\pgfpathlineto{\pgfqpoint{1.246cm}{0.791cm}}
\pgfusepath{stroke}
\end{pgfscope}
\definecolor{eps2pgf_color}{gray}{0}\pgfsetstrokecolor{eps2pgf_color}\pgfsetfillcolor{eps2pgf_color}
\pgfpathmoveto{\pgfqpoint{0.273cm}{0.765cm}}
\pgfpathcurveto{\pgfqpoint{0.273cm}{0.801cm}}{\pgfqpoint{0.259cm}{0.836cm}}{\pgfqpoint{0.233cm}{0.862cm}}
\pgfpathcurveto{\pgfqpoint{0.207cm}{0.888cm}}{\pgfqpoint{0.173cm}{0.902cm}}{\pgfqpoint{0.137cm}{0.902cm}}
\pgfpathcurveto{\pgfqpoint{0.1cm}{0.902cm}}{\pgfqpoint{0.066cm}{0.888cm}}{\pgfqpoint{0.04cm}{0.862cm}}
\pgfpathcurveto{\pgfqpoint{0.014cm}{0.836cm}}{\pgfqpoint{0cm}{0.801cm}}{\pgfqpoint{0cm}{0.765cm}}
\pgfpathcurveto{\pgfqpoint{0cm}{0.729cm}}{\pgfqpoint{0.014cm}{0.694cm}}{\pgfqpoint{0.04cm}{0.668cm}}
\pgfpathcurveto{\pgfqpoint{0.066cm}{0.643cm}}{\pgfqpoint{0.1cm}{0.628cm}}{\pgfqpoint{0.137cm}{0.628cm}}
\pgfpathcurveto{\pgfqpoint{0.173cm}{0.628cm}}{\pgfqpoint{0.207cm}{0.643cm}}{\pgfqpoint{0.233cm}{0.668cm}}
\pgfpathcurveto{\pgfqpoint{0.259cm}{0.694cm}}{\pgfqpoint{0.273cm}{0.729cm}}{\pgfqpoint{0.273cm}{0.765cm}}
\pgfusepath{fill}
\pgfpathmoveto{\pgfqpoint{1.345cm}{0.741cm}}
\pgfpathcurveto{\pgfqpoint{1.345cm}{0.777cm}}{\pgfqpoint{1.331cm}{0.812cm}}{\pgfqpoint{1.305cm}{0.838cm}}
\pgfpathcurveto{\pgfqpoint{1.28cm}{0.863cm}}{\pgfqpoint{1.245cm}{0.878cm}}{\pgfqpoint{1.209cm}{0.878cm}}
\pgfpathcurveto{\pgfqpoint{1.172cm}{0.878cm}}{\pgfqpoint{1.138cm}{0.863cm}}{\pgfqpoint{1.112cm}{0.838cm}}
\pgfpathcurveto{\pgfqpoint{1.087cm}{0.812cm}}{\pgfqpoint{1.072cm}{0.777cm}}{\pgfqpoint{1.072cm}{0.741cm}}
\pgfpathcurveto{\pgfqpoint{1.072cm}{0.704cm}}{\pgfqpoint{1.087cm}{0.67cm}}{\pgfqpoint{1.112cm}{0.644cm}}
\pgfpathcurveto{\pgfqpoint{1.138cm}{0.618cm}}{\pgfqpoint{1.172cm}{0.604cm}}{\pgfqpoint{1.209cm}{0.604cm}}
\pgfpathcurveto{\pgfqpoint{1.245cm}{0.604cm}}{\pgfqpoint{1.28cm}{0.618cm}}{\pgfqpoint{1.305cm}{0.644cm}}
\pgfpathcurveto{\pgfqpoint{1.331cm}{0.67cm}}{\pgfqpoint{1.345cm}{0.704cm}}{\pgfqpoint{1.345cm}{0.741cm}}
\pgfusepath{fill}
\end{pgfscope}
\end{pgfscope}
\end{pgfscope}
\end{tikzpicture}
}}}

\newcommand{\tthreeone}[1]{#1^{\!\resizebox{0.6em}{!}{% Created by Eps2pgf 0.7.0 (build on 2008-08-24) on Wed Apr 25 13:02:21 WEST 2018
\begin{tikzpicture}
\pgfpathmoveto{\pgfqpoint{0cm}{-0.035cm}}
\pgfpathlineto{\pgfqpoint{1.376cm}{-0.035cm}}
\pgfpathlineto{\pgfqpoint{1.376cm}{1.552cm}}
\pgfpathlineto{\pgfqpoint{0cm}{1.552cm}}
\pgfpathclose
\pgfusepath{clip}
\begin{pgfscope}
\begin{pgfscope}
\pgfpathmoveto{\pgfqpoint{0cm}{-0.035cm}}
\pgfpathlineto{\pgfqpoint{1.376cm}{-0.035cm}}
\pgfpathlineto{\pgfqpoint{1.376cm}{1.552cm}}
\pgfpathlineto{\pgfqpoint{0cm}{1.552cm}}
\pgfpathclose
\pgfusepath{clip}
\begin{pgfscope}
\begin{pgfscope}
\pgfsetdash{}{0cm}
\pgfsetlinewidth{0.818mm}
\pgfsetroundcap
\pgfsetroundjoin
\pgfsetmiterlimit{7.0}
\definecolor{eps2pgf_color}{gray}{0}\pgfsetstrokecolor{eps2pgf_color}\pgfsetfillcolor{eps2pgf_color}
\pgfpathmoveto{\pgfqpoint{0.117cm}{1.421cm}}
\pgfpathlineto{\pgfqpoint{0.682cm}{0.671cm}}
\pgfpathlineto{\pgfqpoint{1.246cm}{1.421cm}}
\pgfusepath{stroke}
\end{pgfscope}
\definecolor{eps2pgf_color}{gray}{0}\pgfsetstrokecolor{eps2pgf_color}\pgfsetfillcolor{eps2pgf_color}
\pgfpathmoveto{\pgfqpoint{0.273cm}{1.395cm}}
\pgfpathcurveto{\pgfqpoint{0.273cm}{1.432cm}}{\pgfqpoint{0.259cm}{1.467cm}}{\pgfqpoint{0.233cm}{1.492cm}}
\pgfpathcurveto{\pgfqpoint{0.207cm}{1.518cm}}{\pgfqpoint{0.173cm}{1.532cm}}{\pgfqpoint{0.137cm}{1.532cm}}
\pgfpathcurveto{\pgfqpoint{0.1cm}{1.532cm}}{\pgfqpoint{0.066cm}{1.518cm}}{\pgfqpoint{0.04cm}{1.492cm}}
\pgfpathcurveto{\pgfqpoint{0.014cm}{1.467cm}}{\pgfqpoint{0cm}{1.432cm}}{\pgfqpoint{0cm}{1.395cm}}
\pgfpathcurveto{\pgfqpoint{0cm}{1.359cm}}{\pgfqpoint{0.014cm}{1.324cm}}{\pgfqpoint{0.04cm}{1.299cm}}
\pgfpathcurveto{\pgfqpoint{0.066cm}{1.273cm}}{\pgfqpoint{0.1cm}{1.258cm}}{\pgfqpoint{0.137cm}{1.258cm}}
\pgfpathcurveto{\pgfqpoint{0.173cm}{1.258cm}}{\pgfqpoint{0.207cm}{1.273cm}}{\pgfqpoint{0.233cm}{1.299cm}}
\pgfpathcurveto{\pgfqpoint{0.259cm}{1.324cm}}{\pgfqpoint{0.273cm}{1.359cm}}{\pgfqpoint{0.273cm}{1.395cm}}
\pgfusepath{fill}
\begin{pgfscope}
\pgfsetdash{}{0cm}
\pgfsetlinewidth{0.818mm}
\pgfsetmiterlimit{7.0}
\pgfpathmoveto{\pgfqpoint{0.682cm}{0.671cm}}
\pgfpathlineto{\pgfqpoint{0.679cm}{1.418cm}}
\pgfusepath{stroke}
\end{pgfscope}
\pgfpathmoveto{\pgfqpoint{0.815cm}{1.399cm}}
\pgfpathcurveto{\pgfqpoint{0.815cm}{1.435cm}}{\pgfqpoint{0.801cm}{1.47cm}}{\pgfqpoint{0.775cm}{1.496cm}}
\pgfpathcurveto{\pgfqpoint{0.75cm}{1.521cm}}{\pgfqpoint{0.715cm}{1.536cm}}{\pgfqpoint{0.679cm}{1.536cm}}
\pgfpathcurveto{\pgfqpoint{0.643cm}{1.536cm}}{\pgfqpoint{0.608cm}{1.521cm}}{\pgfqpoint{0.582cm}{1.496cm}}
\pgfpathcurveto{\pgfqpoint{0.557cm}{1.47cm}}{\pgfqpoint{0.542cm}{1.435cm}}{\pgfqpoint{0.542cm}{1.399cm}}
\pgfpathcurveto{\pgfqpoint{0.542cm}{1.363cm}}{\pgfqpoint{0.557cm}{1.328cm}}{\pgfqpoint{0.582cm}{1.302cm}}
\pgfpathcurveto{\pgfqpoint{0.608cm}{1.276cm}}{\pgfqpoint{0.643cm}{1.262cm}}{\pgfqpoint{0.679cm}{1.262cm}}
\pgfpathcurveto{\pgfqpoint{0.715cm}{1.262cm}}{\pgfqpoint{0.75cm}{1.276cm}}{\pgfqpoint{0.775cm}{1.302cm}}
\pgfpathcurveto{\pgfqpoint{0.801cm}{1.328cm}}{\pgfqpoint{0.815cm}{1.363cm}}{\pgfqpoint{0.815cm}{1.399cm}}
\pgfusepath{fill}
\pgfpathmoveto{\pgfqpoint{1.345cm}{1.371cm}}
\pgfpathcurveto{\pgfqpoint{1.345cm}{1.408cm}}{\pgfqpoint{1.331cm}{1.442cm}}{\pgfqpoint{1.305cm}{1.468cm}}
\pgfpathcurveto{\pgfqpoint{1.28cm}{1.494cm}}{\pgfqpoint{1.245cm}{1.508cm}}{\pgfqpoint{1.209cm}{1.508cm}}
\pgfpathcurveto{\pgfqpoint{1.172cm}{1.508cm}}{\pgfqpoint{1.138cm}{1.494cm}}{\pgfqpoint{1.112cm}{1.468cm}}
\pgfpathcurveto{\pgfqpoint{1.087cm}{1.442cm}}{\pgfqpoint{1.072cm}{1.408cm}}{\pgfqpoint{1.072cm}{1.371cm}}
\pgfpathcurveto{\pgfqpoint{1.072cm}{1.335cm}}{\pgfqpoint{1.087cm}{1.3cm}}{\pgfqpoint{1.112cm}{1.274cm}}
\pgfpathcurveto{\pgfqpoint{1.138cm}{1.249cm}}{\pgfqpoint{1.172cm}{1.234cm}}{\pgfqpoint{1.209cm}{1.234cm}}
\pgfpathcurveto{\pgfqpoint{1.245cm}{1.234cm}}{\pgfqpoint{1.28cm}{1.249cm}}{\pgfqpoint{1.305cm}{1.274cm}}
\pgfpathcurveto{\pgfqpoint{1.331cm}{1.3cm}}{\pgfqpoint{1.345cm}{1.335cm}}{\pgfqpoint{1.345cm}{1.371cm}}
\pgfusepath{fill}
\begin{pgfscope}
\pgfsetdash{}{0cm}
\pgfsetlinewidth{0.818mm}
\pgfsetroundcap
\pgfsetmiterlimit{4.0}
\pgfpathmoveto{\pgfqpoint{0.682cm}{0.671cm}}
\pgfpathlineto{\pgfqpoint{0.682cm}{0.042cm}}
\pgfusepath{stroke}
\end{pgfscope}
\end{pgfscope}
\end{pgfscope}
\end{pgfscope}
\end{tikzpicture}
}}}
\newcommand{\ttwoone}[1]{#1^{\!\resizebox{0.6em}{!}{% Created by Eps2pgf 0.7.0 (build on 2008-08-24) on Wed Apr 25 13:00:21 WEST 2018
\begin{tikzpicture}
\pgfpathmoveto{\pgfqpoint{0cm}{0cm}}
\pgfpathlineto{\pgfqpoint{1.376cm}{0cm}}
\pgfpathlineto{\pgfqpoint{1.376cm}{1.588cm}}
\pgfpathlineto{\pgfqpoint{0cm}{1.588cm}}
\pgfpathclose
\pgfusepath{clip}
\begin{pgfscope}
\begin{pgfscope}
\pgfpathmoveto{\pgfqpoint{0cm}{0cm}}
\pgfpathlineto{\pgfqpoint{1.376cm}{0cm}}
\pgfpathlineto{\pgfqpoint{1.376cm}{1.588cm}}
\pgfpathlineto{\pgfqpoint{0cm}{1.588cm}}
\pgfpathclose
\pgfusepath{clip}
\begin{pgfscope}
\begin{pgfscope}
\definecolor{eps2pgf_color}{gray}{0.976471}\pgfsetstrokecolor{eps2pgf_color}\pgfsetfillcolor{eps2pgf_color}
\pgfpathmoveto{\pgfqpoint{0cm}{0cm}}
\pgfpathlineto{\pgfqpoint{1.376cm}{0cm}}
\pgfpathlineto{\pgfqpoint{1.376cm}{1.588cm}}
\pgfpathlineto{\pgfqpoint{0cm}{1.588cm}}
\pgfpathclose
\pgfusepath{fill}
\end{pgfscope}
\begin{pgfscope}
\pgfsetdash{}{0cm}
\pgfsetlinewidth{0.818mm}
\pgfsetroundcap
\pgfsetroundjoin
\pgfsetmiterlimit{7.0}
\definecolor{eps2pgf_color}{gray}{0}\pgfsetstrokecolor{eps2pgf_color}\pgfsetfillcolor{eps2pgf_color}
\pgfpathmoveto{\pgfqpoint{0.117cm}{1.476cm}}
\pgfpathlineto{\pgfqpoint{0.682cm}{0.726cm}}
\pgfpathlineto{\pgfqpoint{1.246cm}{1.476cm}}
\pgfusepath{stroke}
\end{pgfscope}
\definecolor{eps2pgf_color}{gray}{0}\pgfsetstrokecolor{eps2pgf_color}\pgfsetfillcolor{eps2pgf_color}
\pgfpathmoveto{\pgfqpoint{0.273cm}{1.451cm}}
\pgfpathcurveto{\pgfqpoint{0.273cm}{1.487cm}}{\pgfqpoint{0.259cm}{1.522cm}}{\pgfqpoint{0.233cm}{1.547cm}}
\pgfpathcurveto{\pgfqpoint{0.207cm}{1.573cm}}{\pgfqpoint{0.173cm}{1.588cm}}{\pgfqpoint{0.137cm}{1.588cm}}
\pgfpathcurveto{\pgfqpoint{0.1cm}{1.588cm}}{\pgfqpoint{0.066cm}{1.573cm}}{\pgfqpoint{0.04cm}{1.547cm}}
\pgfpathcurveto{\pgfqpoint{0.014cm}{1.522cm}}{\pgfqpoint{0cm}{1.487cm}}{\pgfqpoint{0cm}{1.451cm}}
\pgfpathcurveto{\pgfqpoint{0cm}{1.414cm}}{\pgfqpoint{0.014cm}{1.379cm}}{\pgfqpoint{0.04cm}{1.354cm}}
\pgfpathcurveto{\pgfqpoint{0.066cm}{1.328cm}}{\pgfqpoint{0.1cm}{1.314cm}}{\pgfqpoint{0.137cm}{1.314cm}}
\pgfpathcurveto{\pgfqpoint{0.173cm}{1.314cm}}{\pgfqpoint{0.207cm}{1.328cm}}{\pgfqpoint{0.233cm}{1.354cm}}
\pgfpathcurveto{\pgfqpoint{0.259cm}{1.379cm}}{\pgfqpoint{0.273cm}{1.414cm}}{\pgfqpoint{0.273cm}{1.451cm}}
\pgfusepath{fill}
\pgfpathmoveto{\pgfqpoint{1.345cm}{1.426cm}}
\pgfpathcurveto{\pgfqpoint{1.345cm}{1.463cm}}{\pgfqpoint{1.331cm}{1.497cm}}{\pgfqpoint{1.305cm}{1.523cm}}
\pgfpathcurveto{\pgfqpoint{1.28cm}{1.549cm}}{\pgfqpoint{1.245cm}{1.563cm}}{\pgfqpoint{1.209cm}{1.563cm}}
\pgfpathcurveto{\pgfqpoint{1.172cm}{1.563cm}}{\pgfqpoint{1.138cm}{1.549cm}}{\pgfqpoint{1.112cm}{1.523cm}}
\pgfpathcurveto{\pgfqpoint{1.087cm}{1.497cm}}{\pgfqpoint{1.072cm}{1.463cm}}{\pgfqpoint{1.072cm}{1.426cm}}
\pgfpathcurveto{\pgfqpoint{1.072cm}{1.39cm}}{\pgfqpoint{1.087cm}{1.355cm}}{\pgfqpoint{1.112cm}{1.329cm}}
\pgfpathcurveto{\pgfqpoint{1.138cm}{1.304cm}}{\pgfqpoint{1.172cm}{1.289cm}}{\pgfqpoint{1.209cm}{1.289cm}}
\pgfpathcurveto{\pgfqpoint{1.245cm}{1.289cm}}{\pgfqpoint{1.28cm}{1.304cm}}{\pgfqpoint{1.305cm}{1.329cm}}
\pgfpathcurveto{\pgfqpoint{1.331cm}{1.355cm}}{\pgfqpoint{1.345cm}{1.39cm}}{\pgfqpoint{1.345cm}{1.426cm}}
\pgfusepath{fill}
\begin{pgfscope}
\pgfsetdash{}{0cm}
\pgfsetlinewidth{0.818mm}
\pgfsetroundcap
\pgfsetmiterlimit{4.0}
\pgfpathmoveto{\pgfqpoint{0.682cm}{0.726cm}}
\pgfpathlineto{\pgfqpoint{0.682cm}{0.097cm}}
\pgfusepath{stroke}
\end{pgfscope}
\end{pgfscope}
\end{pgfscope}
\end{pgfscope}
\end{tikzpicture}
}}}

\newcommand{\tthreethree}[1]{#1\tmrsup{\!\resizebox{!}{.8em}{\input{trees-tikz/33_tree.tikz}}}}
\newcommand{\tthreetwo}[1]{#1\tmrsup{\!\resizebox{!}{.8em}{\input{trees-tikz/32_tree.tikz}}}}
\newcommand{\tthreethreer}[1]{#1^{\!\resizebox{!}{.8em}{% Created by Eps2pgf 0.7.0 (build on 2008-08-24) on Wed Apr 25 13:03:22 WEST 2018
\begin{tikzpicture}
\pgfpathmoveto{\pgfqpoint{0cm}{-0.035cm}}
\pgfpathlineto{\pgfqpoint{1.976cm}{-0.035cm}}
\pgfpathlineto{\pgfqpoint{1.976cm}{1.94cm}}
\pgfpathlineto{\pgfqpoint{0cm}{1.94cm}}
\pgfpathclose
\pgfusepath{clip}
\begin{pgfscope}
\begin{pgfscope}
\pgfpathmoveto{\pgfqpoint{0cm}{-0.035cm}}
\pgfpathlineto{\pgfqpoint{1.976cm}{-0.035cm}}
\pgfpathlineto{\pgfqpoint{1.976cm}{1.94cm}}
\pgfpathlineto{\pgfqpoint{0cm}{1.94cm}}
\pgfpathclose
\pgfusepath{clip}
\begin{pgfscope}
\begin{pgfscope}
\pgfsetdash{}{0cm}
\pgfsetlinewidth{0.818mm}
\pgfsetroundcap
\pgfsetroundjoin
\pgfsetmiterlimit{7.0}
\definecolor{eps2pgf_color}{gray}{0}\pgfsetstrokecolor{eps2pgf_color}\pgfsetfillcolor{eps2pgf_color}
\pgfpathmoveto{\pgfqpoint{0.117cm}{1.815cm}}
\pgfpathlineto{\pgfqpoint{0.682cm}{1.065cm}}
\pgfpathlineto{\pgfqpoint{1.246cm}{1.815cm}}
\pgfusepath{stroke}
\end{pgfscope}
\definecolor{eps2pgf_color}{gray}{0}\pgfsetstrokecolor{eps2pgf_color}\pgfsetfillcolor{eps2pgf_color}
\pgfpathmoveto{\pgfqpoint{0.273cm}{1.789cm}}
\pgfpathcurveto{\pgfqpoint{0.273cm}{1.825cm}}{\pgfqpoint{0.259cm}{1.86cm}}{\pgfqpoint{0.233cm}{1.886cm}}
\pgfpathcurveto{\pgfqpoint{0.207cm}{1.912cm}}{\pgfqpoint{0.173cm}{1.926cm}}{\pgfqpoint{0.137cm}{1.926cm}}
\pgfpathcurveto{\pgfqpoint{0.1cm}{1.926cm}}{\pgfqpoint{0.066cm}{1.912cm}}{\pgfqpoint{0.04cm}{1.886cm}}
\pgfpathcurveto{\pgfqpoint{0.014cm}{1.86cm}}{\pgfqpoint{0cm}{1.825cm}}{\pgfqpoint{0cm}{1.789cm}}
\pgfpathcurveto{\pgfqpoint{0cm}{1.753cm}}{\pgfqpoint{0.014cm}{1.718cm}}{\pgfqpoint{0.04cm}{1.692cm}}
\pgfpathcurveto{\pgfqpoint{0.066cm}{1.667cm}}{\pgfqpoint{0.1cm}{1.652cm}}{\pgfqpoint{0.137cm}{1.652cm}}
\pgfpathcurveto{\pgfqpoint{0.173cm}{1.652cm}}{\pgfqpoint{0.207cm}{1.667cm}}{\pgfqpoint{0.233cm}{1.692cm}}
\pgfpathcurveto{\pgfqpoint{0.259cm}{1.718cm}}{\pgfqpoint{0.273cm}{1.753cm}}{\pgfqpoint{0.273cm}{1.789cm}}
\pgfusepath{fill}
\begin{pgfscope}
\pgfsetdash{}{0cm}
\pgfsetlinewidth{0.818mm}
\pgfsetmiterlimit{7.0}
\pgfpathmoveto{\pgfqpoint{0.682cm}{1.065cm}}
\pgfpathlineto{\pgfqpoint{0.679cm}{1.812cm}}
\pgfusepath{stroke}
\end{pgfscope}
\pgfpathmoveto{\pgfqpoint{0.815cm}{1.793cm}}
\pgfpathcurveto{\pgfqpoint{0.815cm}{1.829cm}}{\pgfqpoint{0.801cm}{1.864cm}}{\pgfqpoint{0.775cm}{1.89cm}}
\pgfpathcurveto{\pgfqpoint{0.75cm}{1.915cm}}{\pgfqpoint{0.715cm}{1.93cm}}{\pgfqpoint{0.679cm}{1.93cm}}
\pgfpathcurveto{\pgfqpoint{0.643cm}{1.93cm}}{\pgfqpoint{0.608cm}{1.915cm}}{\pgfqpoint{0.582cm}{1.89cm}}
\pgfpathcurveto{\pgfqpoint{0.557cm}{1.864cm}}{\pgfqpoint{0.542cm}{1.829cm}}{\pgfqpoint{0.542cm}{1.793cm}}
\pgfpathcurveto{\pgfqpoint{0.542cm}{1.756cm}}{\pgfqpoint{0.557cm}{1.722cm}}{\pgfqpoint{0.582cm}{1.696cm}}
\pgfpathcurveto{\pgfqpoint{0.608cm}{1.67cm}}{\pgfqpoint{0.643cm}{1.656cm}}{\pgfqpoint{0.679cm}{1.656cm}}
\pgfpathcurveto{\pgfqpoint{0.715cm}{1.656cm}}{\pgfqpoint{0.75cm}{1.67cm}}{\pgfqpoint{0.775cm}{1.696cm}}
\pgfpathcurveto{\pgfqpoint{0.801cm}{1.722cm}}{\pgfqpoint{0.815cm}{1.756cm}}{\pgfqpoint{0.815cm}{1.793cm}}
\pgfusepath{fill}
\pgfpathmoveto{\pgfqpoint{1.345cm}{1.765cm}}
\pgfpathcurveto{\pgfqpoint{1.345cm}{1.801cm}}{\pgfqpoint{1.331cm}{1.836cm}}{\pgfqpoint{1.305cm}{1.862cm}}
\pgfpathcurveto{\pgfqpoint{1.28cm}{1.887cm}}{\pgfqpoint{1.245cm}{1.902cm}}{\pgfqpoint{1.209cm}{1.902cm}}
\pgfpathcurveto{\pgfqpoint{1.172cm}{1.902cm}}{\pgfqpoint{1.138cm}{1.887cm}}{\pgfqpoint{1.112cm}{1.862cm}}
\pgfpathcurveto{\pgfqpoint{1.087cm}{1.836cm}}{\pgfqpoint{1.072cm}{1.801cm}}{\pgfqpoint{1.072cm}{1.765cm}}
\pgfpathcurveto{\pgfqpoint{1.072cm}{1.728cm}}{\pgfqpoint{1.087cm}{1.694cm}}{\pgfqpoint{1.112cm}{1.668cm}}
\pgfpathcurveto{\pgfqpoint{1.138cm}{1.642cm}}{\pgfqpoint{1.172cm}{1.628cm}}{\pgfqpoint{1.209cm}{1.628cm}}
\pgfpathcurveto{\pgfqpoint{1.245cm}{1.628cm}}{\pgfqpoint{1.28cm}{1.642cm}}{\pgfqpoint{1.305cm}{1.668cm}}
\pgfpathcurveto{\pgfqpoint{1.331cm}{1.694cm}}{\pgfqpoint{1.345cm}{1.728cm}}{\pgfqpoint{1.345cm}{1.765cm}}
\pgfusepath{fill}
\begin{pgfscope}
\pgfsetdash{}{0cm}
\pgfsetlinewidth{0.818mm}
\pgfsetroundcap
\pgfsetroundjoin
\pgfsetmiterlimit{7.0}
\pgfpathmoveto{\pgfqpoint{0.682cm}{1.065cm}}
\pgfpathlineto{\pgfqpoint{1.246cm}{0.315cm}}
\pgfpathlineto{\pgfqpoint{1.811cm}{1.065cm}}
\pgfusepath{stroke}
\end{pgfscope}
\pgfpathmoveto{\pgfqpoint{1.948cm}{1.065cm}}
\pgfpathcurveto{\pgfqpoint{1.948cm}{1.101cm}}{\pgfqpoint{1.933cm}{1.136cm}}{\pgfqpoint{1.907cm}{1.162cm}}
\pgfpathcurveto{\pgfqpoint{1.882cm}{1.187cm}}{\pgfqpoint{1.847cm}{1.202cm}}{\pgfqpoint{1.811cm}{1.202cm}}
\pgfpathcurveto{\pgfqpoint{1.775cm}{1.202cm}}{\pgfqpoint{1.74cm}{1.187cm}}{\pgfqpoint{1.714cm}{1.162cm}}
\pgfpathcurveto{\pgfqpoint{1.689cm}{1.136cm}}{\pgfqpoint{1.674cm}{1.101cm}}{\pgfqpoint{1.674cm}{1.065cm}}
\pgfpathcurveto{\pgfqpoint{1.674cm}{1.029cm}}{\pgfqpoint{1.689cm}{0.994cm}}{\pgfqpoint{1.714cm}{0.968cm}}
\pgfpathcurveto{\pgfqpoint{1.74cm}{0.942cm}}{\pgfqpoint{1.775cm}{0.928cm}}{\pgfqpoint{1.811cm}{0.928cm}}
\pgfpathcurveto{\pgfqpoint{1.847cm}{0.928cm}}{\pgfqpoint{1.882cm}{0.942cm}}{\pgfqpoint{1.907cm}{0.968cm}}
\pgfpathcurveto{\pgfqpoint{1.933cm}{0.994cm}}{\pgfqpoint{1.948cm}{1.029cm}}{\pgfqpoint{1.948cm}{1.065cm}}
\pgfusepath{fill}
\begin{pgfscope}
\pgfsetdash{}{0cm}
\pgfsetlinewidth{0.818mm}
\pgfsetmiterlimit{7.0}
\pgfpathmoveto{\pgfqpoint{1.246cm}{0.315cm}}
\pgfpathlineto{\pgfqpoint{1.244cm}{1.061cm}}
\pgfusepath{stroke}
\end{pgfscope}
\pgfpathmoveto{\pgfqpoint{1.38cm}{1.065cm}}
\pgfpathcurveto{\pgfqpoint{1.38cm}{1.101cm}}{\pgfqpoint{1.366cm}{1.136cm}}{\pgfqpoint{1.34cm}{1.162cm}}
\pgfpathcurveto{\pgfqpoint{1.315cm}{1.187cm}}{\pgfqpoint{1.28cm}{1.202cm}}{\pgfqpoint{1.244cm}{1.202cm}}
\pgfpathcurveto{\pgfqpoint{1.207cm}{1.202cm}}{\pgfqpoint{1.173cm}{1.187cm}}{\pgfqpoint{1.147cm}{1.162cm}}
\pgfpathcurveto{\pgfqpoint{1.121cm}{1.136cm}}{\pgfqpoint{1.107cm}{1.101cm}}{\pgfqpoint{1.107cm}{1.065cm}}
\pgfpathcurveto{\pgfqpoint{1.107cm}{1.029cm}}{\pgfqpoint{1.121cm}{0.994cm}}{\pgfqpoint{1.147cm}{0.968cm}}
\pgfpathcurveto{\pgfqpoint{1.173cm}{0.942cm}}{\pgfqpoint{1.207cm}{0.928cm}}{\pgfqpoint{1.244cm}{0.928cm}}
\pgfpathcurveto{\pgfqpoint{1.28cm}{0.928cm}}{\pgfqpoint{1.315cm}{0.942cm}}{\pgfqpoint{1.34cm}{0.968cm}}
\pgfpathcurveto{\pgfqpoint{1.366cm}{0.994cm}}{\pgfqpoint{1.38cm}{1.029cm}}{\pgfqpoint{1.38cm}{1.065cm}}
\pgfusepath{fill}
\begin{pgfscope}
\pgfsetdash{}{0cm}
\pgfsetlinewidth{0.818mm}
\pgfsetmiterlimit{4.0}
\pgfpathmoveto{\pgfqpoint{1.383cm}{0.178cm}}
\pgfpathcurveto{\pgfqpoint{1.383cm}{0.214cm}}{\pgfqpoint{1.369cm}{0.249cm}}{\pgfqpoint{1.343cm}{0.275cm}}
\pgfpathcurveto{\pgfqpoint{1.317cm}{0.3cm}}{\pgfqpoint{1.283cm}{0.315cm}}{\pgfqpoint{1.246cm}{0.315cm}}
\pgfpathcurveto{\pgfqpoint{1.21cm}{0.315cm}}{\pgfqpoint{1.175cm}{0.3cm}}{\pgfqpoint{1.15cm}{0.275cm}}
\pgfpathcurveto{\pgfqpoint{1.124cm}{0.249cm}}{\pgfqpoint{1.11cm}{0.214cm}}{\pgfqpoint{1.11cm}{0.178cm}}
\pgfpathcurveto{\pgfqpoint{1.11cm}{0.141cm}}{\pgfqpoint{1.124cm}{0.107cm}}{\pgfqpoint{1.15cm}{0.081cm}}
\pgfpathcurveto{\pgfqpoint{1.175cm}{0.055cm}}{\pgfqpoint{1.21cm}{0.041cm}}{\pgfqpoint{1.246cm}{0.041cm}}
\pgfpathcurveto{\pgfqpoint{1.283cm}{0.041cm}}{\pgfqpoint{1.317cm}{0.055cm}}{\pgfqpoint{1.343cm}{0.081cm}}
\pgfpathcurveto{\pgfqpoint{1.369cm}{0.107cm}}{\pgfqpoint{1.383cm}{0.141cm}}{\pgfqpoint{1.383cm}{0.178cm}}
\pgfusepath{stroke}
\end{pgfscope}
\end{pgfscope}
\end{pgfscope}
\end{pgfscope}
\end{tikzpicture}
}}}
\newcommand{\tthreethreerprime}[1]{#1^{\prime\!\resizebox{!}{.8em}{% Created by Eps2pgf 0.7.0 (build on 2008-08-24) on Wed Apr 25 13:03:22 WEST 2018
\begin{tikzpicture}
\pgfpathmoveto{\pgfqpoint{0cm}{-0.035cm}}
\pgfpathlineto{\pgfqpoint{1.976cm}{-0.035cm}}
\pgfpathlineto{\pgfqpoint{1.976cm}{1.94cm}}
\pgfpathlineto{\pgfqpoint{0cm}{1.94cm}}
\pgfpathclose
\pgfusepath{clip}
\begin{pgfscope}
\begin{pgfscope}
\pgfpathmoveto{\pgfqpoint{0cm}{-0.035cm}}
\pgfpathlineto{\pgfqpoint{1.976cm}{-0.035cm}}
\pgfpathlineto{\pgfqpoint{1.976cm}{1.94cm}}
\pgfpathlineto{\pgfqpoint{0cm}{1.94cm}}
\pgfpathclose
\pgfusepath{clip}
\begin{pgfscope}
\begin{pgfscope}
\pgfsetdash{}{0cm}
\pgfsetlinewidth{0.818mm}
\pgfsetroundcap
\pgfsetroundjoin
\pgfsetmiterlimit{7.0}
\definecolor{eps2pgf_color}{gray}{0}\pgfsetstrokecolor{eps2pgf_color}\pgfsetfillcolor{eps2pgf_color}
\pgfpathmoveto{\pgfqpoint{0.117cm}{1.815cm}}
\pgfpathlineto{\pgfqpoint{0.682cm}{1.065cm}}
\pgfpathlineto{\pgfqpoint{1.246cm}{1.815cm}}
\pgfusepath{stroke}
\end{pgfscope}
\definecolor{eps2pgf_color}{gray}{0}\pgfsetstrokecolor{eps2pgf_color}\pgfsetfillcolor{eps2pgf_color}
\pgfpathmoveto{\pgfqpoint{0.273cm}{1.789cm}}
\pgfpathcurveto{\pgfqpoint{0.273cm}{1.825cm}}{\pgfqpoint{0.259cm}{1.86cm}}{\pgfqpoint{0.233cm}{1.886cm}}
\pgfpathcurveto{\pgfqpoint{0.207cm}{1.912cm}}{\pgfqpoint{0.173cm}{1.926cm}}{\pgfqpoint{0.137cm}{1.926cm}}
\pgfpathcurveto{\pgfqpoint{0.1cm}{1.926cm}}{\pgfqpoint{0.066cm}{1.912cm}}{\pgfqpoint{0.04cm}{1.886cm}}
\pgfpathcurveto{\pgfqpoint{0.014cm}{1.86cm}}{\pgfqpoint{0cm}{1.825cm}}{\pgfqpoint{0cm}{1.789cm}}
\pgfpathcurveto{\pgfqpoint{0cm}{1.753cm}}{\pgfqpoint{0.014cm}{1.718cm}}{\pgfqpoint{0.04cm}{1.692cm}}
\pgfpathcurveto{\pgfqpoint{0.066cm}{1.667cm}}{\pgfqpoint{0.1cm}{1.652cm}}{\pgfqpoint{0.137cm}{1.652cm}}
\pgfpathcurveto{\pgfqpoint{0.173cm}{1.652cm}}{\pgfqpoint{0.207cm}{1.667cm}}{\pgfqpoint{0.233cm}{1.692cm}}
\pgfpathcurveto{\pgfqpoint{0.259cm}{1.718cm}}{\pgfqpoint{0.273cm}{1.753cm}}{\pgfqpoint{0.273cm}{1.789cm}}
\pgfusepath{fill}
\begin{pgfscope}
\pgfsetdash{}{0cm}
\pgfsetlinewidth{0.818mm}
\pgfsetmiterlimit{7.0}
\pgfpathmoveto{\pgfqpoint{0.682cm}{1.065cm}}
\pgfpathlineto{\pgfqpoint{0.679cm}{1.812cm}}
\pgfusepath{stroke}
\end{pgfscope}
\pgfpathmoveto{\pgfqpoint{0.815cm}{1.793cm}}
\pgfpathcurveto{\pgfqpoint{0.815cm}{1.829cm}}{\pgfqpoint{0.801cm}{1.864cm}}{\pgfqpoint{0.775cm}{1.89cm}}
\pgfpathcurveto{\pgfqpoint{0.75cm}{1.915cm}}{\pgfqpoint{0.715cm}{1.93cm}}{\pgfqpoint{0.679cm}{1.93cm}}
\pgfpathcurveto{\pgfqpoint{0.643cm}{1.93cm}}{\pgfqpoint{0.608cm}{1.915cm}}{\pgfqpoint{0.582cm}{1.89cm}}
\pgfpathcurveto{\pgfqpoint{0.557cm}{1.864cm}}{\pgfqpoint{0.542cm}{1.829cm}}{\pgfqpoint{0.542cm}{1.793cm}}
\pgfpathcurveto{\pgfqpoint{0.542cm}{1.756cm}}{\pgfqpoint{0.557cm}{1.722cm}}{\pgfqpoint{0.582cm}{1.696cm}}
\pgfpathcurveto{\pgfqpoint{0.608cm}{1.67cm}}{\pgfqpoint{0.643cm}{1.656cm}}{\pgfqpoint{0.679cm}{1.656cm}}
\pgfpathcurveto{\pgfqpoint{0.715cm}{1.656cm}}{\pgfqpoint{0.75cm}{1.67cm}}{\pgfqpoint{0.775cm}{1.696cm}}
\pgfpathcurveto{\pgfqpoint{0.801cm}{1.722cm}}{\pgfqpoint{0.815cm}{1.756cm}}{\pgfqpoint{0.815cm}{1.793cm}}
\pgfusepath{fill}
\pgfpathmoveto{\pgfqpoint{1.345cm}{1.765cm}}
\pgfpathcurveto{\pgfqpoint{1.345cm}{1.801cm}}{\pgfqpoint{1.331cm}{1.836cm}}{\pgfqpoint{1.305cm}{1.862cm}}
\pgfpathcurveto{\pgfqpoint{1.28cm}{1.887cm}}{\pgfqpoint{1.245cm}{1.902cm}}{\pgfqpoint{1.209cm}{1.902cm}}
\pgfpathcurveto{\pgfqpoint{1.172cm}{1.902cm}}{\pgfqpoint{1.138cm}{1.887cm}}{\pgfqpoint{1.112cm}{1.862cm}}
\pgfpathcurveto{\pgfqpoint{1.087cm}{1.836cm}}{\pgfqpoint{1.072cm}{1.801cm}}{\pgfqpoint{1.072cm}{1.765cm}}
\pgfpathcurveto{\pgfqpoint{1.072cm}{1.728cm}}{\pgfqpoint{1.087cm}{1.694cm}}{\pgfqpoint{1.112cm}{1.668cm}}
\pgfpathcurveto{\pgfqpoint{1.138cm}{1.642cm}}{\pgfqpoint{1.172cm}{1.628cm}}{\pgfqpoint{1.209cm}{1.628cm}}
\pgfpathcurveto{\pgfqpoint{1.245cm}{1.628cm}}{\pgfqpoint{1.28cm}{1.642cm}}{\pgfqpoint{1.305cm}{1.668cm}}
\pgfpathcurveto{\pgfqpoint{1.331cm}{1.694cm}}{\pgfqpoint{1.345cm}{1.728cm}}{\pgfqpoint{1.345cm}{1.765cm}}
\pgfusepath{fill}
\begin{pgfscope}
\pgfsetdash{}{0cm}
\pgfsetlinewidth{0.818mm}
\pgfsetroundcap
\pgfsetroundjoin
\pgfsetmiterlimit{7.0}
\pgfpathmoveto{\pgfqpoint{0.682cm}{1.065cm}}
\pgfpathlineto{\pgfqpoint{1.246cm}{0.315cm}}
\pgfpathlineto{\pgfqpoint{1.811cm}{1.065cm}}
\pgfusepath{stroke}
\end{pgfscope}
\pgfpathmoveto{\pgfqpoint{1.948cm}{1.065cm}}
\pgfpathcurveto{\pgfqpoint{1.948cm}{1.101cm}}{\pgfqpoint{1.933cm}{1.136cm}}{\pgfqpoint{1.907cm}{1.162cm}}
\pgfpathcurveto{\pgfqpoint{1.882cm}{1.187cm}}{\pgfqpoint{1.847cm}{1.202cm}}{\pgfqpoint{1.811cm}{1.202cm}}
\pgfpathcurveto{\pgfqpoint{1.775cm}{1.202cm}}{\pgfqpoint{1.74cm}{1.187cm}}{\pgfqpoint{1.714cm}{1.162cm}}
\pgfpathcurveto{\pgfqpoint{1.689cm}{1.136cm}}{\pgfqpoint{1.674cm}{1.101cm}}{\pgfqpoint{1.674cm}{1.065cm}}
\pgfpathcurveto{\pgfqpoint{1.674cm}{1.029cm}}{\pgfqpoint{1.689cm}{0.994cm}}{\pgfqpoint{1.714cm}{0.968cm}}
\pgfpathcurveto{\pgfqpoint{1.74cm}{0.942cm}}{\pgfqpoint{1.775cm}{0.928cm}}{\pgfqpoint{1.811cm}{0.928cm}}
\pgfpathcurveto{\pgfqpoint{1.847cm}{0.928cm}}{\pgfqpoint{1.882cm}{0.942cm}}{\pgfqpoint{1.907cm}{0.968cm}}
\pgfpathcurveto{\pgfqpoint{1.933cm}{0.994cm}}{\pgfqpoint{1.948cm}{1.029cm}}{\pgfqpoint{1.948cm}{1.065cm}}
\pgfusepath{fill}
\begin{pgfscope}
\pgfsetdash{}{0cm}
\pgfsetlinewidth{0.818mm}
\pgfsetmiterlimit{7.0}
\pgfpathmoveto{\pgfqpoint{1.246cm}{0.315cm}}
\pgfpathlineto{\pgfqpoint{1.244cm}{1.061cm}}
\pgfusepath{stroke}
\end{pgfscope}
\pgfpathmoveto{\pgfqpoint{1.38cm}{1.065cm}}
\pgfpathcurveto{\pgfqpoint{1.38cm}{1.101cm}}{\pgfqpoint{1.366cm}{1.136cm}}{\pgfqpoint{1.34cm}{1.162cm}}
\pgfpathcurveto{\pgfqpoint{1.315cm}{1.187cm}}{\pgfqpoint{1.28cm}{1.202cm}}{\pgfqpoint{1.244cm}{1.202cm}}
\pgfpathcurveto{\pgfqpoint{1.207cm}{1.202cm}}{\pgfqpoint{1.173cm}{1.187cm}}{\pgfqpoint{1.147cm}{1.162cm}}
\pgfpathcurveto{\pgfqpoint{1.121cm}{1.136cm}}{\pgfqpoint{1.107cm}{1.101cm}}{\pgfqpoint{1.107cm}{1.065cm}}
\pgfpathcurveto{\pgfqpoint{1.107cm}{1.029cm}}{\pgfqpoint{1.121cm}{0.994cm}}{\pgfqpoint{1.147cm}{0.968cm}}
\pgfpathcurveto{\pgfqpoint{1.173cm}{0.942cm}}{\pgfqpoint{1.207cm}{0.928cm}}{\pgfqpoint{1.244cm}{0.928cm}}
\pgfpathcurveto{\pgfqpoint{1.28cm}{0.928cm}}{\pgfqpoint{1.315cm}{0.942cm}}{\pgfqpoint{1.34cm}{0.968cm}}
\pgfpathcurveto{\pgfqpoint{1.366cm}{0.994cm}}{\pgfqpoint{1.38cm}{1.029cm}}{\pgfqpoint{1.38cm}{1.065cm}}
\pgfusepath{fill}
\begin{pgfscope}
\pgfsetdash{}{0cm}
\pgfsetlinewidth{0.818mm}
\pgfsetmiterlimit{4.0}
\pgfpathmoveto{\pgfqpoint{1.383cm}{0.178cm}}
\pgfpathcurveto{\pgfqpoint{1.383cm}{0.214cm}}{\pgfqpoint{1.369cm}{0.249cm}}{\pgfqpoint{1.343cm}{0.275cm}}
\pgfpathcurveto{\pgfqpoint{1.317cm}{0.3cm}}{\pgfqpoint{1.283cm}{0.315cm}}{\pgfqpoint{1.246cm}{0.315cm}}
\pgfpathcurveto{\pgfqpoint{1.21cm}{0.315cm}}{\pgfqpoint{1.175cm}{0.3cm}}{\pgfqpoint{1.15cm}{0.275cm}}
\pgfpathcurveto{\pgfqpoint{1.124cm}{0.249cm}}{\pgfqpoint{1.11cm}{0.214cm}}{\pgfqpoint{1.11cm}{0.178cm}}
\pgfpathcurveto{\pgfqpoint{1.11cm}{0.141cm}}{\pgfqpoint{1.124cm}{0.107cm}}{\pgfqpoint{1.15cm}{0.081cm}}
\pgfpathcurveto{\pgfqpoint{1.175cm}{0.055cm}}{\pgfqpoint{1.21cm}{0.041cm}}{\pgfqpoint{1.246cm}{0.041cm}}
\pgfpathcurveto{\pgfqpoint{1.283cm}{0.041cm}}{\pgfqpoint{1.317cm}{0.055cm}}{\pgfqpoint{1.343cm}{0.081cm}}
\pgfpathcurveto{\pgfqpoint{1.369cm}{0.107cm}}{\pgfqpoint{1.383cm}{0.141cm}}{\pgfqpoint{1.383cm}{0.178cm}}
\pgfusepath{stroke}
\end{pgfscope}
\end{pgfscope}
\end{pgfscope}
\end{pgfscope}
\end{tikzpicture}
}}}
\newcommand{\ttwothreer}[1]{#1^{\!\resizebox{!}{.8em}{\input{trees-tikz/23_tree_res.tikz}}}}
\newcommand{\tthreetwor}[1]{#1^{\!\resizebox{!}{.8em}{\input{trees-tikz/32_tree_res.tikz}}}}



\newcommand{\CC}{\mathscr{C} \hspace{.1em}}
\newcommand{\CD}{D{\hspace{.1em}}}
\newcommand{\CF}{F{\hspace{.1em}}}
\newcommand{\CB}{B{\hspace{.1em}}}
\newcommand{\CA}{A{\hspace{.1em}}}
\newcommand{\DD}{D{\hspace{.1em}}}
\newcommand{\LL}{\mathscr{L} \hspace{.2em}}
\newcommand{\Q}{\mathscr{Q} \hspace{.2em}}
\newcommand{\SSS}{S{\hspace{.1em}}}
\newcommand{\PP}{P{\hspace{.1em}}}
\newcommand{\XX}{X}
\newcommand{\UU}{\mathscr{U}}
\newcommand{\VV}{V}
%

%

\newcommand{\axiomtext}{Assumption}

\begin{document}

\title{A PDE construction of \\ the Euclidean $\Phi^4_3$ quantum field theory}

\author[1]{Massimiliano Gubinelli}
\author[2]{Martina Hofmanov\'a}
\affil[1]{\small Hausdorff Center for Mathematics\\ \& Institute for Applied Mathematics\\ University of Bonn\\
Endenicher Allee 60\\
53115 Bonn, Germany.  \href{mailto:gubinelli@iam.uni-bonn.de}{gubinelli@iam.uni-bonn.de} }
%\email
%
\affil[2]{\small Fakult\"at f\"ur Mathematik, Universit\"at Bielefeld, Postfach 10 01 31, 33501 Bielefeld, Germany. \href{mailto:hofmanova@math.uni-bielefeld.de}{hofmanova@math.uni-bielefeld.de}}
%\email{hofmanov@math.tu-berlin.de}


\maketitle

\begin{abstract}
  We present a new construction of the Euclidean $\Phi^4$ quantum
  field theory on $\mathbb{R}^3$ based on PDE arguments. More precisely, we
  consider an approximation of the stochastic quantization equation on
  $\mathbb{R}^3$ defined on a periodic lattice of mesh size $\varepsilon$ and
  side length $M$. We introduce 
  a new renormalized energy method in weighted spaces and prove tightness of the corresponding Gibbs measures as
  $\varepsilon \rightarrow 0$, $M \rightarrow \infty$. Every limit point is non-Gaussian and satisfies reflection positivity, translation invariance and stretched exponential integrability. These properties allow to verify the Osterwalder--Schrader axioms for a  Euclidean QFT apart from rotation invariance and clustering. Our argument applies to arbitrary positive coupling constant, to multicomponent models with $O(N)$ symmetry and to some long-range variants. Moreover, we establish an integration by parts formula leading to the hierarchy of Dyson--Schwinger equations for the Euclidean correlation functions. To this end, we identify the renormalized cubic term as a \emph{distribution} on the space of Euclidean fields.
\end{abstract}

\tmkeywords{stochastic quantization, Euclidean quantum field theory,
paracontrolled calculus, integration by parts formulas, Dyson--Schwinger equations}

{\tableofcontents}


\section{Introduction}\label{sec:intro}



Let 
$\Lambda_{M, \varepsilon} = ( (\varepsilon\mathbb{Z})/( M\mathbb{Z}))^3$ be a   periodic lattice with
mesh size $\varepsilon$ and side length $M$ where $M/(2\varepsilon)\in\mathbb{N} .$ Consider the family $(\nu_{M, \varepsilon})_{M,\varepsilon}$ 
of Gibbs measures for the scalar field $\varphi:\Lambda_{M, \varepsilon}\to \mathbb{R}$, given by
\begin{equation}
 \mathd \nu_{M, \varepsilon} \propto \exp \left\{ - 2 \varepsilon^d  \sum_{x \in\Lambda_{M, \varepsilon}} \left[
  \frac{\lambda}{4} | \varphi_x |^4 + \frac{- 3 \lambda a_{M, \varepsilon} + 3
  \lambda^2 b_{M, \varepsilon} + m^2}{2} | \varphi_x |^2 + \frac{1}{2} |
  \nabla_{\varepsilon} \varphi_x |^2 \right] \right\}  \prod_{x \in \Lambda_{M,
  \varepsilon}}\!\! \mathd \varphi_x, \label{eq:gibbs}
\end{equation}
where $\nabla_{\varepsilon}$ denotes the discrete gradient and $a_{M,
\varepsilon}, b_{M, \varepsilon}$ are suitable renormalization constants, $m^2
\in \mathbb{R}$ is called the \tmtextit{mass} and $\lambda > 0$ the
\tmtextit{coupling constant}.  The numerical factor in the exponential is chosen in order to simplify the form of the  stochastic quantization equation \eqref{eq:P4} below. 
The main result of this paper is the following.
\begin{theorem}
  \label{th:main}There exists a choice of the sequence $(a_{M, \varepsilon},
  b_{M, \varepsilon})_{M, \varepsilon}$ such that for any $\lambda > 0$ and
  $m^2 \in \mathbb{R}$, the family of measures $(\nu_{M, \varepsilon})_{M,
  \varepsilon}$ appropriately extended to $\mathcal{S}' (\mathbb{R}^3)$ is tight.
  Every accumulation point $\nu$ is translation invariant, reflection positive
  and non-Gaussian. In addition, for every small $\kappa > 0$ there exists $\sigma >
  0$, $\beta > 0$ and $\upsilon = O (\kappa) > 0$ such that
  \begin{equation}
    \int_{\mathcal{S}' (\mathbb{R}^3)} \exp\{{\beta \| (1 + | \cdot |^2)^{-
    \sigma} \varphi \|_{H^{- 1 / 2 - \kappa}}^{1 - \upsilon}}\} \nu (\mathd
    \varphi) < \infty . \label{eq:exp-int-intro}
  \end{equation}
  Every $\nu$ satisfies an integration by parts formula which leads
  to the hierarchy of the~Dyson--Schwinger equations for $n$-point correlation
  functions.
\end{theorem}


For the precise definition of translation invariance and reflection positivity (RP) we refer the reader to Section~\ref{s:ax}. 


\medskip




The proof of convergence of the family $(\nu_{M, \varepsilon})_{M,  \varepsilon}$ 
has been one of the major achievements of the constructive quantum field theory (CQFT)
program~{\cite{velo_constructive_1973,simon_po2_1974,MR887102,rivasseau_perturbative_1991,baez_introduction_1992,jaffe_constructive_2000,MR2391806,summers_perspective_2012}}
which flourished in the 70s and 80s. 
%
In the two dimensional setting the existence of an
analogous object has been one of the early successes of CQFT, while in four and more dimensions (after a proper normalization) any accumulation point is necessarily Gaussian~{\cite{fernandez_random_1992}}.


The existence of an Euclidean invariant and reflection positive limit $\nu$ (plus some technical conditions) 
implies the existence of a
relativistic quantum field theory in the Minkowski space-time $\mathbb{R}^{1
+ 2}$ which satisfies the Wightman axioms~{\cite{MR0436800}}. This is a minimal set of
axioms capturing the essence of the combination of quantum mechanics and
special relativity. The translation from the commutative probabilistic setting
(Euclidean QFT) to the non-commutative Minkowski QFT setting is operated by a
set of axioms introduced by
Osterwalder--Schrader (OS)~{\cite{osterwalder_axioms_1973,osterwalder_axioms_1975}}
for the correlation functions of the measure $\nu$. These  are called Schwinger
functions or Euclidean correlation functions and shall satisfy: a
regularity axiom, a Euclidean invariance axiom, a reflection
positivity axiom, a symmetry axiom and a cluster property. \

Euclidean invariance and reflection positivity conspire against each other. 
Models which easily satisfy one property hardly satisfy the other  if they
are not Gaussian, or simple transformations thereof, see e.g.~{\cite{albeverio_HC1_2002,albeverio_hida_2009}}. Reflection
positivity itself is a property whose crucial importance for probability theory and
mathematical physics~{\cite{kotecky_reflection_2009,jaffe_reflection_2018}}
and representation theory~{\cite{neeb_reflection_2018,jorgensen_reflection_2018}} has been one of
the byproducts of the constructive effort.



\medskip

The original proof of the OS axioms, along with additional properties of the
limiting measures which are called $\Phi^4_3$
measures, is scattered in a series of works covering almost a decade.
Glimm~{\cite{glimm_boson_1968}} first proved the existence of the Hamiltonian
(with an infrared regularization) in the Minkowski setting. Then Glimm and
Jaffe~{\cite{glimm_positivity_1973}} introduced the \tmtextit{phase cell
expansion} of the regularized Schwinger functions, which revealed itself a
powerful and robust tool (albeit complex to digest) in order to handle the local
 singularities of Euclidean quantum fields and  to prove the ultraviolet stability
in finite volume (i.e. the limit $\varepsilon\to 0$ with $M$ fixed). The proof of existence of the infinite volume limit ($M \to \infty$) and the
verification of Osterwalder--Schrader
axioms
%
was then
completed, for $\lambda$
small and using cluster expansion methods, independently by Feldman and Osterwalder~{\cite{feldman_wightman_1976}} and by Magnen and
S\'en\'eor~{\cite{magnen_infinite_1976}}. Finally
the work of Seiler and Simon~{\cite{seiler_nelsons_1976}} allowed to extend
the existence result to all $\lambda > 0$ (this is claimed in~{\cite{MR887102}} even though we could not find a clear statement in Seiler and Simon's paper). Equations of
motion for the quantum fields were established by Feldman and
R{\c{a}}czka~{\cite{feldman77}}.

\medskip

Since this first, complete, construction, there have been several other
attempts to simplify (both technically and conceptually) the arguments and
the $\Phi^4_3$ measure has been since considered a test bed for various CQFT
techniques. There exists at least six methods of  proof: the original \emph{phase
cell method} of Glimm and Jaffe extended by Feldman and
Osterwalder~{\cite{feldman_wightman_1976}}, Magnen and
S\'en\'eor~{\cite{magnen_infinite_1976}}  and Park~{\cite{park_convergence_1977}}
(among others), the probabilistic approach of Benfatto, Cassandro, Gallavotti,
Nicol{\'o}, Olivieri, Presutti and
Schiacciatelli~{\cite{benfatto_probabilistic_1978}}, the \emph{block average method}
of Ba{\l}aban~{\cite{MR733476}} revisited by Dimock
in~{\cite{dimock_renormalization_2013_1,dimock_renormalization_2013_2,dimock_renormalization_2014_3}},
the wavelet method of Battle--Federbush~{\cite{battle_wavelets_1999}}, the
\tmtextit{skeleton inequalities method} of Brydges, Fr{\"o}hlich,
Sokal~{\cite{MR723546}}, the work of Watanabe on rotation invariance~\cite{watanabe_block_1989} via the renormalization group method of Gaw\k{e}dzki and Kupiainen~\cite{gawpolhk_edzki_asymptotic_1986}, and more recently the {renormalization group method}
 of Brydges, Dimock and Hurd~{\cite{brydges_short_1995}}.

It should be said that, apart from the Glimm--Jaffe--Feldman--Osterwalder--Magnen--S\'en\'eor
result, none of the additional constructions seems to be as complete and to
verify explicitly all the OS axioms. As Jaffe~{\cite{MR2391806}} remarks:

\begin{quotation}
  ``Not only should one give a transparent proof of the dimension $d = 3$ 
  construction, but as explained to me by Gelfand [private communication], one
  should make it sufficiently attractive that probabilists will take
  cognizance of the existence of a wonderful mathematical object.''
\end{quotation}


\medskip
The proof of Theorem~\ref{th:main} uses tools from the PDE theory as
well as recent advances in the field of \tmtextit{singular SPDEs}, without using any input 
 from traditional CQFT.
It applies to all  values of the coupling parameter $\lambda>0$ as well as to natural
extensions to  $N$-dimensional vectorial and long-range variants of the model. 
 
 Our methods are  very different from all the known constructions we
enumerated above. In particular, we do not rely on any of the standard tools
like cluster expansion or correlation inequalities or skeleton inequalities, and therefore our approach
brings a new perspective to this extensively investigated classical problem,
with respect to the removal of both ultraviolet and infrared regularizations.

Showing  invariance under translation, reflection positivity, the regularity
axiom of Osterwalder and Schrader  and the non-Gaussianity of the measure, we go a long
way (albeit not fully reaching the goal) to a complete independent construction of the $\Phi^4_3$ quantum field theory.
Furthermore, the integration by parts formula that we are able to establish leads to the hierarchy of the
Dyson--Schwinger equations for the Schwinger functions of the measure.

The key idea is to use a dynamical description of the approximate measure which
relies on an additional random source term which is Gaussian, in the spirit of the
\tmtextit{stochastic quantization} approach introduced by
Nelson~{\cite{nelson1966,MR0214150}} and Parisi and
Wu~{\cite{parisi_perturbation_1981}} (with a precursor in a technical report
of Symanzik~{\cite{Symanzik1964}}).

The concept of \emph{stochastic quantization} refers to the introduction of a reversible
stochastic dynamics which has the target measure as the invariant measure, here in
particular the $\Phi^4_d$ measure in $d$ dimensions. The rigorous study of
the stochastic quantization for the two dimensional version of the $\Phi^4$ theory
has been first initiated by Jona-Lasinio and
Mitter~{\cite{jona-lasinio_stochastic_1985}} in finite volume and by Borkar,
Chari and Mitter~{\cite{borkar_stochastic_1988}} in infinite volume. A natural
$d = 2$ local dynamics has been subsequently constructed by Albeverio and
R{\"o}ckner~{\cite{albeverio_stochastic_1991}} using Dirichlet forms in
infinite dimensions. Later on, Da Prato and
Debussche~{\cite{da_prato_strong_2003}} have shown for the first time the
existence of strong solutions to the stochastic dynamics in finite volume.
Da Prato and Debussche have introduced an innovative use of a mixture of probabilistic and
PDE techniques and constitute a landmark in the development of PDE techniques
to study stochastic analysis problems.  Similar methods have been used by McKean~\cite{mckean_1995,mckean_1995_err} and Bourgain~\cite{bourgain_invariant_1996} in the context of random data deterministic PDEs. Mourrat and
Weber~{\cite{MW17}} have subsequently shown the existence and
uniqueness of the stochastic dynamics globally in space and time. For the $d =
1$ dimensional variant, which is substantially simpler and does not require
renormalization, global existence and uniqueness have been established by
Iwata~{\cite{iwata_infinite_1987}}.

In the three dimensional setting the progress has been significantly slower due to the more
severe nature of the singularities of  solutions to the stochastic quantization
equation. Only very recently, there has been substantial progress due to the
invention of \tmtextit{regularity structures theory} by
Hairer~{\cite{hairer_theory_2014}} and \tmtextit{paracontrolled distributions}
by Gubinelli, Imkeller, Perkowski~{\cite{GIP}}. These  theories greatly
extend the pathwise approach of Da Prato and Debussche via insights coming
from Lyons' \tmtextit{rough path
theory}~{\cite{lyons_differential_1998,lyons_system_2002,lyons_differential_2007}}
and in particular the concept of \tmtextit{controlled
paths}~{\cite{gubinelli_controlling_2004,friz_course_2014}}. With these new ideas
it became possible to solve certain analytically ill-posed stochastic PDEs, including the
stochastic quantization equation for the $\Phi_3^4$ measure and the
Kardar--Parisi--Zhang equation. The first results were limited to finite
volume: local-in-time well-posedness has been established by
Hairer~{\cite{hairer_theory_2014}} and Catellier, Chouk~{\cite{CC}}.
Kupiainen~{\cite{kupiainen_renormalization_2016}} introduced a method based on the
renormalization group ideas of~\cite{gawpolhk_edzki_asymptotic_1986}. Long-time behavior has been studied by
Mourrat, Weber~{\cite{MWcomedown}}, Hairer,
Mattingly~{\cite{Hairer:2018:10.1214/17-AIHP840}} and a lattice approximation
in finite volume has been given by Hairer and Matetski~\cite{hairer_discretisations_2018} and by Zhu and Zhu~{\cite{ZZ18}}. 
Global in
space and time solutions have been first constructed by Gubinelli and
Hofmanov{\'a} in~{\cite{GH18}}. Local bounds on solutions, independent on
boundary conditions, and stretched exponential integrability have been
recently proven by Moinat and Weber~{\cite{moinat_space_time_2018}}.

However, all these advances  still fell short of giving a complete proof of the
existence of the $\Phi^4_3$ measure on the full space and of its properties.
Indeed they, including essentially all of the two dimensional
results, are principally aimed at studying the dynamics with an \tmtextit{a~priori} knowledge
of the existence and the properties of the invariant measure. For example Hairer and Matetski~\cite{hairer_discretisations_2018} use a discretization of a finite periodic domain to prove that the limiting dynamics leaves the finite volume $\Phi^4_3$ measure  invariant \emph{using} the a priori knowledge of its convergence from the paper of Brydges et al.~\cite{MR723546}. Studying the dynamics, especially globally in space and time   is still
a very complex problem which has siblings in the ever growing literature on
invariant measures for \tmtextit{deterministic} PDEs starting with the work of
Lebowitz, Rose and
Speer~{\cite{lebowitz_statistical_1988,lebowitz_statistical_1989}},
Bourgain~{\cite{bourgain_periodic_1994,bourgain_invariant_1996}}, Burq and
Tzvetkov~{\cite{burq_random_2008,burq_random_2008_1,tzvetkov_random_2016}} and with many following works (see e.g.~\cite{colliander_almost_2012, chatterjee_probabilistic_2012, nahmod_almost_2013, chatterjee_invariant_2014, benyi_probabilistic_2015}) which we cannot exhaustively review here.

The first work proposing a \tmtextit{constructive} use of the dynamics is, to our knowledge, the work
of Albeverio and Kusuoka~{\cite{albeverio_invariant_2017}}, who proved tightness of certain approximations
in a finite volume.  Inspired by this result, our aim here is to show how these
recent ideas connecting probability with PDE theory can be streamlined and
extended to recover a complete and independent proof of
existence of the $\Phi^{4}_{3}$ measure on the full space. In the same spirit see also the work of
Hairer and Iberti~{\cite{hairer_tightness_2018}} on the tightness of the 2d
Ising--Kac model.

\medskip

Soon after Hairer's seminal paper~{\cite{hairer_theory_2014}},
Jaffe~\cite{MR3392505} analyzed the stochastic quantization from the
point of view of reflection positivity and constructive QFT and concluded that
one has to necessarily take the infinite time limit to satisfy RP. Even with
global solution at hand a proof of RP from dynamics seems nontrivial and
actually the only robust tool we are aware of to prove RP is to start from
finite volume lattice Gibbs measures for which RP can be established by elementary arguments.


Taking into account these considerations, our aim is to use an equilibrium dynamics
to derive bounds which are strong enough to prove the 
tightness of the family $(\nu_{M, \varepsilon})_{M, \varepsilon}$. To be more
precise, we study a lattice approximation of the (renormalized) stochastic
quantization equation
\begin{equation}
  (\partial_t + m^2 - \Delta) \varphi + \lambda \varphi^3 - \infty \varphi =
  \xi, \qquad (t, x) \in \mathbb{R}_+ \times \mathbb{R}^3, \label{eq:P4}
\end{equation}
where $\xi$ is a space-time white noise on $\mathbb{R}^3$. The lattice
dynamics is a system of stochastic differential equation which is globally
well-posed and has $\nu_{M, \varepsilon}$ as its unique invariant measure. We
can therefore consider its stationary solution $\varphi_{M, \varepsilon}$
having at each time the law $\nu_{M, \varepsilon}$. We introduce a suitable
decomposition together with an energy method in the framework of weighted
Besov spaces. This allows us, on the one hand, to track down and renormalize
the short scale singularities present in the model as $\varepsilon \rightarrow
0$, and on the other hand, to control the growth of the solutions as $M \to
\infty$. As a result we obtain uniform bounds  which allow us to pass to the
limit in the weak topology of probability measures.

The details of the renormalized energy method rely on recent developments in
the analysis of singular PDEs. In order to make the paper accessible to a wide audience with some PDE background 
 we  implement
 renormalization using the paracontrolled calculus of~{\cite{GIP}} which is based on Bony's paradifferential operators~\cite{bony_calcul_1981, meyer_remarques_1981, BCD}.
 We  also rely on some tools from the paracontrolled analysis in weigthed Besov spaces which we developed in~{\cite{GH18}} and on the results of Martin and Perkowski~{\cite{MP17}} on Besov spaces on the lattice.

%
%Our
%choice of an alternative approach is mostly motivated by the desire to provide an independent
%and accessible  argument.


\begin{remark} Let us comment in detail on  specific aspects of our proof.
  \begin{enumerate}
  
   \item The method we use here  differs from the approach
of~{\cite{GH18}} in that we are initially less concerned with the continuum
dynamics itself. We do not try to obtain estimates for strong solutions and
rely instead on certain cancellations in the energy estimate that permit to
significantly simplify the proof. The resulting bounds are sufficient to
provide a rather clear picture of any limit measure as well as some of its physical
properties. In contrast, in {\cite{GH18}} we provided a
detailed control of the dynamics {\eqref{eq:P4}} (in stationary or
non-stationary situations) at the price of a more involved analysis.
Section~\ref{s:estim} of the present paper could in principle be replaced by
the corresponding analysis of {\cite{GH18}}. However the adaptation of that
analysis to the lattice setting (without which we do not know how to prove RP) would anyway require the further preparatory work which constitutes a large fraction of the present paper. Similarly, the recent results of Moinat and Weber~{\cite{moinat_space_time_2018}} (which appeared after we completed a first version of this paper) can be conceivably used to replace a part of Section~\ref{sec:tight}. 
  
  
    \item The stretched exponential integrability
    in~{\eqref{eq:exp-int-intro}} is also discussed in the work of Moinat and
    Weber~{\cite{moinat_space_time_2018}} (using different norms) and it is
    sufficient to prove the original regularity axiom of Osterwalder and Schrader but not its formulation given in the book of Glimm and Jaffe~\cite{MR887102}.
    
    \item The Dyson--Schwinger equations were  first derived by Feldman
    and R{\c{a}}czka~{\cite{feldman77}} using the results of Glimm,
    Jaffe, Feldman and Osterwalder.
    
    \item As already noted by Albeverio, Liang and Zegarlinski
    {\cite{albeverio_remark_2006}} on the formal level, the integration by
    parts formula gives rise to a cubic term which cannot be interpreted as a
    random variable under the $\Phi^4_3$ measure. Therefore, the crucial
    question that remained unsolved until now is how to make sense of this
    critical term as a well-defined probabilistic object. In the present
    paper, we obtain fine estimates on the approximate stochastic quantization
    equation and construct a coupling of the stationary solution to the
    continuum $\Phi^4_3$ dynamics and the Gaussian free field. This leads to a
    detailed description of the renormalized cubic term as a genuine random
    space-time distribution. Moreover, we approximate this term in the spirit
    of the operator product expansion.
    
    \item To the best of our knowledge, our work provides the first rigorous
    proof of a general integration by parts formula with an exact formula for
    the renormalized cubic term. In addition, the method applies to arbitrary
    values of the coupling constant $\lambda \geqslant 0$ if $m^2 > 0$ and
    $\lambda > 0$ if $m^2 \leqslant 0$  and we state the precise dependence of our
    estimates on $\lambda$. In particular, we show that our energy bounds are
    uniform over $\lambda$ in every bounded subset of $[0, \infty)$ provided $m^2 >0$ (see Remark~\ref{rem:neg-mass}). Let us recall  that for some $m^{2}=m_{c}^{2}(\lambda)$ %the  model has a  critical point when
     the physical mass of the continuum theory is zero and it is said that the model is critical. Existence of such a critical point was shown in \cite[Section 9, Part (4)]{MR723546}. We note that this case is included in our construction, even though we are not able to locate it since we do not have control over correlations. Its large scale limit  is conjectured to correspond to the Ising conformal field theory, recently  actively studied in \cite{MR3942977} using the conformal bootstrap approach.
    
%    Nevertheless, this is an important example since the long distance scaling limit of this measure is conjectured to be the same as the three-dimensional Ising conformal field theory. The latter has been actively studied recently  [11].
%    
%    This point is important since it is  conjectured to correspond to the scaling limit of the critical Ising model
    
    \item By essentially the same arguments, we are able to treat the vector
    version of the model, where the scalar field $\varphi : \mathbb{R}^3
    \rightarrow \mathbb{R}$ is replaced by a vector valued one $\varphi :
    \mathbb{R}^3 \rightarrow \mathbb{R}^N$ for some $N \in \mathbb{N}$ and the
    measures $\nu_{M, \varepsilon}$ are given by a similar expression
    as~{\eqref{eq:gibbs}}, where the norm $| \varphi |$ is understood as the
    Euclidean norm in $\mathbb{R}^N$.
    
    
    \item Our proof also readily extends to the \emph{fractional} variant of $\Phi^4_3$ where
     the base Gaussian measure is obtained from the fractional Laplacian $(-\Delta)^\gamma$ with $\gamma\in(21/22,1)$ (see Section~\ref{sec:fractional} for details). In general this model is sub-critical for $\gamma\in(3/4,1)$ and in the mass-less case 
     it has recently attracted some interest since  it is \emph{bootstrappable}~\cite{poland_conformal_2019, behan_bootstrapping_2019}. 
     
  \end{enumerate}
\end{remark}

To conclude this introductory part, let us compare our result with other
constructions of the $\Phi^4_3$ field theory. The most straightforward and
simplest available proof has been given by Brydges, Fr{\"o}hlich and
Sokal~{\cite{MR723546}} using skeleton  and correlation inequalities.
All the other methods we cited above employ technically involved machineries
and various kinds of expansions (they are however able to obtain very strong
information about the model in the weakly-coupled regime, i.e. when $\lambda$ is small). Compared to the
existing methods, ours bears similarity in conceptual simplicity to that
of~{\cite{MR723546}}, with some advantages and some disadvantages. Both works
construct the continuum $\Phi^4_3$ theory as a subsequence limit of lattice
theories and the rotational invariance remains unproven. The main difference
is that~{\cite{MR723546}} relies on correlation inequalities. On the
one hand, this restricts the applicability to weak couplings and only models with
$N = (0,) 1, 2$ components (note that the $N=0$ models have a meaning only in their formalism but not in ours). But, on the other hand, it allows to establish bounds
on the decay of correlation functions, which we do not have. However, our results hold for every
value of $\lambda > 0$ and $m^2 \in \mathbb{R}$ while the results
in~{\cite{MR723546}} work only in the so-called ``single phase region'',
which  corresponds to  $m^2>m_{c}^{2}(\lambda)$.

\medskip
Our work is intended as a first step in the direction of using PDE methods in the study of
Euclidean QFTs and large scale properties of statistical mechanical models. Another related attempt is the variational approach developed in~\cite{barashkov_variational_2018} for the finite volume $\Phi^4_3$ measure.
As far as the present paper is concerned the main open problem is to establish rotational invariance and to give more information on the limiting measures, in particular to establish uniqueness for small $\lambda$. It is not clear how to deduce anything about correlations from the dynamics but it seems to be a very interesting and challenging problem. 



\paragraph{Plan.} The paper is organized as follows. Section~\ref{s:not} gives a summary of
notation used throughout the paper, Section~\ref{s:strat} presents the main ideas of our strategy and Section~\ref{sec:tight}, Section~\ref{s:ax} and
Section~\ref{s:sd} are devoted to the main results. First, in Section~\ref{sec:tight}
we construct the Euclidean quantum field theory as a limit of the approximate
Gibbs measures $\nu_{M, \varepsilon}$. To this end, we introduce the lattice
dynamics together with its decomposition. The main energy estimate is
established in Theorem~\ref{th:energy-estimate} and consequently the desired
tightness as well as moment bounds are proven in Theorem~\ref{thm:main}.  In
Section~\ref{s:exp} we establish finite stretched exponential moments. Consequently, in
Section~\ref{s:ax} we verify the translation invariance and reflection positivity, the regularity axiom
and non-Gaussianity of any limit measure. Section~\ref{s:sd} is devoted to the
integration by parts formula and the Dyson--Schwinger equations. In Section~\ref{sec:fractional} we discuss the extension of 
our results to a long-range version of the $\Phi^4_3$ model.
Finally, in Appendix~\ref{s:app} we collect a number of technical results
needed in the main body of the paper.

\paragraph{Acknowledgement.} The authors would like to thank the Isaac Newton
Institute for Mathematical Sciences for support and hospitality during the
programme Scaling limits, rough paths, quantum field theory when work on this
paper was undertaken. In particular, we are grateful to Adelmalek Abdesselam, Sergio Albeverio,
David Brydges, J{\"u}rg Fr{\"o}hlich, Stefan Hollands, Seiichiro Kusuoka and
Pronob Mitter for stimulating discussions. We are also deeply grateful to the anonymous referees for their impressively detailed comments on the CQFT literature and on the relations between various set of axioms for Euclidean correlation functions which helped us to  substantially improve this paper and our knowledge.

This work was supported by EPSRC
Grant Number EP/R014604/1. M. G. is partially supported by the German Research
Foundation (DFG) via CRC 1060.


\section{Notation}

\label{s:not}Within this paper we are concerned with the $\Phi^4_3$ model in
discrete as well as continuous setting. In particular, we denote by
$\Lambda_{\varepsilon} = (\varepsilon \mathbb{Z})^d$ for $\varepsilon = 2^{-
N}$, $N \in \mathbb{N}_0$, the rescaled lattice $\mathbb{Z}^d$ and by
$\Lambda_{M, \varepsilon} = \varepsilon \mathbb{Z}^d \cap \mathbb{T}^d_M =
\varepsilon \mathbb{Z}^d \cap \left[ - \frac{M}{2}, \frac{M}{2} \right)^d$ its
periodic counterpart of size $M > 0$ such that  $M/(2\varepsilon)\in\mathbb{N}$. For notational simplicity, we use the
convention that the case $\varepsilon = 0$ always refers to the continuous
setting. For instance, we denote by $\Lambda_0$ the full space $\Lambda_0
=\mathbb{R}^d$ and by $\Lambda_{M, 0}$ the continuous torus $\Lambda_{M, 0}
=\mathbb{T}^d_M$. With the slight abuse of notation, the parameter
$\varepsilon$ is always taken either of the form $\varepsilon = 2^{- N}$ for
some $N \in \mathbb{N}_0$, $N \geqslant N_0$, for certain $N_0 \in
\mathbb{N}_0$ that will be chosen as a consequence of Lemma~\ref{lem:equiv}
below, or $\varepsilon = 0$. Various proofs below will be formulated generally
for $\varepsilon \in \mathcal{A} \assign \{ 0, 2^{- N} ; N \in \mathbb{N}_0, N
\geqslant N_0 \}$ and it is understood that the case $\varepsilon = 0$ or
alternatively $N = \infty$ refers to the continuous setting. All the proportionality constants, unless explicitly signalled, will be independent of $M,\varepsilon,\lambda,m^2$. We will track the explicit dependence on $\lambda$ as far as possible and signal when the constant depends on the value of $m^2>0$. 

\medskip


For $f \in \ell^1
(\Lambda_{\varepsilon})$ and $g \in L^1 (\hat{\Lambda}_{\varepsilon})$, respectively, we define the Fourier and the inverse Fourier transform as
\[ \mathcal{F} f (k) = \varepsilon^d \sum_{x \in \Lambda_{\varepsilon}} f (x)
   e^{- 2 \pi i k \cdummy x}, \qquad \mathcal{F}^{- 1} g (x) = \int_{(\varepsilon^{- 1} \mathbb{T})^d} g (k)
   e^{2 \pi i k \cdummy x} \mathd k, \]
   where $k \in (\varepsilon^{- 1} \mathbb{T})^d
   \backassign \hat{\Lambda}_{\varepsilon}$ and $x \in \Lambda_{\varepsilon}$.
These definitions can be extended to discrete Schwartz distributions in a
natural way, we refer to {\cite{MP17}} for more details. In general, we do not
specify on which lattice the Fourier transform is taken as it will be clear
from the context.

Consider a smooth dyadic partition of unity $(\varphi_j)_{j \geqslant - 1}$
such that $\varphi_{- 1}$ is supported in a ball around $0$ of radius
$\frac{1}{2}$, $\varphi_0$ is supported in an annulus, $\varphi_j (\cdummy) =
\varphi_0 (2^{- j} \cdummy)$ for $j \geqslant 0$ and if $| i - j | > 1$ then
$\tmop{supp} \varphi_i \cap \tmop{supp} \varphi_j = \emptyset$. For the
definition of Besov spaces on the lattice $\Lambda_{\varepsilon}$ for
$\varepsilon = 2^{- N}$, we introduce a suitable periodic partition of unity
on $\hat{\Lambda}_{\varepsilon}$ as follows
\begin{equation}
  \varphi^{\varepsilon}_j (k) \assign \left\{ \begin{array}{lll}
    \varphi_j (k), &  & j < N - J,\\
    1 - \sum_{j < N - J} \varphi_j (k), &  & j = N - J,
  \end{array} \right. \label{eq:p1}
\end{equation}
where $k \in \hat{\Lambda}_{\varepsilon}$ and the parameter $J \in
\mathbb{N}_0$, whose precise value will be chosen below independently on
$\varepsilon \in \mathcal{A}$, satisfies $0 \leqslant N - J \leqslant
J_{\varepsilon} \assign \inf \{ j : \tmop{supp} \varphi_j \not\subseteq
[-\varepsilon^{- 1}/2,\varepsilon^{- 1}/2 )^d \} \rightarrow \infty$ as
$\varepsilon \rightarrow 0$. We note that by construction there exists $\ell
\in \mathbb{Z}$ independent of $\varepsilon = 2^{- N}$ such that
$J_{\varepsilon} = N - \ell$.

Then {\eqref{eq:p1}} yields a periodic partition of unity on
$\hat{\Lambda}_{\varepsilon}$. The reason for choosing the upper index as $N -
J$ and not the maximal choice $J_{\varepsilon}$ will become clear in Lemma
\ref{lem:equiv} below, where it allows us to define suitable localization
operators needed for our analysis. The choices of parameters $N_0$ and $J$ are
related in the following way: A given partition of unity $(\varphi_j)_{j
\geqslant - 1}$ determines the parameters $J_{\varepsilon}$ in the form
$J_{\varepsilon} = N - \ell$ for some $\ell \in \mathbb{Z}$. By the condition
$N - J \leqslant J_{\varepsilon}$ we obtain the first lower bound on $J$. Then
Lemma~\ref{lem:equiv} yields a (possibly larger) value of $J$ which is fixed
throughout the paper. Finally, the condition $0 \leqslant N - J$ implies
the necessary lower bound $N_0$ for $N$, or alternatively the upper bound for
$\varepsilon = 2^{- N} \leqslant 2^{- N_0}$ and defines the set $\mathcal{A}$.
We stress that once the parameters $J, N_0$ are chosen, they remain fixed
throughout the paper.

Remark that according to our convention, $(\varphi^0_j)_{j \geqslant - 1}$
denotes the original partition of unity $(\varphi_j)_{j \geqslant - 1}$ on
$\mathbb{R}^d$, which can be also read from {\eqref{eq:p1}} using the
fact that for $\varepsilon = 0$ we have $J_{\varepsilon} = \infty$.

Now we may define the Littlewood--Paley blocks for distributions on
$\Lambda_{\varepsilon}$ by
\[ \Delta_j^{\varepsilon} f \assign \mathcal{F}^{- 1} (\varphi_j^{\varepsilon}
   \mathcal{F} f), \]
which leads us to the definition of weighted Besov spaces. Throughout the paper, $\rho$ denotes a polynomial weight 
of the form  \begin{equation}\label{eq:weight}
\rho (x) =
\langle h x \rangle^{- \nu} = (1 + |h  x |^2)^{- \nu / 2}
\end{equation}
 for some $\nu \geqslant
0$ and $h>0$. The constant $h$ will be fixed below in Lemma \ref{lemma:bounds-rhs1} in order to produce a small bound for certain terms.
 Such weights satisfy the admissibility condition $\rho(x)/\rho(y)\lesssim \rho^{-1}(x-y)$
for all $ x, y
   \in \mathbb{R}^d . $
For $\alpha \in \mathbb{R}$, $p, q \in [1, \infty]$ and $\varepsilon \in [0,
1]$ we define the weighted Besov spaces on $\Lambda_{\varepsilon}$ by the norm
\[ \| f \|_{B^{\alpha, \varepsilon}_{p, q} (\rho)} = \Bigg( \sum_{- 1
   \leqslant j \leqslant N - J} 2^{\alpha j q} \| \Delta_j^{\varepsilon} f
   \|_{L^{p, \varepsilon} (\rho)}^q \Bigg)^{1 / q} = \Bigg( \sum_{- 1
   \leqslant j \leqslant N - J} 2^{\alpha j q} \| \rho \Delta_j^{\varepsilon}
   f \|_{L^{p, \varepsilon}}^q \Bigg)^{1 / q}, \]
where $L^{p, \varepsilon}$ for $\varepsilon \in \mathcal{A} \setminus \{ 0 \}$
stands for the $L^p$ space on $\Lambda_{\varepsilon}$ given by the norm
\[ \| f \|_{L^{p, \varepsilon}} = \Bigg( \varepsilon^d \sum_{x \in
   \Lambda_{\varepsilon}} | f (x) |^p \Bigg)^{1 / p} \]
(with the usual modification if $p = \infty$). Analogously, we may define the weighted Besov spaces for explosive polynomial weights of the form $\rho^{-1}$. Note that if $\varepsilon = 0$
then $B^{\alpha, \varepsilon}_{p, q} (\rho)$ is the classical weighted Besov
space $B^{\alpha}_{p, q} (\rho)$. In the sequel, we also employ the following
notations
\[ \CC^{\alpha, \varepsilon} (\rho) \assign B^{\alpha, \varepsilon}_{\infty,
   \infty} (\rho), \qquad H^{\alpha, \varepsilon} (\rho) \assign B^{\alpha,
   \varepsilon}_{2, 2} (\rho) . \]
In Lemma~\ref{lem:equiv2}  we show that one can pull the weight
inside the Littlewood--Paley blocks in the definition of the weighted Besov
spaces. Namely, under suitable assumptions on the weight that are satisfied by
polynomial weights we have
$ \| f \|_{B^{\alpha, \varepsilon}_{p, q} (\rho)} \sim \| \rho f
   \|_{B^{\alpha, \varepsilon}_{p, q}} $
in the sense of equivalence of norms, uniformly in $\varepsilon$.   We define the duality product on
$\Lambda_{\varepsilon}$ by
\[ \langle f, g \rangle_{\varepsilon} \assign \varepsilon^d \sum_{x \in
   \Lambda_{\varepsilon}} f (x) g (x)  \]
and Lemma~\ref{lem:dual2} shows that $B^{- \alpha, \varepsilon}_{p', q'} (\rho^{- 1})$ is
included in the topological dual of $B^{\alpha, \varepsilon}_{p, q} (\rho)$
for conjugate exponents $p, p'$ and $q, q'$.






\medskip


We employ the tools from paracontrolled calculus as
introduced in {\cite{GIP}}, the reader is also referred to {\cite{BCD}} for
further details. We shall  freely use the decomposition $f g = f \prec g +
f \circ g + f \succ g$, where $f \succ g = g \succ f$ and $f \circ g$,
respectively, stands for the paraproduct of $f$ and $g$ and the corresponding
resonant term, defined in terms of Littlewood--Paley decomposition. More
precisely, for $f, g \in \mathcal{S}' (\Lambda_{\varepsilon})$ we let
\[ f \prec g \assign \sum_{1 \leqslant i, j \leqslant N - J, i < j - 1}
   \Delta^{\varepsilon}_i f \Delta^{\varepsilon}_j g, \qquad f \circ g \assign
   \sum_{1 \leqslant i, j \leqslant N - J, i \sim j} \Delta^{\varepsilon}_i f
   \Delta^{\varepsilon}_j g. \]
   We also employ the notations $f\preccurlyeq g:= f\prec g+f\circ g$ and $f\Join g:=f\prec g+f\succ g$.
For notational simplicity, we do not stress the dependence of the paraproduct
and the resonant term on $\varepsilon$ in the sequel. These paraproducts
satisfy the usual estimates uniformly in $\varepsilon$, see e.g.
{\cite{MP17}}, Lemma~4.2, which can be naturally extended to general
$B^{\alpha, \varepsilon}_{p, q} (\rho)$ Besov spaces as in {\cite{MW17}},
Theorem~3.17.   



\medskip


Throughout the paper we assume that $m^{2}>0$ and we only discuss in Remark \ref{rem:neg-mass} how to treat the case of $m^{2}\leqslant0$. In addition,  we are only concerned with the 3 dimensional setting and let $d = 3$. We denote by $\Delta_{\varepsilon}$ the discrete Laplacian on $\Lambda_{\varepsilon}$
given by
\[ \Delta_{\varepsilon} f (x) = \varepsilon^{- 2} \sum_{i = 1}^d (f (x +
   \varepsilon e_i) - 2 f (x) + f (x - \varepsilon e_i)), \qquad x \in
   \Lambda_{\varepsilon}, \]
where $(e_i)_{i = 1, \ldots, d}$ is the canonical basis of $\mathbb{R}^d$.  It
can be checked by a direct computation that the integration by parts formula
\[ \langle \Delta_{\varepsilon} f, g \rangle_{M, \varepsilon} = - \langle
   \nabla_{\varepsilon} f, \nabla_{\varepsilon} g \rangle_{M, \varepsilon} = -
   \varepsilon^d \sum_{x \in \Lambda_{M, \varepsilon}} \sum_{i = 1}^d \frac{f
   (x + \varepsilon e_i) - f (x)}{\varepsilon}  \frac{g (x + \varepsilon e_i)
   - g (x)}{\varepsilon} \]
holds for the discrete gradient
\[ \nabla_{\varepsilon} f (x) = \left( \frac{f (x + \varepsilon e_i) - f
   (x)}{\varepsilon} \right)_{i = 1, \ldots, d} . \]
We let $\Q_{\varepsilon} \assign m^{2} - \Delta_{\varepsilon}$,  $\LL_{\varepsilon} \assign \partial_t +
\Q_{\varepsilon}$ and we write $\LL$ for the continuum analogue of $\LL_{\varepsilon}$.  We let $\LL_{\varepsilon}^{- 1}$ to be the inverse of
$\LL_{\varepsilon}$ on $\Lambda_{\varepsilon}$ such that $\LL_{\varepsilon}^{-
1} f = v$ is a solution to $\LL_{\varepsilon} v = f$, $v (0) = 0.$

   


\section{Overview of the strategy}
\label{s:strat}



With the goals and notations being set, let us now outline the main steps of our strategy.

\paragraph{Lattice dynamics.} 




For fixed parameters $\varepsilon \in \mathcal{A}, M > 0$, we consider a stationary
solution $\varphi_{M, \varepsilon}$ to the discrete stochastic quantization
equation
\begin{equation}
  \mathscr{L} \hspace{.2em}_{\varepsilon} \varphi_{M, \varepsilon} + \lambda
  \varphi_{M, \varepsilon}^3 + (- 3 \lambda a_{M, \varepsilon} + 3
  \lambda^2 b_{M, \varepsilon}) \varphi_{M, \varepsilon} = \xi_{M,
  \varepsilon}, \qquad x \in \Lambda_{M, \varepsilon}, \label{eq:moll}
\end{equation}
whose law at every time $t \geqslant 0$ is given by the Gibbs measure
{\eqref{eq:gibbs}}.
%We denote
%\begin{equation}\label{eq:20} V_{M, \varepsilon} (\varphi) = \varepsilon^d \sum_{x \in \Lambda_{M,
%   \varepsilon}} \left[ \frac{\lambda}{4} | \varphi |^4 + \frac{- 3\lambda a_{M,
%   \varepsilon} + 3\lambda^2 b_{M, \varepsilon} + m^{2}}{2} | \varphi |^2 + \frac{1}{2} |
%   \nabla_{\varepsilon} \varphi |^2 \right]  (x) .
%   \end{equation}
%and
%\begin{equation}\label{eq:19}
% \nu_{M, \varepsilon} (\mathd \varphi) = \frac{1}{Z_{M, \varepsilon}} \exp
%   \{ - 2 V_{M, \varepsilon} (\varphi) \} \prod_{x \in \Lambda_{M,
%   \varepsilon}} \mathd \varphi (x),
%   \end{equation}
%where $Z_{M, \varepsilon}$ is a normalization constant  and $\mathd \varphi$ denotes the Lebesgue measure.
Here $\xi_{M, \varepsilon}$ is a discrete approximation of
a space-time white noise $\xi$ on $\mathbb{R}^{d}$ constructed as follows: Let $\xi_M$ denote its periodization on $\mathbb{T}^d_M$
given by
\[ \xi_M (h) \assign \xi (h_M), \qquad \tmop{where} \quad h_M (t, x) \assign
   \tmmathbf{1}_{\left[ - \frac{M}{2}, \frac{M}{2} \right)^d} (x) \sum_{y \in
   M\mathbb{Z}^d} h (t, x + y), \]
where $h\in L^{2}(\mathbb R\times\mathbb R^{d})$ is a test function, and define the corresponding spatial discretization by
\[ \xi_{M, \varepsilon} (t, x) \assign \varepsilon^{- d} \langle \xi_M (t,
   \cdummy), \tmmathbf{1}_{| \cdummy - x | \leqslant \varepsilon / 2} \rangle,
   \qquad (t, x) \in \mathbb{R} \times \Lambda_{M, \varepsilon} . \]
Then {\eqref{eq:moll}} is a finite-dimensional SDE in a gradient form and it has a (unique) invariant measure $\nu_{M, \varepsilon}$ given by~\eqref{eq:gibbs}. Indeed, the global existence of solutions can be proved along the lines of Khasminskii nonexplosion test \cite[Theorem 3.5]{khasminskii2011stochastic} whereas invariance of the measure~\eqref{eq:gibbs} follows from \cite[Theorem 2]{Zab89}.


Recall that due to the irregularity of the
space-time white noise in dimension $3$, a solution to the limit problem
{\eqref{eq:P4}} can only exist as a distribution. Consequently, since products
of distributions are generally not well-defined it is necessary to make sense
of the cubic term. This forces us to introduce a mass renormalization via
constants $a_{M, \varepsilon}, b_{M, \varepsilon} \geqslant 0$ in
{\eqref{eq:moll}} which shall be suitably chosen in order to compensate the
ultraviolet divergencies. In other words, the additional linear term shall
introduce the correct counterterms needed to renormalize the cubic power and
to derive estimates uniform in both parameters $M, \varepsilon$. To this end,
$a_{M, \varepsilon}$ shall diverge linearly whereas $b_{M, \varepsilon}$
logarithmically and these are of course the same divergencies as those
appearing in the other approaches, see e.g. Chapter 23 in {\cite{MR887102}}.

\paragraph{Energy method in a nutshell.}Our aim is to apply the so-called
energy method, which is one of the very basic approaches in the PDE theory. It
relies on testing the equation by the solution itself and estimating all the
terms. To explain the main idea, consider a toy model
\[ \LL u + \lambda u^3 = f, \qquad x \in \mathbb{R}^3,
\]
driven by a sufficiently regular forcing $f$ such that  the solution is
smooth and there are no difficulties in defining the cube. Testing the
equation by $u$ and integrating the Laplace term by parts leads to
\[ \frac{1}{2} \partial_t \| u \|_{L^2}^2 + m^2 \| u \|_{L^2}^2 + \| \nabla u
   \|_{L^2}^2 + \lambda \| u \|_{L^4}^4 = \langle f, u \rangle . \]
Now, there are several possibilities to estimate the right hand side using
duality and Young's inequality, namely,
\[ \langle f, u \rangle \leqslant \left\{ \begin{array}{l}
     \| f \|_{L^2} \| u \|_{L^2} \leqslant C_{ m^2} \| f \|_{L^2}^2 +
     \frac12 m^2 \| u \|_{L^2}^2\\
     \| f \|_{L^{4 / 3}} \| u \|_{L^4} \leqslant C \lambda^{- 1 / 3}
     \| f \|_{L^{4 / 3}}^{4 / 3} + \frac12 \lambda \| u \|_{L^4}^4\\
     \| f \|_{H^{- 1}} \| u \|_{H^1} \leqslant C_{m^2} \| f \|_{H^{-
     1}}^2 + \frac12 (m^2 \| u \|_{L^2}^2 + \| \nabla u \|_{L^2}^2)
   \end{array} . \right. \]
This way, the dependence on $u$ on the right hand side can be absorbed into
the good terms on the left hand side. If in
addition $u$ was stationary hence in particular $t \mapsto \mathbb{E} \| u (t)
\|_{L^2}^2$ is constant, then we obtain
\[ m^2 \mathbb{E} \| u (t) \|_{L^2}^2 +\mathbb{E} \| \nabla u (t) \|_{L^2}^2 +
   \lambda \mathbb{E} \| u (t) \|_{L^4}^4 \leqslant \left\{ \begin{array}{l}
     C_{m^2} \| f \|_{L^2}^2\\
     C \lambda^{- 1 / 3} \| f \|_{L^{4 / 3}}^{4 / 3}\\
     C_{m^2} \| f \|_{H^{- 1}}^2
   \end{array} . \right. \]


To summarize, using the dynamics we are able to obtain moment bounds for the
invariant measure that depend only on the forcing $f$. Moreover, we also see
the behavior of the estimates with respect to the coupling constant $\lambda$.
Nevertheless, even though using the $L^4$-norm of $u$ introduces a blow up for
$\lambda \rightarrow 0$, the right hand side $f$ in our energy estimate below
will always contain certain power of $\lambda$ in order to cancel this blow up
and to obtain bounds that are uniform as $\lambda \rightarrow 0$.

\paragraph{Decomposition and estimates.} Since the forcing $\xi$ on the right
hand side of {\eqref{eq:P4}} does not possess sufficient regularity, the
energy method cannot be applied directly. Following the usual approach within
the field of singular SPDEs, we shall find a suitable decomposition of the
solution $\varphi_{M, \varepsilon}$, isolating parts of different regularity.
In particular, since the equation is subcritical in the sense of Hairer
{\cite{hairer_theory_2014}} (or superrenormalizable in the language of quantum
field theory), we expect the nonlinear equation {\eqref{eq:P4}} to be a
perturbation of the linear problem $ \LL X = \xi .$
This singles out the most irregular part of the limit field $\varphi$. Hence on
the approximate level we set $\varphi_{M, \varepsilon} = X_{M, \varepsilon} +
\eta_{M, \varepsilon}$ where $X_{M, \varepsilon}$ is a stationary solution to
\begin{equation}
 \LL_{\varepsilon} X_{M,\varepsilon} = \xi_{M,\varepsilon} , \label{eq:X}
\end{equation}
and the remainder
$\eta_{M, \varepsilon}$ is expected to be more regular.

To see if it is indeed the case we plug our decomposition into
{\eqref{eq:moll}} to obtain
\begin{equation}
 \LL_{\varepsilon} \eta_{M, \varepsilon} + 3
  \lambda^2 b_{M, \varepsilon} \varphi_{M, \varepsilon} + \lambda \llbracket
  X_{M, \varepsilon}^3 \rrbracket + \lambda 3 \eta_{M, \varepsilon} \llbracket
  X_{M, \varepsilon}^2 \rrbracket + \lambda 3 \eta_{M, \varepsilon}^2 X_{M,
  \varepsilon} + \lambda \eta_{M, \varepsilon}^3 = 0. \label{eq:eta}
\end{equation}
Here $\llbracket X^2_{M, \varepsilon} \rrbracket$ and $\llbracket X^3_{M,
\varepsilon} \rrbracket$ denote the second and third Wick power of the
Gaussian random variable $X_{M, \varepsilon}$ defined by
\begin{equation}\label{eq:X2X3}
\llbracket X^2_{M, \varepsilon} \rrbracket \assign X^2_{M, \varepsilon} -
   a_{M, \varepsilon}, \qquad \llbracket X^3_{M, \varepsilon} \rrbracket
   \assign X^3_{M, \varepsilon} - 3 a_{M, \varepsilon} X_{M, \varepsilon},
   \end{equation}
where $a_{M, \varepsilon} \assign \mathbb{E} [X^2_{M, \varepsilon} (t)]$ is
independent of $t$ due to stationarity. It can be shown by direct computations
that appeared already in a number of works (see {\cite{CC}},
{\cite{hairer_theory_2014}}, {\cite{hairer_regularity_2015}},
{\cite{mourrat_construction_2016}}) that $\llbracket X^2_{M, \varepsilon}
\rrbracket$ is bounded
uniformly in $M, \varepsilon$ as a continuous stochastic process with values
in the weighted Besov space $\mathscr{C} \hspace{.1em}^{- 1 - \kappa,\varepsilon}
(\rho^{\sigma})$ for every $\kappa, \sigma > 0$, whereas  $\llbracket X^3_{M, \varepsilon} \rrbracket$ can only be constructed as a space-time distribution. In addition,
they converge to the Wick power $\llbracket X^2 \rrbracket$ and $\llbracket
X^3 \rrbracket$ of $X$. In other words, the
linearly growing renormalization constant $a_{M, \varepsilon}$ gives
counterterms needed for the Wick ordering.

Note that $X$ is a continuous
stochastic process with values in $\mathscr{C} \hspace{.1em}^{- 1 / 2 -
\kappa} (\rho^{\sigma})$ for every $\kappa, \sigma > 0$. This limits the
regularity that can be obtained for the approximations $X_{M, \varepsilon}$
uniformly in $M, \varepsilon$. Hence the most irregular term in
{\eqref{eq:eta}} is the third Wick power and by Schauder estimates we expect
$\eta_{M, \varepsilon}$ to be 2 degrees of regularity better. Namely, we
expect uniform bounds for $\eta_{M, \varepsilon}$ in $\mathscr{C}
\hspace{.1em}^{1 / 2 - \kappa} (\rho^{\sigma})$ which indeed verifies our
presumption that $\eta_{M, \varepsilon}$ is more regular than $\varphi_{M,
\varepsilon}$. However, the above decomposition introduced new products in
{\eqref{eq:eta}} that are not well-defined under the above discussed uniform
bounds. In particular, both $\eta_{M, \varepsilon} \llbracket X_{M,
\varepsilon}^2 \rrbracket$ and $\eta_{M, \varepsilon}^2 X_{M, \varepsilon}$ do
not meet the condition that the sum of their regularities is strictly
positive, which is a convenient sufficient  condition  for a product of two distributions
to be analytically well-defined.

In order  to continue the decomposition in the same spirit and to cancel the most irregular term in {\eqref{eq:eta}}, namely,
$\llbracket X^3_{M, \varepsilon} \rrbracket$. The usual way, which can be found
basically in all the available works on the stochastic quantization (see e.g.
in \ {\cite{CC}}, {\cite{GH18}}, {\cite{hairer_theory_2014}},
{\cite{hairer_regularity_2015}}, {\cite{MWcomedown}}) is therefore to define
$\tthreeone{X_{M, \varepsilon}}$ as the stationary solution to
\begin{equation}\label{eq:Xt31}
 \LL_{\varepsilon} \tthreeone{X}_{M, \varepsilon} =
   \llbracket X^3_{M, \varepsilon} \rrbracket,
   \end{equation}
leading to the decomposition $\varphi_{M, \varepsilon} = X_{M, \varepsilon} -
\lambda \tthreeone{X_{M, \varepsilon}} + \zeta_{M, \varepsilon}$. Writing down
the dynamics for $\zeta_{M, \varepsilon}$ we observe that the most irregular
term is the paraproduct $\llbracket X_{M, \varepsilon}^2 \rrbracket \succ
\tthreeone{X}_{M, \varepsilon}$ which can be bounded uniformly in $\mathscr{C}
\hspace{.1em}^{- 1 - \kappa,\varepsilon} (\rho^{\sigma})$ and hence this is not yet
sufficient for the energy method outlined above. Indeed,  the expected (uniform)
regularity of $\zeta_{M, \varepsilon}$ is $\mathscr{C} \hspace{.1em}^{1 -
\kappa,\varepsilon} (\rho^{\sigma})$ and so the term $\langle \zeta_{M,\varepsilon},\llbracket X_{M, \varepsilon}^2 \rrbracket \succ
\tthreeone{X}_{M, \varepsilon}\rangle$ cannot be controlled.
However, we point out that not much is missing.

In order to overcome this issue, we proceed differently than the above cited
works and let $Y_{M, \varepsilon}$ be a solution to
\begin{equation}
 \LL_{\varepsilon} Y_{M, \varepsilon} = - \llbracket
  X_{M, \varepsilon}^3 \rrbracket - 3 \lambda (\mathscr{U}^{\varepsilon}_{>}
  \llbracket X_{M, \varepsilon}^2 \rrbracket) \succ Y_{M, \varepsilon}, \qquad
  Y_{M, \varepsilon} (0) = - \lambda \tthreeone{X}_{M, \varepsilon}(0),
  \label{eq:Y1}
\end{equation}
where $\mathscr{U}^{\varepsilon}_{>}$ is the localization operator defined in
Section \ref{s:l1}. With a suitable choice of the constant $L = L (\lambda, M,
\varepsilon)$ determining $\mathscr{U}^{\varepsilon}_{>}$ (cf. Lemma
\ref{lem:loc}, Lemma \ref{lem:Y1}) we are able to construct the unique solution to this problem
via Banach's fixed point theorem. Consequently, we find our decomposition
$\varphi_{M, \varepsilon} = X_{M, \varepsilon} + Y_{M, \varepsilon} + \phi_{M,
\varepsilon}$ together with the dynamics for the remainder
\begin{equation}
  \LL_{\varepsilon}\phi_{M, \varepsilon} + \lambda
  \phi_{M, \varepsilon}^3 = - 3 \lambda \llbracket X_{M, \varepsilon}^2
  \rrbracket \succ \phi_{M, \varepsilon} - 3 \lambda \llbracket X_{M,
  \varepsilon}^2 \rrbracket \circ \phi_{M, \varepsilon} - 3 \lambda^2 b_{M,
  \varepsilon} \phi_{M, \varepsilon} + \Xi_{M, \varepsilon} . \label{eq:ph}
\end{equation}
The first term on the right hand side is the most irregular contribution, the
second term is not controlled uniformly in $M, \varepsilon$, the third term is
needed for the renormalization and $\Xi_{M, \varepsilon}$ contains various
terms that are more regular and in principle not problematic or that can be
constructed as stochastic objects using the remaining counterterm $- 3
\lambda^2 b_{M, \varepsilon} (X_{M, \varepsilon} + Y_{M, \varepsilon})$.

The advantage of this decomposition with $\phi_{M, \varepsilon}$ as opposed to
the usual approach leading to $\zeta_{M, \varepsilon}$ above is that together
with $\llbracket X^3_{M, \varepsilon} \rrbracket$ we cancelled also the second
most irregular contribution $(\mathscr{U}^{\varepsilon}_{>} \llbracket X_{M,
\varepsilon}^2 \rrbracket) \succ Y_{M, \varepsilon}$, which is too irregular
to be controlled as a forcing $f$ using the energy method. The same difficulty
of course comes with $\llbracket X_{M, \varepsilon}^2 \rrbracket \succ
\phi_{M, \varepsilon}$ in {\eqref{eq:ph}}, however, since it depends on the
solution $\phi_{M, \varepsilon}$ we are able to control it using a
paracontrolled ansatz. To explain this, let us also turn our attention to the
resonant product $\llbracket X_{M, \varepsilon}^2 \rrbracket \circ \phi_{M,
\varepsilon}$ which poses problems as well. When applying the energy method to
{\eqref{eq:ph}}, these two terms appear in the form
\[ \langle \rho^4 \phi_{M, \varepsilon}, - 3 \lambda \llbracket X_{M,
   \varepsilon}^2 \rrbracket \circ \phi_{M, \varepsilon} \rangle_{\varepsilon}
   + \langle \rho^4 \phi_{M, \varepsilon}, - 3 \lambda \llbracket X_{M,
   \varepsilon}^2 \rrbracket \succ \phi_{M, \varepsilon}
   \rangle_{\varepsilon}, \]
where we included a polynomial weight $\rho$ as in \eqref{eq:weight}. The key observation is that the
presence of the duality product permits to show that these two terms
{\em{approximately}} coincide, in the sense that their difference denoted by
$D_{\rho^4, \varepsilon} (\phi_{M, \varepsilon}, - 3 \lambda \llbracket
X^2_{M, \varepsilon} \rrbracket, \phi_{M, \varepsilon})$ is controlled by the
expected uniform bounds. This is proven generally in Lemma \ref{lem:dual1}. As a
consequence, we obtain
\[ \frac{1}{2} \partial_t \| \phi_{M, \varepsilon} \|_{L^{2, \varepsilon}}^2 +
   \lambda \| \phi_{M, \varepsilon} \|_{L^{4, \varepsilon}}^4 + \langle
   \phi_{M, \varepsilon}, \Q_{\varepsilon}\phi_{M, \varepsilon}
   \rangle_{\varepsilon} \]
\[ = \langle \rho^4 \phi_{M, \varepsilon}, - 3 \cdummy 2 \lambda \llbracket
   X_{M, \varepsilon}^2 \rrbracket \succ \phi_{M, \varepsilon}
   \rangle_{\varepsilon} + D_{\rho^4, \varepsilon} (\phi_{M, \varepsilon}, - 3
   \lambda \llbracket X^2_{M, \varepsilon} \rrbracket, \phi_{M, \varepsilon})
   + \Xi_{M, \varepsilon} . \]


Finally, since the last term on the left hand side as well as the first term
on the right hand side are diverging, the idea is to couple them by the
following paracontrolled ansatz. We define
\[\Q_{\varepsilon} \psi_{M, \varepsilon} \assign \Q_{\varepsilon} \phi_{M, \varepsilon} + 3 \llbracket X_{M,
   \varepsilon}^2 \rrbracket \succ \phi_{M, \varepsilon} \]
and expect that the sum of the two terms on the right hand side is more
regular than each of them separately. In other words, $\psi_{M,
\varepsilon}$ is (uniformly) more regular than $\phi_{M, \varepsilon}$.
Indeed, with this ansatz we may complete the square and  obtain
\[ \frac{1}{2} \partial_t \| \rho^2 \phi_{M, \varepsilon} \|_{L^{2,
   \varepsilon}}^2 + \lambda \| \rho \phi_{M, \varepsilon} \|_{L^{4,
   \varepsilon}}^4 + m^2 \| \rho^2 \psi_{M, \varepsilon} \|_{L^{2,
   \varepsilon}}^2 + \| \rho^2 \nabla_{\varepsilon} \psi_{M, \varepsilon}
   \|_{L^{2, \varepsilon}}^2 = \Theta_{\rho^4, M, \varepsilon} + \Psi_{\rho^4,
   M, \varepsilon}, \]
where the right hand side, given in Lemma \ref{lem:energy12}, can be controlled
by the norms on the left hand side, in the spirit of the energy method 
discussed above.

These considerations lead  to our first main result proved as Theorem
\ref{th:energy-estimate} below. In what follows,   $Q_{\rho}(\mathbb{X}_{M,\varepsilon})$ denotes a polynomial in the $\rho$-weighted norms of the involved stochastic objects, the precise definition can be found in Section \ref{ssec:stoch}.

\begin{theorem}
  \label{th:energy-estimate-int}Let $\rho$ be a weight such that $\rho^{\iota}
  \in L^{4, 0}$ for some $\iota \in (0, 1)$. There exists a constant $\alpha =
  \alpha (m^2) > 0$ such that
  \[ \frac{1}{2} \partial_t \| \rho^2 \phi_{M, \varepsilon} \|_{L^{2,
     \varepsilon}}^2 + \alpha [\lambda \| \rho \phi_{M, \varepsilon} \|_{L^{4,
     \varepsilon}}^4 + m^2 \| \rho^2 \psi_{M, \varepsilon} \|_{L^{2,
     \varepsilon}}^2 + \| \rho^2 \nabla_{\varepsilon} \psi_{M, \varepsilon}
     \|_{L^{2, \varepsilon}}^2] + \| \rho^2 \phi_{M, \varepsilon} \|_{H^{1 - 2
     \kappa, \varepsilon}}^2 \]
  \[ \leqslant C_{\lambda, t} Q_{\rho} (\mathbb{X}_{M, \varepsilon}), \]
  where $C_{\lambda, t} = \lambda^3 + \lambda^{(12 - \theta) / (2 + \theta)} |
  \log t |^{4 / (2 + \theta)} + \lambda^7$ for $\theta = \frac{1 / 2 - 4
  \kappa}{1 - 2 \kappa}$.
\end{theorem}

Here we observe the precise dependence on $\lambda$ which in particular
implies that the bound is uniform over $\lambda$ in every bounded subset of
$[0, \infty)$ and vanishes as $\lambda \rightarrow 0$.

\paragraph{Tightness.} In order to proceed to the proof of the existence of the
Euclidean $\Phi^4_3$ field theory, we shall employ the extension operator
$\mathcal{E}^{\varepsilon}$ from Section~\ref{s:ext} which permits to extend
discrete distributions to the full space $\mathbb{R}^3$. An additional twist
originates in the fact that by construction the process $Y_{M, \varepsilon}$
given by {\eqref{eq:Y1}} is not stationary and consequently also $\phi_{M,
\varepsilon}$ fails to be stationary. Therefore the energy argument as
explained above does not apply as it stands and we shall go back to the
stationary decomposition $\varphi_{M, \varepsilon} = X_{M, \varepsilon} -
\lambda \tthreeone{X_{M, \varepsilon}} + \zeta_{M, \varepsilon}$, while using
the result of Theorem \ref{th:energy-estimate-int} in order to estimate
$\zeta_{M, \varepsilon}$. Consequently, we deduce tightness of the family of
the joint laws of $( \mathcal{E}^{\varepsilon}\varphi_{M, \varepsilon},\mathcal{E}^{\varepsilon} X_{M, \varepsilon},
\mathcal{E}^{\varepsilon}\tthreeone{X}_{M, \varepsilon} )$ evaluated at any fixed time $t
\geqslant 0$, proven in Theorem \ref{thm:main} below. To this end, we denote
by $(\varphi, X, \tthreeone{X})$ a canonical representative of the random
variables under consideration and let $\zeta \assign \varphi- X + \lambda
\tthreeone{X}$.

\begin{theorem}
  \label{thm:main-int}Let $\rho$ be a weight such that $\rho^{\iota} \in L^{4,
  0}$ for some $\iota \in (0, 1)$. Then the family of joint laws of $(
  \mathcal{E}^{\varepsilon} \varphi_{M, \varepsilon},
  \mathcal{E}^{\varepsilon} X_{M, \varepsilon}, \mathcal{E}^{\varepsilon}
  \tthreeone{X}_{M, \varepsilon} )$, $\varepsilon \in \mathcal{A}$, $M >
  0$, evaluated at an arbitrary time $t \geqslant 0$ is tight. Moreover, any
  limit measure $\mu$ satisfies for all $p \in [1, \infty)$
  \[ \mathbb{E}_{\mu} \| \varphi \|_{H^{- 1 / 2 - 2 \kappa} (\rho^2)}^{2 p}
     \lesssim 1 + \lambda^{3 p}, \qquad \mathbb{E}_{\mu} \| \zeta \|_{L^2
     (\rho^2)}^{2 p} \lesssim  \lambda^p + \lambda^{3p+4} + \lambda^{4p}, \]
  \[ \mathbb{E}_{\mu} \| \zeta \|_{H^{1 - 2 \kappa} (\rho^2)}^2 \lesssim
     \lambda^2 + \lambda^7, \qquad \mathbb{E}_{\mu} \| \zeta \|_{B^0_{4,
     \infty} (\rho)}^4 \lesssim \lambda + \lambda^6 . \]
\end{theorem}

\paragraph{Osterwalder--Schrader axioms.}The projection of a limit measure
$\mu$ onto the first component is the candidate $\Phi^4_3$ measure and we
denote it by $\nu$. Based on Theorem \ref{thm:main-int} we are able to show
that $\nu$ is translation invariant and reflection positive, see Section \ref{ss:OS1} and Section \ref{ss:OS2}. In addition, we prove that the measure is
non-Gaussian. To this end, we make use of the decomposition
$\varphi = X - \lambda \tthreeone{X} + \zeta$ together with the moment bounds
from Theorem \ref{thm:main-int}. Since $X$ is Gaussian whereas $\tthreeone{X}$
is not, the idea is to use the regularity of $\zeta$ to conclude that it
cannot compensate $\tthreeone{X}$ which is less regular. In particular, we
show that the connected $4$-point function is nonzero, see Section \ref{ss:nonG}.

It remains to discuss a stretched exponential integrability of $\varphi$, leading
to the distribution property  shown in Section \ref{ss:OS0}. More precisely, we show the following result
which can be found in Proposition~\ref{lemma:int-bound}.

\begin{proposition}\label{prop:exp}
  Let $\rho$ be a weight such that $\rho^{\iota} \in L^{4, 0}$ for some $\iota
  \in (0, 1)$. For every $\kappa \in (0, 1)$ small there exists $\upsilon = O
  (\kappa) > 0$ small such that
  \[ \int_{\mathcal{S}'(\mathbb{R}^{3})} \exp\{{\beta \| \varphi \|_{H^{- 1 / 2 - 2
     \kappa} (\rho^2)}^{1 - \upsilon}} \} \nu (\mathrm{d}\varphi)< \infty \]
  provided $\beta > 0$ is chosen sufficiently small.
\end{proposition}

In order to obtain this bound we revisit the bounds from Theorem
\ref{th:energy-estimate-int} and track the precise dependence of the
polynomial $Q_{\rho} (\mathbb{X}_{M, \varepsilon})$ on the right hand side of the estimate on
the quantity $\| \mathbb{X}_{M, \varepsilon} \|$ which will be defined  through \eqref{eq:XX1}, \eqref{eq:XX2}, \eqref{eq:XX3} below taking into account the number of copies of $X$ appearing in each stochastic object. However, the estimates in
Theorem \ref{th:energy-estimate-int} are not optimal and consequently the
power of $\| \mathbb{X}_{M, \varepsilon} \|$ in Theorem
\ref{th:energy-estimate-int} is too
large. To optimize we introduce a large momentum cut-off $\llbracket X^3_{M,
\varepsilon} \rrbracket_{\leqslant}$ given by a parameter $K > 0$ and let
$\llbracket X^3_{M, \varepsilon} \rrbracket_{>} \assign \llbracket X^3_{M,
\varepsilon} \rrbracket - \llbracket X^3_{M, \varepsilon}
\rrbracket_{\leqslant}$. Then we modify the dynamics of $Y_{M, \varepsilon}$
to
\[ \LL_{\varepsilon} Y_{M, \varepsilon} = - \llbracket
   X_{M, \varepsilon}^3 \rrbracket_{>} - 3 \lambda
   (\mathscr{U}^{\varepsilon}_{>} \llbracket X_{M, \varepsilon}^2 \rrbracket)
   \succ Y_{M, \varepsilon}, \]
which allows for refined bounds on $Y_{M, \varepsilon}$, yielding optimal
powers of $\| \mathbb{X}_{M, \varepsilon} \|$.

\paragraph{Integration by parts formula.}
The uniform energy estimates from Theorem~\ref{thm:main-int} and Proposition~\ref{prop:exp} are enough
to obtain tightness of the approximate measures and to show that any
accumulation point satisfies the distribution property,  translation invariance, reflection positivity and
non-Gaussianity. However, they do not provide sufficient regularity in order to
identify the continuum dynamics or to establish the hierarchy of
Dyson--Schwinger equations providing relations of various $n$-point
correlation functions. This can be seen easily since neither the resonant
product $\llbracket X_{M, \varepsilon}^2 \rrbracket \circ \phi_{M,
\varepsilon}$ nor $\llbracket X_{M, \varepsilon}^2 \rrbracket \circ \psi_{M,
\varepsilon}$ is well-defined in the limit.
Another and even more severe difficulty lies in the fact that the  third Wick power $\llbracket X^3 \rrbracket$ only exists as a space-time distribution and is  not a well-defined random variable under the $\Phi^{4}_{3}$ measure, cf.~\cite{albeverio_remark_2006}.

To overcome the first issue, we  introduce a new
paracontrolled ansatz
$  \chi_{M, \varepsilon} \assign \phi_{M, \varepsilon} + 3\lambda \ttwoone{X_{M,
  \varepsilon}} \succ \phi_{M, \varepsilon}$
and show that $\chi_{M,\varepsilon}$ possesses enough regularity uniformly in $M,\varepsilon$ in order to pass to the limit in the resonant product $\llbracket X^{2}_{M,\varepsilon}\rrbracket\circ \chi_{M,\varepsilon}$.
Namely,  we establish uniform bounds for $\chi_{M,\varepsilon}$ in $L^1_T B_{1, 1}^{1 + 3 \kappa,
     \varepsilon}(\rho^{4})$. This not only allows to give meaning to the critical resonant product in the continuum, but  it  also leads to a uniform time regularity of the processes $\varphi_{M,\varepsilon}$.
We obtain the following result proved below as Theorem~\ref{thm:phitight}.

\begin{theorem}
  \label{thm:phitight-int}Let $\beta \in (0, 1 / 4)$ and $\sigma\in (0,1)$. Then  for
  all $p \in [1, \infty)$ and $\tau \in (0, T)$
  \[ \sup_{\varepsilon \in \mathcal{A}, M > 0} \mathbb{E} \| \varphi_{M,
     \varepsilon} \|^{2 p}_{W^{\beta, 1}_T B_{1, 1}^{- 1 - 3 \kappa,\varepsilon} (\rho^{4
     + \sigma})} + \sup_{\varepsilon \in \mathcal{A}, M > 0} \mathbb{E} \|
     \varphi_{M, \varepsilon} \|^{2 p}_{L^{\infty}_{\tau, T} H^{- 1 / 2 -2
     \kappa,\varepsilon} (\rho^2)}  < \infty, \]
  where $L^{\infty}_{\tau, T} H^{- 1 / 2 -2 \kappa,\varepsilon} (\rho^2) = L^{\infty}
  (\tau, T ; H^{- 1 / 2 -2 \kappa,\varepsilon} (\rho^2))$.
\end{theorem}


This additional time regularity  is then used in order to treat the second issue raised above
and  to construct a renormalized cubic term $\llbracket
\varphi^3 \rrbracket$. More precisely, we derive an explicit formula for $\llbracket \varphi^{3}\rrbracket$ including  $\llbracket X^3 \rrbracket$ as a space-time distribution, where  {\em time} indeed means the fictitious {\em stochastic} time variable introduced by the stochastic quantization, nonexistent under the $\Phi^{4}_{3}$ measure. In order to control $\llbracket X^3 \rrbracket$ we re-introduce the stochastic time and use stationarity together with the above mentioned time regularity.
Finally, we derive an integration by parts formula
 leading to the hierarchy of Dyson--Schwinger equations connecting the correlation functions. 
To this end, we recall that a cylinder function $F$ on $\mathcal{S}' (\mathbb{R}^3)$ has the form $F
(\varphi) = \Phi (\varphi (f_1), \ldots, \varphi (f_n))$ where $\Phi :
\mathbb{R}^n \rightarrow \mathbb{R}$ and $f_1, \ldots, f_n \in \mathcal{S}
(\mathbb{R}^3)$. Loosely stated, the result proved in Theorem \ref{thm:ibp} says the following.

\begin{theorem}
  \label{thm:ibp-int}
  Let $F : \mathcal{S}' (\mathbb{R}^3) \rightarrow \mathbb{R}$
  be a cylinder function such that
  \[ | F (\varphi) | + \| \mathD F (\varphi) \|_{B_{\infty, \infty}^{1 + 3
     \kappa} (\rho^{- 4 - \sigma})} \leqslant C_F \| \varphi \|_{H^{- 1 / 2 -
    2 \kappa} (\rho^2)}^n \]
  for some $n \in \mathbb{N}$, where $\mathD F (\varphi)$  the $L^2$-gradient of $F$. Any accumulation point $\nu$ of the sequence $(\nu_{M,
  \varepsilon} \circ (\mathcal{E}^{\varepsilon})^{- 1})_{M, \varepsilon}$
  satisfies for all $f\in \mathcal{S}(\mathbb{R}^{3})$
  \begin{equation*}
    \int \langle\mathD F (\varphi),f\rangle \nu (\mathd \varphi) = 2 \int \langle(m^2 - \Delta)
    \varphi,f\rangle F (\varphi) \nu (\mathd \varphi) + 2\lambda \langle\mathcal{J}_{\nu} (F),f\rangle,
  \end{equation*}
  where for a smooth $h : \mathbb{R} \rightarrow \mathbb{R}$ 
with $\tmop{supp} h \subset [\tau, T]$ for some $0 < \tau < T < \infty$ and
 $\int_{\mathbb{R}} h (t) \mathd t = 1$ we have for all $f\in \mathcal{S}(\mathbb{R}^{3})$
  \[ \langle\mathcal{J}_{\nu} (F),f\rangle
   =\mathbb{E}_{\nu} \left[ \int_{\mathbb{R}} h (t) F (\varphi (t))\langle \llbracket
   \varphi^3 \rrbracket (t) ,f\rangle\mathd t \right]
    \]
    and $\llbracket \varphi^{3}\rrbracket$ is given by an explicit formula, namely, \eqref{eq:phi3}.
\end{theorem}

In addition, we are able to characterize $\mathcal{J}_{\nu}(F)$ in the spirit of the operator product expansion, see Lemma \ref{lem:OPE}.




\section{Construction of the Euclidean $\Phi^4$ field theory}

\label{sec:tight}This section is devoted to our main result. More precisely,
we consider \eqref{eq:moll} which is a discrete approximation of {\eqref{eq:P4}} posed on a periodic
lattice $\Lambda_{M, \varepsilon}$. For every $\varepsilon \in (0, 1)$
and $M > 0$ \eqref{eq:moll} possesses a unique invariant measure that is the Gibbs measure $\nu_{M,\varepsilon}$ given by \eqref{eq:gibbs}.
We derive new estimates on stationary solutions sampled from these measures
which hold true uniformly in $\varepsilon$ and $M$. As a consequence, we
obtain tightness of the invariant measures while sending both the mesh size as
well as the volume to their respective limits, i.e. $\varepsilon \rightarrow
0$, $M \rightarrow \infty$.




\subsection{Stochastic terms}

\label{ssec:stoch}
Recall that the stochastic objects $X_{M,\varepsilon},\llbracket X_{M,\varepsilon}^{2}\rrbracket, \llbracket X_{M,\varepsilon}^{3}\rrbracket$ and $\tthreeone{X_{M, \varepsilon}}$ were already defined in \eqref{eq:X}, \eqref{eq:X2X3} and \eqref{eq:Xt31}.
As the next step we provide further details and construct additional stochastic objects needed in the sequel. All the distributions on
$\Lambda_{M, \varepsilon}$ are extended periodically to the full lattice
$\Lambda_{\varepsilon}$. Then $\tthreeone{X_{M, \varepsilon}}$ which is a stationary solution to \eqref{eq:Xt31}
satisfies $\tthreeone{X_{M, \varepsilon}}(t) =
P^{\varepsilon}_{t}\tthreeone{X_{M, \varepsilon}} (0) + \LL_{\varepsilon}^{- 1} \llbracket X_{M,
\varepsilon}^3 \rrbracket$ with $\tthreeone{X_{M, \varepsilon}} (0) = \int_{-
\infty}^0 P^{\varepsilon}_{- s} \llbracket X_{M, \varepsilon}^3 \rrbracket (s)
\mathd s$, where $P^{\varepsilon}_t$ denotes the semigroup generated by
$-\Q_{\varepsilon}$ on $\Lambda_{\varepsilon}$. Then  for every
$\kappa, \sigma > 0$ and some $\beta > 0$ small
\[ \| \tthreeone{X_{M, \varepsilon}} \|_{C_T \CC^{1 / 2 - \kappa,
   \varepsilon} (\rho^{\sigma})} + \| \tthreeone{X_{M, \varepsilon}}
   \|_{C_T^{\beta / 2} L^{\infty, \varepsilon} (\rho^{\sigma})} \lesssim
   1, \]
uniformly in $M, \varepsilon$ thanks to the presence of the weight. For details and further references see e.g. Section 3 in \cite{GH18}. Here and in the sequel, $T\in (0,\infty)$ denotes an arbitrary finite time horizon and $C_{T}$ and $C^{\beta/2}_{T}$ are shortcut notations for $C([0,T])$ and $C^{\beta/2}([0,T])$, respectively. Throughout
our analysis, we fix $\kappa, \beta > 0$ in the above estimate such that
$\beta \geqslant 3 \kappa$. This condition will be needed for the control of
a parabolic commutator  in Lemma
\ref{lemma:bounds-rhs1} below. On the other hand, the parameter $\sigma > 0$
varies from line to line and can be arbitrarily small.

As already discussed  in Section \ref{s:strat}, in particular after equation \eqref{eq:Xt31}, the usual decomposition $\varphi_{M,\varepsilon}=X_{M,\varepsilon}-\lambda \tthreeone{X}_{M,\varepsilon}+\zeta_{{M,\varepsilon}}$ is not suitable for the energy method. Indeed, it would introduce  the term $\llbracket X_{M, \varepsilon}^2 \rrbracket \succ
\tthreeone{X}_{M, \varepsilon}$ which cannot be cancelled or controlled by the available quantities. We overcome this issue by working rather with the decomposition $\varphi_{M,\varepsilon}=X_{M,\varepsilon}+Y_{M,\varepsilon}+\phi_{M,\varepsilon}$ defined in the sequel. Note that a similar modification of the paracontrolled ansatz has been necessary to construct a renormalized control problem for the KPZ equation in \cite{gubinelli_kpz_2017}. Here, the price to pay is that the auxiliary variables $Y_{M,\varepsilon}$, $\phi_{M,\varepsilon}$ are not stationary. Thus, in Section~\ref{ss:tight} we go back to the stationary decomposition $\varphi_{M,\varepsilon}=X_{M,\varepsilon}-\lambda \tthreeone{X}_{M,\varepsilon}+\zeta_{{M,\varepsilon}}$. 

If $\UU^{\varepsilon}_{>}$ is a localizer defined for some given constant $L >
0$ according to Lemma~\ref{lem:loc}, we let $Y_{M, \varepsilon}$ be the
solution of \eqref{eq:Y1} 
hence
\begin{equation}\label{eq:YY}
Y_{M, \varepsilon} = -\lambda
\tthreeone{X_{M, \varepsilon}} - \LL_{\varepsilon}^{- 1} [ 3\lambda (
\UU^{\varepsilon}_{>} \llbracket X_{M, \varepsilon}^2 \rrbracket ) \succ Y_{M,
\varepsilon} ].
\end{equation}
Note that this is an equation for $Y_{M, \varepsilon}$,
which also implies that $Y_{M, \varepsilon}$ is not a polynomial of the
Gaussian noise. However, as shown in the following lemma, $Y_{M, \varepsilon}$
can be constructed as a fixed point provided $L$ is large enough.

\begin{lemma}
  \label{lem:Y1}
  There exists $L_{0}=L_{0}(\lambda)\geqslant 0$ and $L=L(\lambda,M,\varepsilon) \geqslant 0$ with a (not relabeled) subsequence satisfying $L(\lambda,M,\varepsilon)\to L_{0}$ as $\varepsilon\to0$, $M\to\infty$, such that {\eqref{eq:Y1}} with $\UU^{\varepsilon}_{>}$ determined by  $L$ has a unique
  solution $Y_{M, \varepsilon}$ that belongs to $C_T \CC^{1 / 2 - \kappa}
  (\rho^{\sigma}) \cap C_T^{\beta / 2} L^{\infty} (\rho^{\sigma})$.
  Furthermore,
  \[ \| Y_{M, \varepsilon} \|_{C_T \CC^{1 / 2 - \kappa, \varepsilon}
     (\rho^{\sigma})} \lesssim \lambda \| \tthreeone{X_{M, \varepsilon}}
     \|_{C_T \CC^{1 / 2 - \kappa, \varepsilon} (\rho^{\sigma})}, \]
  \[ \| Y_{M, \varepsilon} \|_{C_T^{\beta / 2} L^{\infty, \varepsilon}
     (\rho^{\sigma})} \lesssim \lambda[ \| \tthreeone{X_{M, \varepsilon}}
     \|_{C_T \CC^{1 / 2 - \kappa, \varepsilon} (\rho^{\sigma})} +
     \| \tthreeone{X_{M, \varepsilon}} \|_{C_T^{\beta / 2}
     L^{\infty, \varepsilon} (\rho^{\sigma})}], \]
  where the proportionality constant is independent of $M,\varepsilon$.
\end{lemma}

\begin{proof}
  We define a fixed point map 
  \[
  \mathcal{K} : \tilde{Y} \mapsto Y
  \assign -\lambda \tthreeone{X_{M, \varepsilon}} - \LL_{\varepsilon}^{- 1} [ 3\lambda
  ( \UU^{\varepsilon}_{>} \llbracket X_{M, \varepsilon}^2 \rrbracket ) \succ
  \tilde{Y} ]
  \] for some $L > 0$ to be chosen below. Then in
  view of the Schauder estimates from Lemma~3.4 in {\cite{MP17}}, the
  paraproduct estimates as well as Lemma~\ref{lem:loc}, we have
  \[ \| \mathcal{K} \tilde{Y}_1 - \mathcal{K} \tilde{Y}_2 \|_{C_T \CC^{1 / 2 -
     \kappa, \varepsilon} (\rho^{\sigma})} \lesssim \lambda \| (
     \UU^{\varepsilon}_{>} \llbracket X_{M, \varepsilon}^2 \rrbracket )
     \succ (\tilde{Y_1} - \tilde{Y_2}) \|_{C_T \CC^{- 3 / 2 -
     \kappa, \varepsilon} (\rho^{\sigma})} \]
  \[ \leqslant C \lambda 2^{- L / 2} \| \llbracket X_{M, \varepsilon}^2 \rrbracket
     \|_{C_T \CC^{- 1 - \kappa, \varepsilon} (\rho^{\sigma})} \|
     \tilde{Y_1} - \tilde{Y_2} \|_{C_T L^{\infty, \varepsilon}
     (\rho^{\sigma})} \leqslant \delta \| \tilde{Y_1} - \tilde{Y_2}
     \|_{C_T \CC^{1 / 2 - \kappa, \varepsilon} (\rho^{\sigma})} \]
  for some $\delta \in (0, 1)$ independent of $\lambda, M,\varepsilon$ provided $L=L(\lambda, M,\varepsilon)$ in the definition of the localizer
  $\UU^{\varepsilon}_{>}$ is chosen to be the smallest $L\geqslant 0$ such that
  \[ \lambda \left\| \UU^{\varepsilon}_{>} \llbracket X_{M, \varepsilon}^2 \rrbracket
     \right\|_{C_T \CC^{- 3 / 2 - \kappa, \varepsilon} (\rho^0)} \leqslant C \lambda
     2^{- L / 2} \| \llbracket X_{M, \varepsilon}^2 \rrbracket \|_{C_T \CC^{-
     1 - \kappa, \varepsilon} (\rho^{\sigma})} \leqslant \delta . \]
In particular, we have that
 \begin{equation}
 2^{L/2}= C_{\delta}( 1+\lambda \| \llbracket X_{M, \varepsilon}^2 \rrbracket \|_{C_T \CC^{-
     1 - \kappa, \varepsilon} (\rho^{\sigma})}),
    \label{eq:U11}
  \end{equation}
  which will be used later in order to estimate the complementary operator $\UU^{\varepsilon}_{\leqslant}$ by Lemma~\ref{lem:loc}.
Note that   $L(\lambda,{M,\varepsilon})$ a~priori depends on $M,\varepsilon$. However, due to the uniform bound on
$$\|\llbracket X^{2}_{M,\varepsilon}\rrbracket\|_{C_{T}\CC^{-1-\kappa/2,\varepsilon}(\rho^{\sigma})}+\|\llbracket X^{2}_{M,\varepsilon}\rrbracket\|_{C^{\gamma/2}_{T}L^{\infty,\varepsilon}(\rho^{\sigma})}$$ valid for some $\gamma\in (0,1)$, we may use compactness to deduce that for every fixed $\lambda>0$ there exists a subsequence (not relabeled) such that $L(\lambda,M,\varepsilon)\to L_{0}(\lambda)$. This will also allow to identify the limit of  the localized term below  in Section \ref{s:sd}.



Next, we estimate
  \[ \| \mathcal{K} \tilde{Y} \|_{C_T \CC^{1 / 2 - \kappa, \varepsilon}
     (\rho^{\sigma})} \leqslant  \lambda \| \tthreeone{X_{M, \varepsilon}}
     \|_{C_T \CC^{1 / 2 - \kappa, \varepsilon} (\rho^{\sigma})} + C \lambda
     \| ( \UU^{\varepsilon}_{>} \llbracket X_{M, \varepsilon}^2
     \rrbracket ) \succ \tilde{Y} \|_{C_T \CC^{- 3 / 2 - \kappa,
     \varepsilon} (\rho^{\sigma})} \]
  \[ \leqslant \lambda \| \tthreeone{X_{M, \varepsilon}} \|_{C_T \CC^{1 / 2
     - \kappa, \varepsilon} (\rho^{\sigma})} +  \delta \| \tilde{Y} \|_{C_T
     \CC^{1 / 2 - \kappa, \varepsilon} (\rho^{\sigma})} . \]
  Therefore we deduce that $\mathcal{K}$ leaves balls in $C_T \CC^{1 / 2 -
  \kappa, \varepsilon} (\rho^{\sigma})$ invariant and is a contraction on $C_T
  \CC^{1 / 2 - \kappa, \varepsilon} (\rho^{\sigma})$. Hence there exists a
  unique fixed point $Y_{M, \varepsilon}$ and the first bound follows. Next,
  we use the Schauder estimates (see Lemma 3.10 in {\cite{MP17}}) to bound the
  time regularity as follows
  \[ \| Y_{M, \varepsilon} \|_{C_T^{\beta / 2} L^{\infty, \varepsilon}
     (\rho^{\sigma})} \leqslant \lambda \| \tthreeone{X_{M, \varepsilon}}
     \|_{C_T^{\beta / 2} L^{\infty, \varepsilon} (\rho^{\sigma})} + C \lambda
     \| ( \UU^{\varepsilon}_{>} \llbracket X_{M, \varepsilon}^2
     \rrbracket ) \succ Y_{M, \varepsilon} \|_{C_T \CC^{- 3 / 2 -
     \kappa, \varepsilon} (\rho^{\sigma})} \]
  \[ \leqslant \lambda \| \tthreeone{X_{M, \varepsilon}} \|_{C_T^{\beta /
     2} L^{\infty, \varepsilon} (\rho^{\sigma})} +  \delta \| Y_{M,
     \varepsilon} \|_{C_T \CC^{1 / 2 - \kappa, \varepsilon} (\rho^{\sigma})}
  \]
  \[ \lesssim \lambda \| \tthreeone{X_{M, \varepsilon}} \|_{C_T^{\beta / 2}
     L^{\infty, \varepsilon} (\rho^{\sigma})} + \lambda \| \tthreeone{X_{M,
     \varepsilon}} \|_{C_T \CC^{1 / 2 - \kappa, \varepsilon}
     (\rho^{\sigma})} . \]
  The proof is complete.
\end{proof}



According to this result, we remark that $Y_{M, \varepsilon}$ itself is not a
polynomial in the noise terms, but with our choice of localization it allows
for a polynomial bound of its norm.
As the next step, we introduce further stochastic objects needed below.
Namely,
\[ \ttwoone{X_{M, \varepsilon}} \assign \LL_{\varepsilon}^{- 1} \llbracket
   X_{M, \varepsilon}^2 \rrbracket, \qquad \tthreetwor{X_{M, \varepsilon}} =
   X_{M, \varepsilon} \circ \tthreeone{X_{M, \varepsilon}}, \]
\[ \ttwothreer{X_{M, \varepsilon}} \assign 9 \llbracket X_{M, \varepsilon}^2
   \rrbracket \circ \Q_{\varepsilon}^{- 1} \llbracket X_{M, \varepsilon}^2
   \rrbracket - 3 b_{M, \varepsilon}, \]
\[ \ttwothreer{\tilde{X}_{M, \varepsilon}} = 9 \llbracket X_{M, \varepsilon}^2
   \rrbracket \circ \ttwoone{X_{M, \varepsilon}} - 3 \tilde{b}_{M,
   \varepsilon} (t), \qquad \tthreethreer{X_{M, \varepsilon}} = 3 \llbracket
   X_{M, \varepsilon}^2 \rrbracket \circ \tthreeone{X_{M, \varepsilon}} - 3
   b_{M, \varepsilon} X_{M, \varepsilon}, \]
where $b_{M, \varepsilon}, \tilde{b}_{M, \varepsilon} (t)$ are suitable
renormalization constants. It follows from standard estimates that $|
\tilde{b}_{M, \varepsilon} (t) - b_{M, \varepsilon} | \lesssim | \log t |$
uniformly in $M, \varepsilon$. We denote collectively
\begin{equation}\label{eq:XX}
 \mathbb{X}_{M, \varepsilon} \assign (X_{M, \varepsilon}, \llbracket X_{M,
   \varepsilon}^2 \rrbracket, \tthreeone{X_{M, \varepsilon}},
   \tthreetwor{X_{M, \varepsilon}}, \ttwothreer{X_{M, \varepsilon}},
   \ttwothreer{\tilde{X}_{M, \varepsilon}}, \tthreethreer{X_{M, \varepsilon}})
   .
   \end{equation}
   These objects can be constructed similarly as the usual $\Phi^{4}_{3}$ terms, see e.g. \cite{GH18,hairer_regularity_2015,mourrat_construction_2016}.
Note that we do not include $\ttwoone{X_{M, \varepsilon}}$ in $\mathbb{X}_{M,
\varepsilon}$ since it can be controlled by $\llbracket X_{M, \varepsilon}^2
\rrbracket$ using Schauder estimates. 
In order to have a precise control of the number of copies of $X$ appearing in each  stochastic term we define $\|\mathbb X_{M,\varepsilon}\|$ as the smallest number bigger than 1 and all the quantities
\begin{equation}\label{eq:XX1}
 \| X_{M, \varepsilon} \|_{C_T \CC^{- 1 / 2 - \kappa, \varepsilon}
   (\rho^{\sigma})}, \quad \| \llbracket X_{M, \varepsilon}^2 \rrbracket
   \|^{1/2}_{C_T \CC^{- 1 - \kappa, \varepsilon} (\rho^{\sigma})}, \quad \|
   \tthreeone{X_{M, \varepsilon}} \|^{1/3}_{C_T \CC^{1 / 2 - \kappa,
   \varepsilon} (\rho^{\sigma})},
   \end{equation}
\begin{equation}\label{eq:XX2}
 \quad \| \tthreeone{X_{M, \varepsilon}} \|^{1/3}_{C_{T}^{\beta / 2}
   L^{\infty, \varepsilon} (\rho^{\sigma})}, \qquad \| \tthreetwor{X_{M,
   \varepsilon}} \|^{1/4}_{C_T \CC^{- \kappa, \varepsilon} (\rho^{\sigma})},
\end{equation}
\begin{equation}\label{eq:XX3}
 \| \ttwothreer{X_{M, \varepsilon}} \|^{1/4}_{C_{T} \CC^{- \kappa, \varepsilon}
   (\rho^{\sigma})}, \quad \| \ttwothreer{\tilde{X}_{M, \varepsilon}} \|^{1/4}_{C_T
   \CC^{- \kappa, \varepsilon} (\rho^{\sigma})}, \quad \| \tthreethreer{X_{M,
   \varepsilon}} \|^{1/5}_{C_T \CC^{- 1 / 2 - \kappa, \varepsilon} (\rho^{\sigma})}.
\end{equation}
Note that it is bounded uniformly with respect to $M, \varepsilon$.  Besides, if we do not need to be precise about the exact powers, we denote by $Q_{\rho}
    (\mathbb{X}_{M, \varepsilon})$  a generic polynomial in the above
    norms of the noise terms $\mathbb{X}_{M, \varepsilon}$, whose coefficients
    depend on $\rho$ but are independent of $M,\varepsilon, \lambda$, and change from line to
    line.
    
    
    
\subsection{Decomposition and uniform estimates}
\label{s:estim}

With the above stochastic objects at hand, we let $\varphi_{M, \varepsilon}$
be a stationary solution to {\eqref{eq:moll}} on $\Lambda_{M, \varepsilon}$
having at each time $t \geqslant 0$ the law $\nu_{M, \varepsilon}$. We
consider its decomposition $\varphi_{M, \varepsilon} = X_{M, \varepsilon} + Y_{M, \varepsilon} +
  \phi_{M, \varepsilon}$
and deduce that $\phi_{M, \varepsilon}$ satisfies
\begin{equation}
  \begin{array}{lll}
    \LL_{\varepsilon} \phi_{M, \varepsilon} +\lambda \phi_{M, \varepsilon}^3 & = & -
    3\lambda \llbracket X_{M, \varepsilon}^2 \rrbracket \succ \phi_{M, \varepsilon} -
    3\lambda \llbracket X_{M, \varepsilon}^2 \rrbracket \preccurlyeq (Y_{M,
    \varepsilon} + \phi_{M, \varepsilon})\\
    &  & - 3\lambda^2 b_{M, \varepsilon} (X_{M, \varepsilon} + Y_{M, \varepsilon} +
    \phi_{M, \varepsilon}) - 3\lambda ( \UU^{\varepsilon}_{\leqslant L}
    \llbracket X_{M, \varepsilon}^2 \rrbracket ) \succ Y_{M,
    \varepsilon}\\
    &  & - 3\lambda X_{M, \varepsilon} (Y_{M, \varepsilon} + \phi_{M,
    \varepsilon})^2 -\lambda Y_{M, \varepsilon}^3 - 3\lambda Y_{M, \varepsilon}^2 \phi_{M,
    \varepsilon} - 3\lambda Y_{M, \varepsilon} \phi_{M, \varepsilon}^2 .
  \end{array} \label{eq:phi-eq}
\end{equation}
Our next goal is to derive energy estimates for {\eqref{eq:phi-eq}} which hold
true uniformly in both parameters $M, \varepsilon$. To this end, we recall
that all the distributions above were extended periodically to the full
lattice $\Lambda_{\varepsilon}$. Consequently, apart from the stochastic
objects, the renormalization constants and the initial conditions, all the
operations in {\eqref{eq:phi-eq}} are independent of $M$. Therefore, for
notational simplicity, we fix the parameter $M$ and omit the dependence on $M$
throughout the rest of this subsection. The following series of lemmas serves
as a preparation for our main energy estimate established in Theorem
\ref{th:energy-estimate}. Here, we make use of the approximate duality operator $D_{\rho^{4},\varepsilon}$ as well as the commutators $C_{\varepsilon},\,\tilde C_{\varepsilon}$ and $\bar C_{\varepsilon}$ introduced Section~\ref{s:l2}.

\begin{lemma}
  \label{lem:energy12}It holds
  \begin{equation}
    \frac{1}{2} \partial_t \| \rho^2 \phi_{\varepsilon} \|_{L^{2,
    \varepsilon}}^2 +\lambda \| \rho \phi_{\varepsilon} \|_{L^{4, \varepsilon}}^4 +
    m^{2} \| \rho^2 \psi_{\varepsilon} \|_{L^{2, \varepsilon}}^2 + \| \rho^2
    \nabla_{\varepsilon} \psi_{\varepsilon} \|_{L^{2, \varepsilon}}^2 =
    \Theta_{\rho^4, \varepsilon} + \Psi_{\rho^4, \varepsilon} \label{eq:en12}
  \end{equation}
  with
  \begin{equation}
    \psi_{\varepsilon} \assign \phi_{\varepsilon} + \Q_{\varepsilon}^{- 1} [3 \lambda
    \llbracket X_{\varepsilon}^2 \rrbracket \succ \phi_{\varepsilon}],
    \label{eq:psi1}
  \end{equation}
  \[ \Theta_{\rho^4, \varepsilon} \assign - \langle [\nabla_{\varepsilon}, \rho^4]
     \psi_{\varepsilon}, \nabla_{\varepsilon} \psi_{\varepsilon} \rangle_{\varepsilon} +
     \left\langle \left[ \Q_{\varepsilon}, \rho^4 \right] \Q_{\varepsilon}^{-
     1} [3\lambda \llbracket X_{\varepsilon}^2 \rrbracket \succ \phi_{\varepsilon}],
     \psi_{\varepsilon} \right\rangle_{\varepsilon} + \langle \rho^4
     \phi_{\varepsilon}^2, \lambda^2 \ttwothreer{X_{\varepsilon}} \rangle_{\varepsilon} \]
  \[ + D_{\rho^4, \varepsilon} (\phi_{\varepsilon}, - 3\lambda \llbracket
     X_{\varepsilon}^2 \rrbracket, \phi_{\varepsilon}) + \langle \rho^4
     \phi_{\varepsilon}, \tilde{C}_{\varepsilon} (\phi_{\varepsilon}, 3\lambda
     \llbracket X_{\varepsilon}^2 \rrbracket, 3\lambda \llbracket X_{\varepsilon}^2
     \rrbracket) \rangle_{\varepsilon} \]
  \[ + D_{\rho^4, \varepsilon} \left( \phi_{\varepsilon}, 3\lambda \llbracket
     X_{\varepsilon}^2 \rrbracket, \Q_{\varepsilon}^{- 1} [3\lambda \llbracket
     X_{\varepsilon}^2 \rrbracket \succ \phi_{\varepsilon}] \right), \]
  \[ \Psi_{\rho^4, \varepsilon} \assign \langle \rho^4 \phi_{\varepsilon}, - 3\lambda
     \llbracket X_{\varepsilon}^2 \rrbracket \prec (Y_{\varepsilon} +
     \phi_{\varepsilon}) - 3\lambda X_{\varepsilon} (Y_{\varepsilon} +
     \phi_{\varepsilon})^2 - \lambda Y_{\varepsilon}^3 - 3 \lambda Y_{\varepsilon}^2
     \phi_{\varepsilon} - 3\lambda Y_{\varepsilon} \phi_{\varepsilon}^2 \rangle_{\varepsilon} \]
  \[ + \langle \rho^4 \phi_{\varepsilon}, - 3\lambda ( \UU^{\varepsilon}_{\leqslant}
     \llbracket X_{\varepsilon}^2 \rrbracket ) \succ Y_{\varepsilon} + \lambda^2
     Z_{\varepsilon} \rangle_{\varepsilon}, \]
  and
  \begin{equation}\label{eq:def-Z}
   Z_{\varepsilon} \assign \tthreethreer{X_{\varepsilon}} +
     \ttwothreer{\tilde{X}_{\varepsilon}} Y_{\varepsilon} + 3
     (\tilde{b}_{\varepsilon} - b_{\varepsilon}) Y_{\varepsilon} +
     \bar{C}_{\varepsilon} (Y_{\varepsilon}, 3 \llbracket X_{\varepsilon}^2
     \rrbracket, 3 \llbracket X_{\varepsilon}^2 \rrbracket) - 3 \llbracket
     X_{\varepsilon}^2 \rrbracket \circ \LL_{\varepsilon}^{- 1} \left( 3
     \UU^{\varepsilon}_{\leqslant} \llbracket X_{\varepsilon}^2 \rrbracket
     \succ Y_{\varepsilon} \right) .
     \end{equation}
\end{lemma}

\begin{proof}
  Noting that {\eqref{eq:phi-eq}} is of the form $\LL_{\varepsilon}
  \phi_{\varepsilon} +\lambda \phi_{\varepsilon}^3 = U_{\varepsilon}$, we may test
  this equation by $\rho^4 \phi_{\varepsilon}$ to deduce
  \[ \frac{1}{2} \partial_t \langle \rho^2 \phi_{\varepsilon}, \rho^2
     \phi_{\varepsilon} \rangle_{\varepsilon} + \lambda \langle \rho^2 \phi_{\varepsilon}, \rho^2
     \phi_{\varepsilon}^3 \rangle_{\varepsilon} = \Phi_{\rho^4, \varepsilon} + \Psi_{\rho^4,
     \varepsilon}, \]
  with
  \[ \Phi_{\rho^4, \varepsilon} \assign \langle \rho^4 \phi_{\varepsilon}, -
     \Q_{\varepsilon} \phi_{\varepsilon} - 3\lambda \llbracket X_{\varepsilon}^2
     \rrbracket \succ \phi_{\varepsilon} - 3 \lambda\llbracket X_{\varepsilon}^2
     \rrbracket \circ \phi_{\varepsilon} - 3\lambda^{2} b_{\varepsilon}
     \phi_{\varepsilon} \rangle_{\varepsilon}, \]
  and
  \[ \Psi_{\rho^4, \varepsilon} \assign \langle \rho^4 \phi_{\varepsilon}, - 3\lambda
     \llbracket X_{\varepsilon}^2 \rrbracket \prec (Y_{\varepsilon} +
     \phi_{\varepsilon}) - 3\lambda X_{\varepsilon} (Y_{\varepsilon} +
     \phi_{\varepsilon})^2 -\lambda Y_{\varepsilon}^3 - 3\lambda Y_{\varepsilon}^2
     \phi_{\varepsilon} - 3\lambda Y_{\varepsilon} \phi_{\varepsilon}^2 \rangle_{\varepsilon} \]
  \[ + \langle \rho^4 \phi_{\varepsilon}, - 3\lambda \left(
     \UU^{\varepsilon}_{\leqslant L} \llbracket X_{\varepsilon}^2 \rrbracket
     \right) \succ Y_{\varepsilon} - 3\lambda \llbracket X_{\varepsilon}^2 \rrbracket
     \circ Y_{\varepsilon} - 3\lambda^2 b_{\varepsilon} (X_{\varepsilon} +
     Y_{\varepsilon}) \rangle_{\varepsilon} . \]
  We use the fact that $(f \succ)$ is an approximate adjoint to $(f \circ)$
  according to Lemma~\ref{lem:dual1} to rewrite the resonant term as
  \[ \langle \rho^4 \phi_{\varepsilon}, - 3\lambda \llbracket X_{\varepsilon}^2
     \rrbracket \circ \phi_{\varepsilon} \rangle_{\varepsilon} = \langle \rho^4
     \phi_{\varepsilon}, - 3\lambda \llbracket X_{\varepsilon}^2 \rrbracket \succ
     \phi_{\varepsilon} \rangle_{\varepsilon} + D_{\rho^4, \varepsilon} (\phi_{\varepsilon},
     - 3\lambda \llbracket X_{\varepsilon}^2 \rrbracket, \phi_{\varepsilon}), \]
  and use the definition of $\psi$ in {\eqref{eq:psi1}} to rewrite
  $\Phi_{\rho, \varepsilon}$ as
  \[ \Phi_{\rho^4, \varepsilon} = \langle \rho^4 \psi_{\varepsilon}, -
     \Q_{\varepsilon} \psi_{\varepsilon} \rangle_{\varepsilon} + \left\langle \left[
     \Q_{\varepsilon}, \rho^4 \right] \Q_{\varepsilon}^{- 1} [3\lambda \llbracket
     X_{\varepsilon}^2 \rrbracket \succ \phi_{\varepsilon}],
     \psi_{\varepsilon} \right\rangle_{\varepsilon} \]
  \[ + \langle \rho^4 [3\lambda \llbracket X_{\varepsilon}^2 \rrbracket \succ
     \phi_{\varepsilon}], \Q_{\varepsilon}^{- 1} [3\lambda \llbracket
     X_{\varepsilon}^2 \rrbracket \succ \phi_{\varepsilon}] \rangle_{\varepsilon} - 3\lambda^2
     b_{\varepsilon} \langle \rho^4 \phi_{\varepsilon}, \phi_{\varepsilon}
     \rangle_{\varepsilon} + D_{\rho^4, \varepsilon} (\phi_{\varepsilon}, - 3 \lambda\llbracket
     X_{\varepsilon}^2 \rrbracket, \phi_{\varepsilon}) . \]
  For the first term we write
  \[ \langle \rho^4 \psi_{\varepsilon}, - \Q_{\varepsilon} \psi_{\varepsilon}
     \rangle_{\varepsilon} = - m^{2} \langle \rho^4 \psi_{\varepsilon}, \psi_{\varepsilon}
     \rangle_{\varepsilon} - \langle \rho^4 \nabla_{\varepsilon} \psi_{\varepsilon},
     \nabla_{\varepsilon} \psi_{\varepsilon} \rangle_{\varepsilon} - \langle
     [\nabla_{\varepsilon}, \rho^4] \psi_{\varepsilon}, \nabla_{\varepsilon}
     \psi_{\varepsilon} \rangle_{\varepsilon} . \]
  Next, we use again Lemma~\ref{lem:dual1} to simplify the quadratic term as
  \[ \langle \rho^4 [3\lambda \llbracket X_{\varepsilon}^2 \rrbracket \succ
     \phi_{\varepsilon}], \Q_{\varepsilon}^{- 1} [3\lambda \llbracket
     X_{\varepsilon}^2 \rrbracket \succ \phi_{\varepsilon}] \rangle_{\varepsilon} =
     \left\langle \rho^4 \phi_{\varepsilon}, 3\lambda \llbracket X_{\varepsilon}^2
     \rrbracket \circ \Q_{\varepsilon}^{- 1} [3\lambda \llbracket X_{\varepsilon}^2
     \rrbracket \succ \phi_{\varepsilon}] \right\rangle_{\varepsilon} \]
  \[ + D_{\rho^4,\varepsilon} \left( \phi_{\varepsilon}, 3\lambda \llbracket X_{\varepsilon}^2
     \rrbracket, \Q_{\varepsilon}^{- 1} [3\lambda \llbracket X_{\varepsilon}^2
     \rrbracket \succ \phi_{\varepsilon}] \right), \]
  hence Lemma~\ref{lem:comm1} leads to
  \[ = \left\langle \rho^4 \phi_{\varepsilon}^2, 9\lambda^2 \llbracket
     X_{\varepsilon}^2 \rrbracket \circ \Q_{\varepsilon}^{- 1} \llbracket
     X_{\varepsilon}^2 \rrbracket \right\rangle_{\varepsilon} + \langle \rho^4
     \phi_{\varepsilon}, \tilde{C}_{\varepsilon} (\phi, 3\lambda \llbracket
     X_{\varepsilon}^2 \rrbracket, 3\lambda \llbracket X_{\varepsilon}^2 \rrbracket)
     \rangle_{\varepsilon} \]
  \[ + D_{\rho^4,\varepsilon} \left( \phi_{\varepsilon}, 3\lambda \llbracket X_{\varepsilon}^2
     \rrbracket, \Q_{\varepsilon}^{- 1} [3\lambda \llbracket X_{\varepsilon}^2
     \rrbracket \succ \phi_{\varepsilon}] \right) . \]
  We conclude that
  \[ \Phi_{\rho^4, \varepsilon} = - m^{2} \langle \rho^4 \psi_{\varepsilon},
     \psi_{\varepsilon} \rangle_{\varepsilon} - \langle \rho^4 \nabla_{\varepsilon}
     \psi_{\varepsilon}, \nabla_{\varepsilon} \psi_{\varepsilon} \rangle_{\varepsilon} 
     - \langle [\nabla_{\varepsilon}, \rho^4] \psi_{\varepsilon},
     \nabla_{\varepsilon} \psi_{\varepsilon} \rangle_{\varepsilon} \]
  \[  + \left\langle \left[
     \Q_{\varepsilon}, \rho^4 \right] \Q_{\varepsilon}^{- 1} [3\lambda \llbracket
     X_{\varepsilon}^2 \rrbracket \succ \phi_{\varepsilon}],
     \psi_{\varepsilon} \right\rangle_{\varepsilon} + \left\langle \rho^4
     \phi_{\varepsilon}^2, 9\lambda^2 \llbracket X_{\varepsilon}^2 \rrbracket \circ
     \Q_{\varepsilon}^{- 1} \llbracket X_{\varepsilon}^2 \rrbracket - 3 \lambda^2
     b_{\varepsilon} \right\rangle_{\varepsilon} \]
  \[ + D_{\rho^4, \varepsilon} (\phi_{\varepsilon}, - 3\lambda \llbracket
     X_{\varepsilon}^2 \rrbracket, \phi_{\varepsilon}) + \langle \rho^4
     \phi_{\varepsilon}, \tilde{C}_{\varepsilon} (\phi_{\varepsilon}, 3\lambda
     \llbracket X_{\varepsilon}^2 \rrbracket, 3\lambda \llbracket X_{\varepsilon}^2
     \rrbracket) \rangle_{\varepsilon} \]
  \[ + D_{\rho^4, \varepsilon} \left( \phi_{\varepsilon}, 3\lambda \llbracket
     X_{\varepsilon}^2 \rrbracket, \Q_{\varepsilon}^{- 1} [3\lambda \llbracket
     X_{\varepsilon}^2 \rrbracket \succ \phi_{\varepsilon}] \right) . \]
  As the next step, we justify the definition of the resonant product
  appearing in $\Psi_{\rho^4, \varepsilon}$ and show that it is given by
  $Z_{\varepsilon}$ from the statement of the lemma. To this end, let
  \[ Z_{\varepsilon} \assign - 3\lambda^{-1} \llbracket X_{\varepsilon}^2 \rrbracket \circ
     Y_{\varepsilon} - 3 b_{\varepsilon} (X_{\varepsilon} + Y_{\varepsilon}),
  \]
  and recall the definition of $Y_{M,\varepsilon}$ \eqref{eq:YY}. Hence by Lemma~\ref{lem:comm1}
  \[ Z_{\varepsilon} = 3 \llbracket X_{\varepsilon}^2 \rrbracket \circ
     \tthreeone{X_{\varepsilon}} - 3 b_{\varepsilon} X_{\varepsilon} + 3
     \llbracket X_{\varepsilon}^2 \rrbracket \circ \LL_{\varepsilon}^{- 1} (3
     \llbracket X_{\varepsilon}^2 \rrbracket \succ Y_{\varepsilon}) - 3
     b_{\varepsilon} Y_{\varepsilon} \]
  \[ - 3 \llbracket X_{\varepsilon}^2 \rrbracket \circ \LL_{\varepsilon}^{- 1}
     ( 3 \UU^{\varepsilon}_{\leqslant} \llbracket X_{\varepsilon}^2
     \rrbracket \succ Y_{\varepsilon} ) \]
  \[ = (3 \llbracket X_{\varepsilon}^2 \rrbracket \circ
     \tthreeone{X_{\varepsilon}} - 3 b_{\varepsilon} X_{\varepsilon}) + (
     3 \llbracket X_{\varepsilon}^2 \rrbracket \circ \LL_{\varepsilon}^{- 1} 3
     \llbracket X_{\varepsilon}^2 \rrbracket - 3 \tilde{b}_{\varepsilon}
     ) Y_{\varepsilon} + 3 (\tilde{b}_{\varepsilon} - b_{\varepsilon})
     Y_{\varepsilon} \]
  \[ + \bar{C}_{\varepsilon} (Y_{\varepsilon}, 3 \llbracket X_{\varepsilon}^2
     \rrbracket, 3 \llbracket X_{\varepsilon}^2 \rrbracket) - 3 \llbracket
     X_{\varepsilon}^2 \rrbracket \circ \LL_{\varepsilon}^{- 1} \left( 3
     \UU_{\leqslant} \llbracket X_{\varepsilon}^2 \rrbracket \succ
     Y_{\varepsilon} \right), \]
  which is the desired formula. In this formulation we clearly see
  the structure of the renormalization and the appropriate combinations of
  resonant products and the counterterms.
\end{proof}

As the next step, we estimate the new stochastic terms appearing in Lemma
\ref{lem:energy12}. Here and in the sequel, $\vartheta=O(\kappa)>0$ denotes a generic small constant which changes from line to line.

\begin{lemma}
  \label{lem:Z}
It holds true
  \[ \| Z_{\varepsilon} (t) \|_{\CC^{- 1 / 2 - \kappa, \varepsilon}
     (\rho^{\sigma})} \lesssim (1+\lambda |\log t| +\lambda^2)  \|     \mathbb{X}_{\varepsilon}\|^{7+\vartheta}, \]
  \[ \| X_{\varepsilon} Y_{\varepsilon} \|_{C_T \CC^{- 1 / 2 - \kappa,
     \varepsilon} (\rho^{\sigma})}  \lesssim (\lambda+\lambda^2)
    \|     \mathbb{X}_{\varepsilon}\|^{6}, \]
     \[ \| X_{\varepsilon} Y_{\varepsilon}^2
     \|_{C_T \CC^{- 1 / 2 - \kappa, \varepsilon} (\rho^{\sigma})} \lesssim (\lambda^{2}+\lambda^3)
     \|     \mathbb{X}_{\varepsilon}\|^{9}. \]
\end{lemma}

\begin{proof}
  By definition of $Z_{\varepsilon}$ and the discussion in Section~\ref{ssec:stoch}, Lemma~\ref{lem:Y1}, Lemma~\ref{lem:comm1}, Lemma~\ref{lem:loc} and {\eqref{eq:U11}} we have (since the choice of exponent
  $\sigma > 0$ of the weight corresponding to the stochastic objects is
  arbitrary, $\sigma$ changes from line to line in the sequel)
  \[ \| Z_{\varepsilon} (t) \|_{\CC^{- 1 / 2 - \kappa, \varepsilon} (\rho^{3
     \sigma})} \lesssim \| \tthreethreer{X_{\varepsilon}} \|_{C_T
     \CC^{- 1 / 2 - \kappa, \varepsilon} (\rho^{3 \sigma})} + \|
     \ttwothreer{\tilde{X}_{\varepsilon}} \|_{C_T \CC^{- \kappa,
     \varepsilon} (\rho^{\sigma})} \| Y_{\varepsilon} \|_{C_T \CC^{1 / 2 -
     \kappa, \varepsilon} (\rho^{\sigma})} \]
  \[ + | \log t | \| Y_{\varepsilon} \|_{C_T \CC^{1 / 2 - \kappa, \varepsilon}
     (\rho^{\sigma})} + \big( \| Y_{\varepsilon} \|_{C \CC^{1 / 2 - \kappa,
     \varepsilon} (\rho^{\sigma})} + \| Y_{\varepsilon} \|_{C_T^{\beta / 2}
     L^{\infty, \varepsilon} (\rho^{\sigma})} \big) \| \llbracket
     X_{\varepsilon}^2 \rrbracket \|^2_{C \CC^{- 1 - \kappa, \varepsilon}
     (\rho^{\sigma})} \]
  \[ + (1+\lambda \| \llbracket X_{\varepsilon}^2 \rrbracket \|_{C_T \CC^{- 1 -
     \kappa, \varepsilon} (\rho^{\sigma})})^{6\kappa} \| \llbracket X_{\varepsilon}^2 \rrbracket \|^{2}_{C_T \CC^{- 1 -
     \kappa, \varepsilon} (\rho^{\sigma})} \| Y_{\varepsilon} \|_{C_T \CC^{1
     / 2 - \kappa, \varepsilon} (\rho^{\sigma})}\]
     \[ \lesssim (1+\lambda+\lambda|\log t| +\lambda^{2}) \|
\mathbb{X}_{\varepsilon}\|^{7+\vartheta} \]
  and the first claim follows since $\sigma > 0$ was chosen arbitrarily.
  
  Next, we recall \eqref{eq:YY} and the fact that
  $\tthreetwor{X_{\varepsilon}} = X_{\varepsilon} \circ
  \tthreeone{X_{\varepsilon}}$ can be constructed without any renormalization
  in $C_T \CC^{- \kappa, \varepsilon} (\rho^{\sigma})$. As a consequence, the
  resonant term reads
  \begin{equation}\label{eq:XY-res3}
  X_{\varepsilon} \circ Y_{\varepsilon} = - \lambda \tthreetwor{X_{\varepsilon}} -
     X_{\varepsilon} \circ \LL_{\varepsilon}^{- 1} \left[ 3\lambda\left(
     \UU^{\varepsilon}_{>} \llbracket X_{\varepsilon}^2 \rrbracket \right)
     \succ Y_{\varepsilon} \right],
  \end{equation}
  where the for the second term we have (since $\UU^{\varepsilon}_{>}$ is a contraction) that
  \[ \lambda \left\| X_{\varepsilon} \circ \LL_{\varepsilon}^{- 1} \left[ 3 \left(
     \UU^{\varepsilon}_{>} \llbracket X_{\varepsilon}^2 \rrbracket \right)
     \succ Y_{\varepsilon} \right] \right\|_{C_T \CC^{1 / 2 - 2 \kappa,
     \varepsilon} (\rho^{3 \sigma})} \]
  \[ \lesssim \lambda \| X_{\varepsilon} \|_{C_T \CC^{- 1 / 2 - \kappa, \varepsilon}
     (\rho^{\sigma})} \left\| \left( \UU^{\varepsilon}_{>} \llbracket
     X_{\varepsilon}^2 \rrbracket \right) \succ Y_{\varepsilon} \right\|_{C_T
     \CC^{- 1 - \kappa, \varepsilon} (\rho^{2 \sigma})} \]
  \begin{equation}\label{eq:XY-res}
  \lesssim \lambda \| X_{\varepsilon} \|_{C_T \CC^{- 1 / 2 - \kappa, \varepsilon}
     (\rho^{\sigma})} \| \llbracket X_{\varepsilon}^2 \rrbracket \|_{C_T
     \CC^{- 1 - \kappa, \varepsilon} (\rho^{\sigma})} \| Y_{\varepsilon}
     \|_{C_T L^{\infty, \varepsilon} (\rho^{\sigma})} \lesssim \lambda^2 \|\mathbb{X}_{\varepsilon}\|^{6}.
     \end{equation}
  For the two paraproducts we obtain directly
  \begin{equation}\label{eq:XY-par1}
  \| X_{\varepsilon} \prec Y_{\varepsilon} \|_{C_T \CC^{- 2 \kappa,
     \varepsilon} (\rho^{3 \sigma})} \lesssim \| X_{\varepsilon} \|_{C_T
     \CC^{- 1 / 2 - \kappa, \varepsilon} (\rho^{\sigma})} \| Y_{\varepsilon}
     \|_{C_T \CC^{1 / 2 - \kappa, \varepsilon} (\rho^{\sigma})} \lesssim \lambda
\|\mathbb{X}_{\varepsilon}\|^{4},
\end{equation}
    \begin{equation}\label{eq:XY-par2}
     \| X_{\varepsilon} \succ Y_{\varepsilon} \|_{C_T \CC^{- 1 / 2 - \kappa,
     \varepsilon} (\rho^{3 \sigma})} \lesssim \| X_{\varepsilon} \|_{C_T
     \CC^{- 1 / 2 - \kappa, \varepsilon} (\rho^{\sigma})} \| Y_{\varepsilon}
     \|_{C_T L^{\infty, \varepsilon} (\rho^{\sigma})} \lesssim \lambda \|
     \mathbb{X}_{\varepsilon}\|^{4} .
     \end{equation}
  We proceed similarly for the remaining term, which is quadratic in
  $Y_{\varepsilon}$. We have
  \[ X_{\varepsilon} \circ Y_{\varepsilon}^2 = X_{\varepsilon} \circ (2
     Y_{\varepsilon} \prec Y_{\varepsilon}) + X_{\varepsilon} \circ
     (Y_{\varepsilon} \circ Y_{\varepsilon}) \]
  \[ = - X_{\varepsilon} \circ ( 2 Y_{\varepsilon} \prec \lambda \tthreeone{X}
     ) - X_{\varepsilon} \circ \left( 2 Y_{\varepsilon} \prec
     \LL_{\varepsilon}^{- 1} \left[ 3 \lambda \left( \UU_{>} \llbracket
     X_{\varepsilon}^2 \rrbracket \right) \succ Y_{\varepsilon} \right]
     \right) + X_{\varepsilon} \circ (Y_{\varepsilon} \circ Y_{\varepsilon})
  \]
  \[ = - 2 \lambda \tthreetwor{X_{\varepsilon}} Y_{\varepsilon} - \lambda C_{\varepsilon}
     ( Y_{\varepsilon}, 2 \tthreeone{X_{\varepsilon}}, X_{\varepsilon}
     ) -\lambda  X_{\varepsilon} \circ \left( 2 Y_{\varepsilon} \prec
     \LL_{\varepsilon}^{- 1} \left[ 3 \left( \UU^{\varepsilon}_{>} \llbracket
     X_{\varepsilon}^2 \rrbracket \right) \succ Y_{\varepsilon} \right]
     \right) + X_{\varepsilon} \circ (Y_{\varepsilon} \circ Y_{\varepsilon}) .
  \]
  Accordingly,
  \[ \| X_{\varepsilon} \circ Y_{\varepsilon}^2 \|_{C_T \CC^{- \kappa,
     \varepsilon} (\rho^{4 \sigma})} \lesssim \lambda \|
     \tthreetwor{X_{\varepsilon}} \|_{C_T \CC^{- \kappa, \varepsilon}
     (\rho^{\sigma})} \| Y_{\varepsilon} \|_{C_T \CC^{2 \kappa,
     \varepsilon} (\rho^{\sigma})} \]
  \[ + \lambda \| Y_{\varepsilon} \|_{C \CC^{3\kappa, \varepsilon}
     (\rho^{\sigma})} \| \tthreeone{X_{\varepsilon}} \|_{C_T \CC^{1
     / 2 - \kappa, \varepsilon} (\rho^{\sigma})} \| X_{\varepsilon} \|_{C_T
     \CC^{- 1 / 2 - \kappa, \varepsilon} (\rho^{\sigma})} \]
  \[ + \lambda \| X_{\varepsilon} \|_{C_T \CC^{- 1 / 2 - \kappa, \varepsilon}
     (\rho^{\sigma})} \| Y_{\varepsilon} \|_{C_T L^{\infty,
     \varepsilon} (\rho^{\sigma})}^2 \| \llbracket X_{\varepsilon}^2
     \rrbracket \|_{C_T \CC^{- 1 - \kappa, \varepsilon} (\rho^{\sigma})} \]
  \begin{equation}\label{eq:XY2-res}
  + \| X_{\varepsilon} \|_{C_T \CC^{- 1 / 2 - \kappa, \varepsilon}
     (\rho^{\sigma})} \| Y_{\varepsilon} \|_{C_T \CC^{3\kappa,
     \varepsilon} (\rho^{\sigma})}  \| Y_{\varepsilon} \|_{C_T \CC^{1 / 2 - \kappa,
     \varepsilon} (\rho^{\sigma})} \lesssim (\lambda^2 + \lambda^3) 
\|\mathbb{X}_{\varepsilon}\|^{9} 
  \end{equation}
  and for the paraproducts
  \[ \| X_{\varepsilon} \prec Y_{\varepsilon}^2 \|_{C_T \CC^{- 2 \kappa,
     \varepsilon} (\rho^{4 \sigma})} \lesssim \| X_{\varepsilon} \|_{C_T
     \CC^{- 1 / 2 - \kappa, \varepsilon} (\rho^{\sigma})} \| Y_{\varepsilon}
     \|^2_{C_T \CC^{1 / 2 - \kappa, \varepsilon} (\rho^{\sigma})} \lesssim \lambda^2
     \|\mathbb{X}_{\varepsilon}\|^{7}, \]
  \[ \| X_{\varepsilon} \succ Y_{\varepsilon}^2 \|_{C_T \CC^{- 1 / 2 - \kappa,
     \varepsilon} (\rho^{4 \sigma})} \lesssim \| X_{\varepsilon} \|_{C_T
     \CC^{- 1 / 2 - \kappa, \varepsilon} (\rho^{\sigma})} \| Y_{\varepsilon}
     \|^2_{C_T L^{\infty, \varepsilon} (\rho^{\sigma})} \lesssim \lambda^2 
     \|\mathbb{X}_{\varepsilon}\|^{7} . \]
       This gives the second bound from the statement of the lemma.
\end{proof}

Let us now proceed with our main energy estimate. In view of Lemma
\ref{lem:energy12}, our goal is to control the terms in $\Theta_{\rho^4,
\varepsilon} + \Psi_{\rho^4, \varepsilon}$ by quantities of the from
\[ c(\lambda) Q_{\rho} (\mathbb{X}_{\varepsilon}) + \delta (\lambda \| \rho \phi_{\varepsilon}
   \|_{L^{4, \varepsilon}}^4 + m^{2} \| \rho^2 \psi_{\varepsilon} \|_{L^{2,
   \varepsilon}}^2 + \| \rho^2 \nabla_{\varepsilon} \psi_{\varepsilon}
   \|_{L^{2, \varepsilon}}^2), \]
where $\delta > 0$ is a small constant which can change from line to line.
Indeed, with such a bound in hand it will be possible to absorb the norms of
$\phi_{\varepsilon}, \psi_{\varepsilon}$ from the right hand side of
{\eqref{eq:en12}} into the left hand side and a bound for $\phi_{\varepsilon},
\psi_{\varepsilon}$ in terms of the noise terms will follow.

\begin{lemma}
  \label{lemma:bounds-rhs1}Let $\rho$ be a weight such that $\rho^{\iota} \in
  L^{4, 0}$ for some $\iota \in (0, 1)$. Then
  \[ | \Theta_{\rho^4, \varepsilon} | + | \Psi_{\rho^4, \varepsilon} |
     \leqslant ( \lambda^3+\lambda^{(12 - \theta)
     / (2 + \theta)} | \log t|^{4 / (2 + \theta)}+ \lambda^{7}) Q_{\rho} (\mathbb{X}_{\varepsilon}) \]
  \[ + \delta (\lambda \| \rho \phi_{\varepsilon} \|_{L^{4, \varepsilon}}^4 + \|
     \rho^2 \phi_{\varepsilon} \|_{H^{1 - 2 \kappa, \varepsilon}}^2 + m^{2}\|
     \rho^2 \psi_{\varepsilon} \|^2_{L^{2, \varepsilon}} + \| \rho^2 \nabla_{\varepsilon}
     \psi_{\varepsilon} \|^2_{L^{2, \varepsilon}}), \]
where $\theta=\frac{1/2-4\kappa}{1-2\kappa}$.
\end{lemma}



\begin{proof}
  Since the weight $\rho$ is polynomial and vanishes at infinity, we may
  assume without loss of generality that $0 < \rho \leqslant 1$ and
  consequently $\rho^{\alpha} \leqslant \rho^{\beta}$ whenever $\alpha
  \geqslant \beta \geqslant 0$. We also observe that due to the integrability
  of the weight (see Lemma~\ref{lem:15})
  \[ \| \rho^{1 + \iota} \phi_{\varepsilon} \|_{L^{2, \varepsilon}} \lesssim
     \| \rho \phi_{\varepsilon} \|_{L^{4, \varepsilon}} \]
  with a constant that depends only on $\rho$. In the sequel, we repeatedly
  use various results for discrete Besov spaces established in
  Section~\ref{s:app}. Namely, the equivalent formulation of the Besov norms
  (Lemma~\ref{lem:equiv2}), the duality estimate (Lemma~\ref{lem:dual2}),
  interpolation (Lemma~\ref{lem:int}), embedding (Lemma~\ref{lem:emb}), a
  bound for powers of functions (Lemma~\ref{lem:mult}) as well as bounds for
  the commutators (Lemma~\ref{lem:comm1}).
  
  Even though it is not necessary for the present proof, we keep track of the precise power of the quantity $\|\mathbb {X}_{\varepsilon}\|$ in each of the estimates. This will be used in Section \ref{s:exp} below to establish the stretched exponential integrability of the fields. We recall that $\vartheta=O(\kappa)>0$ denotes a generic small constant which changes from line to line.
  
  In view of Lemma~\ref{lem:energy12} we shall bound each term on the right
  hand side of {\eqref{eq:en12}}. We have
  \[ | \langle [\nabla_{\varepsilon}, \rho^4] \psi_{\varepsilon},
     \nabla_{\varepsilon} \psi_{\varepsilon} \rangle_{\varepsilon} | \leqslant
     C_{\rho}  \| \rho^2 \psi_{\varepsilon} \|_{L^{2, \varepsilon}}  \| \rho^2
     \nabla_{\varepsilon} \psi_{\varepsilon} \|_{L^{2, \varepsilon}} \leqslant
     C_{\delta} C_{\rho}^2  \| \rho^2 \psi_{\varepsilon} \|^2_{L^{2,
     \varepsilon}} + \delta \| \rho^2 \nabla_{\varepsilon} \psi_{\varepsilon}
     \|^2_{L^{2, \varepsilon}} . \]
  This term can be absorbed provided $C_{\rho} = \| \rho^{- 4}
  [\nabla_{\varepsilon}, \rho^4]\|_{L^{\infty, \varepsilon}}$ is sufficiently
  small, such that $C_{\delta} C^2_{\rho} \leqslant m^2$, which can be
  obtained by choosing $h > 0$ small enough (depending only on $m^2$ and
  $\delta$) in the definition~{\eqref{eq:weight}} of the weight $\rho$. Next,
  \[ \left| \left\langle \left[ \Q_{\varepsilon}, \rho^4 \right]
     \Q_{\varepsilon}^{- 1} [3 \lambda \llbracket X_{\varepsilon}^2 \rrbracket
     \succ \phi_{\varepsilon}], \psi_{\varepsilon} \right\rangle_{\varepsilon}
     \right| \leqslant \left| \left\langle \Q_{\varepsilon}^{- 1} [3 \lambda
     \llbracket X_{\varepsilon}^2 \rrbracket \succ \phi_{\varepsilon}], \left[
     \Q_{\varepsilon}, \rho^4 \right] \psi_{\varepsilon}
     \right\rangle_{\varepsilon} \right| \]
  and we estimate explicitly
  \[ \left| \rho^{- 2} \left[ \Q_{\varepsilon}, \rho^4 \right]
     \psi_{\varepsilon} \right|_{L^{2, \varepsilon}} \leqslant C_{\rho}  (\|
     \rho^2 \psi_{\varepsilon} \|_{L^{2, \varepsilon}} +\| \rho^2
     \nabla_{\varepsilon} \psi_{\varepsilon} \|_{L^{2, \varepsilon} (\rho^2)})
  \]
  for another constant $C_{\rho}$ depending only on the weight $\rho$, which
  can be taken smaller than $m^2$ by choosing $h > 0$ small, and consequently
  \[ \left| \left\langle \left[ \Q_{\varepsilon}, \rho^4 \right]
     \Q_{\varepsilon}^{- 1} [3 \lambda \llbracket X_{\varepsilon}^2 \rrbracket
     \succ \phi_{\varepsilon}], \psi_{\varepsilon} \right\rangle_{\varepsilon}
     \right| \lesssim \lambda \|\mathbb{X}_{\varepsilon} \|^2  \| \rho^{2 -
     \sigma} \phi_{\varepsilon} \|_{L^{2, \varepsilon}}  (m^2 \| \rho^2
     \psi_{\varepsilon} \|_{L^{2, \varepsilon}} +\| \rho^2
     \nabla_{\varepsilon} \psi_{\varepsilon} \|_{L^{2, \varepsilon}}) \]
  \[ \leqslant \lambda^3 C_{\delta} \|\mathbb{X}_{\varepsilon} \|^8 + \delta
     (\lambda \| \rho \phi_{\varepsilon} \|^4_{L^{4, \varepsilon}} + m^2 \|
     \rho^2 \psi_{\varepsilon} \|^2_{L^{2, \varepsilon}} +\| \rho^2
     \nabla_{\varepsilon} \psi_{\varepsilon} \|^2_{L^{2, \varepsilon}}), \]
  since $\sigma$ is sufficiently small.
  
  Using Lemma~\ref{lem:dual2}, Lemma~\ref{lem:mult}, interpolation from Lemma~\ref{lem:int} with for $\theta = \frac{1 -
  4 \kappa}{1 - 2 \kappa}$ and Young's inequality 
  we obtain  
  \[ | \lambda^2 \langle \rho^4 \phi_{\varepsilon}^2,
     \ttwothreer{X_{\varepsilon}} \rangle_{\varepsilon} | \lesssim \lambda^2
     \| \rho^{\sigma} \ttwothreer{X_{\varepsilon}} \|_{\CC^{- \kappa,
     \varepsilon}}  \| \rho^{4 - \sigma} \phi_{\varepsilon}^2 \|_{B^{\kappa,
     \varepsilon}_{1, 1}} \lesssim \lambda^2 \| \rho^{\sigma}
     \ttwothreer{X_{\varepsilon}} \|_{\CC^{- \kappa, \varepsilon}}  \| \rho^{1
     + \iota} \phi_{\varepsilon} \|_{L^{2, \varepsilon}}  \| \rho^{3 - \iota -
     \sigma} \phi_{\varepsilon} \|_{H^{2 \kappa, \varepsilon}} \]
  \[ \lesssim \lambda^2 \|\mathbb{X}_{\varepsilon} \|^4  \| \rho
     \phi_{\varepsilon} \|^{1 + \theta}_{L^{4, \varepsilon}}  \| \rho^2
     \phi_{\varepsilon} \|^{1 - \theta}_{H^{1 - 2 \kappa, \varepsilon}}
     \leqslant \lambda^{(7 - \theta) / (1 + \theta)} C_{\rho}
     \|\mathbb{X}_{\varepsilon} \|^{8 + \vartheta} + \delta (\lambda \| \rho
     \phi_{\varepsilon} \|_{L^{4, \varepsilon}}^4 +\| \rho^2
     \phi_{\varepsilon} \|_{H^{1 - 2 \kappa, \varepsilon}}^2) . \]
  Recall that since $\sigma$ is chosen small, we have the interpolation
  inequality (see Lemma~\ref{lem:int})
  \[ \| \phi_{\varepsilon} \|_{H^{1 / 2 + \kappa, \varepsilon} (\rho^{2 -
     \sigma / 2})} \leqslant \| \phi_{\varepsilon} \|_{L^{2, \varepsilon}
     (\rho^{1 + \iota})}^{\theta} \| \phi_{\varepsilon} \|_{H^{1 - 2 \kappa,
     \varepsilon} (\rho^2)}^{1 - \theta} \]
  where $\theta = \frac{1 / 2 - 3 \kappa}{1 - 2 \kappa}$. Similar
  interpolation inequalities will also be employed below. Then, in view of
  Lemma~\ref{lem:dual1} and Young's inequality, we have
  \[ \lambda |D_{\rho^4, \varepsilon} (\phi_{\varepsilon}, - 3 \llbracket
     X_{\varepsilon}^2 \rrbracket, \phi_{\varepsilon}) | \lesssim \lambda \|
     \rho^{\sigma} \llbracket X_{\varepsilon}^2 \rrbracket \|_{\CC^{- 1 -
     \kappa, \varepsilon}}  \| \rho^{2 - \sigma / 2} \phi_{\varepsilon}
     \|_{H^{1 / 2 + \kappa, \varepsilon}}^2 \]
  \[ \lesssim \lambda \| \rho^{\sigma} \llbracket X_{\varepsilon}^2 \rrbracket
     \|_{\CC^{- 1 - \kappa, \varepsilon}} \| \rho^{1 + \iota}
     \phi_{\varepsilon} \|_{L^{2, \varepsilon}}^{2 \theta}  \| \rho^2
     \phi_{\varepsilon} \|_{H^{1 - 2 \kappa, \varepsilon}}^{2 (1 - \theta)} \]
  \[ \lesssim \lambda \|\mathbb{X}_{\varepsilon} \|^2  \| \rho
     \phi_{\varepsilon} \|_{L^{4, \varepsilon}}^{2 \theta}  \| \rho^2
     \phi_{\varepsilon} \|_{H^{1 - 2 \kappa, \varepsilon}}^{2 (1 - \theta)}
     \leqslant \lambda^{2 / \theta - 1} C_{\delta} \|\mathbb{X}_{\varepsilon}
     \|^{8 + \vartheta} + \delta (\lambda \| \rho \phi_{\varepsilon} \|_{L^{4,
     \varepsilon}}^4 +\| \rho^2 \phi_{\varepsilon} \|_{H^{1 - 2 \kappa,
     \varepsilon}}^2) . \]
  Similarly,
  \[ \lambda^2 \left| D_{\rho^4, \varepsilon} \left( \phi_{\varepsilon}, 3
     \llbracket X_{\varepsilon}^2 \rrbracket, \Q_{\varepsilon}^{- 1} [3
     \llbracket X_{\varepsilon}^2 \rrbracket \succ \phi_{\varepsilon}] \right)
     \right| \]
  \[ \lesssim \lambda^2  \| \rho^{\sigma} \llbracket X_{\varepsilon}^2
     \rrbracket \|_{\CC^{- 1 - \kappa, \varepsilon}} \| \rho^{3 - \iota - 2
     \sigma} \phi_{\varepsilon} \|_{H^{4 \kappa, \varepsilon}} \left| \rho^{1
     + \iota + \sigma} \Q_{\varepsilon}^{- 1} [3 \llbracket X_{\varepsilon}^2
     \rrbracket \succ \phi_{\varepsilon}] \right|_{H^{1 - 2 \kappa,
     \varepsilon}}, \]
  where we further estimate by Schauder and paraproduct estimates
  \[ \left| \rho^{1 + \iota + \sigma} \Q_{\varepsilon}^{- 1} [3 \llbracket
     X_{\varepsilon}^2 \rrbracket \succ \phi_{\varepsilon}] \right|_{H^{1 - 2
     \kappa, \varepsilon}} \lesssim \| \rho^{1 + \iota + \sigma} \llbracket
     X_{\varepsilon}^2 \rrbracket \succ \phi_{\varepsilon} \|_{H^{- 1 - 2
     \kappa, \varepsilon}} \]
  \[ \lesssim \| \rho^{\sigma} \llbracket X_{\varepsilon}^2 \rrbracket
     \|_{\CC^{- 1 - \kappa, \varepsilon}}  \| \rho^{1 + \iota}
     \phi_{\varepsilon} \|_{L^{2, \varepsilon}} \]
  and hence we deduce by interpolation with $\theta = \frac{1 - 6 \kappa}{1 -
  2 \kappa}$ and embedding that
  \[ \lambda^2 \left| D_{\rho^4, \varepsilon} \left( \phi_{\varepsilon}, 3
     \llbracket X_{\varepsilon}^2 \rrbracket, \Q_{\varepsilon}^{- 1} [3
     \llbracket X_{\varepsilon}^2 \rrbracket \succ \phi_{\varepsilon}] \right)
     \right| \lesssim \lambda^2 \|\mathbb{X}_{\varepsilon} \|^4  \| \rho^{1 +
     \iota} \phi_{\varepsilon} \|_{L^{2, \varepsilon}}  \| \rho^2
     \phi_{\varepsilon} \|_{H^{4 \kappa, \varepsilon}} \]
  \[ \lesssim \lambda^2 \|\mathbb{X}_{\varepsilon} \|^4  \| \rho
     \phi_{\varepsilon} \|^{1 + \theta}_{L^{2, \varepsilon}}  \| \rho^2
     \phi_{\varepsilon} \|^{1 - \theta}_{H^{1 - 2 \kappa, \varepsilon}} \]
  \[ \leqslant \lambda^{(7 - \theta) / (1 + \theta)} C_{\delta}
     \|\mathbb{X}_{\varepsilon} \|^{8 + \vartheta} + \delta (\lambda \| \rho
     \phi_{\varepsilon} \|_{L^{4, \varepsilon}}^4 +\| \rho^2
     \phi_{\varepsilon} \|_{H^{1 - 2 \kappa, \varepsilon}}^2) . \]
  Due to Lemma~\ref{lem:comm1} and interpolation with $\theta =
  \frac{1 - 5 \kappa}{1 - 2 \kappa}$, we obtain 
  \[ \lambda^2 | \langle \rho^4 \phi_{\varepsilon}, \tilde{C}
     (\phi_{\varepsilon}, 3 \llbracket X_{\varepsilon}^2 \rrbracket, 3
     \llbracket X_{\varepsilon}^2 \rrbracket) \rangle_{\varepsilon} | \lesssim
     \lambda^2 \| \rho^{\sigma} \llbracket X_{\varepsilon}^2 \rrbracket
     \|_{\CC^{- 1 - \kappa, \varepsilon}}^2  \| \rho^{2 - \sigma}
     \phi_{\varepsilon} \|^2_{H^{3 \kappa, \varepsilon}} \]
  \[ \leqslant \lambda^2 C_{\delta} \|\mathbb{X}_{\varepsilon} \|^4  \|
     \rho^{1 + \iota} \phi_{\varepsilon} \|^{2 \theta}_{L^{2, \varepsilon}} 
     \| \rho^2 \phi_{\varepsilon} \|^{2 (1 - \theta)}_{H^{1 - 2 \kappa,
     \varepsilon}} \]
  \[ \leqslant \lambda^{4 / \theta - 1} C_{\delta} \|\mathbb{X}_{\varepsilon}
     \|^{8 + \vartheta} + \delta (\lambda \| \rho \phi_{\varepsilon} \|_{L^{4,
     \varepsilon}}^4 +\| \rho^2 \phi_{\varepsilon} \|_{H^{1 - 2 \kappa,
     \varepsilon}}^2) . \]
  Then we use the paraproduct estimates, the embedding $\CC^{1 / 2 - \kappa,
  \varepsilon} (\rho^{\sigma}) \subset H^{1 / 2 - 2 \kappa, \varepsilon}
  (\rho^{2 - \sigma / 2})$ (which holds due to the integrability of $\rho^{4
  \iota}$ for some $\iota \in (0, 1)$ and the fact that $\sigma$ can be chosen
  small), together with Lemma~\ref{lem:Y1} and interpolation to deduce for
  $\theta = \frac{1 / 2 - 5 \kappa}{1 - 2 \kappa}$ that
  \[ \lambda | \langle \rho^4 \phi_{\varepsilon}, - 3 \llbracket
     X_{\varepsilon}^2 \rrbracket \prec (Y_{\varepsilon} + \phi_{\varepsilon})
     \rangle_{\varepsilon} | \]
  \[ \lesssim \lambda \| \rho^{\sigma} \llbracket X_{\varepsilon}^2 \rrbracket
     \|_{\CC^{- 1 - \kappa, \varepsilon}} \| \rho^{2 - \sigma / 2}
     (Y_{\varepsilon} + \phi_{\varepsilon})\|_{H^{1 / 2 - 2 \kappa,
     \varepsilon}}  \| \rho^{2 - \sigma / 2} \phi_{\varepsilon} \|_{H^{1 / 2 +
     3 \kappa, \varepsilon}} \]
  \[ \lesssim \lambda \| \rho^{\sigma} \llbracket X_{\varepsilon}^2 \rrbracket
     \|_{\CC^{- 1 - \kappa, \varepsilon}} \| \rho^{2 - \sigma / 2}
     Y_{\varepsilon} \|_{H^{1 / 2 - 2 \kappa, \varepsilon}}  \| \rho^{2 -
     \sigma / 2} \phi_{\varepsilon} \|_{H^{1 / 2 + 3 \kappa, \varepsilon}} \]
  \[ + \lambda \| \rho^{\sigma} \llbracket X_{\varepsilon}^2 \rrbracket
     \|_{\CC^{- 1 - \kappa, \varepsilon}}  \| \rho^{2 - \sigma / 2}
     \phi_{\varepsilon} \|_{H^{1 / 2 + 3 \kappa, \varepsilon}}^2 \]
  \[ \lesssim \lambda  (\lambda \|\mathbb{X}_{\varepsilon} \|^5 \|
     \rho^{1 + \iota} \phi_{\varepsilon} \|_{L^{2, \varepsilon}}^{\theta} \|
     \rho^2 \phi_{\varepsilon} \|_{H^{1 - 2 \kappa, \varepsilon}}^{1 - \theta}
     +\|\mathbb{X}_{\varepsilon} \|^2 \| \rho^{1 + \iota} \phi_{\varepsilon}
     \|_{L^{2, \varepsilon}}^{2 \theta} \| \rho^2 \phi_{\varepsilon} \|_{H^{1
     - 2 \kappa, \varepsilon}}^{2 (1 - \theta)}) \]
  \[ \leqslant (\lambda^{(8 - \theta) / (2 + \theta)} + \lambda^{2 / \theta -
     1}) C_{\delta} \|\mathbb{X}_{\varepsilon} \|^{8 + \vartheta} + \delta
     (\lambda \| \rho \phi_{\varepsilon} \|_{L^{4, \varepsilon}}^4 +\| \rho^2
     \phi_{\varepsilon} \|_{H^{1 - 2 \kappa, \varepsilon}}^2) . \]
  Next, we have
  \[ \lambda | \langle \rho^4 \phi_{\varepsilon}, - 3 X_{\varepsilon}
     (Y_{\varepsilon} + \phi_{\varepsilon})^2 \rangle_{\varepsilon} | \lesssim
     \lambda \| \rho^{\sigma} X_{\varepsilon} \|_{\CC^{- 1 / 2 - \kappa,
     \varepsilon}}  \| \rho^{4 - \sigma} \phi_{\varepsilon}^3 \|_{B_{1, 1}^{1
     / 2 + \kappa, \varepsilon}} \]
  \[ + \lambda \| \rho^{\sigma} X_{\varepsilon} Y_{\varepsilon} \|_{\CC^{- 1 /
     2 - \kappa, \varepsilon}}  \| \rho^{4 - \sigma} \phi_{\varepsilon}^2
     \|_{B_{1, 1}^{1 / 2 + \kappa, \varepsilon}} + \lambda \| \rho^{\sigma}
     X_{\varepsilon} Y_{\varepsilon}^2 \|_{\CC^{- 1 / 2 - \kappa,
     \varepsilon}}  \| \rho^{4 - \sigma} \phi_{\varepsilon} \|_{B_{1, 1}^{1 /
     2 + \kappa, \varepsilon}} . \]
  Here we employ Lemma~\ref{lem:mult} and interpolation to obtain for $\theta
  = \frac{1 / 2 - 4 \kappa}{1 - 2 \kappa}$
  \[ \lambda \| \rho^{\sigma} X_{\varepsilon} \|_{\CC^{- 1 / 2 - \kappa,
     \varepsilon}}  \| \rho^{4 - \sigma} \phi_{\varepsilon}^3 \|_{B_{1, 1}^{1
     / 2 + \kappa, \varepsilon}} \lesssim \lambda \| \rho^{\sigma}
     X_{\varepsilon} \|_{\CC^{- 1 / 2 - \kappa, \varepsilon}}  \| \rho
     \phi_{\varepsilon} \|^2_{L^{4, \varepsilon}}  \| \rho^{2 - \sigma}
     \phi_{\varepsilon} \|_{H^{1 / 2 + 2 \kappa, \varepsilon}} \]
  \[ \lesssim \lambda \|\mathbb{X}_{\varepsilon} \|  \| \rho
     \phi_{\varepsilon} \|_{L^{4, \varepsilon}}^{2 + \theta}  \| \rho^2
     \phi_{\varepsilon} \|_{H^{1 - 2 \kappa, \varepsilon}}^{1 - \theta}
     \leqslant \lambda^{(2 - \theta) / \theta} C_{\delta}
     \|\mathbb{X}_{\varepsilon} \|^{8 + \vartheta} + \delta (\lambda \| \rho
     \phi_{\varepsilon} \|_{L^{4, \varepsilon}}^4 +\| \rho^2
     \phi_{\varepsilon} \|_{H^{1 - 2 \kappa, \varepsilon}}^2) \]
  and similarly for the other two terms, where we also use Lemma~\ref{lem:Z}
  and the embedding $H^{1 - 2 \kappa, \varepsilon} (\rho^2) \subset H^{1 / 2 +
  2 \kappa, \varepsilon} (\rho^{3 - \iota - \sigma})$ and $H^{1 / 2 + 2
  \kappa, \varepsilon} (\rho^2) = B_{2, 2}^{1 / 2 + 2 \kappa, \varepsilon}
  (\rho^2) \subset B_{1, 1}^{1 / 2 + \kappa, \varepsilon} (\rho^{4 - \sigma})$
  together with interpolation with $\theta = \frac{1 / 2 - 4 \kappa}{1 - 2
  \kappa}$
  \[ \lambda \| \rho^{\sigma} X_{\varepsilon} Y_{\varepsilon} \|_{\CC^{- 1 / 2
     - \kappa, \varepsilon}}  \| \rho^{4 - \sigma} \phi_{\varepsilon}^2
     \|_{B_{1, 1}^{1 / 2 + \kappa, \varepsilon}} + \lambda \| \rho^{\sigma}
     X_{\varepsilon} Y_{\varepsilon}^2 \|_{\CC^{- 1 / 2 - \kappa,
     \varepsilon}}  \| \rho^{4 - \sigma} \phi_{\varepsilon} \|_{B_{1, 1}^{1 /
     2 + \kappa, \varepsilon}} \]
  \[ \lesssim (\lambda^2 + \lambda^3) \|\mathbb{X}_{\varepsilon} \|^6  \|
     \rho^{1 + \iota} \phi_{\varepsilon} \|_{L^{2, \varepsilon}}  \| \rho^{3 -
     \iota - \sigma} \phi_{\varepsilon} \|_{H^{1 / 2 + 2 \kappa, \varepsilon}}
     + (\lambda^3 + \lambda^4) \|\mathbb{X}_{\varepsilon} \|^9  \| \rho^2
     \phi_{\varepsilon} \|_{H^{1 / 2 + 2 \kappa, \varepsilon}} \]
  \[ \lesssim (\lambda^2 + \lambda^3) \|\mathbb{X}_{\varepsilon} \|^6  \| \rho
     \phi_{\varepsilon} \|^{1 + \theta}_{L^{4, \varepsilon}}  \| \rho^2
     \phi_{\varepsilon} \|^{1 - \theta}_{H^{1 - 2 \kappa, \varepsilon}} +
     (\lambda^3 + \lambda^4) \|\mathbb{X}_{\varepsilon} \|^9  \| \rho
     \phi_{\varepsilon} \|^{\theta}_{L^{4, \varepsilon}}  \| \rho^2
     \phi_{\varepsilon} \|^{1 - \theta}_{H^{1 - 2 \kappa, \varepsilon}} \]
  \begin{equation}
  \label{eq:XY}
   \leqslant (\lambda^{(11 - \theta) / (2 + \theta)} + \lambda^{(12 -
     \theta) / (2 + \theta)}) C_{\delta} \|\mathbb{X}_{\varepsilon} \|^{16 +
     \vartheta} + \delta (\lambda \| \rho \phi_{\varepsilon} \|_{L^{4,
     \varepsilon}}^4 +\| \rho^2 \phi_{\varepsilon} \|_{H^{1 - 2 \kappa,
     \varepsilon}}^2) .
     \end{equation}
  Next, we obtain
  \begin{equation}\label{eq:Y3}
   \lambda | \langle \rho^4 \phi_{\varepsilon}, - Y_{\varepsilon}^3
     \rangle_{\varepsilon} | \lesssim \lambda \| \rho^{\sigma} Y_{\varepsilon}
     \|_{L^{\infty, \varepsilon}}^3  \| \rho^{4 - 3 \sigma} \phi_{\varepsilon}
     \|_{L^{1, \varepsilon}} \lesssim \lambda^4 \|\mathbb{X}_{\varepsilon}
     \|^9  \| \rho \phi_{\varepsilon} \|_{L^{4, \varepsilon}} \leqslant
     \lambda^5 C_{\delta} \|\mathbb{X}_{\varepsilon} \|^{12} + \delta \lambda
     \| \rho \phi_{\varepsilon} \|^4_{L^{4, \varepsilon}}, 
     \end{equation}
  and similarly
  \[ \lambda | \langle \rho^4 \phi_{\varepsilon}, - 3 Y_{\varepsilon}^2
     \phi_{\varepsilon} \rangle_{\varepsilon} | \lesssim \lambda \|
     \rho^{\sigma} Y_{\varepsilon} \|_{L^{\infty, \varepsilon}}^2  \| \rho^{4
     - \sigma} \phi_{\varepsilon}^2 \|_{L^{1, \varepsilon}} \]
    \begin{equation}\label{eq:Y2}
     \lesssim \lambda^3 \|\mathbb{X}_{\varepsilon} \|^6  \| \rho
     \phi_{\varepsilon} \|^2_{L^{4, \varepsilon}} \leqslant \lambda^5
     C_{\delta} \|\mathbb{X}_{\varepsilon} \|^{12} + \delta \lambda \| \rho
     \phi_{\varepsilon} \|_{L^{4, \varepsilon}}^4, 
     \end{equation}
  \[ \lambda | \langle \rho^4 \phi_{\varepsilon}, - 3 Y_{\varepsilon}
     \phi_{\varepsilon}^2 \rangle_{\varepsilon} | \lesssim \lambda \|
     \rho^{\sigma} Y_{\varepsilon} \|_{L^{\infty, \varepsilon}}  \| \rho^{4 -
     \sigma} \phi_{\varepsilon}^3 \|_{L^{1, \varepsilon}} \lesssim \lambda \|
     \rho^{\sigma} Y_{\varepsilon} \|_{L^{\infty, \varepsilon}}  \| \rho
     \phi_{\varepsilon} \|^3_{L^{4, \varepsilon}} \]
  \begin{equation}\label{eq:Y11}
   \lesssim \lambda^2 \|\mathbb{X}_{\varepsilon} \|^3  \| \rho
     \phi_{\varepsilon} \|_{L^{4, \varepsilon}}^3 \leqslant \lambda^5
     C_{\delta} \|\mathbb{X}_{\varepsilon} \|^{12} + \delta \lambda \| \rho
     \phi_{\varepsilon} \|_{L^{4, \varepsilon}}^4 .
     \end{equation}
  Then, by {\eqref{eq:U11}}
  \[ \lambda \left| \langle \rho^4 \phi_{\varepsilon}, - 3
     (\UU^{\varepsilon}_{\leqslant} \llbracket X^2 \rrbracket) \succ
     Y_{\varepsilon} \rangle_{\varepsilon} \right| \lesssim \lambda \|
     \rho^{\sigma} \UU^{\varepsilon}_{\leqslant} \llbracket X_{\varepsilon}^2
     \rrbracket \|_{\CC^{- 1 + 3 \kappa, \varepsilon}} \| \rho^{\sigma}
     Y_{\varepsilon} \|_{L^{\infty, \varepsilon}}  \| \rho^{4 - 3 \sigma}
     \phi_{\varepsilon} \|_{B^{1 - 3 \kappa, \varepsilon}_{1, 1}} \]
  \[ \lesssim \lambda (1 + \lambda \| \rho^{\sigma} \llbracket
     X_{\varepsilon}^2 \rrbracket \|_{\CC^{- 1 - \kappa, \varepsilon}})^{8
     \kappa}  \| \rho^{\sigma} \llbracket X_{\varepsilon}^2 \rrbracket
     \|_{\CC^{- 1 - \kappa, \varepsilon}} \| \rho^{\sigma} Y_{\varepsilon}
     \|_{L^{\infty, \varepsilon}}  \| \rho^2 \phi_{\varepsilon} \|_{H^{1 - 2
     \kappa, \varepsilon}} \]
  \begin{equation}\label{eq:U0}
   \lesssim (\lambda^{2} + \lambda^{2 + 8 \kappa}) \|\mathbb{X}_{\varepsilon}
     \|^{5 + 16 \kappa}  \| \rho^2 \phi_{\varepsilon} \|_{H^{1 - 2 \kappa,
     \varepsilon}} \leqslant (\lambda^4 + \lambda^5) C_{\delta}
     \|\mathbb{X}_{\varepsilon} \|^{10 + \vartheta} + \delta \| \rho^2
     \phi_{\varepsilon} \|_{H^{1 - 2 \kappa, \varepsilon}}^2,
     \end{equation}
  and finally for $\theta = \frac{1 / 2 - 4 \kappa}{1 - 2 \kappa}$
  \[ \lambda^2 | \langle \rho^4 \phi_{\varepsilon}, Z_{\varepsilon}
     \rangle_{\varepsilon} | \lesssim \lambda^2  \| \rho^{\sigma}
     Z_{\varepsilon} \|_{\CC^{- 1 / 2 - \kappa, \varepsilon}}  \| \rho^{4 -
     \sigma} \phi_{\varepsilon} \|_{B_{1, 1}^{1 / 2 + \kappa, \varepsilon}} \]
  \[ \lesssim (\lambda^2 + \lambda^3 | \log t| + \lambda^4)
     \|\mathbb{X}_{\varepsilon} \|^{7+\vartheta}  \| \rho \phi_{\varepsilon} \|_{L^{4,
     \varepsilon}}^{\theta}  \| \rho^2 \phi \|_{H^{1 - 2 \kappa}}^{1 - \theta}
  \]
 \[
   \leqslant (\lambda^{(8 - \theta) / (2 + \theta)} + \lambda^{(12 - \theta)
     / (2 + \theta)} | \log t|^{4 / (2 + \theta)} + \lambda^{(16 - \theta) /
     (2 + \theta)}) C_{\delta} \|\mathbb{X}_{\varepsilon} \|^{12} \]
   \begin{equation}\label{eq:Z0}
    + \delta (\lambda \| \rho \phi_{\varepsilon} \|_{L^{4, \varepsilon}}^4
     +\| \rho^2 \phi_{\varepsilon} \|_{H^{1 - 2 \kappa, \varepsilon}}^2) .
     \end{equation}
  The proof is complete.
\end{proof}


Now we have all in hand to establish our main energy estimate.

\begin{theorem}
  \label{th:energy-estimate}
  Let $\rho$ be a weight such that $\rho^{\iota} \in
  L^{4, 0}$ for some $\iota \in (0, 1)$. There exists a constant $\alpha=\alpha(m^{2}) \in (0,1)$  such that for    $\theta=\frac{1/2-4\kappa}{1-2\kappa}$
  \begin{equation}
    \frac{1}{2} \partial_t \| \rho^2 \phi_{\varepsilon} \|_{L^{2,
    \varepsilon}}^2 + \alpha [\lambda \| \rho \phi_{\varepsilon} \|_{L^{4,
    \varepsilon}}^4 +  m^2 \| \rho^2 \psi_{\varepsilon} \|_{L^{2,
    \varepsilon}}^2 +  \| \rho^2 \nabla_{\varepsilon} \psi_{\varepsilon}
    \|_{L^{2, \varepsilon}}^2] + \| \rho^2 \phi_{\varepsilon} \|_{H^{1 - 2
    \kappa, \varepsilon}}^2 \label{eq:d18}
  \end{equation}
  \[ \leqslant (\lambda^3 + \lambda^{(12 - \theta)
     / (2 + \theta)} | \log t|^{4 / (2 + \theta)} + \lambda^{7}) Q_{\rho} (\mathbb{X}_{\varepsilon}) . \]
\end{theorem}

\begin{proof}
  As a consequence of {\eqref{eq:psi1}}, we have according to Lemma~\ref{lem:grad}, Lemma~\ref{lem:emb}, Lemma~\ref{lem:equiv2}
  \[ \| \rho^2 \phi_{\varepsilon} \|_{H^{1 - 2 \kappa, \varepsilon}}^2
     \lesssim \left\| \rho^2 \Q_{\varepsilon}^{- 1} [3\lambda\llbracket
     X_{\varepsilon}^2 \rrbracket \succ \phi_{\varepsilon}] \right\|_{H^{1 - 2
     \kappa, \varepsilon}}^2 + \| \rho^2 \psi_{\varepsilon} \|_{H^{1 - 2
     \kappa, \varepsilon}}^2 \]
  \[ \lesssim \lambda^2 \| \rho^{\sigma} \llbracket X_{\varepsilon}^2 \rrbracket
     \|^2_{\CC^{- 1 - \kappa, \varepsilon}} \| \rho^{2 - \sigma}
     \phi_{\varepsilon} \|^2_{L^{2, \varepsilon}} + \| \rho^2
     \psi_{\varepsilon} \|_{H^{1 - \kappa, \varepsilon}}^2 \]
  \begin{equation}
    \lesssim \lambda^3 Q_{\rho} (\mathbb{X}_{\varepsilon}) + \lambda \| \rho \phi_{\varepsilon}
    \|_{L^{4, \varepsilon}}^4 + \| \rho^2 \psi_{\varepsilon} \|_{L^{2,
    \varepsilon}}^2 + \| \rho^2 \nabla_{\varepsilon} \psi_{\varepsilon}
    \|_{L^{2, \varepsilon}}^2 . \label{eq:17}
  \end{equation}
  Therefore, according to Lemma~\ref{lemma:bounds-rhs1} we obtain that
  \[ \frac{1}{2} \partial_t \| \rho^2 \phi_{\varepsilon} \|_{L^{2,
     \varepsilon}}^2 +\lambda \| \rho \phi_{\varepsilon} \|_{L^{4, \varepsilon}}^4 +
     m^{2} \| \rho^2 \psi_{\varepsilon} \|_{L^{2, \varepsilon}}^2 + \| \rho^2
     \nabla_{\varepsilon} \psi_{\varepsilon} \|_{L^{2, \varepsilon}}^2  \]
  \[ \leqslant(\lambda^3 + \lambda^{(12 - \theta)
     / (2 + \theta)} | \log t|^{4 / (2 + \theta)} + \lambda^{7})Q_{\rho} (\mathbb{X}_{\varepsilon}) + \delta C
     (\lambda\| \rho \phi_{\varepsilon} \|_{L^{4, \varepsilon}}^4 + \| \rho^2
     \psi_{\varepsilon} \|_{L^{2, \varepsilon}}^2 + \| \rho^2
     \nabla_{\varepsilon} \psi_{\varepsilon} \|_{L^{2, \varepsilon}}^2). \]
Choosing $\delta > 0$ sufficiently small (depending  on $m^2$ and the implicit constant $C$ from Lemma~\ref{lem:grad}) allows to absorb the norms of
  $\phi_{\varepsilon}, \psi_{\varepsilon}$ from the right hand side into the
  left hand side and the claim follows.
\end{proof}

\begin{remark}\label{rem:neg-mass}
We point out that the requirement of a strictly positive mass $m^{2}>0$ is to some extent superfluous for our approach. To be more precise, if $m^{2}\leqslant 0$ then we may rewrite the mollified stochastic quantization equation as
$$
(\partial_{t}-\Delta_{\varepsilon} +1)\varphi_{\varepsilon} +\lambda\varphi_{\varepsilon}^{3}=\xi_{\varepsilon}+(1-m^{2})\varphi_{\varepsilon}
$$
and the same decomposition as above introduces an additional term on the right hand side of \eqref{eq:en12}. This can be controlled by
$$
|(1-m^{2})\langle \rho^{4}\phi_{\varepsilon},X_{\varepsilon}+Y_{\varepsilon}+\phi_{\varepsilon}\rangle|\lesssim C_{\delta,\lambda^{-1}}Q_{\rho}(\mathbb{X}_{\varepsilon})+\delta(\lambda\|\rho\phi_{\varepsilon}\|_{L^{4,\varepsilon}}^{4}+\|\rho^{2}\phi_{\varepsilon}\|_{H^{1-2\kappa,\varepsilon}}^{2}),
$$
where we write $C_{\delta,\lambda^{-1}}$ to stress that the constant is not uniform over small $\lambda$.
As a consequence, we obtain an analogue of Theorem \ref{th:energy-estimate} but the uniformity  for small $\lambda$ is not valid anymore.
\end{remark}



\begin{corollary}
  \label{cor:Lp}Let $\rho$ be a weight such that $\rho^{\iota} \in L^{4, 0}$
  for some $\iota \in (0, 1)$. Then
  for all $p \in [1, \infty)$ and    $\theta=\frac{1/2-4\kappa}{1-2\kappa}$
  \begin{equation}\label{eq:lp} \frac{1}{2 p} \partial_t \| \rho^2 \phi_{\varepsilon} \|_{L^{2,
     \varepsilon}}^{2 p} + \lambda \| \rho^2 \phi_{\varepsilon} \|_{L^{2,
     \varepsilon}}^{2 p + 2} \leqslant  \lambda [(\lambda^{2} + \lambda^{(10 - 2\theta)
     / (2 + \theta)} | \log t|^{4 / (2 + \theta)} + \lambda^{6}) Q_{\rho}
     (\mathbb{X}_{\varepsilon})]^{(p + 1) / 2} .
     \end{equation}
\end{corollary}

\begin{proof}
  Based on {\eqref{eq:d18}} we obtain
  \[ \frac{1}{2 p} \partial_t \| \rho^2 \phi_{\varepsilon} \|_{L^{2,
     \varepsilon}}^{2 p} + \lambda \| \rho^2 \phi_{\varepsilon} \|_{L^{2,
     \varepsilon}}^{2 (p - 1)} \| \rho \phi_{\varepsilon} \|_{L^{4,
     \varepsilon}}^4 
     \]
     \[
     \leqslant (\lambda^3 + \lambda^{(12 - \theta)
     / (2 + \theta)} | \log t|^{4 / (2 + \theta)} + \lambda^{7}) \| \rho^2 \phi_{\varepsilon}
     \|_{L^{2, \varepsilon}}^{2 (p - 1)} Q_{\rho} (\mathbb{X}_{\varepsilon}) .
  \]
  The $L^4$-norm on the left hand side can be estimated from below by the
  $L^2$-norm, whereas on the right hand side we use Young's inequality to
  deduce
  \[ \frac{1}{2 p} \partial_t \| \rho^2 \phi_{\varepsilon} \|_{L^{2,
     \varepsilon}}^{2 p} + \lambda \| \rho^2 \phi_{\varepsilon} \|_{L^{2,
     \varepsilon}}^{2 p + 2}
     \]
     \[
      \leqslant \lambda [(\lambda^{2} + \lambda^{(10 - 2\theta)
     / (2 + \theta)} | \log t|^{4 / (2 + \theta)} + \lambda^{6}) Q_{\rho}
     (\mathbb{X}_{\varepsilon})]^{(p + 1) / 2} + \delta \lambda \| \rho^2
     \phi_{\varepsilon} \|_{L^{2, \varepsilon}}^{2 p + 2} .\]
Hence we may absorb the second term from the right hand side into the left hand
  side.
\end{proof}

\subsection{Tightness of the invariant measures}

\label{ss:tight}Recall that $\varphi_{M, \varepsilon}$ is a stationary
solution to {\eqref{eq:moll}} having at time $t \geqslant 0$ law given by the
Gibbs measure $\nu_{M, \varepsilon}$. Moreover, we have the decomposition
$\varphi_{M, \varepsilon} = X_{M, \varepsilon} + Y_{M, \varepsilon} + \phi_{M,
\varepsilon}$, where $X_{M, \varepsilon}$ is stationary as well. By our construction, all equations are solved on a common
probability space, say $(\Omega, \mathcal{F}, \mathbb{P})$, and we denote by
$\mathbb{E}$ the corresponding expected value. In addition, we assume that the processes $\varphi_{M,\varepsilon}$ and $X_{M,\varepsilon}$ are  jointly stationary. This could be achieved for instance by considering a solution to the coupled SDE for $(\varphi_{M,\varepsilon},X_{M,\varepsilon})$ starting from the product of the corresponding marginal invariant measures, and applying Krylov--Bogoliubov's argument.

\begin{theorem}
  \label{thm:tight}Let $\rho$ be a weight such that $\rho^{\iota} \in L^{4,
  0}$ for some $\iota \in (0, 1)$. Then for every $p \in [1, \infty)$
  \[ \sup_{\varepsilon \in \mathcal{A}, M > 0} (\mathbb{E} \| \varphi_{M,
     \varepsilon} (0) - X_{M, \varepsilon} (0) \|_{H^{1 / 2 - 2 \kappa,
     \varepsilon} (\rho^2)}^2)^{1/2}\lesssim {\lambda} +  \lambda^{7/2},\]
  \[    \sup_{\varepsilon \in \mathcal{A}, M > 0}   (\mathbb{E} \| \varphi_{M,
     \varepsilon} (0) - X_{M, \varepsilon} (0) \|_{L^{2, \varepsilon}
     (\rho^2)}^{2 p})^{1/2p}\lesssim {\lambda^{1/2}} +  {\lambda^{3/2}}.\]
\end{theorem}

\begin{proof}
  Let us  show the first claim. Due to stationarity of $\varphi_{M,
  \varepsilon} - X_{M, \varepsilon} = Y_{M, \varepsilon} + \phi_{M,
  \varepsilon}$ we obtain
  \[ \mathbb{E} \| \rho^2 (\varphi_{M, \varepsilon} (0) - X_{M, \varepsilon}
     (0)) \|_{H^{1 / 2 - 2 \kappa, \varepsilon}}^2 = \frac{1}{\tau}
     \int_0^{\tau} \mathbb{E} \| \rho^2 (\varphi_{M, \varepsilon} (s) - X_{M,
     \varepsilon} (s)) \|_{H^{1 / 2 - 2 \kappa, \varepsilon}}^2 \mathd s \]
  \[ = \frac{1}{\tau} \int_0^{\tau} \mathbb{E} \| \rho^2 (\phi_{M,
     \varepsilon} (s) + Y_{M, \varepsilon} (s)) \|_{H^{1 / 2 - 2 \kappa,
     \varepsilon}}^2 \mathd s \]
  \[ \lesssim \frac{1}{\tau} \int_0^{\tau} \mathbb{E} \| \rho^2 \phi_{M,
     \varepsilon} (s) \|_{H^{1 / 2 - 2 \kappa, \varepsilon}}^2 \mathd s +
     \frac{1}{\tau} \int_0^{\tau} \mathbb{E} \| \rho^2 Y_{M, \varepsilon} (s)
     \|_{H^{1 / 2 - 2 \kappa, \varepsilon}}^2 \mathd s. \]
  In order to estimate the  right hand side, we employ Theorem \ref{th:energy-estimate} together with Lemma~\ref{lem:Y1} to deduce
  \[ \mathbb{E} \| \rho^2 (\varphi_{M, \varepsilon} (0) - X_{M, \varepsilon}
     (0)) \|_{H^{1 / 2 - 2 \kappa, \varepsilon}}^2 \]
  \[ \lesssim C_{\tau}(\lambda^2 +  \lambda^{7})\mathbb{E}Q_{\rho}
     (\mathbb{X}_{M, \varepsilon}) + \frac{1}{2\tau} \mathbb{E} \| \rho^2 \phi_{M,
     \varepsilon} (0) \|_{L^{2, \varepsilon}}^2 +\mathbb{E}\|\rho^{\sigma}Y_{M,\varepsilon}\|_{C_{T}\CC^{1/2-\kappa,\varepsilon}}^{2}\]
  \[ \leqslant C_{\tau} (\lambda^2 +  \lambda^{7})\mathbb{E}Q_{\rho} (\mathbb{X}_{M, \varepsilon}) +\frac{C}{\tau}
     \mathbb{E} \| \rho^2 (\varphi_{M, \varepsilon} (0) - X_{M, \varepsilon}
     (0)) \|_{L^{2, \varepsilon}}^2 + \frac{C}{\tau}\mathbb{E} \| \rho^2 Y_{M,\varepsilon}
     (0) \|_{L^{2, \varepsilon}}^2 \]
  \[ \leqslant C_{\tau}(\lambda^2 +  \lambda^{7}) \mathbb{E}Q_{\rho} (\mathbb{X}_{M, \varepsilon}) +
     \frac{C}{\tau}\mathbb{E} \| \rho^2 (\varphi_{M, \varepsilon} (0) - X_{M, \varepsilon}
     (0)) \|_{L^{2, \varepsilon}}^2 .\]
  Finally, taking $\tau > 0$ large enough, we may absorb the second term from
  the right hand side into the left hand side to deduce
  \[ \mathbb{E} \| \rho^2 (\varphi_{M, \varepsilon} (0) - X_{M, \varepsilon}
     (0)) \|_{H^{1 / 2 - 2 \kappa, \varepsilon}}^2 \leqslant C_{\tau} ({\lambda^2} +  \lambda^{7})
     \mathbb{E}Q_{\rho} (\mathbb{X}_{M, \varepsilon}) . \]
  Observing that the right hand side is bounded uniformly in $M,\varepsilon$, completes the proof of the first claim.
  
  Now, we show the second claim for $p \in [2, \infty)$. The case $p \in [1,
  2)$ then follows easily from the bound for $p=2$. Using stationarity as above we have
  \[ \mathbb{E} \| \rho^2 (\varphi_{M, \varepsilon} (0) - X_{M, \varepsilon}
     (0)) \|_{L^{2, \varepsilon}}^{2 p} = \frac{1}{\tau} \int_0^{\tau}
     \mathbb{E} \| \rho^2 (\phi_{M, \varepsilon} (s) + Y_{M, \varepsilon} (s))
     \|_{L^{2, \varepsilon}}^{2 p} \mathd s \]
  \[ \  \]
  \begin{equation}
    \lesssim \frac{1}{\tau} \int_0^{\tau} \mathbb{E} \| \rho^2 \phi_{M,
    \varepsilon} (s) \|_{L^{2, \varepsilon}}^{2 p} \mathd s + \frac{1}{\tau}
    \int_0^{\tau} \mathbb{E} \| \rho^2 Y_{M, \varepsilon} (s) \|_{L^{2,
    \varepsilon}}^{2 p} \mathd s. \label{eq:d21}
  \end{equation}
  Due to Corollary~\ref{cor:Lp} applied to $p - 1$ and the fact that for any $\sigma>0$ and  $\tau\geqslant 1$
  $$\int_{0}^{\tau}| \log s|^{2p/(2+\theta)}\mathrm{d} s\leqslant C_{p,\sigma}\tau^{1+\sigma},$$
  we deduce
  \[ \begin{aligned}
  \alpha \int_0^{\tau} \mathbb{E} \| \rho^2 \phi_{M, \varepsilon} (s)
     \|_{L^{2, \varepsilon}}^{2 p} \mathd s & \leqslant C_{p,\sigma} [\tau(\lambda^{2} + \lambda^{6})^{p/2}+\tau^{1+\sigma}\lambda^{p(5-\theta)/(2+\theta)}] \mathbb{E}
     [Q_{\rho} (\mathbb{X}_{M,\varepsilon})] \\
     &\qquad + \frac{\lambda^{-1}}{2 (p - 1)}
     \mathbb{E} \| \rho^2 \phi_{M, \varepsilon} (0) \|_{L^{2, \varepsilon}}^{2
     (p - 1)} \\
  & \leqslant C_{p,\sigma}[\tau(\lambda^{2} + \lambda^{6})^{p/2}+\tau^{1+\sigma}\lambda^{p(5-\theta)/(2+\theta)}]  \mathbb{E} [Q_{\rho} (\mathbb{X}_{M,\varepsilon})] 
     \\ & \qquad + C_p  \lambda^{-1}\mathbb{E} \| \rho^2 (\varphi_{M, \varepsilon} (0) - X_{M,
     \varepsilon} (0)) \|_{L^{2, \varepsilon}}^{2 (p - 1)} 
     \\ & \qquad + C_p \lambda^{-1}\mathbb{E} \|
     \rho^2 Y_{M, \varepsilon} (0) \|_{L^{2, \varepsilon}}^{2 (p - 1)} . \end{aligned}\]
  Plugging this back into {\eqref{eq:d21}} and using Young's inequality we
  obtain
  \[ \mathbb{E} \| \rho^2 (\varphi_{M, \varepsilon} (0) - X_{M, \varepsilon}
     (0)) \|_{L^{2, \varepsilon}}^{2 p} \leqslant \frac{C_{p,\sigma}}{\alpha}[(\lambda^{2} + \lambda^{6})^{p/2}+\tau^{\sigma}\lambda^{p(5-\theta)/(2+\theta)}] \mathbb{E}
     [Q_{\rho} (\mathbb{X}_{M,\varepsilon})]  \]
  \[ + \delta \frac{C_p}{\alpha
     \tau}  \mathbb{E} \| \rho^2 (\varphi_{M, \varepsilon} (0) -
     X_{M, \varepsilon} (0)) \|_{L^{2, \varepsilon}}^{2 p} + \frac{1}{\lambda^p \tau}C_{\delta, p} {+\frac{C_{p}\lambda^{2p}}{\alpha\tau}\mathbb{E}[Q_{\rho}(\mathbb{X}_{M,\varepsilon})]}. \]
Taking $\tau = {\max}(1,\lambda^{{-2p}})$ leads to
 \[ \mathbb{E} \| \rho^2 (\varphi_{M, \varepsilon} (0) - X_{M, \varepsilon}
     (0)) \|_{L^{2, \varepsilon}}^{2 p} \leqslant \frac{C_{p,\sigma}}{\alpha}[(\lambda^{2}+ \lambda^{6})^{p/2}+\tau^{\sigma}\lambda^{p(5-\theta)/(2+\theta)}] \mathbb{E}
     [Q_{\rho} (\mathbb{X}_{M,\varepsilon})]  \]
  \[ + \delta C_{p,\alpha}  \mathbb{E} \| \rho^2 (\varphi_{M, \varepsilon} (0) -
     X_{M, \varepsilon} (0)) \|_{L^{2, \varepsilon}}^{2 p} + {\lambda^{p}}C_{\delta, p} {+C_{p,\alpha}\lambda^{2p}\mathbb{E}[Q_{\rho}(\mathbb{X}_{M,\varepsilon})]} \]
and choosing $\delta>0$ small enough, we may absorb the second term on the right hand side into the left hand side and the claim follows
\end{proof}

The above result directly implies the desired tightness of the approximate
Gibbs measures $\nu_{M, \varepsilon}$. To formulate this precisely we make use
of the extension operators $\mathcal{E}^{\varepsilon}$ for distributions on
$\Lambda_{\varepsilon}$ constructed in Section~\ref{s:ext}.
We recall that on
the approximate level the stationary process $\varphi_{M, \varepsilon}$ admits the decomposition
$ \varphi_{M, \varepsilon} = X_{M, \varepsilon} + Y_{M, \varepsilon} +
   \phi_{M, \varepsilon}, $
where $X_{M,\varepsilon}$ is stationary and $Y_{M,\varepsilon}$ is given by \eqref{eq:YY} with $\tthreeone{X_{M, \varepsilon}}$ being also stationary.
Accordingly, letting
\[ \zeta_{M, \varepsilon} \assign - \LL_{\varepsilon}^{- 1} \left[ 3\lambda \left(
   \UU^{\varepsilon}_{>} \llbracket X_{M, \varepsilon}^2 \rrbracket \right)
   \succ Y_{M, \varepsilon} \right] + \phi_{M, \varepsilon} \backassign
   \eta_{M, \varepsilon} + \phi_{M, \varepsilon} \]
we obtain $\varphi_{M, \varepsilon} = X_{M, \varepsilon} - \lambda \tthreeone{X_{M,
\varepsilon}} + \zeta_{M, \varepsilon}$, where all the summands are
stationary.


The next result shows that the family of joint laws of $( \mathcal{E}^{\varepsilon} \varphi_{M, \varepsilon},
\mathcal{E}^{\varepsilon} X_{M, \varepsilon},\mathcal{E}^{\varepsilon} \tthreeone{X}_{M, \varepsilon})$ at any chosen time $t\geqslant 0$ is tight. In addition, we obtain bounds for arbitrary moments of the limiting measure. To this end, we denote by $(\varphi,X, \tthreeone{X})$  a canonical representative of the random variables under consideration and let 
$
\zeta\assign\varphi-X+\lambda\tthreeone{X}.
$


\begin{theorem}\label{thm:main}
Let $\rho$ be a weight such that $\rho^{\iota} \in L^4$ for
  some $\iota \in (0, 1)$. Then 
  the family of joint laws of $( \mathcal{E}^{\varepsilon} \varphi_{M, \varepsilon}(t),
\mathcal{E}^{\varepsilon} X_{M, \varepsilon}(t),\mathcal{E}^{\varepsilon} \tthreeone{X}_{M, \varepsilon}(t))$, $\varepsilon\in\mathcal{A},M>0,$ evaluated at an arbitrary time $t \geqslant 0$ is tight on $H^{-1/2-3\kappa}(\rho^{2+\kappa})\times \CC^{-1/2-\kappa}(\rho^{\sigma})\times \CC^{1/2-\kappa}(\rho^{\sigma})$. Moreover, any limit probability measure $\mu$    satisfies for all $p \in [1, \infty)$
  \[ \mathbb{E}_{\mu} \| \varphi \|_{H^{- 1 / 2 - 2\kappa} (\rho^2)}^{2 p} \lesssim  {1+\lambda^{3p}}, \qquad \mathbb{E}_{\mu} \| \zeta  \|^{2p}_{L^{2}
   (\rho^2)} \lesssim {\lambda^{p}+\lambda^{3p+4}+\lambda^{4p}}, \]
\[\mathbb{E}_{\mu} \| \zeta \|_{H^{1 - 2 \kappa}
   (\rho^2)}^2  \lesssim \lambda^2 +\lambda^{7},\qquad
   \mathbb{E}_{\mu} \| \zeta  \|_{B^{0}_{4, \infty}
   (\rho)}^4  \lesssim {\lambda +\lambda^{6}}.
\]


\end{theorem}

\begin{proof}
 Since by Lemma~\ref{lem:ext}
  \[ \mathbb{E} \|
     \mathcal{E}^{\varepsilon} X_{M, \varepsilon} (0) \|_{H^{- 1 / 2 - 2
     \kappa} (\rho^2)}^{2p} \lesssim 
     \mathbb{E} \| X_{M, \varepsilon} (0) \|_{\CC^{- 1 / 2 - \kappa,
     \varepsilon} (\rho^{\sigma})}^{2p} \lesssim 1, \]
     uniformly in $M,\varepsilon$,  we deduce from Theorem~\ref{thm:tight} that
   \[ 
   \mathbb{E} \|\mathcal{E}^{\varepsilon}
     \varphi_{M, \varepsilon} (0) \|_{H^{- 1 / 2 - 2
     \kappa} (\rho^2)}^{2 p} \lesssim  {1+\lambda^{3p}}  \]
 uniformly in $M,\varepsilon$.
  Integrating \eqref{eq:lp} in time and using the decomposition of $\varphi_{M,\varepsilon}$ leads to
\[ \| \rho^2 \phi_{M,\varepsilon} (t) \|_{L^{2, \varepsilon}}^{2 p} \leqslant \|
   \rho^2 \phi_{M,\varepsilon} (0) \|_{L^{2, \varepsilon}}^{2 p} + C_t \lambda (\lambda^{2}+\lambda^{6})^{(p+1)/2} Q_{\rho}
    (\mathbb{X}_{M,\varepsilon})^{(p + 1) / 2} \]
\[ \leqslant C_p \| \rho^2 (\varphi_{M,\varepsilon} (0) - X_{M,\varepsilon} (0))
    \|_{L^{2, \varepsilon}}^{2 p} + C_p \| \rho^2 Y_{M,\varepsilon} (0) \|_{L^{2,
    \varepsilon}}^{2 p} + C_t \lambda (\lambda^{2}+\lambda^{6})^{(p+1)/2} Q_{\rho} (\mathbb{X}_{M,\varepsilon})^{(p + 1) / 2}.
\]
Hence due to Theorem~\ref{thm:tight} we obtain a uniform bound 
\[
 \mathbb{E} \| \rho^2 \phi_{M,\varepsilon} (t) \|_{L^{2, \varepsilon}}^{2 p} \lesssim_t {\lambda^{p}+\lambda^{3p+4}},\]
for all $t\geqslant 0$.
In addition, the following expressions are bounded uniformly in
$M, \varepsilon$ according to Lemma~\ref{lem:Y1} and Theorem~\ref{th:energy-estimate}
\[ \mathbb{E} \| \eta_{M, \varepsilon} \|_{C_T \CC^{1 - \kappa, \varepsilon}
   (\rho^{\sigma})}^{2p} \lesssim \lambda^{{4p}},
\]
\[ \lambda \int_0^T \mathbb{E} \| \phi_{M, \varepsilon} (t)
   \|_{L^{4, \varepsilon} (\rho)}^4 \mathd t + \int_0^T \mathbb{E} \|
   \phi_{M, \varepsilon} (t) \|_{H^{1 - 2 \kappa, \varepsilon} (\rho^2)}^2
   \mathd t \lesssim_T \lambda^2 + \lambda^{7},\]
whenever the weight $\rho$ is such
that $\rho^{\iota} \in L^4$ for some $\iota \in (0, 1)$. In view of
stationarity of $\zeta_{M, \varepsilon}$ and the embedding $\CC^{1 -
\kappa, \varepsilon} (\rho^{\sigma}) \subset H^{1 - 2 \kappa, \varepsilon}
(\rho^2)$, we therefore obtain a uniform bound
$ \mathbb{E} \| \zeta_{M, \varepsilon} (t) \|_{H^{1 - 2 \kappa, \varepsilon}
   (\rho^2)}^2 \lesssim \lambda^2 + \lambda^{7}$ as well as $ \mathbb{E} \| \zeta_{M, \varepsilon} (t) \|_{L^{2,\varepsilon}
   (\rho^2)}^{2p} \lesssim {\lambda^{p}+\lambda^{3p+4}+\lambda^{4p}}$
for every $t \geqslant 0$. Similarly, using stationarity together with the
embedding $\CC^{1 - \kappa, \varepsilon} (\rho^{\sigma}) \subset B^{0,
\varepsilon}_{4, \infty} (\rho)$ as well as $L^{4, \varepsilon} (\rho) \subset
B^{0, \varepsilon}_{4, \infty}(\rho)$ we deduce a uniform bound
$\mathbb{E} \| \zeta_{M, \varepsilon} (t) \|_{B^{0, \varepsilon}_{4, \infty}
   (\rho)}^4  \lesssim {\lambda + \lambda^{6}}$
for every $t \geqslant 0$.


Consequently, by Lemma~\ref{lem:ext} the same
bounds hold for the corresponding extended distributions and hence the family joint laws of
$( \mathcal{E}^{\varepsilon} \varphi_{M, \varepsilon}(t),
\mathcal{E}^{\varepsilon} X_{M, \varepsilon}(t),\mathcal{E}^{\varepsilon} \tthreeone{X}_{M, \varepsilon}(t))$ at any time $t\geqslant 0$ is tight on $H^{-1/2-3\kappa}(\rho^{2+\kappa})\times \CC^{-1/2-\kappa}(\rho^{\sigma})\times \CC^{1/2-\kappa}(\rho^{\sigma})$. Indeed, this is a consequence of the compact embedding
$$
H^{-1/2-2\kappa}(\rho^{2})\times \CC^{-1/2-\kappa/2}(\rho^{2\sigma})\times \CC^{1/2-\kappa/2}(\rho^{2\sigma})\subset H^{-1/2-3\kappa}(\rho^{2+\kappa})\times \CC^{-1/2-\kappa}(\rho^{\sigma})\times \CC^{1/2-\kappa}(\rho^{\sigma}).
$$
Therefore up to a subsequence we may pass to the
limit as $\varepsilon \rightarrow 0$, $M \rightarrow \infty$ and the uniform moment bounds are preserved for every limit point.
\end{proof}


The marginal of $\mu$ corresponding to $\varphi$ is the desired  $\Phi^{4}_3$ measure, which we denote by $\nu$. According to the above result, $\nu$ is obtained as a limit (up to a subsequence) of the continuum extensions of the Gibbs measures $\nu_{M,\varepsilon}$ given by \eqref{eq:gibbs} as $\varepsilon \rightarrow 0$, $M \rightarrow \infty$.




\subsection{Stretched exponential integrability}
\label{s:exp}



The goal of this section is to establish better probabilistic properties of the $\Phi^{4}_{3}$ measure. Namely, we show that  $\| \rho^2 \varphi_{M,
\varepsilon} \|_{H^{- 1 / 2 - 2\kappa, \varepsilon}}^{1 - \upsilon}$ is uniformly (in $M,\varepsilon$)
exponentially integrable for every  $\upsilon=O(\kappa) > 0$, hence we recover the same stretched exponential moment bound for any limit measure $\nu$.
To this end, we revisit the energy estimate in Section \ref{s:estim} and take a particular care  to optimize the power of the quantity $\|\mathbb{X}_{M,\varepsilon}\|$ appearing in the estimates. 
Recall that it can be shown that
\begin{equation}
  \mathbb{E} [e^{\beta \| \mathbb{X}_{M, \varepsilon} \|^2}] < \infty
  \label{eq:exp-int}
\end{equation}
uniformly in $M, \varepsilon$ for a small parameter $\beta > 0$ (see
{\cite{moinat_space_time_2018}}). Accordingly, it turns out that the polynomial $Q_{\rho}(\mathbb X_{M,\varepsilon})$ on the right hand side of the bound in Lemma \ref{lemma:bounds-rhs1} shall not contain higher powers of $\|\mathbb X_{M,\varepsilon}\|$ than $8+O(\kappa)$. In the proof of Lemma \ref{lemma:bounds-rhs1} we already see what the problematic terms are. In order to allow for a refined treatment of these terms, we introduce an additional large momentum cut-off and modify the definition of $Y_{M,\varepsilon}$ from \eqref{eq:Y1}, leading to better uniform estimates and consequently to the desired stretched exponential integrability.



More precisely, let $K
> 0$ and take a compactly supported, smooth function $v : \mathbb{R} \rightarrow
\mathbb{R}_+$ such that $\| v \|_{L^1} = 1$. We define
\[ \llbracket X_{M, \varepsilon}^3 \rrbracket_{\leqslant} \assign v_K \ast_t
   \Delta^{\varepsilon}_{\leqslant K} \llbracket X_{M, \varepsilon}^3
   \rrbracket, \]
where the convolution is in the time variable and $v_K (t) \assign 2^K v (2^K
t)$. With standard arguments one can prove that
\[ \sup_{K \in \mathbb{N}} (2^{- K (3 / 2 + \kappa)} \| \llbracket X_{M,
   \varepsilon}^3 \rrbracket_{\leqslant} \|_{C_T L^{\infty, \varepsilon}})^{2
   / 3} \]
is exponentially integrable for a small parameter and therefore we can modify the definition of $\| \mathbb{X}_{M, \varepsilon} \|$ to obtain
\begin{equation}
  \| \llbracket X_{M, \varepsilon}^3 \rrbracket_{\leqslant} \|_{C_T L^{\infty,
  \varepsilon}} \lesssim 2^{K (3 / 2 + \kappa)} \| \mathbb{X}_{M, \varepsilon}
  \|^3  \label{eq:bound-X3}
\end{equation}
while still keeping the validity of~{\eqref{eq:exp-int}}. Moreover, we let $\llbracket X_{M,
\varepsilon}^3 \rrbracket_{>} \assign \llbracket X_{M, \varepsilon}^3
\rrbracket - \llbracket X_{M, \varepsilon}^3 \rrbracket_{\leqslant}$ and
define $\tthreeone{X_{M, \varepsilon, >}}$ to be the stationary solution of
\[ \LL_{\varepsilon} \tthreeone{X_{M, \varepsilon, >}} = \llbracket X_{M,
   \varepsilon}^3 \rrbracket - \llbracket X_{M, \varepsilon}^3
   \rrbracket_{\leqslant}. \]
By choosing $K$ we can have that
\[ \| \tthreeone{X_{M, \varepsilon, >}} \|_{C_T L^{\infty, \varepsilon}
   (\rho^{\sigma})} \lesssim 2^{- K (1 / 2 - 2 \kappa)} \| \tthreeone{X_{M,
   \varepsilon, >}} \|_{C_T \mathscr{C} \hspace{.1em}^{1 / 2 - \kappa,
   \varepsilon} (\rho^{\sigma})} \lesssim 2^{- K (1 / 2 - 2 \kappa)} \|
   \mathbb{X}_{M, \varepsilon} \|^3 \lesssim \| \mathbb{X}_{M, \varepsilon}
   \|^2 \]
which holds true provided
\[ 2^{K / 2} = \| \mathbb{X}_{M, \varepsilon} \|^{1 / (1 - 4 \kappa)} .
\]
Next, we redefine $Y_{M, \varepsilon}$ to solve
\[  Y_{M, \varepsilon} = - \lambda \tthreeone{X_{M,
   \varepsilon, >}} - \LL_{\varepsilon}^{- 1} [3 \lambda
   (\UU^{\varepsilon}_{>} \llbracket X_{M, \varepsilon}^2 \rrbracket) \succ
   Y_{M, \varepsilon}] . \]
The estimates of Lemma~\ref{lem:Y1} are still valid with obvious
modifications. In addition, we obtain
\[ \| \rho^{\sigma} Y_{M, \varepsilon} \|_{C_T L^{\infty, \varepsilon}
   (\rho^{\sigma})} \lesssim \lambda \| \mathbb{X}_{M, \varepsilon} \|^2,
   \qquad \| \rho^{\sigma} Y_{M, \varepsilon} \|_{C_T \mathscr{C}
   \hspace{.1em}^{1 / 2 - \kappa, \varepsilon} (\rho^{\sigma})} \lesssim
   \lambda \| \mathbb{X}_{M, \varepsilon} \|^3, \]
and by interpolation it follows for $a \in [0, 1 / 2 - \kappa]$ that
\begin{equation}
  \| \rho^{\sigma} Y_{M, \varepsilon} \|_{C_T \mathscr{C} \hspace{.1em}^{a,
  \varepsilon} (\rho^{\sigma})} \lesssim \lambda \| \mathbb{X}_{M,
  \varepsilon} \|^{2 + a / (1 / 2 - \kappa)} . \label{eq:interp-Y}
\end{equation}
From now on we avoid, as usual, to specify explicitly the dependence on $M$
since it does not play any role in the estimates. The energy
equality~{\eqref{eq:en12}} in Lemma~\ref{lem:energy12} now reads
\begin{equation}
  \frac{1}{2} \partial_t \| \rho^2 \phi_{\varepsilon} \|_{L^{2,
  \varepsilon}}^2 + \Upsilon_{\varepsilon} = \Theta_{\rho^4, \varepsilon} +
  \Psi_{\rho^4, \varepsilon} + \langle \rho^4 \phi_{\varepsilon}, - \lambda
  \llbracket X_{\varepsilon}^3 \rrbracket_{\leqslant} \rangle_{\varepsilon},
  \label{eq:en12-int}
\end{equation}
where
\[
\Upsilon_{\varepsilon}:=\lambda\|\rho\phi_{\varepsilon}\|_{L^{4,\varepsilon}}^{4}+m^{2}\|\rho^{2}\psi_{\varepsilon}\|_{L^{2,\varepsilon}}^{2}+\|\rho^{2}\nabla_{\varepsilon}\psi_{\varepsilon}\|_{L^{2,\varepsilon}}^{2}
\]
and $\Theta_{\rho^4, \varepsilon}, \Psi_{\rho^4, \varepsilon}$ where defined in Lemma~\ref{lem:energy12}.
Our goal is to bound the right hand side of \eqref{eq:en12-int}
with no more than a factor $\| \mathbb{X}_{M, \varepsilon} \|^{8 + \vartheta}$
for some $\vartheta = O (\kappa)$. In view of the estimates within the proof of Lemma \ref{lemma:bounds-rhs1} we
observe that the bounds {\eqref{eq:XY}}, {\eqref{eq:Y3}}, {\eqref{eq:Y2}},
{\eqref{eq:Y11}}, {\eqref{eq:U0}} and {\eqref{eq:Z0}} need to be improved.

\begin{lemma}
  \label{lemma:bounds-rhs1-int}Let $\rho$ be a weight such that $\rho^{\iota}
  \in L^{4, 0}$ for some $\iota \in (0, 1)$. Then there is $\vartheta = O
  (\kappa)>0$ such that
  \[ | \Theta_{\rho^4, \varepsilon} | + | \Psi_{\rho^4, \varepsilon} |+|\langle \rho^4 \phi_{\varepsilon}, - \lambda
\llbracket X_{\varepsilon}^3 \rrbracket_{\leqslant} \rangle_{\varepsilon}|
     \leqslant C_{\delta}  (\lambda + \lambda^{7 / 3} | \log t|^{4 / 3} +
     \lambda^5) \| \mathbb{X}_{\varepsilon} \|^{8 + \vartheta} + \delta
     \Upsilon_{\varepsilon} . \]
\end{lemma}

\begin{proof}
  Let us begin with a new bound for the term with $X_{\varepsilon}
  Y_{\varepsilon}^2$ appearing in {\eqref{eq:XY}}. For the resonant term we
  get from the interpolation estimate~{\eqref{eq:interp-Y}} that the
  bound~{\eqref{eq:XY2-res}} can be updated as
  \[ \| \rho^{\sigma} X_{\varepsilon} \circ Y_{\varepsilon}^2 \|_{C_T
     \mathscr{C} \hspace{.1em}^{- \kappa, \varepsilon}} \lesssim \lambda^2 \|
     \mathbb{X}_{\varepsilon} \|^{6 + \vartheta} + \lambda^3 \|
     \mathbb{X}_{\varepsilon} \|^{5 + \vartheta} \lesssim (\lambda^2 +
     \lambda^3) \| \mathbb{X}_{\varepsilon} \|^{6 + \vartheta} \]
  where we used that, due to the presence of the localizer (see~{\eqref{eq:U11}}), we can bound
  \begin{equation}
    \left\| \rho^{\sigma} \UU_{>} \llbracket X_{\varepsilon}^2 \rrbracket
    \right\|_{\mathscr{C} \hspace{.1em}^{- 3 / 2 + 2 \kappa, \varepsilon}}
    \lesssim \| \rho^{\sigma} \llbracket X_{\varepsilon}^2 \rrbracket
    \|_{\mathscr{C} \hspace{.1em}^{- 1 - \kappa, \varepsilon}} \left( 1 +
    \lambda \| \rho^{\sigma} \llbracket X_{\varepsilon}^2 \rrbracket
    \|_{\mathscr{C} \hspace{.1em}^{- 1 - \kappa, \varepsilon}} \right)^{- (1 -
    6 \kappa)} \lesssim \| \mathbb{X}_{\varepsilon} \|^{\vartheta}
    \label{eq:impro-X2}
  \end{equation}
  giving an improved bound for the paracontrolled term which reads as follows
  \[ \left\| \rho^{4 \sigma} X_{\varepsilon} \circ \left( 2 Y_{\varepsilon}
     \prec \LL_{\varepsilon}^{- 1} \left[ 3 \lambda \left( \UU_{>} \llbracket
     X_{\varepsilon}^2 \rrbracket \right) \succ Y_{\varepsilon} \right]
     \right) \right\|_{\mathscr{C} \hspace{.1em}^{- \kappa, \varepsilon}} \]
  \[ \lesssim \lambda \| \rho^{\sigma} X_{\varepsilon} \|_{\mathscr{C}
     \hspace{.1em}^{- 1 / 2 - \kappa, \varepsilon}} \| \rho^{\sigma}
     Y_{\varepsilon} \|_{L^{\infty, \varepsilon}}^2 \left\| \rho^{\sigma}
     \UU_{>} \llbracket X_{\varepsilon}^2 \rrbracket \right\|_{\mathscr{C}
     \hspace{.1em}^{- 3 / 2 + 2 \kappa, \varepsilon}} \lesssim \lambda^3 \|
     \mathbb{X}_{\varepsilon} \|^{5 + \vartheta} . \]
  Consequently, for $\theta = \frac{1 - 4 \kappa}{1 - 2 \kappa}$
  \[ \lambda | \langle \rho^4 \phi_{\varepsilon}, X_{\varepsilon} \circ
     Y_{\varepsilon}^2 \rangle_{\varepsilon} | \lesssim \lambda \|
     \rho^{\sigma} X_{\varepsilon} \circ Y_{\varepsilon}^2 \|_{\mathscr{C}
     \hspace{.1em}^{- \kappa, \varepsilon}} \| \rho^{4 - \sigma}
     \phi_{\varepsilon} \|_{B^{\kappa, \varepsilon}_{1, 1}} \lesssim
     (\lambda^3 + \lambda^4) \| \mathbb{X}_{\varepsilon} \|^{6 + \vartheta} \|
     \rho \phi_{\varepsilon} \|^{\theta}_{L^{4, \varepsilon}} \| \rho^2
     \phi_{\varepsilon} \|_{H^{1 - 2 \kappa, \varepsilon}}^{1 - \theta} \]
  \[ \leqslant (\lambda^{(12 - \theta) / (2 + \theta)} + \lambda^{(16 -
     \theta) / (2 + \theta)}) C_{\delta} \| \mathbb{X}_{\varepsilon} \|^{8 +
     \vartheta} + \delta \Upsilon_{\varepsilon} . \]
  For the paraproducts we have for $\theta = \frac{1 / 2 - 4 \kappa}{1 - 2
  \kappa}$
  \[ \lambda | \langle \rho^4 \phi_{\varepsilon}, X_{\varepsilon} \Join
     Y_{\varepsilon}^2 \rangle_{\varepsilon} | \lesssim \lambda \| \rho^{4 - 2
     \sigma} \phi_{\varepsilon} \|_{B^{1 / 2 + \kappa,\varepsilon}_{1, 1}} \|
     \rho^{\sigma} X_{\varepsilon} \|_{\mathscr{C} \hspace{.1em}^{- 1 / 2 -
     \kappa, \varepsilon}} \| \rho^{\sigma} Y_{\varepsilon} \|_{L^{\infty,
     \varepsilon}}^2 \]
  \[ \lesssim \lambda^3 \| \mathbb{X}_{\varepsilon} \|^5 \| \rho
     \phi_{\varepsilon} \|^{\theta}_{L^{4, \varepsilon}} \| \rho^2
     \phi_{\varepsilon} \|^{1 - \theta}_{H^{1 - 2 \kappa,\varepsilon}} \leqslant
     \lambda^{(12 - \theta) / (2 + \theta)} C_{\delta} \|
     \mathbb{X}_{\varepsilon} \|^8 + \delta \Upsilon_{\varepsilon} . \]
  Let us now consider the term with $X_{\varepsilon} Y_{\varepsilon}$ always
  in {\eqref{eq:XY}}. In view of {\eqref{eq:XY-res}}, {\eqref{eq:XY-par1}},
  {\eqref{eq:XY-par2}} we shall modify the bound of the resonant
  product which using the decomposition {\eqref{eq:XY-res3}} together with
  {\eqref{eq:XY-res}} and the bound~{\eqref{eq:impro-X2}}. We obtain
  \[ \| \rho^{\sigma} X_{\varepsilon} \circ Y_{\varepsilon} \|_{\mathscr{C}
     \hspace{.1em}^{- \kappa, \varepsilon}} \lesssim \lambda \|
     \mathbb{X}_{\varepsilon} \|^4 + \lambda^2 \| \mathbb{X}_{\varepsilon}
     \|^{3 + \vartheta} \lesssim (\lambda + \lambda^2) \|
     \mathbb{X}_{\varepsilon} \|^4 , \]
  and consequently, for $\theta = \frac{1 - 4 \kappa}{1 - 2 \kappa}$,
  \[ \lambda | \langle \rho^4 \phi_{\varepsilon}^2, X_{\varepsilon} \circ
     Y_{\varepsilon} \rangle_{\varepsilon} | \lesssim \lambda \| \rho^{\sigma}
     X_{\varepsilon} \circ Y_{\varepsilon} \|_{\mathscr{C} \hspace{.1em}^{-
     \kappa, \varepsilon}} \| \rho^{4 - \sigma} \phi_{\varepsilon}^2
     \|_{B^{\kappa, \varepsilon}_{1, 1}} \lesssim (\lambda^2 + \lambda^3) \|
     \mathbb{X}_{\varepsilon} \|^4 \| \rho \phi_{\varepsilon} \|^{1 +
     \theta}_{L^{4, \varepsilon}} \| \rho^2 \phi_{\varepsilon} \|_{H^{1 - 2
     \kappa, \varepsilon}}^{1 - \theta} \]
  \[ \leqslant (\lambda^{(7 - \theta) / (1 + \theta)} + \lambda^{(11 - \theta)
     / (1 + \theta)}) C_{\delta} \| \mathbb{X}_{\varepsilon} \|^8 + \delta
     \Upsilon_{\varepsilon} . \]
  For the paraproducts we have for $\theta = \frac{1 / 2 - 4 \kappa}{1 - 2
  \kappa}$
  \[ \lambda | \langle \rho^4 \phi_{\varepsilon}^2, X_{\varepsilon} \Join
     Y_{\varepsilon} \rangle_{\varepsilon} | \lesssim \lambda \| \rho^{4 - 2
     \sigma} \phi_{\varepsilon}^2 \|_{B^{1 / 2 + \kappa, \varepsilon}_{1, 1}}
     \| \rho^{\sigma} X_{\varepsilon} \|_{\mathscr{C} \hspace{.1em}^{- 1 / 2 -
     \kappa, \varepsilon}} \| \rho^{\sigma} Y_{\varepsilon} \|_{L^{\infty,
     \varepsilon}} \]
  \[ \lesssim \lambda^2 \| \mathbb{X}_{\varepsilon} \|^3 \| \rho
     \phi_{\varepsilon} \|^{1 + \theta}_{L^{4, \varepsilon}} \| \rho^2
     \phi_{\varepsilon} \|^{1 - \theta}_{H^{1 - 2 \kappa, \varepsilon}}
     \leqslant \lambda^{(7 - \theta) / (1 + \theta)} C_{\delta} \|
     \mathbb{X}_{\varepsilon} \|^8 + \delta \Upsilon_{\varepsilon} . \]
  With the improved bound for $Y$, {\eqref{eq:Y3}}, {\eqref{eq:Y2}}, {\eqref{eq:Y11}}
  can be updated as follows
  \[ | \langle \rho^4 \phi_{\varepsilon}, \lambda Y_{\varepsilon}^3
     \rangle_{\varepsilon} | \lesssim \lambda \| \rho \phi_{\varepsilon}
     \|_{L^{4, \varepsilon}} \| \rho^{\sigma} Y_{\varepsilon} \|_{C_T
     L^{\infty, \varepsilon}}^3 \lesssim \lambda^4 \| \rho \phi_{\varepsilon}
     \|_{L^{4, \varepsilon}} \| \mathbb{X}_{\varepsilon} \|^6 \leqslant \delta
     \lambda \| \rho \phi_{\varepsilon} \|_{L^{4, \varepsilon}}^4 + C_{\delta}
     \lambda^5 \| \mathbb{X}_{\varepsilon} \|^8, \]
  \[ | \langle \rho^4 \phi_{\varepsilon}, 3 \lambda Y_{\varepsilon}^2
     \phi_{\varepsilon} \rangle_{\varepsilon} | \lesssim \lambda \| \rho
     \phi_{\varepsilon} \|_{L^{4, \varepsilon}}^2 \| \rho^{\sigma}
     Y_{\varepsilon} \|_{C_T L^{\infty, \varepsilon}}^2 \lesssim \lambda^3 \|
     \rho \phi_{\varepsilon} \|_{L^{4, \varepsilon}}^2 \|
     \mathbb{X}_{\varepsilon} \|^4 \leqslant \delta \lambda \| \rho
     \phi_{\varepsilon} \|_{L^{4, \varepsilon}}^4 + C_{\delta} \lambda^5 \|
     \mathbb{X}_{\varepsilon} \|^8, \]
  \[ | \langle \rho^4 \phi_{\varepsilon}, 3 \lambda Y_{\varepsilon}
     \phi_{\varepsilon}^2 \rangle_{\varepsilon} | \lesssim \lambda \| \rho
     \phi_{\varepsilon} \|_{L^{4, \varepsilon}}^3 \| \rho^{\sigma}
     Y_{\varepsilon} \|_{C_T L^{\infty, \varepsilon}} \lesssim \lambda^2 \|
     \rho \phi_{\varepsilon} \|_{L^{4, \varepsilon}}^3 \|
     \mathbb{X}_{\varepsilon} \|^2 \leqslant \delta \lambda \| \rho
     \phi_{\varepsilon} \|_{L^{4, \varepsilon}}^4 + C_{\delta} \lambda^5 \|
     \mathbb{X}_{\varepsilon} \|^8 . \]
  Now, let us update the bound~{\eqref{eq:U0}} as
    \[ \lambda \left| \langle \rho^4 \phi_{\varepsilon}, - 3
     (\UU^{\varepsilon}_{\leqslant} \llbracket X^2 \rrbracket) \succ
     Y_{\varepsilon} \rangle_{\varepsilon} \right| \leqslant (\lambda^4 +
     \lambda^5) C_{\delta} \|\mathbb{X}_{\varepsilon} \|^{8 + \vartheta} +
     \delta \| \rho^2 \phi_{\varepsilon} \|_{H^{1 - 2 \kappa, \varepsilon}}^2
     . \]
  Next, we shall improve the bound~{\eqref{eq:Z0}}. Here we need to use a different modification
  for each term appearing in $\langle \rho^4 \phi_{\varepsilon}, \lambda^2
  Z_{\varepsilon} \rangle_{\varepsilon}$ as defined in~{\eqref{eq:def-Z}}. For
  $\theta = \frac{1 / 2 - 4 \kappa}{1 - 2 \kappa}$ we bound
  \[ | \langle \rho^4 \phi_{\varepsilon}, \lambda^2
     \tthreethreer{X_{\varepsilon}} \rangle_{\varepsilon} | \lesssim \lambda^2
     \| \rho^{4 - \sigma} \phi_{\varepsilon} \|_{B^{1 / 2 + \kappa,
     \varepsilon}_{1, 1}} \| \rho^{\sigma} \tthreethreer{X_{\varepsilon}}
     \|_{C_T \mathscr{C} \hspace{.1em}^{- 1 / 2 - \kappa, \varepsilon}} \]
  \[ \lesssim \lambda^2 \| \rho \phi_{\varepsilon} \|_{L^{4,
     \varepsilon}}^{\theta} \| \rho^2 \phi_{\varepsilon} \|_{H^{1 - 2 \kappa,
     \varepsilon}}^{1 - \theta} \| \mathbb{X}_{\varepsilon} \|^5 \leqslant
     \lambda^{(8 - \theta) / (2 + \theta)} C_{\delta} \|
     \mathbb{X}_{\varepsilon} \|^8 + \delta \Upsilon_{\varepsilon} \]
  \[ \leqslant (\lambda^3 + \lambda^4) C_{\delta} \| \mathbb{X}_{\varepsilon}
     \|^8 + \delta \Upsilon_{\varepsilon}. \]
  Next, we have
  \[ \lambda^2 | \langle \rho^4 \phi_{\varepsilon},
     \ttwothreer{\tilde{X}_{\varepsilon}} Y \rangle_{\varepsilon} | \leqslant
     \lambda^2 | \langle \rho^4 \phi_{\varepsilon},
     \ttwothreer{\tilde{X}_{\varepsilon}} \Join Y \rangle_{\varepsilon} | +
     \lambda^2 | \langle \rho^4 \phi_{\varepsilon},
     \ttwothreer{\tilde{X}_{\varepsilon}} \circ Y \rangle_{\varepsilon} | \]
  where, for $\theta = \frac{1 - 4 \kappa}{1 - 2 \kappa}$, we bound
  \[ \lambda^2 | \langle \rho^4 \phi_{\varepsilon},
     \ttwothreer{\tilde{X}_{\varepsilon}} \Join Y_{\varepsilon}
     \rangle_{\varepsilon} | \lesssim \lambda^2 \| \rho^{4 - 2 \sigma}
     \phi_{\varepsilon} \|_{B^{\kappa, \varepsilon}_{1, 1}} \| \rho^{2 \sigma}
     \ttwothreer{\tilde{X}_{\varepsilon}} \Join Y_{\varepsilon}
     \|_{\mathscr{C} \hspace{.1em}^{- \kappa, \varepsilon}} \]
  \[ \lesssim \lambda^2 \| \rho \phi_{\varepsilon} \|^{\theta}_{L^{4,
     \varepsilon}} \| \rho^2 \phi_{\varepsilon} \|^{1 - \theta}_{H^{1 - 2
     \kappa, \varepsilon}} \| \rho^{\sigma}
     \ttwothreer{\tilde{X}_{\varepsilon}} \|_{\mathscr{C} \hspace{.1em}^{-
     \kappa, \varepsilon}} \| \rho^{\sigma} Y_{\varepsilon} \|_{L^{\infty,
     \varepsilon}} \leqslant \lambda^{(8 - \theta) / (2 + \theta)}
     C_{\delta}^{} \| \mathbb{X}_{\varepsilon} \|^{8 + \vartheta} + \delta
     \Upsilon_{\varepsilon} \]
  \[ \leqslant (\lambda^2 + \lambda^3) C_{\delta}^{} \|
     \mathbb{X}_{\varepsilon} \|^{8 + \vartheta} + \delta
     \Upsilon_{\varepsilon} . \]
  and the resonant term is bounded as
  \[ \lambda^2 | \langle \rho^4 \phi_{\varepsilon},
     \ttwothreer{\tilde{X}_{\varepsilon}} \circ Y_{\varepsilon}
     \rangle_{\varepsilon} | \lesssim \lambda^2 \| \rho^{4 - 2 \sigma}
     \phi_{\varepsilon} \|_{L^{1, \varepsilon}} \| \rho^{\sigma}
     \ttwothreer{\tilde{X}_{\varepsilon}} \|_{\mathscr{C} \hspace{.1em}^{-
     \kappa, \varepsilon}} \| \rho^{\sigma} Y_{\varepsilon} \|_{\mathscr{C}
     \hspace{.1em}^{2 \kappa, \varepsilon}} \lesssim \lambda^3 \| \rho
     \phi_{\varepsilon} \|_{L^{4, \varepsilon}} \| \mathbb{X}_{\varepsilon}
     \|^{6 + \vartheta} \]
  \[ \leqslant C_{\delta} \lambda^{11 / 3} \| \mathbb{X}_{\varepsilon} \|^{8 +
     \vartheta} + \delta \Upsilon_{\varepsilon} \leqslant (\lambda^3 +
     \lambda^4) C_{\delta}^{} \| \mathbb{X}_{\varepsilon} \|^{8 + \vartheta} +
     \delta \Upsilon_{\varepsilon} . \]
  Now,
  \[ \lambda^2 | \langle \rho^4 \phi_{\varepsilon}, (\tilde{b}_{\varepsilon} -
     b_{\varepsilon}) Y_{\varepsilon} \rangle_{\varepsilon} | \lesssim | \log
     t | \lambda^2 \| \rho^{4 - \sigma} \phi_{\varepsilon} \|_{L^{1,
     \varepsilon}} \| \rho^{\sigma} Y_{\varepsilon} \|_{L^{\infty,
     \varepsilon}} \lesssim | \log t |^{4 / 3} \lambda^{7 / 3} C_{\delta} \|
     \mathbb{X}_{\varepsilon} \|^{8 / 3} + \delta \Upsilon_{\varepsilon} . \]
  Next, for $\theta = \frac{1 - 5 \kappa}{1 - 2 \kappa}$,
  \[ \lambda^2 | \langle \rho^4 \phi_{\varepsilon}, \bar{C}_{\varepsilon}
     (Y_{\varepsilon}, 3 \llbracket X_{\varepsilon}^2 \rrbracket, 3 \llbracket
     X_{\varepsilon}^2 \rrbracket) \rangle_{\varepsilon} | \lesssim \lambda^2
     \| \rho^{4 - 3 \sigma} \phi_{\varepsilon} \|_{B^{2 \kappa,
     \varepsilon}_{1, 1}} \| \rho^{\sigma} Y_{\varepsilon} \|_{\mathscr{C}
     \hspace{.1em}^{2 \kappa, \varepsilon}} \| \rho^{\sigma} \llbracket
     X_{\varepsilon}^2 \rrbracket \|^2_{\mathscr{C} \hspace{.1em}^{- 1 -
     \kappa, \varepsilon}} \]
  \[ \lesssim \lambda^3 \| \rho \phi_{\varepsilon} \|_{L^{4,
     \varepsilon}}^{\theta} \| \rho^2 \phi_{\varepsilon} \|_{H^{1 - 2 \kappa,
     \varepsilon}}^{1 - \theta} \| \mathbb{X}_{\varepsilon} \|^{6 + \vartheta}
     \leqslant \lambda^{(12 - \theta) / (2 + \theta)} C_{\delta} \|
     \mathbb{X}_{\varepsilon} \|^{8 + \vartheta} + \delta
     \Upsilon_{\varepsilon} \]
  \[ \leqslant (\lambda^3 + \lambda^4) C_{\delta}^{} \|
     \mathbb{X}_{\varepsilon} \|^{8 + \vartheta} + \delta
     \Upsilon_{\varepsilon} . \]
  At last, we have
  \[ \lambda^2 \left| \left\langle \rho^4 \phi_{\varepsilon}, - 3 \llbracket
     X_{\varepsilon}^2 \rrbracket \circ \LL_{\varepsilon}^{- 1} \left( 3
     \UU^{\varepsilon}_{\leqslant} \llbracket X_{\varepsilon}^2 \rrbracket
     \succ Y_{\varepsilon} \right) \right\rangle_{\varepsilon} \right| \]
  \[ \lesssim \lambda^2 \| \rho^{4 - 3 \sigma} \phi_{\varepsilon} \|_{L^{1,
     \varepsilon}} \| \rho^{\sigma} Y_{\varepsilon} \|_{L^{\infty,
     \varepsilon}} \| \rho^{\sigma} \llbracket X_{\varepsilon}^2 \rrbracket
     \|_{\mathscr{C} \hspace{.1em}^{- 1 - \kappa, \varepsilon}} \left\|
     \rho^{\sigma} \UU^{\varepsilon}_{\leqslant} \llbracket X_{\varepsilon}^2
     \rrbracket \right\|_{\mathscr{C} \hspace{.1em}^{- 1 + 2 \kappa,
     \varepsilon}} \]
  \[ \lesssim \lambda^3 \| \rho^{4 - 3 \sigma} \phi_{\varepsilon} \|_{L^{1,
     \varepsilon}} \| \mathbb{X}_{\varepsilon} \|^{4 + \vartheta} \leqslant
     \lambda^{11 / 3} C_{\delta} \| \mathbb{X}_{\varepsilon} \|^{16 / 3 +
     \vartheta} + \delta \Upsilon_{\varepsilon} \leqslant (\lambda^3 +
     \lambda^4) C_{\delta}^{} \| \mathbb{X}_{\varepsilon} \|^{8 + \vartheta} +
     \delta \Upsilon_{\varepsilon} \]
  This concludes the estimation of $\langle \rho^4 \phi_{\varepsilon},
  \lambda^2 Z_{\varepsilon} \rangle_{\varepsilon}$ giving us
  \[ | \langle \rho^4 \phi_{\varepsilon}, \lambda^2 Z_{\varepsilon}
     \rangle_{\varepsilon} | \leqslant (\lambda^2 + \lambda^4) C_{\delta}^{}
     \| \mathbb{X}_{\varepsilon} \|^{8 + \vartheta} + \delta
     \Upsilon_{\varepsilon} . \]
  Finally, we arrive to the additional term introduced by the localization. Using \eqref{eq:bound-X3} we obtain
  \[ | \langle \rho^4 \phi_{\varepsilon}, - \lambda \llbracket X_{M,
     \varepsilon}^3 \rrbracket_{\leqslant} \rangle_{\varepsilon} | \lesssim
     \lambda \| \rho \phi_{\varepsilon} \|_{L^{4, \varepsilon}} \|
     \rho^{\sigma} \llbracket X_{M, \varepsilon}^3 \rrbracket_{\leqslant}
     \|_{C_T L^{\infty, \varepsilon}} \lesssim \lambda \| \rho
     \phi_{\varepsilon} \|_{L^{4, \varepsilon}} 2^{K (3 / 2 + \kappa)} \|
     \mathbb{X}_{\varepsilon} \|^3 \]
  \[ \leqslant \lambda C_{\delta}^{} \| \mathbb{X}_{\varepsilon} \|^{8 +
     \vartheta} + \delta \Upsilon_{\varepsilon}, \]
  where we also see that the power $8+\vartheta$ is optimal for this decomposition.
\end{proof}

Let $\langle \phi_{\varepsilon} \rangle \assign (1 + \| \rho^2
\phi_{\varepsilon} \|_{L^{2, \varepsilon}}^2)^{1 / 2}$ and $\langle
\varphi_{\varepsilon} \rangle_{\ast} \assign (1 + \| \rho^2
\varphi_{\varepsilon} \|_{H^{- 1 / 2 - 2 \kappa, \varepsilon}}^2)^{1 / 2}$.
With Lemma~\ref{lemma:bounds-rhs1-int} in hand we can proceed to the proof of the stretched exponential
integrability.



\begin{proposition}
  \label{lemma:int-bound}
  There exists an $\alpha
  > 0$, $0 < C < 1$ and  $\upsilon=O(\kappa)>0$  such that for every $\beta>0$ 
  \[ \partial_t e^{\beta \langle t \phi_{\varepsilon} \rangle^{1 - \upsilon}}
     + \alpha e^{\beta \langle t \phi_{\varepsilon} \rangle^{1 - \upsilon}} (1-\upsilon)\beta
     \langle t \phi_{\varepsilon} \rangle^{- \upsilon - 1} t^2
     \Upsilon_{\varepsilon} \lesssim 1 + e^{(\beta / C) \|
     \mathbb{X}_{\varepsilon} \|^2}. \]
 Consequently, for any accumulation point $\nu$ we have
  \[ \int_{\mathcal{S}'(\mathbb{R}^{3})} e^{\beta \langle \varphi_{} \rangle_{\ast}^{1 - \upsilon}} \nu
     (\mathd \varphi) < \infty \]
provided  $\beta>0$ is sufficiently small.
\end{proposition}

\begin{proof}
We apply \eqref{eq:en12-int} and Lemma~\ref{lemma:bounds-rhs1-int} to obtain
  \[ \langle t \phi_{\varepsilon} \rangle^{1+ \upsilon} \frac{\partial_t e^{\beta \langle t \phi_{\varepsilon} \rangle^{1 - \upsilon}}}{(1 -
     \upsilon) \beta}
     = e^{\beta \langle t \phi_{\varepsilon} \rangle^{1 - \upsilon}} 
     \frac{1}{2} \partial_t (t^2 \| \rho^2 \phi_{\varepsilon} \|_{L^{2,
     \varepsilon}}^2) \]
  \[ = e^{\beta \langle t \phi_{\varepsilon} \rangle^{1 - \upsilon}}
     [t^2 (- \Upsilon_{\varepsilon} + \Theta_{\rho^4, \varepsilon} +
     \Psi_{\rho^4, \varepsilon}+\langle \rho^4 \phi_{\varepsilon}, - \lambda \llbracket X_{
     \varepsilon}^3 \rrbracket_{\leqslant} \rangle_{\varepsilon}) + t \| \rho^2 \phi_{\varepsilon} \|_{L^{2,
     \varepsilon}}^2] \]
  \[ \leqslant e^{\beta \langle t \phi_{\varepsilon} \rangle^{1 - \upsilon}}
     [t^2 (- \Upsilon_{\varepsilon} + \Theta_{\rho^4, \varepsilon} +
     \Psi_{\rho^4, \varepsilon}+\langle \rho^4 \phi_{\varepsilon}, - \lambda \llbracket X_{
     \varepsilon}^3 \rrbracket_{\leqslant} \rangle_{\varepsilon}) + \delta t^2\lambda \| \rho \phi_{\varepsilon}
     \|_{L^{4, \varepsilon}}^4 + C_{\delta,\lambda^{-1}}] \]
  \[ \leqslant e^{\beta \langle t \phi_{\varepsilon} \rangle^{1 - \upsilon}}
      [- t^2 (1 - 2 \delta) \Upsilon_{\varepsilon} + C_{\lambda} t^2 (| \log
     t|^{4 / 3} + 1) \| \mathbb{X}_{\varepsilon} \|^{8 + \vartheta} +
     C_{\delta,\lambda^{-1}}], \]
     where by writing $C_{\delta,\lambda^{-1}}$ we point out that the constant is not uniform over small $\lambda$.
  Therefore by absorbing the constant term $C_{\delta,\lambda^{-1}}$ in $\|
  \mathbb{X}_{\varepsilon} \|^{8 + \vartheta}$ we have
  \begin{equation}
    \begin{array}{l}
      \partial_t e^{\beta \langle t \phi_{\varepsilon} \rangle^{1 - \upsilon}}
      + e^{\beta \langle t \phi_{\varepsilon} \rangle^{1 - \upsilon}} (1 -
      \upsilon) \beta \langle t \phi_{\varepsilon} \rangle^{- \upsilon - 1} (1
      - 2 \delta) t^2 \Upsilon_{\varepsilon}\\
      \qquad \leqslant C_{\delta,\lambda^{-1}} e^{\beta \langle t
      \phi_{\varepsilon} \rangle^{1 - \upsilon}} (1 - \upsilon) \beta \langle
      t \phi_{\varepsilon} \rangle^{- \upsilon - 1} t^2 (| \log t|^{4 / 3} +
      1) \| \mathbb{X}_{\varepsilon} \|^{8 + \vartheta}
    \end{array} \label{eq:int-bound}
  \end{equation}
  Now we can have two situations at any given time, either $\|
  \mathbb{X}_{\varepsilon} \|^2 \leqslant \varsigma \|t \rho
  \phi_{\varepsilon} \|_{L^{4, \varepsilon}}^{1 - \upsilon}$ or $\|
  \mathbb{X}_{\varepsilon} \|^2 > \varsigma \|t \rho \phi_{\varepsilon}
  \|_{L^{4, \varepsilon}}^{1 - \upsilon}$ for some fixed and small $\varsigma
  > 0$. In the first case the right hand side of~{\eqref{eq:int-bound}} is bounded by
  \[ C_{\delta,\lambda^{-1}} e^{\beta \langle t \phi_{\varepsilon} \rangle^{1 -
     \upsilon}} (1 - \upsilon)  \beta \langle t \phi_{\varepsilon} \rangle^{-
     \upsilon - 1} \varsigma^{4 + \vartheta / 2} t^2 (| \log t|^{4 / 3} + 1)
     \|t \rho \phi_{\varepsilon} \|_{L^{4, \varepsilon}}^{(4 + \vartheta / 2)
     (1 - \upsilon)}, \]
  and we can choose $\upsilon = \upsilon (\kappa)$ so that $(4 + \vartheta /
  2) (1 - \upsilon) = 4$ and by taking $\varsigma$ small (depending on $\delta, \lambda$ through $C_{\delta,\lambda^{-1}}$) we can absorb this
  term into the left hand side since for $t \in (0, 1)$ it will be bounded by
  \[ C_{\delta,\lambda^{-1}} e^{\beta \langle t \phi_{\varepsilon} \rangle^{1 - \upsilon}}(1 - \upsilon)  \beta \langle t \phi_{\varepsilon} \rangle^{-
     \upsilon - 1}
     \varsigma^{4 + \vartheta / 2} \tmcolor{red}{} t^2 \| \rho
     \phi_{\varepsilon} \|_{L^{4, \varepsilon}}^4 . \]
  In the case $\| \mathbb{X}_{\varepsilon} \|^2 > \varsigma \|t \rho
  \phi_{\varepsilon} \|_{L^{4, \varepsilon}}^{1 - \upsilon}$ we have
  \[ \| \mathbb{X}_{\varepsilon} \|^2 > \varsigma \|t \rho \phi_{\varepsilon}
     \|_{L^{4, \varepsilon}}^{1 - \upsilon} \gtrsim \varsigma \|t \rho^2
     \phi_{\varepsilon} \|_{L^{2, \varepsilon}}^{1 - \upsilon} \gtrsim
     \varsigma (\langle t \phi_{\varepsilon} \rangle^{1 - \upsilon} - 1), \]
  provided $\rho$ is chosen to be of sufficient decay, and therefore we simply
  bound the right hand side of {\eqref{eq:int-bound}} by
  \[ \lesssim C_{\delta,\lambda^{-1}} e^{(\beta / C \varsigma) \|
     \mathbb{X}_{\varepsilon} \|^2} \| \mathbb{X}_{\varepsilon} \|^{8 +
     \vartheta} \lesssim 1 + e^{(2 \beta / C \varsigma) \|
     \mathbb{X}_{\varepsilon} \|^2} . \]
    The first claim is proven.
     
  It remains to prove the bound for $\varphi_{\varepsilon}$. By H{\"o}lder's inequality, we have
  \[ \mathbb{E} [e^{\beta \langle \varphi_{\varepsilon} (0) - X_{\varepsilon}
     (0) \rangle^{1 - \upsilon}}] =\mathbb{E} [e^{\beta \langle
     \varphi_{\varepsilon} (1) - X_{\varepsilon} (1) \rangle^{1 - \upsilon}}]
     \leqslant \mathbb{E} [e^{ \beta \tmcolor{red}{} \langle Y_{\varepsilon}
     (1) \rangle^{1 - \upsilon} +  \beta \tmcolor{red}{} \langle
     \phi_{\varepsilon} (1) \rangle^{1 - \upsilon}}] \]
  \[ \leqslant [\mathbb{E} [e^{2  \beta \langle Y_{\varepsilon} (1)
     \rangle^{1 - \upsilon}}]]^{1 / 2} [\mathbb{E} [e^{2 \beta  \langle
     \phi_{\varepsilon} (1) \rangle^{1 - \upsilon}}]]^{1 / 2} \]
  and we observe that $\langle Y_{\varepsilon} (1) \rangle^{1 - \upsilon}
  \lesssim 1 + \| \mathbb{X}_{\varepsilon} \|^2$ so the first term on the
  right hand side is integrable uniformly in $\varepsilon$ by {\eqref{eq:exp-int}}. On
  the other hand, using Lemma~\ref{lemma:int-bound} we have
  \[ \mathbb{E} [e^{2  \beta \langle t \phi_{\varepsilon} (t) \rangle^{1 -
     \upsilon}}] + \int_0^t \mathbb{E} [\alpha e^{2  \beta \langle s
     \phi_{\varepsilon} (s) \rangle^{1 - \upsilon}} (1-\upsilon)2\beta \langle s
     \phi_{\varepsilon} (s) \rangle^{- \upsilon - 1} s^2
     \Upsilon_{\varepsilon} (s)] \mathd s \lesssim \mathbb{E} [1 + e^{(2 \beta
     / C) \| \mathbb{X}_{\varepsilon} \|^2}] \]
  and therefore
  \[ \mathbb{E} [e^{2  \beta \langle \phi_{\varepsilon} (1) \rangle^{1 -
     \upsilon}}] \lesssim \mathbb{E} [1 + e^{(2 \beta / C) \|
     \mathbb{X}_{\varepsilon} \|^2}] . \]
  We conclude that
  \[ \sup_{\varepsilon\in\mathcal A}\mathbb{E} [e^{\beta \langle \varphi_{\varepsilon} (0) - X_{\varepsilon}
     (0) \rangle^{1 - \upsilon}}] \lesssim [\mathbb{E} [e^{2 \beta  (1+\|\mathbb X_{\varepsilon}\|^{2})}]]^{1 / 2} [\mathbb{E} [1 +
     e^{(2 \beta / C) \| \mathbb{X}_{\varepsilon} \|^2}]]^{1 / 2} < \infty \]
  uniformly in $\varepsilon$ by {\eqref{eq:exp-int}}, from which the claim
   follows.
\end{proof}



\section{The Osterwalder--Schrader axioms and non-Gaussianity}\label{s:ax}

The goal of this section is to establish several important properties of any
limit measure $\nu$ obtained in the previous section. 
Let us first introduce Osterwalder and
Schrader axioms~{\cite{osterwalder_axioms_1973,osterwalder_axioms_1975}} in the stronger variant of Eckmann and Epstein~\cite{eckmann_time_ordered_1979} for the family of distributions $(S_n \in \mathcal{S}'
(\mathbb{R}^{3n}))_{n \in\mathbb{N}_{0}}$.

\begin{description}
  \item[OS0] (Distribution property) It holds $S_0 = 1$. There is a Schwartz
  norm $\| \cdot \|_s$ on $\mathcal{S}' (\mathbb{R}^3)$ and $\beta > 0$ such
  that for all $n \in\mathbb{N}$ and $f_1, \ldots, f_n \in \mathcal{S}
  (\mathbb{R}^3)$
  \begin{equation}
    | S_n (f_1 \otimes \ldots \otimes f_n) | \leqslant (n!)^{\beta} \prod_{i =
    1}^n \| f_i \|_s . \label{eq:os-reg}
  \end{equation}
  \item[OS1] (Euclidean invariance) For each $n\in\mathbb{N}$, $g = (a, R) \in
  \mathbb{R}^3 \times \mathrm{O} (3)$, $f_1, \ldots, f_n \in \mathcal{S}
  (\mathbb{R}^3)$
  \[ S_n ((a, R) .f_1 \otimes \ldots \otimes (a, R) .f_n) = S_n (f_1 \otimes
     \ldots \otimes f_n), \]
  where $(a, R) .f_n (x) = f_n (a + R x)$ and where $\mathrm{O}(3)$ is the orthogonal
  group of $\mathbb{R}^3$.
  
  \item[OS2] (Reflection positivity) Let  $\mathbb{R}^{3 n}_{+} = \{ (x^{(1)}, \ldots, x^{(n)}) \in
(\mathbb{R}^3)^n : x_1^{(j)}>0, j=1,\dots,n \}$ and
\[ \mathcal{S}_{\mathbb{C}} (\mathbb{R}^{3 n}_{+}) \assign \{ f \in \mathcal{S}
   (\mathbb{R}^{3 n};\mathbb{C}) : \tmop{supp} (f) \subset \mathbb{R}^{3 n}_{+} \} . \]
 For all sequences $(f_n \in
  \mathcal{S}_{\mathbb{C}} (\mathbb{R}^{3 n}_{+}))_{n \in\mathbb{N}_{0}}$ with
  finitely many nonzero elements
  \begin{equation}
    \sum_{n, m \in\mathbb{N}_{0}} S_{n + m} (\overline{\Theta f_n} \otimes f_m)
    \geqslant 0, \label{eq:OS2}
  \end{equation}
  where $\Theta f_n (x^{(1)}, \ldots, x^{(n)}) = f (\theta x^{(1)}, \ldots,
  \theta x^{(n)})$ and $\theta (x_1, x_{2}, x_3) = (- x_1, x_{2}, x_3)$ is
  the reflection with respect to the plane $x_1 = 0$.
  
  \item[OS3] (Symmetry) For all $n \in\mathbb{N}$, $f_1, \ldots, f_n \in
  \mathcal{S} (\mathbb{R}^3)$ and $\pi$ a permutation of $n$ elements
  \[ S_n (f_1 \otimes \cdots \otimes f_n) = S_n (f_{\pi (1)} \otimes \cdots
     \otimes f_{\pi (n)}) . \]
\end{description}

The reconstruction theorem of Eckmann and Epstein (Theorem 2 and Corollary 3 in~\cite{eckmann_time_ordered_1979}) asserts that
distributions $(S_n)_{n\in\mathbb{N}_{0}}$ which satisfy OS0--3 are the
 Schwinger functions of a uniquely determined system of time-ordered products of relativistic quantum fields.  Note that if Euclidean invariance in OS1 is replaced with translation invariance with respect to the first coordinate (the Euclidean time), then the reconstruction theorem gives anyway a quantum theory with a unitary time evolution, possibly lacking the full Poincar\'e invariance.
  
For any measure $\mu$ on $\mathcal{S}' (\mathbb{R}^3)$ we define $S_n^{\mu}
\in (\mathcal{S}' (\mathbb{R}^3))^{\otimes n}$ as
\[ S_n^{\mu} (f_1 \otimes \cdots \otimes f_n) \assign \int_{\mathcal{S}'
   (\mathbb{R}^3)} \varphi (f_1) \cdots \varphi (f_n) \mu (\mathd \varphi),
   \qquad n\in \mathbb{N}, f_1, \ldots, f_n \in \mathcal{S} (\mathbb{R}^3) . \]
In this case OS3 is trivially satisfied. 
Along this section we will prove
that, for any accumulation point $\nu$, the functions $(S^{\nu}_n)_n$ satisfy
additionally OS0, OS2 and OS1 with the exception of invariance with respect to $\tmop{SO}
(3)$ (but including reflections) and moreover that $\nu$  is not a Gaussian measure.

\subsection{Distribution property}
\label{ss:OS0}

Here we are concerned with proving the bound~{\eqref{eq:os-reg}} for
correlation functions of $\nu$.

\begin{proposition}
  \label{prop:os-reg-bound}There exists $\beta > 1$ and $K > 0$ such that  any limit measure $\nu$ constructed via the procedure in Section~\ref{sec:tight} satisfies: for
  all $n \in\mathbb{N}$ and all $f_1, \ldots, f_n \in H^{1 / 2 +2 \kappa}
  (\rho^{- 2})$ we have
  \[ | \mathbb{E}_{\nu} [\varphi (f_1) \cdots \varphi (f_n)] | \leqslant K^n
     (n!)^{\beta} \prod_{i = 1}^n \| f_i \|_{H^{1 / 2 +2 \kappa} (\rho^{- 2})}
     . \]
     In particular, it satisfies {\em OS0}.
\end{proposition}

\begin{proof}
  For any $\alpha \in (0, 1)$ and any $n \in\mathbb{N}$ we obtain with  the notation $\langle
\varphi \rangle_{\ast} \assign (1 + \| 
\varphi \|_{H^{- 1 / 2 - 2 \kappa}(\rho^{2})}^2)^{1 / 2}$
  \[ \mathbb{E}_{\nu} [\| \varphi \|_{H^{- 1 / 2 -2 \kappa} (\rho^2)}^n] \leqslant
     \mathbb{E}_{\nu} [\langle \varphi \rangle^{\alpha (n / \alpha)}] \leqslant
     \mathbb{E}_{\nu} [\langle \varphi \rangle^{\alpha \lceil n / \alpha \rceil}]
     \leqslant \beta^{- \lceil n / \alpha \rceil} (\lceil n / \alpha \rceil !)
     \mathbb{E}_{\nu} [e^{\beta \langle \varphi \rangle^{\alpha}}] \]
  \[ \leqslant K^n (n!)^{1 / \alpha} \mathbb{E}_{\nu} [e^{\beta \langle \varphi
     \rangle^{\alpha}}] ,\]
  where we used the fact that Stirling's asymptotic approximation of the
  factorial allows to estimate
  \[ \lceil n / \alpha \rceil ! \leqslant C \left( \frac{\lceil n / \alpha
     \rceil}{e} \right)^{\lceil n / \alpha \rceil} (2 \pi \lceil n / \alpha
     \rceil)^{1 / 2} \leqslant C \left( \frac{2 (n / \alpha)}{e} \right)^{n /
     \alpha + 1} (2 \pi \lceil n / \alpha \rceil)^{1 / 2} \]
  \[ \leqslant K^n \left[ \left( \frac{n}{e} \right)^n (2 \pi n)^{1 / 2}
     \right]^{1 / \alpha} \leqslant K^n (n!)^{1 / \alpha} \]
  for some constants $C, K$, uniformly in $n$ (we allow $K$ to change from
  line to line). From this we can conclude using Proposition \ref{lemma:int-bound}.
\end{proof}







\subsection{Translation invariance}
\label{ss:OS1}

For $h \in \mathbb{R}^3$ we denote by $\mathcal{T}_h : \mathcal{S}'
(\mathbb{R}^3) \rightarrow \mathcal{S}' (\mathbb{R}^3)$ the translation
operator, namely, $\mathcal{T}_h f (x) \assign f (x - h)$. Analogically, for a
measure $\mu$ on $\mathcal{S}' (\mathbb{R}^3)$ we define its translation by
$\mathcal{T}_h \mu (F) \assign \mu (F \circ \mathcal{T}_h)$ where $F \in C_b
(\mathcal{S}' (\mathbb{R}^3))$. We say that $\mu$ is translation invariant if
for all $h \in \mathbb{R}^3$ it holds $\mathcal{T}_h \mu = \mu$.

\begin{proposition}
 Any limit measure $\nu$ constructed via the procedure in Section~\ref{sec:tight} is translation
  invariant.
\end{proposition}

\begin{proof}
  By their definition in {\eqref{eq:gibbs}}, the approximate
  measures $\nu_{M, \varepsilon}$ are translation invariant under lattice
  shifts. That is, for $h_{\varepsilon} \in \Lambda_{\varepsilon}$ it holds
  $\mathcal{T}_{h_{\varepsilon}} \nu_{M, \varepsilon} = \nu_{M, \varepsilon}$.
  In other words, the processes $\varphi_{M, \varepsilon}$ and
  $\mathcal{T}_{h_{\varepsilon}} \varphi_{M, \varepsilon}$ coincide in law. In
  addition, since the translation $\mathcal{T}_{h_{\varepsilon}}$ commutes
  with the extension operator $\mathcal{E}^{\varepsilon}$, it follows that
  $\mathcal{E}^{\varepsilon} \varphi_{M, \varepsilon}$ and
  $\mathcal{T}_{h_{\varepsilon}} \mathcal{E}^{\varepsilon} \varphi_{M,
  \varepsilon}$ coincide in law. Now we recall that the limiting measure $\nu$
  was obtained as a weak limit of the laws of $\mathcal{E}^{\varepsilon}
  \varphi_{M, \varepsilon}$ on $H^{- 1 / 2 - 2\kappa} (\rho^{2 + \gamma})$. If
  $h \in \mathbb{R}^d$ is given, there exists a sequence $h_{\varepsilon} \in
  \Lambda_{\varepsilon}$ such that $h_{\varepsilon} \rightarrow h$. Let
  $\kappa \in (0, 1)$ be small and arbitrary. Then we have for $F \in C^{0,
  1}_b (H^{- 1 / 2 - 3 \kappa} (\rho^{2 + \gamma}))$ that
  \[ \mathcal{T}_h \nu (F) = \nu (F \circ \mathcal{T}_h) = \lim_{\varepsilon
     \rightarrow 0, M \rightarrow \infty} \mathbb{P} \circ
     (\mathcal{E}^{\varepsilon} \varphi_{M, \varepsilon})^{- 1} (F \circ
     \mathcal{T}_h) = \lim_{\varepsilon \rightarrow 0, M \rightarrow \infty}
     \mathbb{E} [F (\mathcal{T}_h \mathcal{E}^{\varepsilon} \varphi_{M,
     \varepsilon})] \]
  \[ \  \]
  \[ = \lim_{\varepsilon \rightarrow 0, M \rightarrow \infty} \mathbb{E} [F
     (\mathcal{T}_{h_{\varepsilon}} \mathcal{E}^{\varepsilon} \varphi_{M,
     \varepsilon})] = \lim_{\varepsilon \rightarrow 0, M \rightarrow \infty}
     \mathbb{E} [F (\mathcal{E}^{\varepsilon} \varphi_{M, \varepsilon})] = \nu
     (F), \]
  where in the third inequality we used the regularity of $F$ and Theorem~\ref{thm:tight} as follows
  \[ \mathbb{E} [F (\mathcal{T}_h \mathcal{E}^{\varepsilon} \varphi_{M,
     \varepsilon}) - F (\mathcal{T}_{h_{\varepsilon}}
     \mathcal{E}^{\varepsilon} \varphi_{M, \varepsilon})] \leqslant \| F
     \|_{C^{0, 1}_b} \mathbb{E} \| \mathcal{T}_h \mathcal{E}^{\varepsilon}
     \varphi_{M, \varepsilon} - \mathcal{T}_{h_{\varepsilon}}
     \mathcal{E}^{\varepsilon} \varphi_{M, \varepsilon} \|_{H^{- 1 / 2 - 3
     \kappa} (\rho^{2 + \gamma})} \]
  \[ \lesssim (h - h_{\varepsilon})^{\kappa} \mathbb{E} \|
     \mathcal{E}^{\varepsilon} \varphi_{M, \varepsilon} \|_{H^{- 1 / 2 -
     2\kappa} (\rho^{2 + \gamma})} \lesssim (h - h_{\varepsilon})^{\kappa}
     \rightarrow 0 \quad \tmop{as} \quad \varepsilon \rightarrow 0. \]
  If $F \in C_b (H^{- 1 / 2 - 3 \kappa} (\rho^{2 + \gamma}))$, then by
  approximation and dominated convergence theorem we also get $\mathcal{T}_h
  \nu (F) = \nu (F)$, which completes the proof.
\end{proof}

\subsection{Reflection positivity}
\label{ss:OS2}

As the next step we
establish reflection positivity of
$\nu$ with respect to the reflection given by any of the hyperplanes $\{x_i=0\}\subset \mathbb{R}^3$ for $i \in \{1, 2,3\}$. 
Fix a small $\delta>0$  and $i\in\{1,2,3\}$ and  define
the space of functionals depending on fields restricted to
$\mathbb{R}^3_{+, \delta}  := \{x \in \mathbb{R}^3 ; x_i > \delta\}$, $\delta\geqslant0$, by
\[ \mathcal{H}_{+, \delta} \assign \left\{ \sum_{k = 1}^K c_k e^{i \varphi
   (f_k)} ; c_k \in \mathbb{C}, f_k \in C^{\infty}_0 (\mathbb{R}^3_{+,\delta}), K \in
   \mathbb{N} \right\} \]
and let $\mathcal{H}_+ =\mathcal{H}_{+, 0}$. For a function $f : \mathbb{R}^3
\rightarrow \mathbb{R}$ we define its reflection
\[ (\theta f) (x) \assign (\theta^i f) (x) \assign f (x_1, \ldots, x_{i - 1},
   - x_i, x_{i + 1}, \ldots, x_3) \]
and extend it to $F \in \mathcal{H}_+$ by $\theta F (\varphi (f_1), \ldots,
\varphi (f_K)) \assign F (\varphi (\theta f_1), \ldots, \varphi (\theta
f_K))$. Hence for $F \in \mathcal{H}_{+, \delta}$ the reflection $\theta F$
depends on $\varphi$ evaluated at $x \in \mathbb{R}^3$ with $x_i < - \delta$.

A measure $\mu$ on $\mathcal{S}'(\mathbb{R}^3)$ is \emph{reflection positive} if
\[ \mathbb{E}_{\mu}  [\overline{\theta F} F] = \int_{\mathcal{S}'
   (\mathbb{R}^3)} \overline{\theta F (\varphi)} F (\varphi) \mu (\mathd
   \varphi) 
   \geqslant 0, \]
for all $F = \sum_{k = 1}^K c_k e^{i \varphi (f_k)} \in \mathcal{H}_+$.
A similar definition applies to measures on functions on the periodic lattice ${\Lambda_{M, \varepsilon}}$ replacing the space $\mathcal{H}_+$ with the appropriate modification $\mathcal{H}_+^{M, \varepsilon}$ given by
$$
\mathcal{H}^{M,\varepsilon}_{+} \assign \left\{ \sum_{k = 1}^K c_k e^{i \varphi
   (f_k)} ; c_k \in \mathbb{C}, f_k :\Lambda_{M,\varepsilon}\cap\mathbb{R}^{3}_{+,0}\to\mathbb{R}\right\}.
$$
The reflection $\theta$ is then defined as on the full space.
Here and also in the proof of Proposition~\ref{prop:RP} below we implicitly assume that $\varepsilon$ is small enough and $M$ is large enough.

An important fact is that for every  $\varepsilon, M$ the Gibbs  measures $\nu_{M, \varepsilon}$ are reflection
  positive see~{\cite[Theorem 7.10.3]{MR887102} or  \cite[Lemma 10.8]{friedli2017statistical}}. The key point of the next proposition is that this property is preserved along the passage to the limit $M\to\infty$, $\varepsilon\to0$.

\begin{proposition}\label{prop:RP}
  Any limit measure $\nu$ constructed via the procedure in
  Section~\ref{sec:tight} is reflection positive with respect to all
  reflections $\theta = \theta^i$, $i \in \{1, 2,3\}$. In particular, its correlation functions
  satisfy {\em OS2}.
\end{proposition}


\begin{proof}
  We recall that the measure $\nu$ was obtained as a
  limit of suitable continuum extensions of the measures $\nu_{M,
  \varepsilon}$ given by {\eqref{eq:gibbs}}. 
 Therefore, up to a subsequence, we have
  \[ \mathbb{E}_{\nu}  [\overline{\theta F} F]  = \lim_{\varepsilon
     \rightarrow 0, M \rightarrow \infty} \mathbb{E} [\overline{F (\theta
     \mathcal{E}^{\varepsilon} \varphi_{M, \varepsilon})} F
     (\mathcal{E}^{\varepsilon} \varphi_{M, \varepsilon})] . \]
 Recall  that the function $w$ in the definition of the
  extension operator $\mathcal{E}^{\varepsilon}$ is radially
  symmetric. Hence, we have
$
(\theta
     \mathcal{E}^{\varepsilon} \varphi_{M, \varepsilon})(f)= 
     \varphi_{M, \varepsilon}(\mathcal{E}^{\varepsilon,*} \theta f) =\varphi_{M, \varepsilon}(\theta \mathcal{E}^{\varepsilon,*}  f)
$  
for any function $f\in C^\infty_0(\mathbb{R}^3)$  supported in $\{x\in\mathbb{R}^{3};|x_i| <  M/2 -\delta\}$. Here $\mathcal{E}^{\varepsilon,*}$ is the adjoint of the extension operator. For a fixed $F  \in \mathcal{H}_{+, \delta}$ we have therefore $F (\theta
     \mathcal{E}^{\varepsilon} \varphi_{M, \varepsilon})=(F \circ \mathcal{E}^{\varepsilon})(
      \theta\varphi_{M, \varepsilon})$  provided $\varepsilon$ is small enough and $M$ large enough depending  on $F$ and
  $\delta$. Hence,
    \[ \mathbb{E}_{\nu}  [\overline{\theta F} F] = \lim_{\varepsilon \rightarrow
     0, M \rightarrow \infty} \mathbb{E} [\overline{F
     (\mathcal{E}^{\varepsilon} \theta \varphi_{M, \varepsilon})} F
     (\mathcal{E}^{\varepsilon} \varphi_{M, \varepsilon})]. \]
     However, since the extension operator is defined as a convolution with a non-compactly supported function $w^{\varepsilon}$, it is generally not true that $F \circ \mathcal{E}^{\varepsilon}
  \in \mathcal{H}_{+}^{M, \varepsilon}$. Thus, in order to be able to use the reflection positivity of the measures $\nu_{M,\varepsilon}$, we need to introduce an additional cut-off: let  $H_{\delta}:\mathbb{R}^{3}\to[0,1]$ be smooth and supported on $\mathbb{R}^{3}_{+,0}$ such that $H_{\delta}=1$ on $\mathbb{R}^{3}_{+,\delta/2}$. We denote by $H_{\delta,\varepsilon}$ its restriction to $\Lambda_{\varepsilon}$ and write
  \[ 
 R_\varepsilon  :=  F(\mathcal{E}^{\varepsilon}\varphi_{M,\varepsilon})-  F(\mathcal{E}^{\varepsilon}(H_{
  \delta,\varepsilon}\varphi_{M,\varepsilon}))
  .\] 
  Our goal is to show that $R_\varepsilon$ vanishes a.s. as $\varepsilon\to 0$. In view of the fact that $F$ is cylindrical and then regularity of $\varphi_{M,\varepsilon}$, it is enough to show that
  \begin{equation}
  \label{eq:H-lim}
\lim_{\varepsilon\to 0} \| (1-H_{\delta,\varepsilon})\mathcal{E}^{\varepsilon,*}f\|_{H^{1/2+\kappa,\varepsilon}(\rho^{-2})}=0
\end{equation}
%for some $\alpha>0$ and   
for any function $f\in C^{\infty}_{0}(\mathbb{R}^{3}_{+,\delta}).$ It holds
  \begin{equation}\label{eq:lll}
  [(1-H_{\delta,\varepsilon})\mathcal{E}^{\varepsilon,*}f](x)=(1-H_{\delta,\varepsilon})(x)\int_{y\in \mathbb{R}^{3}:y_{i}>\delta}w^{\varepsilon}(x-y)f(y)\mathrm{d} y,
  \end{equation}
  where $1-H_{\delta,\varepsilon}(x)\neq 0$ only when $x_{i}\leq \delta/2$. Since $w^{\varepsilon}(\cdot)=\varepsilon^{-d}w(\varepsilon^{-1}\cdot)$ with $w\in\mathcal{S}(\mathbb{R}^{3}),$ we have for an arbitrary $K>0$ and $m\in\mathbb{N}$
  $$| \nabla^{m}w^{\varepsilon} (x - y) | \lesssim \varepsilon^{- d -m} | \varepsilon^{-1}(x - y)  |^{- K}.$$
In addition,  we know that the relevant $|x-y|$ on the right hand side of \eqref{eq:lll} satisfy $|x_{i}-y_{i}|>\delta/2$. Hence, choosing $K,L$ sufficiently large will give us a decay as $\varepsilon\to 0$ for every fixed $\delta>0$. Indeed, we also have
$
|\nabla_{\varepsilon}^{m}(1-H_{\delta,\varepsilon})(x)|\lesssim \delta^{-1}
$
uniformly in $\varepsilon$.
Thus,   we may estimate 
$$
   \| (1-H_{\delta,\varepsilon})\mathcal{E}^{\varepsilon,*}f \|_{H^{1/2+\kappa}(\rho^{-2})}\leq c(\varepsilon,\delta)\|f\|_{L^{\infty}}
%   \varepsilon^{-
%m} \left(\frac{\varepsilon}{\delta}\right)^{K}\frac{1}{\delta}\left\|\varepsilon^{-d}\left(\varepsilon^{-1}|\cdot|\right)^{-L}\right\|_{L^{1}}\|f\|_{L^{\infty}}
 , $$
where $c(\varepsilon,\delta)\to 0$ as $\varepsilon\to\infty$ for every fixed  $\delta>0$. This concludes the proof of~\eqref{eq:H-lim}.
 
 
 On the other hand,   $F (\mathcal{E}^{\varepsilon} ( H_{\delta,\varepsilon} \cdot))
  \in \mathcal{H}_{+}^{M, \varepsilon}$  and consequently
  \[ \mathbb{E}_{\nu}  [\overline{\theta F} F] = \lim_{\varepsilon \rightarrow
     0, M \rightarrow \infty} \mathbb{E} [\overline{F
     (\mathcal{E}^{\varepsilon} (H_{\delta,\varepsilon}\theta \varphi_{M, \varepsilon}))} F
     (\mathcal{E}^{\varepsilon}(H_{\delta,\varepsilon} \varphi_{M, \varepsilon}))] \]
  \[ = \lim_{\varepsilon \rightarrow 0, M \rightarrow \infty} \mathbb{E}_{\nu_{M, \varepsilon}}
     [\overline{\theta (F (\mathcal{E}^{\varepsilon}( H_{\delta,\varepsilon} \cdot)))} F (\mathcal{E}^{\varepsilon}( H_{\delta,\varepsilon} \cdot))] \geqslant 0, \]
  where we used the reflection positivity  of the  measure $\nu_{M, \varepsilon}$.
  Using the support properties of $\nu$ we can now approximate any $F \in
  \mathcal{H}_+$ by functions in $\mathcal{H}_{+, \delta}$ and therefore
  obtain the first claim. Let us now show that~{\eqref{eq:OS2}} holds.
  Thanks to the exponential integrability satisfied by $\nu$, any
  polynomial of the form $G = \sum_{n \in\mathbb{N}_{0}} \varphi^{\otimes n} (f_n)$
  for sequences $(f_n \in \mathcal{S}_{\mathbb{C}} (\mathbb{R}^{3 n}_{+}))_{n
  \in\mathbb{N}_{0}}$ with finitely many nonzero elements, belongs to $L^2 (\nu)$.
  In particular it can be approximated in $L^2 (\nu)$ by  a sequence
  $(F_n)_n$ of cylinder functions in $\mathcal{H}_+$. Therefore
  $\mathbb{E}_{\nu}  [\overline{\theta G} G] = \lim_{n \rightarrow \infty}
  \mathbb{E}_{\nu}  [\overline{\theta F_n} F_n] \geqslant 0$ and we conclude
  that
  \[ \sum_{n, m \in\mathbb{N}_{0}} S_{n + m}^{\nu} (\overline{\theta f_n} \otimes
     f_m) = \sum_{n, m \in\mathbb{N}_{0}} \mathbb{E}_{\nu} [\varphi^{\otimes n}
     (\overline{\theta f_n}) \varphi^{\otimes m} (f_m)] =\mathbb{E}_{\nu} 
     [\overline{\theta G} G] \geqslant 0. \]
\end{proof}



\subsection{Non-Gaussianity}
\label{ss:nonG}



\begin{theorem}
 If $\lambda> 0$ then any limit measure $\nu$ constructed via the procedure in Section~\ref{sec:tight} is non-Gaussian.
\end{theorem}

\begin{proof}
  In order to show that the limiting measure $\nu$ is non-Gaussian, it is
  sufficient to prove that the connected four-point function is nonzero, see \cite{MR723546}. In other words, we shall prove that the
  distribution
  \[ U^{\nu}_4 (x_1, \ldots, x_4) \assign \mathbb{E}_{\nu} [\varphi (x_1)
     \cdots \varphi (x_4)] \]
  \[ -\mathbb{E}_{\nu} [\varphi (x_1) \varphi (x_2)] \mathbb{E}_{\nu} [\varphi
     (x_3) \varphi (x_4)] -\mathbb{E}_{\nu} [\varphi (x_1) \varphi (x_3)]
     \mathbb{E}_{\nu} [\varphi (x_2) \varphi (x_4)] \]
  \[ -\mathbb{E}_{\nu} [\varphi (x_1) \varphi (x_4)] \mathbb{E}_{\nu} [\varphi
     (x_2) \varphi (x_3)], \qquad x_1, \ldots, x_4 \in \mathbb{R}^d, \]
  is nonzero.
  
  Recall that in Theorem~\ref{thm:main} we obtained a limit measure $\mu$ which is the joint law of $(\varphi,X,\tthreeone{X})$ and that $\nu$ is the marginal corresponding to the first component. Let $K_i = \mathcal{F}^{- 1} \varphi_i$ be a Littlewood--Paley
  projector and consider the connected four-point function $U^{\nu}_4$
  convolved with $(K_i, K_i, K_i, K_i)$ and evaluated at $(x_{1},\dots,x_{4})=(0,\dots, 0)$, that
  is,
  \[ U^{\nu}_4 \ast (K_i, K_i, K_i, K_i) (0, 0, 0, 0) =\mathbb{E}_{\nu}
     [(\Delta_i \varphi)^4 (0)] - 3\mathbb{E}_{\nu} [(\Delta_i \varphi)^2
     (0)]^2 \]
  \[ =\mathbb{E}_{\mu} [(\Delta_i \varphi)^4 (0)] - 3\mathbb{E}_{\mu}
     [(\Delta_i \varphi)^2 (0)]^2 \backassign L (\varphi, \varphi, \varphi,
     \varphi), \]
  where $L$ is a quadrilinear form. Since under the limit $\mu$ we have the
   decomposition $\varphi = X -\lambda \tthreeone{X} + \zeta$, we may write
  \begin{equation}
    L (\varphi, \varphi, \varphi, \varphi) = L (X, X, X, X) - 4 \lambda L ( X, X,
    X, \tthreeone{X} ) + R \label{eq:L24}
  \end{equation}
  where $R$ contains terms which are at least bilinear in $\tthreeone{X}$ or
  linear in $\zeta$. Due to Gaussianity of $X$, the first term on the right
  hand side of {\eqref{eq:L24}} vanishes. Our goal is to show that the second
  term behaves like $2^i$ whereas the terms in $R$ are more regular, namely,
  bounded by $2^{i (1 / 2 + \kappa)}$. In other words, $R$ cannot compensate
  $4\lambda L ( X, X, X, \tthreeone{X} )$ and as a consequence $L (\varphi,
  \varphi, \varphi, \varphi) \neq 0$ if $\lambda> 0$.
  
  Let us begin with $L ( X, X, X, \tthreeone{X} )$. To this end, we denote $k_{[123]}=k_{1}+k_{2}+k_{3}$ and
  recall that
  \[ (\Delta_i X) (0) = \int_{\mathbb{R}^d} \varphi_i (k) \int_{- \infty}^0
     e^{- [m^{2} + | k |^2] (- s)} \hat{\xi} (\mathd s, \mathd k), \]
  \[ ( \Delta_i \tthreeone{X} ) (0) = \int^0_{- \infty} \mathd s
     \int_{\mathbb{R}^d} \int_{\mathbb{R}^d} \int_{\mathbb{R}^d} \varphi_i
     (k_{[123]}) e^{- [m^{2} + | k_{[123]} |^2] (- s)} \]
  \[ \times \left\llbracket \prod_{l = 1, 2, 3} \int^s_{- \infty} e^{- [m^{2}+ |
     k_l |^2] (s - s_l)} \hat{\xi} (\mathd s_l, \mathd k_l) \right\rrbracket,
  \]
  where $\llbracket\cdot\rrbracket$ denotes Wick's product.
Hence denoting $H\assign 
     [4m^{2} + | k_{[123]} |^2+|k_{1}|^{2}+|k_{2}|^{2}+|k_{3}|^{2}] $ we obtain
  \[ L ( X, X, X, \tthreeone{X} ) =\mathbb{E} \left[ (\Delta_i X)
     (0) (\Delta_i X) (0) (\Delta_i X) (0) ( \Delta_i \tthreeone{X}
     ) (0) \right] \]
  \[ = 3! \int^0_{- \infty} \mathd s \int_{\mathbb{R}^d}
     \int_{\mathbb{R}^d} \int_{\mathbb{R}^d} \varphi_i (k_{[123]})  e^{-H(-s)}  \prod_{l = 1, 2, 3} \left[ \int_{- \infty}^s e^{- 
     2[m^{2}+| k_l |^2] (s - s_l)} \varphi_i (k_l) \mathd s_l \mathd k_l \right]
  \]
  \[ = \frac{3!}{8} \int^0_{- \infty} \mathd s
     \int_{\mathbb{R}^d} \int_{\mathbb{R}^d} \int_{\mathbb{R}^d} \varphi_i
     (k_{[123]}) e^{-H(-s)} \prod_{l = 1,
     2, 3} \left[ \varphi_i (k_l) \frac{\mathd k_l}{m^{2} + | k_l |^2} \right] \]
  \[ \  \]
  \[ = \frac{3!}{8} \int_{\mathbb{R}^d} \int_{\mathbb{R}^d}
     \int_{\mathbb{R}^d} \frac{\varphi_i (k_{[123]})}{H} \prod_{l = 1, 2, 3} \left[ \varphi_i (k_l) \frac{\mathd k_l}{m^{2} +
     | k_l |^2} \right] \approx 2^{i (- 8 + 9)} \approx 2^i . \]
  
  
  Let us now estimate various terms in $R$. The terms containing only
  combinations of $X, \tthreeone{X}$ can be estimated directly whereas for
  terms where $\zeta$ appears it is necessary to use stationarity due to the
  limited integrability in space. For instance,
  \[ \left| \mathbb{E} \left[ (\Delta_i X) (0) (\Delta_i X) (0) (
     \Delta_i \tthreeone{X} ) (0) ( \Delta_i \tthreeone{X} )
     (0) \right] \right| \]
  \[ \lesssim 2^{- 2 i (- 1 / 2 - \kappa)} 2^{- 2 i (1 / 2 - \kappa)}
     \mathbb{E} \left[ \| X \|_{\CC^{- 1 / 2 - \kappa} (\rho^{\sigma})}^2
     \| \tthreeone{X} \|_{\CC^{1 / 2 - \kappa} (\rho^{\sigma})}^2
     \right] \lesssim 2^{i4 \kappa} \]
  and similarly for the other terms without $\zeta$ which are collectively of order $2^{i4 \kappa} (\lambda^2+\lambda^4)$. For the remaining terms,
  we fix a weight $\rho$ as above and use stationarity. In addition, we shall
  be careful about having the necessary integrability. For instance, for the
  most irregular term we have
  \[ \mathbb{E} [(\Delta_i X)^3 (0) (\Delta_i \zeta) (0)] =
     \int_{\mathbb{R}^d} \rho^4 (x) \mathbb{E} [(\Delta_i X)^3 (x) (\Delta_i
     \zeta) (x)] \mathd x =\mathbb{E} \langle \rho^4, (\Delta_i X)^3 (\Delta_i
     \zeta) \rangle \]
  and we bound this quantity as
  \[\begin{aligned}
   | \mathbb{E} [(\Delta_i X)^3 (0) (\Delta_i \zeta) (0)] | & \leqslant
     \mathbb{E} [\| \Delta_i X_{\varepsilon} \|_{L^{\infty} (\rho^{\sigma})}^3
     \| \Delta_i \zeta \|_{L^1 (\rho^{4 - 3 \sigma})}] \lesssim \mathbb{E} [\|
     \Delta_i X_{\varepsilon} \|_{L^{\infty} (\rho^{\sigma})}^3 \| \Delta_i
     \zeta \|_{L^2 (\rho^2)}] \\
  & \lesssim 2^{- 3 i (- 1 / 2 - \kappa)} 2^{i (- 1 + 2 \kappa)} \mathbb{E}
     \left[ \| X \|^3_{\CC^{- 1 / 2 - \kappa} (\rho^{\sigma})} \| \zeta
     \|_{B^{1 - 2 \kappa}_{2, 2} (\rho^2)} \right]  \\
  & \lesssim 2^{- 3 i (- 1 / 2 - \kappa)} 2^{i (- 1 + 2 \kappa)} (\mathbb{E}
     [ \| X \|^6_{\CC^{- 1 / 2 - \kappa} (\rho^{\sigma})}])^{1/2} (\mathbb{E}[\| \zeta
     \|^2_{B^{1 - 2 \kappa}_{2, 2} (\rho^2)}])^{1/2} 
     \\ & \lesssim 2^{i (1 / 2 + 5
     \kappa)} (\lambda+\lambda^{7/2}).
     \end{aligned} 
     \]
 where we used Theorem~\ref{thm:main}. Next,
  \[ | \mathbb{E} [(\Delta_i X)^2 (0) (\Delta_i \zeta)^2 (0)] | \leqslant
     \mathbb{E} [\| \Delta_i X \|^2_{L^{\infty} (\rho^{\sigma})} \| \Delta_i
     \zeta \|_{L^2 (\rho^{1 + \iota})} \| \Delta_i \zeta \|_{L^2 (\rho^2)}] \]
  \[ \leqslant 2^{- 2 i (- 1 / 2 - \kappa)} 2^{- i (1 - 2 \kappa)} \mathbb{E}
     [\| X \|^2_{\CC^{-1/2-\kappa} (\rho^{\sigma})} \| \zeta \|_{B^0_{4, \infty}
     (\rho)} \|  \zeta \|_{H^{1 - 2 \kappa} (\rho^2)}] \lesssim 2^{i 4
     \kappa} ({\lambda^{5/4}+\lambda^{5}}), \]
  and
  \[ 
  \begin{aligned}
  | \mathbb{E} [(\Delta_i X) (0) (\Delta_i \zeta)^3 (0)] | & \leqslant
     \mathbb{E} [\| \Delta_i X \|_{L^{\infty} (\rho^{\sigma})} \| \Delta_i
     \zeta \|^3_{L^3 (\rho^{(4 - \sigma) / 3})}] \\
  & \leqslant \mathbb{E} [\| \Delta_i X \|_{L^{\infty} (\rho^{\sigma})} \|
     \Delta_i \zeta \|^3_{L^4 (\rho)}] \\ & \quad \lesssim 2^{- i (- 1 / 2 - \kappa)}
     \mathbb{E} \left[ \| X \|_{\CC^{- 1 / 2 - \kappa} (\rho^{\sigma})} \|
     \zeta \|^3_{B^0_{4, \infty} (\rho)} \right] \\
   & \lesssim 2^{i (1 / 2 +
     \kappa)}({\lambda^{3/4}+\lambda^{9/2}}),
     \end{aligned} \]
  \[ | \mathbb{E} [(\Delta_i \zeta)^4 (0)] | = | \mathbb{E} \langle \rho^4,
     (\Delta_i \zeta)^4 \rangle | \leqslant \mathbb{E} \| (\Delta_i \zeta)
     \|^4_{L^4 (\rho)} \leqslant \mathbb{E} [\| \zeta \|^4_{B^0_{4, \infty}
     (\rho)}] \lesssim ({\lambda+\lambda^{6}}). \]
  Proceeding similarly for the other terms we finally obtain the  bound   \[ | R | \lesssim  2^{i (1 / 2 + 5 \kappa)} ({\lambda^{3/4}+\lambda^{7}}). \]
  Therefore for a fixed $\lambda>0$ there exists  a sufficiently large $i$  such that
  \[ \mathbb{E} [(\Delta_i \varphi)^4 (0)] - 3 (\mathbb{E} [(\Delta_i
     \varphi)^2 (0)^2])^2 \lesssim -2^i \lambda < 0, \]
  and the proof is complete.
\end{proof}

\begin{remark}\label{rem:rev1}
To our knowledge, the proof of non-Gaussianity given above, is new. In particular the pathwise estimates of the PDE methods allow to probe correlation functions at high-momenta and check that they are, at leading order, given by perturbative contributions irrespective of the size of the coupling $\lambda$. This seems to be a substantial improvement with respect to the perturbative strategy of~\cite{MR723546} which requires small $\lambda$.
\end{remark}

\section{Integration by parts formula and Dyson--Schwinger equations}
\label{s:sd}



The goal of this section is twofold. First, we introduce a new
paracontrolled ansatz, which allows to prove higher regularity and in
particular to give meaning to the critical resonant product in the continuum.
Second, the higher regularity is used in order to improve the tightness 
and  to construct a renormalized cubic term $\llbracket
\varphi^3 \rrbracket$. 
Finally, we derive an integration by parts formula
together with the Dyson--Schwinger equations and we identify the continuum dynamics.




\subsection{Improved tightness}
\label{s:reg}

In this section we establish  higher order regularity and a better tightness which is needed in order
to define the resonant product $\llbracket X^2 \rrbracket \circ \phi$ in the
continuum limit. Recall that the equation {\eqref{eq:phi-eq}} satisfied by
$\phi_{M, \varepsilon}$ has the form
\begin{equation}
  \LL_{\varepsilon} \phi_{M, \varepsilon} = - 3 \lambda \llbracket X_{M,
  \varepsilon}^2 \rrbracket \succ \phi_{M, \varepsilon} + U_{M, \varepsilon},
  \label{eq:phiU}
\end{equation}
where
\[ \begin{array}{lll}
     U_{M, \varepsilon} & \assign & - 3 \lambda \llbracket X_{M, \varepsilon}^2
     \rrbracket \preccurlyeq (Y_{M, \varepsilon} + \phi_{M, \varepsilon}) - 3
     \lambda^2 b_{M, \varepsilon} (X_{M, \varepsilon} + Y_{M, \varepsilon} + \phi_{M,
     \varepsilon})\\
     &  & - 3 \lambda ( \UU^{\varepsilon}_{\leqslant} \llbracket X_{M,
     \varepsilon}^2 \rrbracket ) \succ Y_{M, \varepsilon} - 3\lambda X_{M,
     \varepsilon} (Y_{M, \varepsilon} + \phi_{M, \varepsilon})^2 -\lambda Y_{M,
     \varepsilon}^3 \\
     & & - 3\lambda Y_{M, \varepsilon}^2 \phi_{M, \varepsilon} - 3\lambda Y_{M,
     \varepsilon} \phi_{M, \varepsilon}^2 -\lambda \phi_{M, \varepsilon}^3 .
   \end{array} \]
If we let
\begin{equation}
  \chi_{M, \varepsilon} \assign \phi_{M, \varepsilon} + 3\lambda \ttwoone{X_{M,
  \varepsilon}} \succ \phi_{M, \varepsilon} \label{eq:chi1},
\end{equation}
we obtain by the commutator lemma, Lemma~\ref{lem:comm1},
\begin{equation*}
  \begin{aligned}
    3\lambda \llbracket X_{M, \varepsilon}^2 \rrbracket \circ \phi_{M, \varepsilon} +
    3\lambda^2 b_{M, \varepsilon} \phi_{M, \varepsilon} & =  - 3\lambda \llbracket X_{M,
    \varepsilon}^2 \rrbracket \circ (3\lambda \ttwoone{X_{M, \varepsilon}} \succ
    \phi_{{M, \varepsilon} }) + 3\lambda^2 b_{M, \varepsilon} \phi_{M, \varepsilon} 
    \\ & \quad + 3\lambda \llbracket X_{M,
    \varepsilon}^2 \rrbracket \circ \chi_{M, \varepsilon}\\
    & =  -\lambda^2 \ttwothreer{\widetilde{X_{}}_{M, \varepsilon}} \phi_{M,
    \varepsilon} + 3\lambda^2 (b_{M, \varepsilon} - \tilde{b}_{M, \varepsilon} (t))
    \phi_{M, \varepsilon}\\
    &  \quad +\lambda^2 C_{\varepsilon} (\phi_{M, \varepsilon}, - 3 \ttwoone{X_{M,
    \varepsilon}}, 3 \llbracket X_{M, \varepsilon}^2 \rrbracket) + 3\lambda
    \llbracket X_{M, \varepsilon}^2 \rrbracket \circ \chi_{M, \varepsilon} .
  \end{aligned} %\label{eq:res-prod}
\end{equation*}
Recalling that $Z_{M, \varepsilon} = - 3\lambda^{-1} \llbracket X_{M, \varepsilon}^2
\rrbracket \circ Y_{M, \varepsilon} - 3 b_{M, \varepsilon} (X_{M, \varepsilon}
+ Y_{M, \varepsilon})$ can be rewritten as {\eqref{eq:def-Z}} and controlled
due to Lemma~\ref{lem:Z}, where we also estimated $X_{M, \varepsilon} Y_{M,
\varepsilon}$ and $X_{M, \varepsilon} Y^2_{M, \varepsilon}$, we deduce
\[ \begin{array}{lll}
     U_{M, \varepsilon} & = & -\lambda^2 \ttwothreer{\widetilde{X_{}}_{M, \varepsilon}}
     \phi_{M, \varepsilon} + 3\lambda^2 (b_{M, \varepsilon} - \tilde{b}_{M,
     \varepsilon} (t)) \phi_{M, \varepsilon} +\lambda^2  C_{\varepsilon} (\phi_{M,
     \varepsilon}, - 3 \ttwoone{X_{M, \varepsilon}}, 3 \llbracket X_{M,
     \varepsilon}^2 \rrbracket) 
     \\ & &  + 3\lambda \llbracket X_{M, \varepsilon}^2 \rrbracket
     \circ \chi_{M, \varepsilon}
     \\
     &  & +\lambda^2 Z_{M, \varepsilon} - 3\lambda \llbracket X_{M, \varepsilon}^2 \rrbracket \prec (Y_{M,
     \varepsilon} + \phi_{M, \varepsilon}) - 3\lambda (
     \UU^{\varepsilon}_{\leqslant} \llbracket X_{M, \varepsilon}^2 \rrbracket
     ) \succ Y_{M, \varepsilon} - 3\lambda X_{M, \varepsilon} Y^2_{M,
     \varepsilon} \\
     &  & - 6\lambda X_{M, \varepsilon} Y_{M, \varepsilon} \phi_{M,
     \varepsilon} - 3\lambda X_{M, \varepsilon} \phi_{M, \varepsilon}^2 -\lambda Y_{M,
     \varepsilon}^3 - 3\lambda Y_{M, \varepsilon}^2 \phi_{M, \varepsilon} - 3\lambda Y_{M,
     \varepsilon} \phi_{M, \varepsilon}^2 -\lambda \phi_{M, \varepsilon}^3 .
   \end{array} \]
Consequently, the equation satisfied by $\chi_{M, \varepsilon}$ reads
\begin{equation}
  \begin{array}{lll}
    \LL_{\varepsilon} \chi_{M, \varepsilon} & = & \LL_{\varepsilon} \phi_{M,
    \varepsilon} + 3\lambda \llbracket X_{M, \varepsilon}^2 \rrbracket \succ \phi_{M,
    \varepsilon} + 3 \lambda\ttwoone{X_{M, \varepsilon}} \succ \LL_{\varepsilon}
    \phi_{M, \varepsilon} - 6\lambda \nabla_{\varepsilon} \ttwoone{X_{M,
    \varepsilon}} \succ \nabla_{\varepsilon} \phi_{M, \varepsilon}\\
    & = & U_{M, \varepsilon} + 3\lambda \ttwoone{X_{M, \varepsilon}} \succ
    \LL_{\varepsilon} \phi - 6\lambda \nabla_{\varepsilon} \ttwoone{X_{M,
    \varepsilon}} \succ \nabla_{\varepsilon} \phi_{M, \varepsilon}\\
    & = & U_{M, \varepsilon} + 3\lambda \ttwoone{X_{M, \varepsilon}} \succ (- 3\lambda
    \llbracket X_{M, \varepsilon}^2 \rrbracket \succ \phi_{M, \varepsilon} +
    U_{M, \varepsilon}) - 6\lambda \nabla_{\varepsilon} \ttwoone{X_{M, \varepsilon}}
    \succ \nabla_{\varepsilon} \phi_{M, \varepsilon},
  \end{array} \label{eq:chi11}
\end{equation}
where the bilinear form $\nabla_{\varepsilon} f \prec \nabla_{\varepsilon} g$
is defined by
\[ \nabla_{\varepsilon} f \prec \nabla_{\varepsilon} g \assign \frac{1}{2}
   (\Delta_{\varepsilon} (f \prec g) - \Delta_{\varepsilon} f \prec g - f
   \prec \Delta_{\varepsilon} g) \]
and can be controlled as in the proof of Lemma~\ref{lem:comm1}.

Next, we state a regularity result for $\chi_{M,\varepsilon}$, proof of which is postponed to Appendix \ref{s:chi-reg}. While it is in principle possible to keep track of the exact dependence of the bounds on $\lambda$ we do not pursue it any further since there seems to be no interesting application of such bounds. Nevertheless, it can be checked that the bounds in this section remain uniform over $\lambda$ belonging to any bounded subset of $[0,\infty)$.

\begin{proposition}
  \label{prop:reg}Let $\rho$ be a weight such that $\rho^{\iota} \in L^{4, 0}$
  for some $\iota \in (0, 1)$. Let $\phi_{M, \varepsilon}$ be a solution to
  {\eqref{eq:phiU}} and let $\chi_{M, \varepsilon}$ be given by
  {\eqref{eq:chi1}}. Then
  \[ \| \rho^4 \chi_{M, \varepsilon} \|_{L^1_T B_{1, 1}^{1 + 3 \kappa,
     \varepsilon}} \leqslant C_{T,m^2,\lambda} Q_{\rho} (\mathbb{X}_{M,\varepsilon}) (1+\| \rho^2 \phi_{
     M,\varepsilon} (0)\|_{L^{2, \varepsilon}}).\]
\end{proposition}



We apply this result in order to
deduce tightness of the sequence $(\varphi_{M, \varepsilon})_{M, \varepsilon}$
as time-dependent stochastic processes. In other words, in contrast to
Theorem~\ref{thm:tight}, where we only proved tightness for a fixed time $t
\geqslant 0$, it is necessary to establish uniform time regularity of
$(\varphi_{M, \varepsilon})_{M, \varepsilon}$. To this end, we recall the
decompositions \[\varphi_{M, \varepsilon} = X_{M, \varepsilon} + Y_{M,
\varepsilon} + \phi_{M, \varepsilon} =
X_{M, \varepsilon} - \lambda \tthreeone{X}_{M, \varepsilon} + \zeta_{M, \varepsilon}\]
with
\begin{equation}
  \zeta_{M, \varepsilon} = Y_{M, \varepsilon} + \lambda \tthreeone{X_{M, \varepsilon}}
  + \phi_{M, \varepsilon} = - \LL_{\varepsilon}^{- 1} [3\lambda
  (\UU_{>}^{\varepsilon} \llbracket X^2_{M, \varepsilon} \rrbracket \succ
  Y_{M, \varepsilon}] + \phi_{M, \varepsilon}. \label{eq:24z}
\end{equation}
 

\begin{theorem}
  \label{thm:phitight}Let $\beta \in (0, 1 / 4)$. Then  for
  all $p \in [1, \infty)$ and $\tau \in (0, T)$
  \[ \sup_{\varepsilon \in \mathcal{A}, M > 0} \mathbb{E} \| \varphi_{M,
     \varepsilon} \|^{2 p}_{W^{\beta, 1}_T B_{1, 1}^{- 1 - 3 \kappa,\varepsilon} (\rho^{4
     + \sigma})} + \sup_{\varepsilon \in \mathcal{A}, M > 0} \mathbb{E} \|
     \varphi_{M, \varepsilon} \|^{2 p}_{L^{\infty}_{\tau, T} H^{- 1 / 2 -2
     \kappa,\varepsilon} (\rho^2)} \leqslant C_\lambda < \infty, \]
  where $L^{\infty}_{\tau, T} H^{- 1 / 2 - 2\kappa,\varepsilon} (\rho^2) = L^{\infty}
  (\tau, T ; H^{- 1 / 2 - 2\kappa,\varepsilon} (\rho^2))$.
\end{theorem}

\begin{proof}
  Let us begin with the first bound. According to Proposition~\ref{prop:reg} and
  Theorem~\ref{thm:tight} we obtain that
  \[ \mathbb{E} \| \chi_{M, \varepsilon} \|^{2 p}_{L^1_T B_{1, 1}^{1 + 3
     \kappa, \varepsilon} (\rho^4)} \leqslant C_{T,\lambda} \mathbb{E}Q_{\rho}
     (\mathbb{X}_{M, \varepsilon}) (1 +\mathbb{E}\| \rho^2 \phi_{M,
     \varepsilon} (0)\|^{2 p}_{L^{2, \varepsilon}}) \]
  \[ \leqslant C_{T,\lambda} \mathbb{E}Q_{\rho} (\mathbb{X}_{\varepsilon}) (1
     +\mathbb{E}\| \rho^2 (\varphi_{M, \varepsilon} (0) - X_{M, \varepsilon}
     (0)) \|^{2 p}_{L^{2, \varepsilon}} +\mathbb{E}\| \rho^2 Y_{M,
     \varepsilon} (0)\|^{2 p}_{L^{2, \varepsilon}}) \]
  is bounded uniformly in $M, \varepsilon$. In addition, the computations in
  the proof of Proposition~\ref{prop:reg} imply that also $\mathbb{E} \left\|
  \LL_{\varepsilon} \chi_{M, \varepsilon} \right\|^{2 p}_{L_T^1 B_{1, 1}^{- 1
  + 3 \kappa, \varepsilon} (\rho^4)}$ is bounded uniformly in $M,
  \varepsilon$. As a consequence, we deduce that
  \[ \mathbb{E} \| \partial_t \chi_{M, \varepsilon} \|^{2 p}_{L^1_T B^{- 1 + 3
     \kappa, \varepsilon}_{1, 1} (\rho^4)} \leqslant \mathbb{E} \|
     (\Delta_{\varepsilon} - m^2) \chi_{M, \varepsilon} \|^{2 p}_{L^1_T B^{- 1
     + 3 \kappa, \varepsilon}_{1, 1} (\rho^4)} +\mathbb{E} \left\|
     \LL_{\varepsilon} \chi_{M, \varepsilon} \right\|^{2 p}_{L^1_T B^{- 1 + 3
     \kappa, \varepsilon}_{1, 1} (\rho^4)} \]
  is also bounded uniformly in $M, \varepsilon$.
  
  Next, we apply a similar approach to derive uniform time regularity of
  $\phi_{M, \varepsilon}$. To this end, we study the right hand side of
  {\eqref{eq:phiU}}. Observe that due to the energy estimate from Theorem~\ref{th:energy-estimate} and the bound from Proposition~\ref{prop:reg} together
  with Theorem~\ref{thm:tight} the following are bounded uniformly in $M,
  \varepsilon$
  \[ \mathbb{E} \| \llbracket X_{M, \varepsilon}^2 \rrbracket \succ \phi_{M,
     \varepsilon} \|^{2 p}_{L^2_T H^{- 1 - \kappa, \varepsilon} (\rho^{2 +
     \sigma})}, \hspace{1em} \mathbb{E} \| \llbracket X_{M, \varepsilon}^2
     \rrbracket \circ \chi_{M, \varepsilon} \|^{2 p}_{L^1_T B^{2 \kappa,\varepsilon}_{1,
     1} (\rho^{4 + \sigma})}, \]
  whereas all the other terms on the right hand side of {\eqref{eq:phiU}} are
  uniformly bounded in better function spaces. Hence we deduce that
  \[ \mathbb{E} \| \partial_t \phi_{M, \varepsilon} \|^{2 p}_{L_T^1 B^{- 1 - 3
     \kappa, \varepsilon}_{1, 1} (\rho^{4 + \sigma})} \leqslant \mathbb{E} \|
     (\Delta_{\varepsilon} - m^2) \phi_{M, \varepsilon} \|^{2 p}_{L^1_T B^{- 1
     - 3 \kappa, \varepsilon}_{1, 1} (\rho^{4 + \sigma})} +\mathbb{E} \left\|
     \LL_{\varepsilon} \phi_{M, \varepsilon} \right\|^{2 p}_{L^1_T B^{- 1 - 3
     \kappa, \varepsilon}_{1, 1} (\rho^{4 + \sigma})} \]
  is bounded uniformly in $M, \varepsilon$.
  
  Now we have all in hand to derive a uniform time regularity of $\zeta_{M,
  \varepsilon}$. Using Schauder estimates together with {\eqref{eq:24z}} it
  holds that
  \[ \mathbb{E} \| \zeta_{M, \varepsilon} \|^{2 p}_{W^{(1 - 2 \kappa) / 2,
     1}_T B_{1, 1}^{- 1 - 3 \kappa, \varepsilon} (\rho^{4 + \sigma})}
     \leqslant \mathbb{E} \left\| \LL_{\varepsilon}^{- 1} [3\lambda
     (\UU_{>}^{\varepsilon} \llbracket X^2_{M, \varepsilon} \rrbracket \succ
     Y_{M, \varepsilon}] \right\|^{2 p}_{C^{(1 - \kappa) / 2}_T L^{\infty,\varepsilon}
     (\rho^{\sigma})} \]
  \[ +\mathbb{E} \| \phi_{M, \varepsilon} \|^{2 p}_{W_T^{1, 1} B^{- 1 - 3
     \kappa, \varepsilon}_{1, 1} (\rho^{4 + \sigma})} \]
  is bounded uniformly in $M, \varepsilon$.
  
  Finally, since for all $\beta \in (0, 1)$ we have that both
  \[ \mathbb{E} \| X_{M, \varepsilon} \|^{2 p}_{C^{\beta}_T \CC^{- 1 / 2 -
     \kappa - 2 \beta,\varepsilon} (\rho^{\sigma})}, \hspace{1em} \mathbb{E} \|
     \tthreeone{X}_{M, \varepsilon} \|^{2 p}_{C^{\beta}_T \CC^{1 / 2 -
     \kappa - 2 \beta,\varepsilon} (\rho^{\sigma})} \]
  are bounded uniformly in $M, \varepsilon$, we conclude that so is
  $\mathbb{E} \| \varphi_{M, \varepsilon} \|^{2 p}_{W^{\beta, 1}_T B_{1, 1}^{-
  1 - 3 \kappa,\varepsilon} (\rho^{4 + \sigma})}$ for $\beta \in (0, 1 / 4)$, which
  completes the proof of the first bound.
  
  In order to establish the second bound we recall the decomposition
  $\varphi_{M, \varepsilon} = X_{M, \varepsilon} + Y_{M, \varepsilon} +
  \phi_{M, \varepsilon}$ and make use of the energy estimate from Corollary~\ref{cor:Lp}. Taking supremum over $t \in [\tau, T]$ and expectation implies
  \[ \sup_{\varepsilon \in \mathcal{A}, M > 0} \mathbb{E} \| \phi_{M,
     \varepsilon} \|^{2 p}_{L^{\infty}_{\tau, T} L^{2,\varepsilon} (\rho^2)} < \infty . \]
  The claim now follows using the bound for $X_{M, \varepsilon}$ together with
  the bound for $Y_{M, \varepsilon}$ in Lemma~\ref{lem:Y1}.
\end{proof}


Even though the uniform bound in the previous result is far from being
  optimal, it is sufficient for our purposes below.


\begin{corollary}
  \label{cor:t}Let $\rho$ be a weight such that $\rho^{\iota} \in L^4$ for
  some $\iota \in (0, 1)$. Let $\beta \in (0, 1 / 4)$ and $\alpha \in (0,
  \beta)$. Then the family of joint laws of $(\mathcal{E}^{\varepsilon}
  \varphi_{M, \varepsilon}, \mathcal{E}^{\varepsilon} \mathbb{X}_{M,
  \varepsilon})$ is tight on $W^{\alpha, 1}_{\tmop{loc}} B_{1, 1}^{- 1 - 4
  \kappa} (\rho^{4 + \sigma}) \times C^{\kappa / 2}_{\tmop{loc}}
  \mathcal{X}^{}$, where
  \[ \mathcal{X} \assign \prod_{i = 1, \ldots, 7} \CC^{\alpha (i) - \kappa}
     (\rho^{\sigma}) \]
  with $\alpha (1) = \alpha (7) = - 1 / 2,$ $\alpha (2) = - 1,$ $\alpha (3) =
  1 / 2,$ $\alpha (4) = \alpha (5) = \alpha (6) = 0$.
\end{corollary}

\begin{proof}
  According to Theorem 6.31 in {\cite{T06}} we have the compact embedding
  \[ B_{1, 1}^{- 1 - 3 \kappa} (\rho^{4 + \sigma}) \subset B_{1, 1}^{- 1 - 4
     \kappa} (\rho^{4 + 2 \sigma}) \]
  and consequently since $\alpha < \beta$ the embedding
  \[ W^{\beta, 1}_{\tmop{loc}} B_{1, 1}^{- 1 - 3 \kappa} (\rho^{4 + \sigma})
     \subset W^{\alpha, 1}_{\tmop{loc}} B_{1, 1}^{- 1 - 4 \kappa} (\rho^{4 + 2
     \sigma}) \]
  is compact, see e.g. Theorem 5.1 {\cite{A00}}. Hence the desired tightness
  of $\mathcal{E}^{\varepsilon} \varphi_{M, \varepsilon}$ follows from Theorem~\ref{thm:phitight} and Lemma~\ref{lem:ext}. The tightness of
  $\mathcal{E}^{\varepsilon} \mathbb{X}_{M, \varepsilon}$ follows from the
  usual arguments and does not pose any problems.
\end{proof}

As a consequence, we may extract a converging subsequence of the joint laws of
the  processes $(\mathcal{E}^{\varepsilon} \varphi_{M, \varepsilon},
\mathcal{E}^{\varepsilon} \mathbb{X}_{M, \varepsilon})_{M, \varepsilon}$ in
$W^{\alpha, 1}_{\tmop{loc}} B_{1, 1}^{- 1 - 4 \kappa} (\rho^{4 + \sigma})
\times C^{\kappa / 2}_{\tmop{loc}} \mathcal{X}^{}$. Let $\hat{\mu}$ denote any
limit point. We recall that $\mathbb{X}_{M,\varepsilon}$ denotes the collection of all the necessary stochastic objects, see \eqref{eq:XX}. We denote by $(\varphi, \mathbb{X})$ the canonical process on
$W^{\alpha, 1}_{\tmop{loc}} B_{1, 1}^{- 1 - 4 \kappa} (\rho^{4 + \sigma})
\times C^{\kappa / 2}_{\tmop{loc}} \mathcal{X}^{}$  and let $\mu$ be the law
of the pair $(\varphi, X)$ under $\hat{\mu}$ (i.e. the projection of $\hat{\mu}$
to the first two components). Observe that there exists a measurable map $\Psi :
(\varphi, X) \mapsto (\varphi, \mathbb{X})$ such that $\hat{\mu} = \mu \circ
\Psi^{- 1}$. Therefore we can represent expectations under $\hat{\mu}$ as
expectations under $\mu$ with the understanding that the elements of
$\mathbb{X}$ are constructed canonically from $X$ via $\Psi$. Furthermore, $Y,\phi,\zeta,\chi$ are defined analogously as on the approximate level as measurable functions of the pair $(\varphi,X)$. In particular, the limit localizer $\UU_{>}$ is determined by the constant $L_{0}$ obtained in Lemma \ref{lem:Y1}. Consequently, all the above uniform estimates are preserved for the limiting measure and the convergence of the corresponding lattice approximations to $Y,\phi,\zeta,\chi$ follows. In addition, the limiting process $\varphi$ is stationary in the following distributional sense: for all $f\in C^{\infty}_{c}(\mathbb{R}_{+})$ and all $\tau>0$, the laws of
$$
\varphi(f)\quad\text{and}\quad \varphi(f(\cdot-\tau))\quad\text{on}\quad \mathcal{S}'(\mathbb{R}^{3})
$$
coincide. Based on the time regularity of $\varphi$ it can be shown that this implies that the laws of $\varphi(t)$ and $\varphi(t+\tau)$ coincide for all $\tau>0$ and a.e. $t\in[0,\infty)$. The projection of $\mu$ on $\varphi(t)$ taken from this set of full measure is the measure $\nu$ as obtained in Theorem~\ref{thm:main}.



\subsection{Integration by parts formula}

The goal of his section is to derive an integration by parts formula for the
$\Phi^4_3$ measure on the full space. To this end, we begin with the
corresponding integration by parts formula on the approximate level, that is,
for the measures $\nu_{M, \varepsilon}$ and pass to the limit.

Let $F$ be a cylinder functional on $\mathcal{S}' (\mathbb{R}^3)$, that is, $F
(\varphi) = \Phi (\varphi (f_1), \ldots, \varphi (f_n))$ for some polynomial $\Phi :
\mathbb{R}^n \rightarrow \mathbb{R}$ and $f_1, \ldots, f_n \in \mathcal{S}
(\mathbb{R}^3)$. Let $\mathD F (\varphi)$ denote the $L^2$-gradient of $F$.
Then  for fields $\varphi_{\varepsilon}$ defined on
$\Lambda_{\varepsilon}$ we have
\[ \frac{\partial F (\mathcal{E}^{\varepsilon}
   \varphi_{\varepsilon})}{\partial \varphi_{\varepsilon} (x)} = \varepsilon^d
   \sum_{i = 1}^n \partial_i \Phi ((\mathcal{E}^{\varepsilon}
   \varphi_{\varepsilon}) (f_1), \ldots, (\mathcal{E}^{\varepsilon}
   \varphi_{\varepsilon}) (f_n)) (w_{\varepsilon} \ast f_i) (x) =
   \varepsilon^d [w_{\varepsilon} \ast \mathD F (\mathcal{E}^{\varepsilon}
   \varphi_{\varepsilon})] (x), \]
where $x\in \Lambda_{\varepsilon}$ and $w_{\varepsilon}$ is the kernel involved in the definition of the
extension operator $\mathcal{E}^{\varepsilon}$ from Section~\ref{s:ext}. By integration by parts it
follows that
\[ \int [w_{\varepsilon} \ast \mathD F (\mathcal{E}^{\varepsilon} \varphi)]
   (x) \nu_{M, \varepsilon} (\mathd \varphi) = \frac{1}{\varepsilon^d} \int
   \frac{\partial F (\mathcal{E}^{\varepsilon} \varphi)}{\partial \varphi (x)}
   \nu_{M, \varepsilon} (\mathd \varphi) = \frac{2}{\varepsilon^d} \int F
   (\mathcal{E}^{\varepsilon} \varphi) \frac{\partial V_{M, \varepsilon}
   (\varphi)}{\partial \varphi (x)} \nu_{M, \varepsilon} (\mathd \varphi) \]
\begin{equation}
  = 2 \int F (\mathcal{E}^{\varepsilon} \varphi) [\lambda \varphi (x)^3 + (- 3\lambda a_{M,
  \varepsilon} + 3\lambda^2 b_{M, \varepsilon}) \varphi (x)] \nu_{M, \varepsilon}
  (\mathd \varphi) + 2 \int F (\mathcal{E}^{\varepsilon} \varphi) [m^2 -
  \Delta_{\varepsilon}] \varphi (x) \nu_{M, \varepsilon} (\mathd \varphi) .
  \label{eq:ibp1}
\end{equation}
According to Theorem~\ref{thm:main}, we can already pass to the limit on the
left hand side as well as in the second term on the right hand side of
{\eqref{eq:ibp1}}. Namely, we obtain for any accumulation point $\nu$ and any
(relabeled) subsequence $(\nu_{M, \varepsilon} \circ
(\mathcal{E}^{\varepsilon})^{- 1})_{M, \varepsilon}$ converging to $\nu$ that
the following convergences hold in the sense of distributions in the variable
$x \in \mathbb{R}^3$
\[ \int \mathcal{E}^{\varepsilon} [w_{\varepsilon} \ast \mathD F
   (\mathcal{E}^{\varepsilon} \varphi)] (x) \nu_{M, \varepsilon} (\mathd
   \varphi) \rightarrow \int \mathD F (\mathcal{E}^{\varepsilon} \varphi) (x)
   \nu (\mathd \varphi), \]
\[ \int F (\mathcal{E}^{\varepsilon} \varphi) \mathcal{E}^{\varepsilon} [m^2 -
   \Delta_{\varepsilon}] \varphi (x) \nu_{M, \varepsilon} (\mathd \varphi)
   \rightarrow \int F (\varphi) [m^2 - \Delta] \varphi (x) \nu (\mathd
   \varphi) . \]
The remainder of this section is devoted to the passage to the limit in
{\eqref{eq:ibp1}}, leading to the integration by parts formula for the
limiting measure in Theorem~\ref{thm:ibp} below. In particular, it is
necessary to find a way to control the convergence of the cubic term and to
interpret the limit under the $\Phi^4_3$ measure.

Let us denote
\[ \llbracket \varphi^3 \rrbracket_{M, \varepsilon} (y) \assign \varphi (y)^3
   + (- 3 a_{M, \varepsilon} + 3\lambda b_{M, \varepsilon}) \varphi (y) . \]
We shall analyze carefully the distributions $\mathcal{J}_{M, \varepsilon} (F)
\in \mathcal{S}' (\Lambda_{\varepsilon})$ given by
\[ \mathcal{J}_{M, \varepsilon} (F) \assign x \mapsto \int F
   (\mathcal{E}^{\varepsilon} \varphi) \llbracket \varphi^3 \rrbracket_{M,
   \varepsilon} (x) \nu_{M, \varepsilon} (\mathd \varphi), \]
in order to determine the limit of $\mathcal{E}^{\varepsilon} \mathcal{J}_{M,
\varepsilon} (F)$ (as a distribution in $x \in \mathbb{R}^3$) as $(M,
\varepsilon) \rightarrow (\infty, 0)$. Unfortunately, even for the Gaussian
case when $\lambda = 0$ one cannot give a well-defined meaning to the random
variable $\varphi^3$ under the measure $\nu$. Additive renormalization is not
enough to cure this problem since it is easy to see that the variance of the
putative Wick renormalized limiting field
\[ \llbracket \varphi^3 \rrbracket = \lim_{\varepsilon \rightarrow 0, M
   \rightarrow \infty} \mathcal{E}^{\varepsilon} \llbracket \varphi^3
   \rrbracket_{M, \varepsilon} \]
is infinite. In the best of the cases one can hope that the renormalized cube
$\llbracket \varphi^3 \rrbracket$ makes sense once integrated against smooth
cylinder functions $F (\varphi)$. Otherwise stated, one could try to prove
that $(\mathcal{J}_{M, \varepsilon})_{M, \varepsilon}$ converges as a linear
functional on cylinder test functions over $\mathcal{S}' (\mathbb{R}^3)$.

To this end, we work with the stationary solution $\varphi_{M, \varepsilon}$
and introduce the additional notation
\[ \llbracket \varphi_{M, \varepsilon}^3 \rrbracket (t, y) \assign \varphi_{M,
   \varepsilon} (t, y)^3 + (- 3 a_{M, \varepsilon} + 3\lambda b_{M, \varepsilon})
   \varphi_{M, \varepsilon} (t, y) . \]
As the next step, we employ the decomposition
\[ \varphi_{M, \varepsilon} = X_{M, \varepsilon} -\lambda \tthreeone{X_{M,
   \varepsilon}} + \zeta_{M, \varepsilon} \]
in order to find a decomposition that can be controlled by our estimates. We
rewrite
\[ \begin{array}{lll}
     \llbracket \varphi_{M, \varepsilon}^3 \rrbracket & = & \llbracket X_{M,
     \varepsilon}^3 \rrbracket + 3 \llbracket X_{M, \varepsilon}^2 \rrbracket
     (-\lambda \tthreeone{X_{M, \varepsilon}} + \zeta_{M, \varepsilon}) + 3\lambda b_{M,
     \varepsilon} \varphi_{M, \varepsilon}\\
     &  & + 3 X_{M, \varepsilon} (- \lambda\tthreeone{X_{M, \varepsilon}} +
     \zeta_{M, \varepsilon})^2 + (-\lambda \tthreeone{X_{M, \varepsilon}} + \zeta_{M,
     \varepsilon})^3 .
   \end{array} \]
Next, we use the paraproducts and paracontrolled ansatz to control the various
resonant products. For the renormalized resonant product $3 \llbracket X_{M,
\varepsilon}^2 \rrbracket \circ (-\lambda \tthreeone{X_{M, \varepsilon}} + \zeta_{M,
\varepsilon}) + 3\lambda b_{M, \varepsilon} \varphi_{M, \varepsilon}$ we first recall
that
\[ \varphi_{M, \varepsilon} = X_{M, \varepsilon} + Y_{M, \varepsilon} +
   \phi_{M, \varepsilon}, \qquad \phi_{M, \varepsilon} = - 3\lambda \ttwoone{X_{M,
   \varepsilon}} \succ \phi_{M, \varepsilon} + \chi_{M, \varepsilon} . \]
Therefore using the definition of $Z_{M, \varepsilon}$ in~{\eqref{eq:def-Z}}
we have
\[ \begin{array}{lll}
     3 \llbracket X_{M, \varepsilon}^2 \rrbracket \circ (-\lambda \tthreeone{X_{M,
     \varepsilon}} + \zeta_{M, \varepsilon}) + 3\lambda b_{M, \varepsilon}
     \varphi_{M, \varepsilon} & = & 3 \llbracket X_{M, \varepsilon}^2
     \rrbracket \circ (Y_{M, \varepsilon} + \phi_{M, \varepsilon}) + 3\lambda b_{M,
     \varepsilon} \varphi_{M, \varepsilon}\\
     & = & \underbrace{3 \llbracket X_{M, \varepsilon}^2 \rrbracket \circ
     Y_{M, \varepsilon} + 3\lambda b_{M, \varepsilon} (X_{M, \varepsilon} + Y_{M,
     \varepsilon})}_{- \lambda Z_{M, \varepsilon}}\\
     &  & + 3 \llbracket X_{M, \varepsilon}^2 \rrbracket \circ \phi_{M,
     \varepsilon} + 3\lambda b_{M, \varepsilon} \phi_{M, \varepsilon}
   \end{array} \]
and
\[ 3 \llbracket X_{M, \varepsilon}^2 \rrbracket \circ \phi_{M, \varepsilon} +
   3\lambda b_{M, \varepsilon} \phi_{M, \varepsilon} = 3 \llbracket X_{M,
   \varepsilon}^2 \rrbracket \circ ( - 3\lambda \ttwoone{X_{M, \varepsilon}}
   \succ \phi_{M, \varepsilon} ) + 3\lambda b_{M, \varepsilon} \phi_{M,
   \varepsilon} + 3 \llbracket X_{M, \varepsilon}^2 \rrbracket \circ \chi_{M,
   \varepsilon} \]
\[ = - \lambda\ttwothreer{\tilde{X}_{M, \varepsilon}} \phi_{M, \varepsilon} + 3\lambda
   (b_{M, \varepsilon} - \tilde{b}_{M, \varepsilon} (t)) \phi_{M, \varepsilon}
   +\lambda C_{\varepsilon} (\phi_{M, \varepsilon}, - 3 \ttwoone{X_{M, \varepsilon}},
   3 \llbracket X_{M, \varepsilon}^2 \rrbracket) + 3 \llbracket X_{M,
   \varepsilon}^2 \rrbracket \circ \chi_{M, \varepsilon} . \]
The remaining resonant product that requires a decomposition can be treated as
\[ \begin{array}{ccl}
     3 X_{M, \varepsilon} \circ (-\lambda \tthreeone{X_{M, \varepsilon}} + \zeta_{M,
     \varepsilon})^2 & = & 3 \lambda^2 X_{M, \varepsilon} \circ (\tthreeone{X_{M,
     \varepsilon}})^2 - 6\lambda X_{M, \varepsilon} \circ ( \tthreeone{X}_{M,
     \varepsilon} \zeta_{M, \varepsilon} ) + 3 X_{M, \varepsilon} \circ
     \zeta_{M, \varepsilon}^2\\
     & = & 6\lambda^2 X_{M, \varepsilon} \circ (\tthreeone{X_{M, \varepsilon}} \succ
     \tthreeone{X_{M, \varepsilon}}) + 3\lambda^2 X_{M, \varepsilon} \circ (
     \tthreeone{X_{M, \varepsilon}} \circ \tthreeone{X_{M, \varepsilon}}
     )\\
     &  & - 6\lambda X_{M, \varepsilon} \circ ( \tthreeone{X_{M, \varepsilon}}
     \succ \zeta_{M, \varepsilon} ) - 6\lambda X_{M, \varepsilon} \circ (
     \tthreeone{X_{M, \varepsilon}} \preccurlyeq \zeta_{M, \varepsilon}
     ) \\
     & &+ 3 X_{M, \varepsilon} \circ \zeta_{M, \varepsilon}^2\\
     & = & 6\lambda (\lambda\tthreeone{X_{M, \varepsilon}} - \zeta_{M, \varepsilon})
     \tthreetwor{X_{M, \varepsilon}} + 6\lambda C_{\varepsilon} (\lambda
     \tthreeone{X_{M, \varepsilon}} - \zeta_{M, \varepsilon}, \tthreeone{X_{M,
     \varepsilon}}, X_{M, \varepsilon} )\\
     &  & + 3\lambda^2 X_{M, \varepsilon} \circ ( \tthreeone{X_{M, \varepsilon}}
     \circ \tthreeone{X_{M, \varepsilon}} ) - 6\lambda X_{M, \varepsilon} \circ
     ( \tthreeone{X_{M, \varepsilon}} \preccurlyeq \zeta_{M, \varepsilon}
     ) \\
     & &+ 3 X_{M, \varepsilon} \circ \zeta_{M, \varepsilon}^2,
   \end{array} \]
where we used the notation $f \preccurlyeq g = f \prec g + f \circ g$.

These decompositions and our estimates show that the products are all are
controlled in the space $L^1 (0, T, B^{- 1 - 3 \kappa, \varepsilon}_{1, 1}
(\rho^{4 + \sigma}))$. The term $\llbracket X^3_{M, \varepsilon} \rrbracket$
requires some care since it cannot be defined as a function of $t$. Indeed,
standard computations show that $\mathcal{E}^{\varepsilon} \llbracket X^3_{M,
\varepsilon} \rrbracket \rightarrow \llbracket X^3 \rrbracket$ in $W^{-
\kappa, \infty}_T \CC^{- 3 / 2 - \kappa, \varepsilon} (\rho^{\sigma})$,
namely, it requires just a mild regularization in time to be well defined and
it is the only one among the contributions to $\llbracket \varphi_{M,
\varepsilon}^3 \rrbracket$ which has negative time regularity. In particular,
we may write $\llbracket \varphi_{M, \varepsilon}^3 \rrbracket = \llbracket
X_{M, \varepsilon}^3 \rrbracket + H_{\varepsilon} (\varphi_{M, \varepsilon},
\mathbb{X}_{M, \varepsilon})$ where for $p \in [1, \infty)$
\[ \sup_{\varepsilon \in \mathcal{A}, M > 0} \mathbb{E} \| \llbracket X_{M,
   \varepsilon}^3 \rrbracket \|^{2 p}_{W^{- \kappa, \infty}_T \CC^{- 3 / 2 -
   \kappa, \varepsilon} (\rho^{\sigma})} + \sup_{\varepsilon \in \mathcal{A},
   M > 0} \mathbb{E} \| H_{\varepsilon} (\varphi_{M, \varepsilon},
   \mathbb{X}_{M, \varepsilon}) \|^{2 p}_{L^1_T B^{- 1 - 3 \kappa,
   \varepsilon}_{1, 1} (\rho^{4 + \sigma})} < \infty \]
is uniformly bounded in $M, \varepsilon$. The dependence of the function $H_{\varepsilon}$ on $\varepsilon$ comes from the corresponding dependence of the paraproducts as well as the resonant product on $\varepsilon$.

Now, let $h : \mathbb{R} \rightarrow \mathbb{R}$ be a smooth test function
with $\tmop{supp} h \subset [\tau, T]$ for some $0 < \tau < T < \infty$ and
such that $\int_{\mathbb{R}} h (t) \mathd t = 1$. Then by stationarity we can
rewrite the Littlewood--Paley blocks $\Delta^{\varepsilon}_j \mathcal{J}_{M,
\varepsilon} (F)$ as
\[ \Delta^{\varepsilon}_j \mathcal{J}_{M, \varepsilon} (F) = \int_{\mathbb{R}}
   h (t) \mathbb{E} [F (\mathcal{E}^{\varepsilon} \varphi_{M, \varepsilon}
   (t)) \Delta_j^{\varepsilon} \llbracket \varphi_{M, \varepsilon}^3 (t)
   \rrbracket_{M, \varepsilon}] \mathd t \]
\[ =\mathbb{E} \left[ \int_{\mathbb{R}} h (t) F (\mathcal{E}^{\varepsilon}
   \varphi_{M, \varepsilon} (t)) \Delta_j^{\varepsilon} \llbracket X_{M,
   \varepsilon}^3 \rrbracket (t) \mathd t \right] +\mathbb{E} \left[
   \int_{\mathbb{R}} h (t) F (\mathcal{E}^{\varepsilon} \varphi_{M,
   \varepsilon} (t)) \Delta_j^{\varepsilon} H_{\varepsilon} (\varphi_{M,
   \varepsilon}, \mathbb{X}_{M, \varepsilon}) (t) \mathd t \right] \]
\[ \backassign \Delta^{\varepsilon}_j \mathcal{J}^X_{M, \varepsilon} (F) +
   \Delta^{\varepsilon}_j \mathcal{J}^H_{M, \varepsilon} (F) . \]
As a consequence of Corollary~\ref{cor:t} and the discussion afterwards we
extract a subsequence converging in law and using the uniform bounds together with the $(\mathcal{E})$ property of our nonlinearities as defined on page 2073 in \cite{MP17},  we may
pass to the limit and conclude
\[ \lim_{\varepsilon \rightarrow 0, M \to \infty}
   \mathcal{E}^{\varepsilon} \mathcal{J}_{M, \varepsilon} (F)
   =\mathbb{E}_{\mu} \left[ \int_{\mathbb{R}} h (t) F (\varphi (t)) \llbracket
   \varphi^3 \rrbracket (t) \mathd t \right] \backassign \mathcal{J}_{\mu} (F)
   . \]
Here $\llbracket \varphi^3 \rrbracket$ is expressed (as $\llbracket
\varphi^3_{M, \varepsilon} \rrbracket$ before) as a measurable function of
$(\varphi, X)$ given by
\begin{equation}
  \begin{array}{lll}
    \llbracket \varphi^3 \rrbracket & \assign & \llbracket X^3 \rrbracket + 3
    \llbracket X^2 \rrbracket \Join (-\lambda \tthreeone{X} + \zeta) - \lambda Z -
  \lambda  \ttwothreer{\tilde{X}} \phi + 3\lambda B(t) \phi\\
    &  & + \lambda C (\phi, - 3 \ttwoone{X}, 3 \llbracket X^2 \rrbracket) + 3
    \llbracket X^2 \rrbracket \circ \chi + 3 X \Join (-\lambda \tthreeone{X} +
    \zeta)^2 + 6\lambda (\lambda\tthreeone{X} - \zeta) \tthreetwor{X}\\
    &  & + 6\lambda C (\lambda \tthreeone{X} - \zeta, \tthreeone{X}, X ) + 3\lambda^2 X
    \circ ( \tthreeone{X} \circ \tthreeone{X} ) - 6\lambda X \circ (
    \tthreeone{X} \preccurlyeq \zeta ) + 3 X \circ \zeta^2\\
    & &+ (-\lambda
    \tthreeone{X} + \zeta)^3,
  \end{array} \label{eq:phi3}
\end{equation}
where we used the notation $f \Join g = f \prec g + f \succ g$ and $\zeta,
\phi, Y$ are defined as starting from $(\varphi, \mathbb{X}) = \Psi (\varphi,
X)$ as
\[ \varphi = X -\lambda \tthreeone{X} + \zeta, \qquad \zeta = - \LL^{- 1} [ 3\lambda
   ( \UU_{>} \llbracket X^2 \rrbracket ) \succ Y ] + \phi ,
\]
the operator $C$ is the continuum analog of the commutator $C_{\varepsilon}$ defined in \eqref{eq:ce}, the localizer $\UU_{>}$ is given by the constant $L_{0}$ from Lemma \ref{lem:Y1} and $B(\cdot)$ (appearing also in the limit $Z$, cf. \eqref{eq:def-Z}) is the uniform  limit of $b_{M,\varepsilon}-\tilde{b}_{M,\varepsilon}(\cdot)$ on $[\tau,T]$.
Let us denote $H (\varphi, X) \assign \llbracket \varphi^3 \rrbracket - \llbracket
X^3 \rrbracket$.

\

Remark that our uniform bounds remain valid for the limiting measure $\mu$.
As a consequence we obtain the following result.

\begin{lemma}
  \label{lemma:IF}Let $F : \mathcal{S}' (\mathbb{R}^3) \rightarrow \mathbb{R}$
  be a cylinder function such that
  \[ | F (\varphi) | + \| \mathD F (\varphi) \|_{B_{\infty, \infty}^{1 + 3
     \kappa} (\rho^{- 4 - \sigma})} \leqslant C_F \| \varphi \|_{H^{- 1 / 2 -2
     \kappa} (\rho^2)}^n \]
  for some $n \in \mathbb{N}$. Let $\mu$ be an accumulation point of the
  sequence of laws of $(\mathcal{E}^{\varepsilon} \varphi_{M, \varepsilon},
  \mathcal{E}^{\varepsilon} X_{M, \varepsilon})$. Then  (along a
  subsequence)  $\mathcal{E}^{\varepsilon} \mathcal{J}_{M, \varepsilon}
  (F) \rightarrow \mathcal{J}_{\mu} (F)$ in $\mathcal{S}' (\mathbb{R}^d)$,
  where $\mathcal{J}_{\mu} (F)$ is given by
  \[ \mathcal{J}_{\mu} (F) =\mathbb{E}_{\mu} \left[ \int_{\mathbb{R}} h (t) F
     (\varphi (t)) \llbracket X^3 \rrbracket (t) \mathd t \right]
     +\mathbb{E}_{\mu} \left[ \int_{\mathbb{R}} h (t) F (\varphi (t)) H
     (\varphi, X) (t) \mathd t \right] \backassign \mathcal{J}^X_{\mu} (F)
     +\mathcal{J}^H_{\mu} (F), \]
for any function $h$ as above, which is understood as an equality of distributions and the expectation is in the weak sense. Moreover, we have the estimate
  \[ \| \mathcal{J}^X_{\mu} (F) \|_{\CC^{- 3 / 2 - \kappa} (\rho^{\sigma})} +
     \| \mathcal{J}^H_{\mu} (F) \|_{B_{1, 1}^{- 1 - 3 \kappa} (\rho^{4+\sigma})}
     \lesssim_{\mu, h} C_F \]
  where the implicit constant depends on $\mu, h$ but not on $F$.
\end{lemma}

\begin{proof}
  For any cylinder function $F$ satisfying the assumptions and since
  $\tmop{supp} h \in [\tau, T]$ we have the following estimate for arbitrary
  conjugate exponents $p, p' \in (1, \infty)$
  \[ \| \mathcal{J}^X_{\mu} (F) \|_{\CC^{- 3 / 2 - \kappa} (\rho^{\sigma})}
     \lesssim_h \mathbb{E}_{\mu} \left[ \| t \mapsto F (\varphi (t))
     \|_{W^{\kappa, 1}_T} \| \llbracket X^3 \rrbracket \|_{W^{- \kappa,
     \infty}_T \CC^{- 3 / 2 - \kappa} (\rho^{\sigma})} \right] \]
  \[ \lesssim (\mathbb{E}_{\mu} [\| t \mapsto F (\varphi (t)) \|_{W^{\kappa,
     1}_T}^p])^{1 / p}  \left( \mathbb{E}_{\mu} \left[ \| \llbracket X^3
     \rrbracket \|_{W^{- \kappa, \infty}_T \CC^{- 3 / 2 - \kappa}
     (\rho^{\sigma})}^{p'} \right] \right)^{1 / p'} \]
  \[ \lesssim (\mathbb{E}_{\mu} [\| t \mapsto F (\varphi (t)) \|_{W^{\kappa,
     1}_T}^p])^{1 / p} \lesssim \left( \int_{[0, T]^2} \frac{\mathbb{E}_{\mu}
     | F (\varphi (t)) - F (\varphi (s)) |^p}{| t - s |^{(1 + \kappa) p}}
     \mathd t \mathd s \right)^{1 / p} . \]
  Since for arbitrary conjugate exponents $q, q' \in (1, \infty)$
  \[ \mathbb{E}_{\mu} | F (\varphi (t)) - F (\varphi (s)) |^p \leqslant
     \int_0^1 \mathbb{E}_{\mu} | \langle \mathD F (\varphi (s) + \tau (\varphi
     (t) - \varphi (s))), \varphi (t) - \varphi (s) \rangle |^p \mathd \tau \]
  \[ \leqslant \int_0^1 \mathd \tau (\mathbb{E}_{\mu} \| \mathD F (\varphi (s)
     + \tau (\varphi (t) - \varphi (s))) \|^{p q'}_{B_{\infty, \infty}^{1 + 3
     \kappa} (\rho^{- 4 - \sigma})})^{1 / q'}  (\mathbb{E}_{\mu} \| \varphi
     (t) - \varphi (s) \|^{p q}_{B_{1, 1}^{- 1 - 3 \kappa} (\rho^{4 +
     \sigma})})^{1 / q} \]
  \[ \lesssim C^p_F (\mathbb{E}_{\mu} \| \varphi (0) \|_{H^{- 1 / 2 -2 \kappa}
     (\rho^2)}^{n p q'})^{1 / q'} (\mathbb{E}_{\mu} \| \varphi (t) - \varphi
     (s) \|^{p q}_{B_{1, 1}^{- 1 - 3 \kappa} (\rho^{4 + \sigma})})^{1 / q}, \]
  we obtain due to Theorem~\ref{thm:tight} that
  \[ \| \mathcal{J}^X_{\mu} (F) \|_{\CC^{- 3 / 2 - \kappa} (\rho^{\sigma})}
     \lesssim C_F \left( \int_{[0, T]^2} \frac{\mathbb{E}_{\mu} \| \varphi (t)
     - \varphi (s) \|^{p q}_{B_{1, 1}^{- 1 - 3 \kappa} (\rho^{4 +
     \sigma})}}{| t - s |^{(1 + \kappa) p q}} \mathd t \mathd s \right)^{1
     / (p q)} \]
  \[ \lesssim C_F (\mathbb{E}_{\mu} \| \varphi \|^{p q}_{W_T^{\alpha, p q}
     B_{1, 1}^{- 1 - 3 \kappa} (\rho^{4 + \sigma})})^{1 / (p q)}, \]
  where $\alpha = 1 + \kappa - 1 / (p q)$. Finally, choosing $p, q \in (1,
  \infty)$ sufficiently small and $\kappa \in (0, 1)$ appropriately, we may
  apply the Sobolev embedding $W_T^{\beta, 1} \subset W_T^{\alpha, p q}$
  together with the uniform bound from Theorem~\ref{thm:phitight} (which
  remains valid in the limit) to deduce
  \[ \| \mathcal{J}^X_{\mu} (F) \|_{\CC^{- 3 / 2 - \kappa} (\rho^{\sigma})}
     \lesssim C_F (\mathbb{E}_{\mu} \| \varphi \|^{p q}_{W_T^{\beta, 1} B_{1,
     1}^{- 1 - 3 \kappa} (\rho^{4 + \sigma})})^{1 / (p q)} \lesssim C_F . \]
  
  
  To show the second bound in the statement of the lemma, we use the fact that
  $\tmop{supp} h \subset [\tau, T]$ for some $0 < \tau < T < \infty$ to
  estimate
  \[ \| \mathcal{J}^H_{\mu} (F) \|_{B_{1, 1}^{- 1 - 3 \kappa} (\rho^{4 +
     \sigma})} \leqslant \mathbb{E}_{\mu} [\| t \mapsto F (\varphi (t))
     \|_{L^{\infty}_{\tau, T}} \| H (\varphi, X) \|_{L^1_T B_{1, 1}^{- 1 - 3
     \kappa} (\rho^{4 + \sigma})}] \]
  \[ \leqslant C_F (\mathbb{E}_{\mu} \| \varphi \|_{L^{\infty}_{\tau, T} H^{-
     1 / 2 - 2\kappa} (\rho^2)}^{2 n})^{1 / 2} (\mathbb{E}_{\mu} \| H (\varphi,
     X) \|^2_{L^1_T B_{1, 1}^{- 1 - 3 \kappa} (\rho^{4 + \sigma})})^{1 / 2}
     \lesssim C_F, \]
  where the last inequality follows from Theorem~\ref{thm:phitight} and the
  bounds in the proof of Proposition~\ref{prop:reg}.
\end{proof}

\

Heuristically we can think of $\mathcal{J}_{\mu} (F)$ as given by
\[ \mathcal{J}_{\mu} (F) \approx \int F (\varphi) \llbracket \varphi^3
   \rrbracket (0) \nu (\mathd \varphi) . \]
However, as we have seen above, this expression is purely formal since
$\llbracket \varphi^3 \rrbracket$ is only a space-time distribution with
respect to $\mu$ and therefore $\llbracket \varphi^3 \rrbracket (0)$ is not a
well defined random variable. One has to consider $F \mapsto \mathcal{J}_{\mu}
(F)$ as a linear functional on cylinder functions taking values in
$\mathcal{S}' (\mathbb{R}^3)$ and satisfying the above properties.
Lemma~\ref{lemma:IF} presents a concrete probabilistic representation based on
the stationary stochastic quantization dynamics of the $\Phi^{4_{}}_3$
measure.

\

Alternatively, the distribution $\mathcal{J}_{\mu} (F)$ can be characterized
in terms of $\varphi (0)$ without using the dynamics, in particular, in the
spirit of the operator product expansion as follows.

\begin{lemma}\label{lem:OPE}
  Let $F$ be a cylinder function as in Lemma~\ref{lemma:IF} and $\nu$ the
  first marginal of $\mu$. Then there exists a sequence of constants $(c_N)_{N
  \in \mathbb{N}}$ tending to $\infty$ as $N \rightarrow \infty$ such that
  \[ \mathcal{J}_{\mu} (F) = \lim_{N \rightarrow \infty} \int F (\varphi)
     [(\Delta_{\leqslant N} \varphi)^3 - c_N (\Delta_{\leqslant N} \varphi)]
     \nu (\mathd \varphi) \]
     in the sense of distributions. Moreover, the renormalization constants are given by
     $$
     c_{N}=3\lambda\mathbb{E}\big[\llbracket (\Delta_{\leqslant N}X)^{2}\rrbracket(t,0)\big]-18\lambda^{2}\mathbb{E}\big[\llbracket (\Delta_{\leqslant N}X)^{2}\rrbracket\circ\Q^{-1}\llbracket (\Delta_{\leqslant N}X)^{2}\rrbracket(t,0)\big],
     $$
     for some $t\geqslant 0$, where
     $$
     \llbracket (\Delta_{\leqslant N}X)^{2}\rrbracket=(\Delta_{\leqslant N}X)^{2}-\mathbb{E}\big[\llbracket (\Delta_{\leqslant N}X)^{2}\rrbracket(t,0)\big].
     $$
\end{lemma}

\begin{proof}
  Let
  \[ \mathcal{J}_{\nu, N} (F) \assign \int F (\varphi) [(\Delta_{\leqslant N}
     \varphi)^3 - c_N (\Delta_{\leqslant N} \varphi)] \nu (\mathd \varphi) .
  \]
  Then by stationarity of $\varphi$ under $\mu$ we have for a function $h$
  satisfying the above properties
  \[ \mathcal{J}_{\nu, N} (F) =\mathbb{E}_{\mu} \left[ \int_{\mathbb{R}} h (t)
     F (\varphi (t)) [(\Delta_{\leqslant N} \varphi (t))^3 - c_N
     (\Delta_{\leqslant N} \varphi (t))] \mathd t \right] . \]
  At this point is not difficult to proceed as above and find suitable
  constants $(c_N)_{N \in \mathbb{N}}$ which deliver the appropriate
  renormalizations so that
  \[ [(\Delta_{\leqslant N} \varphi)^3 - c_N (\Delta_{\leqslant N} \varphi)]
     \rightarrow \llbracket \varphi^3 \rrbracket, \]
  and therefore, using the control of the moments, prove that
  \[ \mathcal{J}_{\nu, N} (F) \rightarrow \mathbb{E}_{\mu} \left[
     \int_{\mathbb{R}} h (t) F (\varphi (t)) \llbracket \varphi^3 \rrbracket
     (t) \mathd t \right] =\mathcal{J}_{\mu} (F) . \]
\end{proof}

\begin{remark}
  By the previous lemma it is now clear that $\mathcal{J}_{\mu}$ does not
  depends on $\mu$ but only on its first marginal $\nu$. So in the following
  we will write $\mathcal{J}_{\nu} \assign \mathcal{J}_{\mu}$ to stress this
  fact.
\end{remark}



Using these informations we can pass to the limit in the approximate
integration by parts formula {\eqref{eq:ibp1}} and obtain an integration by
parts formula for the $\Phi^4_3$ measure in the full space. This is the main
result of this section.

\begin{theorem}
  \label{thm:ibp}Any accumulation point $\nu$ of the sequence $(\nu_{M,
  \varepsilon} \circ (\mathcal{E}^{\varepsilon})^{- 1})_{M, \varepsilon}$
  satisfies
  \begin{equation}
    \int \mathD F (\varphi) \nu (\mathd \varphi) = 2 \int [(m^2 - \Delta)
    \varphi] F (\varphi) \nu (\mathd \varphi) + 2\lambda \mathcal{J}_{\nu} (F)
    \label{eq:IBP}
  \end{equation}
  in the sense of distributions.
\end{theorem}

When interpreted in terms of $n$-point correlation functions, the integration
by parts formula {\eqref{eq:IBP}} gives rise to the hierarchy of
Dyson--Schwinger equations for any limiting measure $\nu$.

\begin{corollary}
  \label{cor:SD}Let $n \in \mathbb{N}$. Any accumulation point $\nu$ of the
  sequence $(\nu_{M, \varepsilon} \circ (\mathcal{E}^{\varepsilon})^{- 1})_{M,
  \varepsilon}$ satisfies
  \[ \sum_{i = 1}^n \delta (x - x_i) \mathbb{E}_{\nu} [\varphi (x_1) \cdots
     \varphi (x_{i - 1}) \varphi (x_{i + 1}) \cdots \varphi (x_n)]
     =\mathbb{E}_{\nu} [[(m^2 - \Delta_x) \varphi (x)] \varphi (x_1) \cdots
     \varphi (x_n)] \]
  \[ - \lambda \lim_{N \rightarrow \infty} \mathbb{E}_{\nu} [\varphi (x_1) \cdots
     \varphi (x_n) ((\Delta_{\leqslant N} \varphi (x))^3 - c_N
     \Delta_{\leqslant N} \varphi (x))]_{} \]
  as an equality for distributions in $\mathcal{S}' (\mathbb{R}^3)^{\otimes (n
  + 1)}$.
\end{corollary}

In particular, this allow to express the (space-homogeneous) two-point
function $S_{2}^{\nu} (x - y) \assign \mathbb{E}_{\nu} [\varphi (x) \varphi
(y)]$ of $\nu$ as the solution to
\[ \delta (x - y) = (m^2 - \Delta_x) S^{\nu}_{2} (x - y) -\lambda \lim_{N \rightarrow
   \infty} [((\mathbb{I} \otimes \Delta_{\leqslant N}^{\otimes 3}) S^{\nu}_{4})
   (y, x, x, x) - c_N (\Delta_{\leqslant N} S^{\nu}_{2}) (x - y)], \]
where the right hand side includes the  four point function $S^{\nu}_{4}
(x_1, \ldots, x_4) \assign \mathbb{E}_{\nu} [\varphi (x_1) \cdots \varphi
(x_4)].$

Finally, we observe that the above arguments also allow us to pass to the
limit in the stochastic quantization equation and to identify the continuum
dynamics. To be more precise, we use Skorokhod's representation theorem to
obtain a new probability space together with (not relabeled) processes
$(\varphi_{M, \varepsilon}, \mathbb{X}_{M, \varepsilon})$ defined on some
probability space and converging in the appropriate topology determined above
to some $(\varphi, \mathbb{X})$. We deduce the following result.

\begin{corollary}
The couple $(\varphi, \mathbb{X})$ solves the continuum stochastic quantization
equation
\[ \LL \varphi +\lambda \llbracket \varphi^3 \rrbracket = \xi
   \qquad \tmop{in} \qquad \mathcal{S}' (\mathbb{R}_+ \times \mathbb{R}^d), \]
where $\xi = \LL X$ and $\llbracket \varphi^3 \rrbracket$ is given by
{\eqref{eq:phi3}}.
\end{corollary}



\section{Fractional $\Phi^4_3$}
\label{sec:fractional}

In this section we discuss the extension of the results of this  paper to the \emph{fractional $\Phi^4_3$ model}, namely to the limit  of the following discrete Gibbs measures. Let $\gamma \in (0, 1)$ and set
{\small{\begin{equation}
  \mathd \nu_{M, \varepsilon}^{\gamma} \propto \exp \left\{ - 2 \varepsilon^d 
  \sum_{x \in \Lambda_{M, \varepsilon}} \left[ \frac{\lambda}{4} | \varphi_x
  |^4 + \frac{- 3 \lambda a_{M, \varepsilon} + 3 \lambda^2 b_{M, \varepsilon}
  + m^2}{2} | \varphi_x |^2 + \frac{1}{2} | (- \Delta_{\varepsilon})^{\gamma /
  2} \varphi_x |^2 \right] \right\}  \prod_{x \in \Lambda_{M, \varepsilon}}
  \hspace{-0.17em} \hspace{-0.17em} \mathd \varphi_x,
  \label{eq:fractional-gibbs}
\end{equation}}}
where $(- \Delta_{\varepsilon})^{\gamma}$ is the (discrete) fractional
Laplacian operator given through Fourier transform by
\[ \mathcal{F} ((- \Delta_{\varepsilon})^{\gamma} f) (k) = l_{\varepsilon}
   (k)^{\gamma}  \hat{f} (k), \]
with $l_{\varepsilon} (k) := \sum_{j = 1}^3 4 \sin^2 (\varepsilon \pi k_j) /
\varepsilon^2$. The kernel of the operator $(- \Delta_{\varepsilon})^{\gamma}$ on the lattice $(\varepsilon \mathbb{Z})^3$ has power-law decay in  space and therefore the above measure corresponds to a non-Gaussian unbounded-spin system with long-range interactions. Varying $\gamma$ at fixed space dimension allows to explore a range of super-renormalizable models which approach the critical dimension  as $\gamma$ is lowered. These and similar models have been considered in~\cite{brydges_non_gaussian_1998, MR2004988, MR2350436, slade_critical_2018, MR3874867} as rigorous ways to implement Wilson's and Fisher's $\varepsilon$-expansion idea, namely the study of critical models perturbatively in the distance to the critical dimension.

%
Let us first observe  that the measure $\nu_{M, \varepsilon}^{\gamma}$ is reflection positive. Albeit this result
seems to belong to the folklore of the mathematical physics community, we could
not find a clear reference to this fact and therefore we will give a sketch  of the proof. We
start from the observation that the fractional Laplacian generates a
reflection positive Gaussian measure. The proof we report below is due to
A.~Abdesselam (private communication). Recall that on $\Lambda_{M,\varepsilon}$ we define reflections $\theta^i$ with  $i=1,2,3$ and the reflection positivity as in Section~\ref{ss:OS2}. 
Below, the reflection positivity is always understood with respect to $\theta=\theta^1$. Of course,  similar considerations hold for the other directions as well.


\begin{theorem}
  Let $a > 0$, $\gamma \in
  (0, 1)$ and let  $\mu_{M, \varepsilon}^{\gamma}$  be the Gaussian measure on  $\Lambda_{M, \varepsilon}$ with
  covariance given by $(a - \Delta_{\varepsilon})^{- \gamma}$. Then $\mu_{M, \varepsilon}^{\gamma}$  is reflection positive.
\end{theorem}

\begin{proof}
  Let $\rho > 0$ and let $K_{\gamma} (\rho) \assign \int_0^{\infty}
  \frac{\mathd t}{t^{\gamma} (t + \rho)}$, so that $K_{\gamma} (\rho) =
  \rho^{- \gamma} K_{\gamma} (1)$. As a consequence we have the formula (as
  Fourier multipliers)
  \[ (a - \Delta_{\varepsilon})^{- \gamma} = \frac{1}{K_{\gamma} (1)}
     \int_0^{\infty} (t + a - \Delta_{\varepsilon})^{- 1} \frac{\mathd
     t}{t^{\gamma}} . \]
  Now the Gaussian measure with covariance $(t + a - \Delta_{\varepsilon})^{-
  1}$ corresponds to a spin-spin nearest neighbors interaction and is well
  known to be reflection positive (see the discussion in Section~\ref{ss:OS2}). In
  particular,
  \[ \sum_{x, y \in \Lambda_{M, \varepsilon}} \overline{\theta f (x)} f (y) (t
     + a - \Delta_{\varepsilon})^{- 1} (x, y) \geqslant 0, \]
  for all $f : \Lambda_{M, \varepsilon} \rightarrow \mathbb{C}$ supported on
  $\Lambda_{M, \varepsilon}^+ = \{ x \in \Lambda_{M, \varepsilon} : 0 < x_1
   < M / 2 \}$. Taking the appropriate integral over $t$ we get
  \[ \sum_{x, y \in \Lambda_{M, \varepsilon}} \overline{\theta f (x)} f (y) (a
     - \Delta_{\varepsilon})^{- \gamma} (x, y) \geqslant 0. \]
  From this we can deduce that, for all cylinder functions $F$ supported on
  $\Lambda_{M, \varepsilon}^+$ we have
  \[ \mathbb{E} [\overline{\theta F (\phi)} F (\phi)] \geqslant 0, \]
  where $\phi$ is the Gaussian field with covariance $(a -
  \Delta_{\varepsilon})^{- \gamma}$. This follows from taking $F$ as a linear
  combination of exponentials and then using Schur-Hadamard product theorem to
  deduce positivity and finally concluding by a density argument (see e.g. \cite[Thm 6.2.2]{MR887102}).
\end{proof}

\begin{corollary}
  The fractional $\Phi^4_3$ measure~{\eqref{eq:fractional-gibbs}} on
  $\Lambda_{M, \varepsilon}$ is reflection positive.
\end{corollary}

\begin{proof}
  Take $a > 0$ and consider the measure
  \[ \nu_{M, \varepsilon}^{\gamma, a} (\mathd \phi) = \frac{1}{Z^{\gamma,
     a}_{M, \varepsilon}} \rho_{\Lambda_{M, \varepsilon}} (\phi) \mu_{M,
     \varepsilon}^{\gamma} (\mathd \phi), \]
  where $\mu_{M, \varepsilon}^{\gamma}$ is, as above, the Gaussian measure with
  covariance $(a - \Delta_{\varepsilon})^{- \gamma}$ and
  \[ \rho_{\Lambda_{M, \varepsilon}} (\varphi) \assign \exp \left\{ - 2
     \varepsilon^d  \sum_{x \in \Lambda_{M, \varepsilon}} \left[
     \frac{\lambda}{4} | \varphi_x |^4 + \frac{- 3 \lambda a_{M, \varepsilon}
     + 3 \lambda^2 b_{M, \varepsilon} + m^2}{2} | \varphi_x |^2 \right]
     \right\} . \]
  Note that $\rho_{\Lambda_{M, \varepsilon}} (\varphi) = \rho_{\Lambda_{M,
  \varepsilon}^+} (\varphi) (\theta \rho_{\Lambda_{M, \varepsilon}^+})
  (\varphi)$ and that we can write
  \[ \int \overline{\theta F (\phi)} F (\phi) \nu_{M, \varepsilon}^{\gamma, a}
     (\mathd \phi) = \frac{1}{Z^{\gamma, a}_{M, \varepsilon}} \int
     \overline{\theta F (\phi)} F (\phi) \rho_{\Lambda_{M, \varepsilon}}
     (\phi) \mu_{M, \varepsilon}^{\gamma, a} (\mathd \phi) \]
  \[ = \frac{1}{Z^{\gamma, a}_{M, \varepsilon}} \int \overline{\theta
     (\rho_{\Lambda_{M, \varepsilon}^+} F) (\phi)} (\rho_{\Lambda_{M,
     \varepsilon}^+} F) (\phi) \mu_{M, \varepsilon}^{\gamma, a} (\mathd \phi)
     \geqslant 0, \]
  since we already proved that $\mu_{M, \varepsilon}^{\gamma, a}$ is
  reflection positive. Now, observe also that as $a \rightarrow 0$ the measures
  $(\nu_{M, \varepsilon}^{\gamma, a})_a$ converge weakly to $\nu_{M,
  \varepsilon}^{\gamma}$ and as a consequence we deduce that $\nu_{M,
  \varepsilon}^{\gamma}$ is reflection positive.
\end{proof}



The equilibrium stochastic dynamics associated to the measure $\nu_{M,
\varepsilon}^{\gamma}$ reads
\begin{equation}
  \LL _{\varepsilon}^{\gamma} \varphi_{M, \varepsilon} + \lambda \varphi_{M,
  \varepsilon}^3 + (- 3 \lambda a_{M, \varepsilon} + 3 \lambda^2 b_{M,
  \varepsilon}) \varphi_{M, \varepsilon} = \xi_{M, \varepsilon}, \qquad x \in
  \Lambda_{M, \varepsilon}, \label{eq:frac-moll}
\end{equation}
where $\LL _{\varepsilon}^{\gamma} = \partial_t + \Q_{\varepsilon}^{\gamma} $ and
$\Q_{\varepsilon}^{\gamma} = m^2 + (- \Delta_{\varepsilon})^{\gamma}$. 
%
We have to take into
account the different regularization properties of the fractional Laplacian, and the related modified space-time scaling
for the fractional heat equation. This  implies that the stochastic terms are of lower regularity. 
%
In particular,
 $X_{M,\varepsilon},\llbracket X_{M,\varepsilon}^{2}\rrbracket, \llbracket X_{M,\varepsilon}^{3}\rrbracket$ and $\tthreeone{X_{M, \varepsilon}}$ have respectively the spatial regularities  $(2\gamma-3)/2-$,  $(2\gamma-3)-$,  $3(2\gamma-3)/2-$, $(10\gamma-9)/2-$. 
%
 It is clear that using only the first order paracontrolled expansion developed in this  paper 
it is not possible to cover the full range of $\gamma$ for which the model is still  
subcritical (i.e. super-renormalizable). From eq.~\eqref{eq:frac-moll} one can readily compute that criticality in three-dimensions is reached when $\gamma=3/4$ at  which point the term $\llbracket X_{M,\varepsilon}^{2}\rrbracket$ scales like the fractional Laplacian. 

For large enough values of $\gamma \in (3/4,1)$ the analysis proceeds exactly in the case $\gamma = 1$. 
 Consequently  $Y_{M,\varepsilon}$ will also be of regularity $(10\gamma-9)/2-$ (cf. Lemma~\ref{lem:Y1}).  Since based on \eqref{eq:17}, $\phi_{\varepsilon}$ will have regularity $(4\gamma-3)-$, the various commutators 
      $D_{\rho^4, \varepsilon} (\phi_{M,\varepsilon}, - 3\lambda \llbracket
     X_{M,\varepsilon}^2 \rrbracket, \phi_{M,\varepsilon})$, $\langle \rho^4
     \phi_{M,\varepsilon}, \tilde{C}_{\varepsilon} (\phi_{M,\varepsilon}, 3\lambda
     \llbracket X_{M,\varepsilon}^2 \rrbracket, 3\lambda \llbracket X_{M,\varepsilon}^2
     \rrbracket) \rangle_{\varepsilon}$,
     and
  $ D_{\rho^4, \varepsilon} ( \phi_{M,\varepsilon}, 3\lambda \llbracket
     X_{M,\varepsilon}^2 \rrbracket, (\Q_{\varepsilon}^{\gamma})^{- 1} [3\lambda \llbracket
     X_{M,\varepsilon}^2 \rrbracket \succ \phi_{M,\varepsilon}] )$
 will be under control as  soon as 
 $
 (8\gamma-6)+(2\gamma-3)=10\gamma-9>0
 $
 namely when $\gamma > 9/10$. However, the term $Z_{M,\varepsilon}$ now has the regularity of the tree $\tthreethreer{X_{M,\varepsilon}}$ namely $(14\gamma-15)/2-$ and therefore in order to control $\langle\phi_{M,\varepsilon},Z_{M,\varepsilon}\rangle$ we must require $ \gamma>21/22$. 
 In this case
   the fractional energy estimate of Theorem~\ref{th:energy-estimate-int} carries through and provides a priori estimates for $\psi_{M,\varepsilon}$ in weighted $H^{\gamma}$ and as a consequence a similar estimate holds for $\zeta_{M,\varepsilon}$ in the same space. The proof of the stretched exponential integrability works as well but the exponent becomes worse due to the limited regularity of the stochastic terms. Moreover, the improved tightness in Section~\ref{s:reg} remains unchanged and yields the corresponding regularity.
%
Therefore, mutatis mutandis we conclude the following results.


\begin{theorem}
  \label{th:main-frac}Let $\gamma \in (21/22,1)$. There exists a choice of the sequence $(a_{M, \varepsilon},
  b_{M, \varepsilon})_{M, \varepsilon}$ such that for any $\lambda > 0$ and
  $m^2 \in \mathbb{R}$, the family of measures $(\nu^\gamma_{M, \varepsilon})_{M,
  \varepsilon}$ appropriately extended to $\mathcal{S}' (\mathbb{R}^3)$ is tight. 
  All the consequences stated  in Theorem~\ref{th:main} carry on to  
  every accumulation point $\nu$ of this family of measures except from the fact that the exponential integrability holds for some $\upsilon\in(0,1)$ not necessarily of order $\kappa$. 
\end{theorem}



If $\gamma \leqslant 21/22$ an additional renormalization is needed to treat the divergence of
$$(\Q^{\gamma}_{\varepsilon})^{-1}\llbracket X_{M,\varepsilon}^{2}\rrbracket\circ \tthreethreer{X_{M,\varepsilon}}.$$
In general, when $\gamma \in (3/4,21/22]$ more complex expansions and renormalizations are needed, either by exploiting the iterated commutator methods of Bailleul and Bernicot~\cite{bailleul_high_2016} or full fledged regularity structures~\cite{hairer_theory_2014,hairer_discretisations_2018}. While it is not clear that the local estimates of Moinat and Weber~\cite{moinat_space_time_2018} apply to the fractional Laplacian (which is a non-local operator), our energy method could be conceivably adapted to the regularity  structures framework. 
%However, the  details are non-trivial and they would require additional work. Thus, w
We prefer to leave these more substantial extensions to further  investigations.
%:edit-location 


\appendix
\section{Technical results}

\label{s:app}
In this section we present  auxiliary results needed in
the main body of the paper.

\subsection{Besov spaces}
First, we cover various properties of the
discrete weighted Besov spaces such as an equivalent formulation of the norms,
duality, interpolation, embeddings, bounds for powers of functions and a
weighted Young's inequality.

\begin{lemma}
  \label{lem:equiv2}Let $\alpha \in \mathbb{R}$, $p, q \in [1, \infty]$. Fix
  $n > | \alpha |$ and assume that $\rho$ is a weight such that
  \[ \| \rho \|_{B^{n + 1, \varepsilon}_{\infty, \infty} (\rho^{- 1})} + \|
     \rho^{- 1} \|_{B^{n + 1, \varepsilon}_{\infty, \infty} (\rho)} \lesssim 1
  \]
  uniformly in $\varepsilon$. Then
  \[ \| f \|_{B^{\alpha, \varepsilon}_{p, q} (\rho)} \sim \| \rho f
     \|_{B^{\alpha, \varepsilon}_{p, q}}, \]
  where the proportionality constant does not depend on $\varepsilon$.
\end{lemma}

\begin{proof}
  We write $\rho f = \rho \prec f + \rho \succcurlyeq f$ and estimate by
  paraproduct estimates
  \[ \| \rho \prec f \|_{B^{\alpha, \varepsilon}_{p, q}} = \| \rho \prec f
     \|_{B^{\alpha, \varepsilon}_{p, q} (\rho^{- 1} \rho)} \lesssim \| \rho
     \|_{L^{\infty, \varepsilon} (\rho^{- 1})} \| f \|_{B^{\alpha,
     \varepsilon}_{p, q} (\rho)} \lesssim \| f \|_{B^{\alpha, \varepsilon}_{p,
     q} (\rho)}, \]
  \[ \| \rho \succcurlyeq f \|_{B^{\alpha, \varepsilon}_{p, q}} = \| \rho
     \succcurlyeq f \|_{B^{\alpha, \varepsilon}_{p, q} (\rho^{- 1} \rho)}
     \lesssim \| f \|_{B^{\alpha, \varepsilon}_{p, \infty} (\rho)} \| \rho
     \|_{B^{n, \varepsilon}_{\infty, q} (\rho^{- 1})} \lesssim \| f
     \|_{B^{\alpha, \varepsilon}_{p, q} (\rho)} \| \rho \|_{B^{n + 1,
     \varepsilon}_{\infty, \infty} (\rho^{- 1})} \]
  \[ \lesssim \| f \|_{B^{\alpha, \varepsilon}_{p, q} (\rho)}, \]
  which implies one inequality. For the converse one, we write $f = \rho^{- 1}
  \prec (\rho f) + \rho^{- 1} \succcurlyeq (\rho f)$, and estimate
  \[ \| \rho^{- 1} \prec (\rho f) \|_{B^{\alpha, \varepsilon}_{p, q} (\rho)}
     \lesssim \| \rho^{- 1} \|_{L^{\infty, \varepsilon} (\rho)} \| \rho f
     \|_{B^{\alpha, \varepsilon}_{p, q}}, \]
  \[ \| \rho^{- 1} \succcurlyeq (\rho f) \|_{B^{\alpha, \varepsilon}_{p, q}
     (\rho)} \lesssim \| \rho f \|_{B^{\alpha, \varepsilon}_{p, \infty}} \|
     \rho^{- 1} \|_{B^{n, \varepsilon}_{\infty, q} (\rho)} \lesssim \| \rho f
     \|_{B^{\alpha, \varepsilon}_{p, q}} \| \rho^{- 1} \|_{B^{n + 1,
     \varepsilon}_{\infty, \infty} (\rho)} . \]
\end{proof}

\begin{lemma}
  \label{lem:dual2}Let $\alpha \in \mathbb{R}$, $p, p', q, q' \in [1, \infty]$
  such that $p, p'$ and $q, q'$ are conjugate exponents. Let $\rho$ be a
  weight as in Lemma~\ref{lem:equiv2}. Then
  \[ \langle f, g \rangle_{\varepsilon} \lesssim \| f \|_{B_{p, q}^{\alpha,
     \varepsilon} (\rho)} \| g \|_{B_{p', q'}^{- \alpha, \varepsilon}
     (\rho^{- 1})} \]
  with a proportionality constant independent of $\varepsilon$. Consequently,
  $B^{- \alpha, \varepsilon}_{p', q'} (\rho^{- 1}) \subset (B^{\alpha,
  \varepsilon}_{p, q} (\rho^{- 1}))^{\ast}$.
\end{lemma}

\begin{proof}
  In view of Lemma~\ref{lem:equiv2} it is sufficient to consider the
  unweighted case. Let $f \in B^{\alpha, \varepsilon}_{p, q}$ and $g \in B^{-
  \alpha, \varepsilon}_{p', q'}$. Then by Parseval's theorem and H{\"o}lder's
  inequality we have
  \[ \varepsilon^d \sum_{x \in \Lambda_{\varepsilon}} f (x) g (x) = \sum_{- 1
     \leqslant i, j \leqslant N - J} \varepsilon^d \sum_{x \in
     \Lambda_{\varepsilon}} \Delta_i^{\varepsilon} f (x)
     \Delta_j^{\varepsilon} g (x) \]
  \[ = \sum_{- 1 \leqslant i, j \leqslant N - J, i \sim j}
     \int_{\hat{\Lambda}_{\varepsilon}} \varphi_i (k) \mathcal{F} f (k)
     \varphi_j (k) \mathcal{F} g (k) \mathd k \]
  \[ = \sum_{- 1 \leqslant i, j \leqslant N - J, i \sim j} 2^{\alpha j} 2^{-
     \alpha j} \varepsilon^d \sum_{x \in \Lambda_{\varepsilon}}
     \Delta_i^{\varepsilon} f (x) \Delta_j^{\varepsilon} g (x) \lesssim \| f
     \|_{B_{p, q}^{\alpha, \varepsilon}} \| g \|_{B_{p', q'}^{- \alpha,
     \varepsilon}} . \]
\end{proof}

\begin{lemma}
  \label{lem:int}Let $\varepsilon \in \mathcal{A}$. Let $\alpha, \alpha_0,
  \alpha_1, \beta, \beta_0, \beta_1 \in \mathbb{R}$, $p, p_0, p_1, q, q_0, q_1
  \in [1, \infty]$ and $\theta \in [0, 1]$ such that
  \[ \alpha = \theta \alpha_0 + (1 - \theta) \alpha_1, \quad \beta = \theta
     \beta_0 + (1 - \theta) \beta_1, \quad \frac{1}{p} = \frac{\theta}{p_0} +
     \frac{1 - \theta}{p_1}, \quad \frac{1}{q} = \frac{\theta}{q_0} + \frac{1
     - \theta}{q_1} . \]
  Then
  \[ \| f \|_{B^{\alpha, \varepsilon}_{p, q} (\rho^{\beta})} \leqslant \| f
     \|^{\theta}_{B^{\alpha_0, \varepsilon}_{p_0, q_0} (\rho^{\beta_0})} \| f
     \|^{1 - \theta}_{B^{\alpha_1, \varepsilon}_{p_1, q_1} (\rho^{\beta_1})} .
  \]
\end{lemma}

\begin{proof}
  The proof is a consequence of H{\"o}lder's inequality. Let us show the claim
  for $p$, $p_0$, $p_1$, $q$, $q_0$, $q_1 \in [1, \infty)$ and $\varepsilon \in
  \mathcal{A} \setminus \{ 0 \}$. If some of the exponents $p, p_0, p_1, q,
  q_0, q_1$ are infinite or we are in the continuous setting, the proof
  follows by obvious modifications. We write
  \[ \| \rho^{\beta} \Delta_j^{\varepsilon} f \|_{L^{p, \varepsilon}}^p =
     \varepsilon^d \sum_{x \in \Lambda_{\varepsilon}} | \rho^{\beta}
     \Delta_j^{\varepsilon} f (x) |^p = \varepsilon^d \sum_{k \in
     \Lambda_{\varepsilon}} (\rho^{\theta \beta_0 p} | \Delta_j^{\varepsilon}
     f (x) |^{\theta p}) (\rho^{(1 - \theta) \beta_1 p} |
     \Delta_j^{\varepsilon} f (x) |^{(1 - \theta) p}) \]
  and apply H{\"o}lder's inequality to the conjugate exponents
  $\frac{p_0}{\theta p}$ and $\frac{p_1}{(1 - \theta) p}$ to obtain
  \[ \| \rho^{\beta} \Delta_j^{\varepsilon} f \|_{L^{p, \varepsilon}}^p
     \leqslant \left( \varepsilon^d \sum_{x \in \Lambda_{\varepsilon}}
     \rho^{\beta_0 p_0} | \Delta_j^{\varepsilon} f |^{p_0} \right)^{\theta p /
     p_0} \left( \varepsilon^d \sum_{x \in \Lambda_{\varepsilon}}
     \rho^{\beta_1 p_1} | \Delta_j^{\varepsilon} f |^{p_1} \right)^{(1 -
     \theta) p / p_1} \]
  \[ = \| \Delta_j^{\varepsilon} f \|^{\theta p}_{L^{p_0, \varepsilon}
     (\rho^{\beta_0})} \| \Delta_j^{\varepsilon} f \|^{(1 - \theta)
     p}_{L^{p_1, \varepsilon} (\rho^{\beta_1})} . \]
  Consequently,
  \[ \| f \|^q_{B^{\alpha, \varepsilon}_{p, q} (\rho^{\beta})} \leqslant
     \sum_{- 1 \leqslant j \leqslant N - J} 2^{\alpha k q} \| \rho^{\beta}
     \Delta_j^{\varepsilon} f \|_{L^{p, \varepsilon}}^q \]
  \[ \leqslant \sum_{- 1 \leqslant j \leqslant N - J} \left( 2^{\theta
     \alpha_0 k q} \| \Delta_j^{\varepsilon} f \|^{\theta q}_{L^{p_0,
     \varepsilon} (\rho^{\beta_0})} \right) \left( 2^{(1 - \theta) \alpha_1 k
     q} \| \Delta_j^{\varepsilon} f \|^{(1 - \theta) q}_{L^{p_1, \varepsilon}
     (\rho^{\beta_1})} \right) \]
  and by H{\"o}lder's inequality to the conjugate exponents $\frac{q_0}{\theta
  q}$ and $\frac{q_1}{(1 - \theta) q}$
  \[ \| f \|^q_{B^{\alpha, \varepsilon}_{p, q} (\rho^{\beta})} \]
  \[ \leqslant \left( \sum_{- 1 \leqslant j \leqslant N - J} 2^{\alpha_0 k
     q_0} \| \Delta_j^{\varepsilon} f \|^{q_0}_{L^{p_0, \varepsilon}
     (\rho^{\beta_0})} \right)^{\theta q / q_0} \left( \sum_{- 1 \leqslant j
     \leqslant N - J} 2^{\alpha_1 k q_1} \| \Delta_j^{\varepsilon} f
     \|^{q_1}_{L^{p_1, \varepsilon} (\rho^{\beta_1})} \right)^{(1 - \theta) q
     / q_1} \]
  \[ = \| f \|^{\theta q}_{B^{\alpha_0, \varepsilon}_{p_0, q_0}
     (\rho^{\beta_0})} \| f \|^{(1 - \theta) q}_{B^{\alpha_1,
     \varepsilon}_{p_1, q_1} (\rho^{\beta_1})} . \]
\end{proof}

We note that by our construction of the Littlewood--Paley projectors on
$\Lambda_{\varepsilon}$, in each of the cases $j = - 1$, $j \in \{ 0, \ldots,
N - J - 1 \}$ and $j = N - J$, there exists an $L^1$-kernel $\mathcal{K}$ such
that the Littlewood--Paley block $\Delta^{\varepsilon}_j f$ is given by a
convolution with $2^{j d} \mathcal{K} (2^j \cdummy)$.  For notational simplicity we omit the
dependence of $\mathcal{K}$ on the three cases above.

\begin{lemma}
  \label{lem:emb}Let $\varepsilon \in \mathcal{A}$ and let $\beta > 0$. Then
  \[ L^{2, \varepsilon} (\rho) = B^{0, \varepsilon}_{2, 2} (\rho), \qquad
     L^{4, \varepsilon} (\rho) \subset B^{0, \varepsilon}_{4, \infty} (\rho)
  \]
  and the proportional constants do not depend on $\varepsilon$.
\end{lemma}

\begin{proof}
  Due to Lemma~\ref{lem:equiv2} together with Parseval's equality we directly
  obtain the first claim. Consequently, by Young's inequality together with
  the fact that $\frac{\rho (y)}{\rho (x)} \lesssim \rho^{- 1} (x - y)$ (for a
  universal proportionality constant that depends only on $\rho$) we have that
  \[ \| f \|_{B^{0, \varepsilon}_{4, \infty} (\rho)} = \sup_{- 1 \leqslant
     j \leqslant N - J} \| \Delta_j^{\varepsilon} f \|_{L^{4, \varepsilon}
     (\rho)} = \sup_{- 1 \leqslant j \leqslant N - J} \| 2^{j d} \mathcal{K}
     (2^j \cdummy) \ast f \|_{L^{4, \varepsilon} (\rho)} \]
  \[ \lesssim \sup_{- 1 \leqslant j \leqslant N - J} \| 2^{j d} \mathcal{K}
     (2^j \cdummy) \|_{L^{1, \varepsilon} (\rho^{- 1})} \| f \|_{L^{4,
     \varepsilon} (\rho)} \lesssim \| f \|_{L^{4, \varepsilon} (\rho)} . \]
\end{proof}

\begin{lemma}
  \label{lem:grad}Let $\kappa \in (0, 1)$, $p \in [1, \infty]$ and let $\rho$
  be a polynomial weight
  \[ \| f \|_{B^{1 - \kappa, \varepsilon}_{p, p} (\rho)} \lesssim \| f
     \|_{B^{- \kappa, \varepsilon}_{p, p} (\rho)} + \| \nabla_{\varepsilon} f
     \|_{B^{- \kappa, \varepsilon}_{p, p} (\rho)}, \]
  where the proportionality constant does not depend on $\varepsilon$.
\end{lemma}

\begin{proof}
  Let $j \geqslant 0$. Let $K_j = K_{j, \varepsilon} = \mathcal{F}^{- 1}
  \varphi^{\varepsilon}_j$ and denote $\bar{K}_j = \bar{K}_{j, \varepsilon} =
  \sum_{i \sim j} K_{i, \varepsilon}$. Then 
  $\Delta^{\varepsilon}_j f = \bar{K}_j \ast \Delta^{\varepsilon}_j f$ and we
  write
  \[ \bar{K}_j \ast \Delta^{\varepsilon}_j f = (\tmop{Id} -
     \Delta_{\varepsilon})^{- 1} (\tmop{Id} - \Delta_{\varepsilon}) (\bar{K}_j
     \ast \Delta^{\varepsilon}_j f) \]
  \begin{equation}
    = (\tmop{Id} - \Delta_{\varepsilon})^{- 1} (\bar{K}_j \ast
    \Delta^{\varepsilon}_j f) - (\tmop{Id} - \Delta_{\varepsilon})^{- 1}
    \nabla^{\ast}_{\varepsilon} \nabla_{\varepsilon} (\bar{K}_j \ast
    \Delta^{\varepsilon}_j f) . \label{eq:16}
  \end{equation}
  For the second term we use translation invariance of
  $\nabla_{\varepsilon}$ to obtain
  \[ (\tmop{Id} - \Delta_{\varepsilon})^{- 1} \nabla^{\ast}_{\varepsilon}
     \nabla_{\varepsilon} (\bar{K}_j \ast \Delta^{\varepsilon}_j f) =
     ((\tmop{Id} - \Delta_{\varepsilon})^{- 1} \nabla^{\ast}_{\varepsilon}
     \bar{K}_j) \ast (\Delta^{\varepsilon}_j \nabla_{\varepsilon} f) , \]
  hence by Young inequality
  \[ \| ((\tmop{Id} - \Delta_{\varepsilon})^{- 1} \nabla^{\ast}_{\varepsilon}
     \bar{K}_j) \ast (\Delta^{\varepsilon}_j \nabla_{\varepsilon} f) \|_{L^{p,
     \varepsilon} (\rho)} \lesssim \| (\tmop{Id} - \Delta_{\varepsilon})^{- 1}
     \nabla^{\ast}_{\varepsilon} \bar{K}_j \|_{L^{1, \varepsilon} (\rho^{-
     1})} \| \Delta^{\varepsilon}_j \nabla_{\varepsilon} f \|_{L^{p,
     \varepsilon} (\rho)} \]
  The kernel $\mathcal{V}_{j, \ell} \assign (\tmop{Id} -
  \Delta_{\varepsilon})^{- 1} \nabla^{\ast}_{\varepsilon, \ell} \bar{K}_j$ is
  given by
  \[ \mathcal{V}_{j, \ell} (k) = \int_{\hat{\Lambda}_{\varepsilon}} e^{2 \pi
     ik \cdot x} \frac{\varepsilon^{- 1}  (1 - e^{- 2 \pi i \varepsilon
     x_{\ell} })}{1 + 2 \sum_{p = 1}^d \varepsilon^{- 2} \sin^2 (\pi i
     \varepsilon x_p)} \bar{\varphi}^{\varepsilon}_j (x) \mathd x \]
  where $\bar{\varphi}_j^{\varepsilon} = \sum_{i \sim j}
  \varphi^{\varepsilon}_i$. Now using $(1 - 2^{2 j} \Delta_x)^M e^{2 \pi ik
  \cdot x} = (1 + 2^{2 j} | 2 \pi k |^2)^M e^{2 \pi ik \cdot x}$ and
  integrating by parts $(1 - \Delta_x)^M$ we have
  \[ | (1 + 2^{2 j} | 2 \pi k |^2)^M \mathcal{V}_{j, \ell} (k) | \leqslant
     \int_{\hat{\Lambda}_{\varepsilon}} \left| (1 - 2^{2 j} \Delta_x)^M \left[
     \frac{\varepsilon^{- 1}  (1 - e^{- 2 \pi i \varepsilon x_{\ell} })}{1 + 2
     \sum_{p = 1}^d \varepsilon^{- 2} \sin^2 (\pi i \varepsilon x_p)}
     \bar{\varphi}^{\varepsilon}_j (x) \right] \right| \mathd x \]
  and it is possible to check that (using that $\varepsilon 2^j \lesssim 1$)
  \[ \left| (1 - 2^{2 j} \Delta_x)^M \left[ \frac{\varepsilon^{- 1}  (1 - e^{-
     2 \pi i \varepsilon x_{\ell} })}{1 + 2 \sum_{p = 1}^d \varepsilon^{- 2}
     \sin^2 (\pi i \varepsilon x_p)} \bar{\varphi}^{\varepsilon}_j (x) \right]
     \right| \lesssim 2^{- j} \mathbb{I}_{2^j \tilde{\mathcal{A}}} \]
  uniformly in $j$ where $\tilde{\mathcal{A}}$ is an annulus centered at the
  origin. Therefore
  \[ | \mathcal{V}_{j, \ell} (k) | \lesssim 2^{- j} 2^{d j} (1 + 2^{2 j} | 2
     \pi k |^2)^{- M} \]
  and from this is easy to deduce that $\| \mathcal{V}_{j, \ell} \|_{L^{1,
  \varepsilon} (\rho^{- 1})} \lesssim 2^{- j}$ uniformly in $j$ and
  $\varepsilon$.
  
  A similar computation applies to the first term in {\eqref{eq:16}} to obtain
  \[ \| (\tmop{Id} - \Delta_{\varepsilon})^{- 1} (\bar{K}_j \ast
     \Delta^{\varepsilon}_j f) \|_{L^{p, \varepsilon} (\rho)} \lesssim \|
     (\tmop{Id} - \Delta_{\varepsilon})^{- 1} \bar{K}_j \|_{L^{1, \varepsilon}
     (\rho^{- 1})} \| \Delta^{\varepsilon}_j f \|_{L^{p, \varepsilon} (\rho)}
     \lesssim 2^{- 2 j} \| \Delta^{\varepsilon}_j f \|_{L^{p, \varepsilon}
     (\rho)} \]
  and the proof is complete.
\end{proof}

\begin{lemma}
  \label{lem:15}Let $\varepsilon \in \mathcal{A}$ and let $\iota > 0$. Let
  $\rho$ be a weight such that $\rho^{\iota} \in L^{4, 0}$. Then
  \[ \| \rho^{1 + \iota} f \|_{L^{2, \varepsilon}} \lesssim \| \rho f
     \|_{L^{4, \varepsilon}}, \]
  where the proportionality constant does not depend on $\varepsilon$.
\end{lemma}

\begin{proof}
  By H{\"o}lder's inequality
  \[ \| \rho^{1 + \iota} f \|_{L^{2, \varepsilon}} \leqslant \| \rho^{\iota}
     \|_{L^{4, \varepsilon}} \| \rho f \|_{L^{4, \varepsilon}}, \]
  and since for $| x - y | \leqslant 1$ the quotient $\frac{\rho (x)}{\rho
  (y)}$ is uniformly bounded above and below, it follows from Lemma A.2
  {\cite{MP17}} that
  \[ \| \rho^{\iota} \|_{L^{4, \varepsilon}}^4 = \varepsilon^d \sum_{x \in
     \Lambda_{\varepsilon}} \rho^{4 \iota} (x) \lesssim \int_{\mathbb{R}^d}
     \rho^{4 \iota} (x) \mathd x < \infty, \]
  where the proportional constant only depends on $\rho$.
\end{proof}

\begin{lemma}
  \label{lem:mult}Let $\alpha > 0$. Let $\rho_1, \rho_2$ be weights. Then for
  every $\beta > 0$
  \[ \| f^2 \|_{B^{\alpha, \varepsilon}_{1, 1} (\rho_1 \rho_2)} \lesssim \| f
     \|_{L^{2, \varepsilon} (\rho_1)} \| f \|_{H^{\alpha + 2 \beta,
     \varepsilon} (\rho_2)}, \]
  \[ \| f^3 \|_{B^{\alpha, \varepsilon}_{1, 1} (\rho_1^2 \rho_2)} \lesssim \|
     f \|_{L^{4, \varepsilon} (\rho_1)}^2 \| f \|_{H^{\alpha + 2 \beta,
     \varepsilon} (\rho_2)}, \]
  where the proportionality constants do not depend on $\varepsilon$.
\end{lemma}

\begin{proof}
  Due to the paraproduct estimates and the embeddings of Besov spaces, we have
  for every $\beta > 0$
  \[ \| f^2 \|_{B^{\alpha, \varepsilon}_{1, 1} (\rho_1 \rho_2)} \lesssim \| f
     \|_{B_{2, \infty}^{- \beta, \varepsilon} (\rho_1)} \| f \|_{B_{2,
     1}^{\alpha + \beta, \varepsilon} (\rho_2)} \lesssim \| f \|_{B_{2, 2}^{-
     \beta, \varepsilon} (\rho_1)} \| f \|_{B_{2, 2}^{\alpha + 2 \beta,
     \varepsilon} (\rho_2)} \]
  \[ \lesssim \| f \|_{L^{2, \varepsilon} (\rho_1)} \| f \|_{H^{\alpha + 2
     \beta, \varepsilon} (\rho_2)} . \]
  For the cubic term, we write
  \[ \| f^3 \|_{B^{\alpha, \varepsilon}_{1, 1} (\rho_1^2 \rho_2)} \lesssim \|
     f \prec f^2 \|_{B^{\alpha, \varepsilon}_{1, 1} (\rho_1^2 \rho_2)} + \| f
     \succ f^2 \|_{B^{\alpha, \varepsilon}_{1, 1} (\rho_1^2 \rho_2)} + \| f
     \circ f^2 \|_{B^{\alpha, \varepsilon}_{1, 1} (\rho_1^2 \rho_2)} \]
  and estimate each term separately. The second and the third term can be
  estimated directly by
  \[ \| f \succ f^2 \|_{B^{\alpha, \varepsilon}_{1, 1} (\rho_1^2 \rho_2)} + \|
     f \circ f^2 \|_{B^{\alpha, \varepsilon}_{1, 1} (\rho_1^2 \rho_2)}
     \lesssim \| f^2 \|_{B^{- \beta, \varepsilon}_{2, \infty} (\rho_1^2)} \| f
     \|_{B^{\alpha + \beta, \varepsilon}_{2, 1} (\rho_2)} \]
  \[ \lesssim \| f^2 \|_{B^{- \beta, \varepsilon}_{2, 2} (\rho_1^2)} \| f
     \|_{B^{\alpha + 2 \beta, \varepsilon}_{2, 2} (\rho_2)} \lesssim \| f
     \|_{L^{4, \varepsilon} (\rho_1)}^2 \| f \|_{H^{\alpha + 2 \beta,
     \varepsilon} (\rho_2)} . \]
  For the remaining term, we have
  \[ \| f \prec f^2 \|_{B^{\alpha, \varepsilon}_{1, 1} (\rho_1^2 \rho_2)}
     \lesssim \| f \|_{B^{- \beta, \varepsilon}_{4, \infty} (\rho_1)} \| f^2
     \|_{B^{\alpha + \beta, \varepsilon}_{4 / 3, 1} (\rho_1 \rho_2)} \]
  where by the paraproduct estimates and Lemma~\ref{lem:emb}
  \[ \| f^2 \|_{B^{\alpha + \beta, \varepsilon}_{4 / 3, 1} (\rho_1 \rho_2)}
     \lesssim \| f \|_{B^{- \beta, \varepsilon}_{4, \infty} (\rho_1)} \| f
     \|_{B^{\alpha + 2 \beta, \varepsilon}_{2, 1} (\rho_2)} \lesssim \| f
     \|_{L^{4, \varepsilon} (\rho_1)} \| f \|_{H^{\alpha + 2 \beta,
     \varepsilon} (\rho_2)} \]
  which completes the proof.
\end{proof}

\begin{lemma}
  \label{lem:young}Let $\rho$ be a polynomial weight. Let $p, q, r \in [1,
  \infty]$ be such that $\frac{1}{r} + 1 = \frac{1}{p} + \frac{1}{q}$. Then
  \[ \| f \ast_{\varepsilon} g \|_{L^{r, \varepsilon} (\rho)} \lesssim \| f \|_{L^{p,
     \varepsilon} (\rho^{- 1})} \| g \|_{L^{q, \varepsilon} (\rho)}, \]
  \[ \| f \ast_{\varepsilon} g \|_{L^{r, 0} (\rho)} \lesssim \sup_{y \in \mathbb{R}^d} \|
     (\rho^{- 1} f) (y - \cdummy) \|_{L^{p, \varepsilon}}^{\frac{r - p}{r}} \|
     f \|^{\frac{p}{r}}_{L^{p, 0} (\rho^{- 1})} \| g \|_{L^{q,
     \varepsilon} (\rho)}, \]
     where  $\ast_{\varepsilon}$  denotes the convolution on 
$\Lambda_{\varepsilon}$ and the proportionality constants are independent of $\varepsilon$.
\end{lemma}

\begin{proof}
  We observe that for a polynomial weight of the form $\rho (x) = \langle x
  \rangle^{- \nu}$ for some $\nu \geqslant 0$, we have that $\rho (y)
  \lesssim \rho (x) \rho^{- 1} (x - y)$. Accordingly,
  \[ | f \ast g (y) \rho (y) | = \left| \varepsilon^d \sum_{x \in
     \Lambda_{\varepsilon}} f (y - x) g (x) \rho (y) \right| \lesssim
     \varepsilon^d \sum_{x \in \Lambda_{\varepsilon}} | \rho f (y - x) |
     \rho^{- 1} (x - y) | g (x) | \rho (x) \]
  hence the claim follows by (unweighted) Young's inequality. For the second
  bound, we write
  \[ | f \ast g (y) \rho (y) | \lesssim \varepsilon^d \sum_{x \in
     \Lambda_{\varepsilon}} (| (\rho^{- 1} f) (y - x) |^p | (\rho g) (x)
     |^q)^{\frac{1}{r}} | (\rho^{- 1} f) (y - x) |^{\frac{r - p}{r}} | (\rho
     g) (x) |^{\frac{r - q}{r}} \]
  and apply H{\"o}lder's inequality with exponents $r, \frac{r p}{r - p},
  \frac{r q}{r - q}$
  \[ | f \ast g (y) \rho (y) | \lesssim \left( \varepsilon^d \sum_{x \in
     \Lambda_{\varepsilon}} | (\rho^{- 1} f) (y - x) |^p | \rho g (x) |^q
     \right)^{\frac{1}{r}} \| (\rho^{- 1} f) (y - \cdummy) \|_{L^{p,
     \varepsilon}}^{\frac{r - p}{r}} \| \rho g \|_{L^{q,
     \varepsilon}}^{\frac{r - q}{r}} \]
  \[ \leqslant \left( \varepsilon^d \sum_{x \in \Lambda_{\varepsilon}} |
     (\rho^{- 1} f) (y - x) |^p | \rho g (x) |^q \right)^{\frac{1}{r}} \sup_{y
     \in \mathbb{R}^d} \| (\rho^{- 1} f) (y - \cdummy) \|_{L^{p,
     \varepsilon}}^{\frac{r - p}{r}} \| \rho g \|_{L^{q,
     \varepsilon}}^{\frac{r - q}{r}} . \]
  Finally, taking the $r$th power and integrating completes the proof.
\end{proof}




  
  \subsection{Localizers}
  \label{s:l1}
As the next step, we introduce another equivalent formulation of the weighted
Besov spaces $B^{\alpha, \varepsilon}_{\infty, \infty} (\rho)$ in terms of
suitable point evaluation of the Littlewood--Paley decomposition. First, for
$J \in \mathbb{N}_0$ such that $N - J \leqslant J_{\varepsilon}$, $\alpha \in
\mathbb{R}$ and $\varepsilon \in \mathcal{A}$ we define the Besov space
$b^{\alpha, \varepsilon}_{\infty, \infty} (\rho)$ of sequences $\lambda =
(\lambda_{j, m})_{- 1 \leqslant j \leqslant N - J, m \in \mathbb{Z}^d}$ by the
norm
\[ \| \lambda \|_{b^{\alpha, \varepsilon}_{\infty, \infty} (\rho)} \assign
   \sup_{- 1 \leqslant j \leqslant N - J} 2^{\alpha j} \sup_{m \in
   \mathbb{Z}^d} \rho (2^{- j - J} m) | \lambda_{j, m} | . \]
Note that we do not stress the dependence of $b^{\alpha, \varepsilon}_{\infty,
\infty} (\rho)$ on the parameter $J$ as in the sequel we only consider one
fixed $J$ for all $\varepsilon \in \mathcal{A}$ given by Lemma~\ref{lem:equiv}
below.
The next result shows  the desired equivalence.

\begin{lemma}
  \label{lem:equiv}Let $\alpha \in \mathbb{R}$, $\varepsilon \in \mathcal{A}$
  and let $\rho$ be a weight. There exists $J \in \mathbb{N}_0$ (independent
  of $\varepsilon$) with the following property: $f \in B^{\alpha,
  \varepsilon}_{\infty, \infty} (\rho)$ if and only if it is represented by
  $\lambda = (\lambda_{j, m})_{- 1 \leqslant j \leqslant N - J, m \in
  \mathbb{Z}^d} \in b^{\alpha, \varepsilon}_{\infty, \infty} (\rho)$ such that
  \begin{equation}
    \| f \|_{B_{\infty, \infty}^{\alpha, \varepsilon} (\rho)} \sim \| \lambda
    \|_{b^{\alpha, \varepsilon}_{\infty, \infty} (\rho)}, \label{eq:d3}
  \end{equation}
  where the proportionality constants do not depend on $\varepsilon$. In
  particular, given $f \in B^{\alpha, \varepsilon}_{\infty, \infty} (\rho)$
  the coefficients $\lambda$ are defined by
  \begin{equation}
    \lambda_{j, m} (f) \assign \Delta_j^{\varepsilon} f (2^{- j - J} m),
    \qquad - 1 \leqslant j \leqslant N - J, \hspace{1em} m \in \mathbb{Z}^d,
    \label{eq:d1}
  \end{equation}
  and given $\lambda \in b^{\alpha, \varepsilon}_{\infty, \infty} (\rho)$ the
  distribution $f$ is recovered via the formula
  \begin{equation}
    f = \sum_{- 1 \leqslant j \leqslant N - J} 
    \mathcal{F}^{- 1} (\mathcal{F}_{2^{- j - J} \mathbb{Z}^d} (\lambda_{j,
    \cdot})), \label{eq:d2}
  \end{equation}
  where $\mathcal{F}_{2^{- j - J} \mathbb{Z}^d}$ denotes the Fourier transform
  on the lattice $2^{- j - J} \mathbb{Z}^d$.
\end{lemma}

  
 \begin{proof} 
  Let us first discuss the decomposition {\eqref{eq:d2}}. We recall that if $f
  \in \mathcal{S}' (\Lambda_{\varepsilon})$ then $\mathcal{F} f = \sum_{- 1
  \leqslant j \leqslant N - J} \varphi^{\varepsilon}_j \mathcal{F} f$ where
  for $j < N - J$ the function $\varphi^{\varepsilon}_j \mathcal{F} f$ is
  supported in a ball of radius proportional to $2^j$. Let $j < N - J$ and let
  $B_j \subset \mathbb{R}^d$ be a cube centered at the origin with length
  $2^{j + J}$. We choose $J \in \mathbb{N}_0$ such that $\tmop{supp}
  \varphi^{\varepsilon}_j \subset B_j$. Next, we identify $B_j$ with $(2^{j +
  J} \mathbb{T})^d \subset (2^N \mathbb{T})^d$ and regard
  $\varphi^{\varepsilon}_j \mathcal{F} f$ as a periodic function on $(2^{j +
  J} \mathbb{T})^d$. Then using a Fourier series expansion we may write
  \[ (\varphi^{\varepsilon}_j \mathcal{F} f) (z) = 2^{(- j - J) d} \sum_{m
     \in \mathbb{Z}^d} \lambda_{j, m} (f) e^{- 2 \pi i 2^{- j - J} m \cdummy
     z} = \mathcal{F}_{2^{- j - J} \mathbb{Z}^d} (\lambda_{j, \cdot} (f)) (z) \]
  where
  \[ \lambda_{j, m} (f) \assign \int_{B_j} (\varphi^{\varepsilon}_j
     \mathcal{F} f) (y) e^{2 \pi i 2^{- j - J} m \cdummy y} \mathd y =
     \mathcal{F}^{- 1} (\varphi^{\varepsilon}_j \mathcal{F} f) (2^{- j - J} m)
     = \Delta_j^{\varepsilon} f (2^{- j - J} m) . \]
  If $j = N - J$ then by definition of $\varphi^{\varepsilon}_j$ we see that
  $\varphi^{\varepsilon}_j \mathcal{F} f$ is a periodic function on $(2^N
  \mathbb{T})^d$. Hence we obtain the same formula (since $- j - J = - N$)
  \[ \lambda_{j, m} (f) \assign \int_{(2^N \mathbb{T})^d}
     (\varphi^{\varepsilon}_j \mathcal{F} f) (y) e^{2 \pi i 2^{- j - J} m
     \cdummy y} \mathd y = \Delta_j^{\varepsilon} f (2^{- j - J} m) . \]
  Therefore, we have derived the decomposition {\eqref{eq:d2}} with
  coefficients given by {\eqref{eq:d1}}.
  
  It remains to establish the equivalence of norms {\eqref{eq:d3}}. One
  direction is immediate, namely, for every $N - J \leqslant J_{\varepsilon}$
  we have
  \[ \sup_{- 1 \leqslant j \leqslant N - J} 2^{\alpha j} \sup_{m \in
     \mathbb{Z}^d} \rho (2^{- j - J} m) | \lambda_{j, m} (f) | = \sup_{- 1
     \leqslant j \leqslant N - J} 2^{\alpha j} \sup_{m \in \mathbb{Z}^d} \rho
     (2^{- j - J} m) | \Delta^{\varepsilon}_j f (2^{- j - J} m) | \]
  \[ \leqslant \sup_{- 1 \leqslant j \leqslant N - J} 2^{\alpha j} \sup_{x \in
     \Lambda_{\varepsilon}} \rho (x) | \Delta^{\varepsilon}_j f (x) | . \]
  Conversely, if $x \in \Lambda_{\varepsilon}$ belongs to the cube of size
  $2^{- j - J}$ centered at $2^{- j - J} m$, we write
  \begin{equation}
    | \Delta^{\varepsilon}_j f (x) | \leqslant | \Delta^{\varepsilon}_j f (x)
    - \Delta^{\varepsilon}_j f (2^{- j - J} m) | + | \Delta^{\varepsilon}_j f
    (2^{- j - J} m) |, \label{eq:25}
  \end{equation}
  Now we shall multiply the above inequality by $\rho (x)$ and estimate. To
  this end, we recall that due to the admissibility condition for polynomial
  weights there exists $\nu \geqslant 0$ and $c_1 > 0$ (depending only on
  $\rho$) such that
  \[ \frac{\rho (x)}{\rho (z)} \lesssim \big( 1 + \big| \sqrt{d} 2^{- j - J
     - 1} \big|^2 \big)^{\nu / 2} \lesssim c_1 \quad \tmop{whenever} \quad
     | x - z | \leqslant \sqrt{d} 2^{- j - J - 1} . \]
  In addition, to estimate the first term in {\eqref{eq:25}}, we recall that
  for $- 1 \leqslant j < N - J$ the Fourier transform of
  $\Delta^{\varepsilon}_j f$ is supported in a ball of radius proportional to
  $2^j$ hence by a computation similar to Bernstein's lemma (since by our
  construction $| x - 2^{- j - J} m | \leqslant \sqrt{d} 2^{- j - J - 1}$)
  \[ \rho (x) | \Delta^{\varepsilon}_j f (x) - \Delta^{\varepsilon}_j f (2^{-
     j - J} m) | \leqslant c_2 2^{- J - 1} \| \Delta^{\varepsilon}_j f
     \|_{L^{\infty, \varepsilon} (\rho)}, \]
  for some universal constant $c_2 > 0$ independent of $f$ and $\varepsilon$.
  If $j = N - J$ then $\Lambda_{\varepsilon}$ coincides with the lattice $2^{-
  j - J} \mathbb{Z}^d$ and therefore we do not need to do anything.
  Consequently it follows from {\eqref{eq:25}} that
  \[ \| \Delta^{\varepsilon}_j f \|_{L^{\infty, \varepsilon} (\rho)} \leqslant
     c_2 2^{- J - 1} \| \Delta^{\varepsilon}_j f \|_{L^{\infty, \varepsilon}
     (\rho)} + c_1 \sup_{m \in \mathbb{Z}^d} \rho (2^{- j - J} m) |
     \Delta^{\varepsilon}_j f (2^{- j - J} m) | . \]
  Hence, making $J \in \mathbb{N}_0$ possibly larger such that $c_2 2^{- J -
  1} < 1$, we may absorb the first term on the right hand side into the left
  hand side and the claim follows.
  \end{proof}
  

\begin{remark}
  Throughout the paper, the parameter $J \in \mathbb{N}_0$ is fixed as in Lemma~\ref{lem:equiv}. Consequently, from the condition $0 \leqslant N - J$ we
  obtain the necessary lower bound $N_0$ for $N$, or alternatively the upper
  bound for $\varepsilon = 2^{- N} \leqslant 2^{- N_0}$ and defines the set
  $\mathcal{A}$. These parameters remain fixed for the rest of the paper.
\end{remark}

\begin{remark}
  \label{rem:3}Note that the formulas {\eqref{eq:d1}}, {\eqref{eq:d2}} depend
  on the chosen partition of unity $(\varphi_j)_{j \geqslant - 1}$ and our
  construction of the associated periodic partitions of unity on
  $\hat{\Lambda}_{\varepsilon}$ via $\eqref{eq:p1} .$
\end{remark}

It follows from the previous lemma that we may identify $f \in B^{\alpha,
\varepsilon}_{\infty, \infty} (\rho)$ with its coefficients $(\lambda_{j, m}
(f))_{- 1 \leqslant j \leqslant N - J, m \in \mathbb{Z}^d} \in b^{\alpha,
\varepsilon}_{\infty, \infty} (\rho)$. This consideration leads us to the
definition of localization operators needed for the analysis of the $\Phi^4_3$
model. Although the principle idea is similar to Section 2.3 in {\cite{GH18}},
we present a different definition of the localizers here. It is based on the
equivalent description of the Besov spaces from Lemma~\ref{lem:equiv} and is
better suited for the discrete setting.

Given $(L_k)_{k \geqslant - 1} \subset (0, \infty)$ and $f \in \mathcal{S}'
(\Lambda_{\varepsilon})$ we define
\[ \UU_{>}^{\varepsilon} f \assign \left( \lambda_{j, m} \left(
   \UU^{\varepsilon}_{>} f \right) \right)_{- 1 \leqslant j \leqslant N - J, m
   \in \mathbb{Z}^d}, \qquad \UU_{\leqslant}^{\varepsilon} f \assign \left(
   \lambda_{j, m} \left( \UU_{\leqslant}^{\varepsilon} f \right) \right)_{- 1
   \leqslant j \leqslant N - J, m \in \mathbb{Z}^d} \]
where
\[ \lambda_{j, m} \left( \UU_{>}^{\varepsilon} f \right) \assign \left\{
   \begin{array}{lll}
     \lambda_{j, m} (f), &  & \tmop{if} | m | \sim 2^k \tmop{and} j > L_k \text{ for some } k\in \{-1,0,1,\dots\},\\
     0, &  & \tmop{otherwise},
   \end{array} \right. \]
\[ \lambda_{j, m} \left( \UU_{\leqslant}^{\varepsilon} f \right) \assign
   \left\{ \begin{array}{lll}
     \lambda_{j, m} (f), &  & \tmop{if} | m | \sim 2^k \tmop{and} j \leqslant
     L_k \text{ for some } k\in \{-1,0,1,\dots\},\\
     0, &  & \tmop{otherwise} .
   \end{array} \right. \]
We observe that by definition $f = \UU_{>}^{\varepsilon} f +
\UU_{\leqslant}^{\varepsilon} f$ and the localizers $\UU_{>}^{\varepsilon},
\UU_{\leqslant}^{\varepsilon}$ will only depend on $\varepsilon$ through the
cut-off of the coefficients $\lambda$ (and consequently on the construction of
the partition of unity on $\hat{\Lambda}_{\varepsilon}$, cf. Remark
\ref{rem:3}), whereas the sequence $(L_k)_{k \geqslant - 1}$ will be chosen
uniformly for all $\varepsilon \in \mathcal{A}$.

\begin{lemma}
  \label{lem:loc}Let $\rho$ be a weight. Let $\alpha, \beta, \gamma \in
  \mathbb{R}$ and $a, b, c \in \mathbb{R}$ such that $\alpha < \beta <
  \gamma$, $a < b < c$ and $r \assign (b - a) / (\beta - \alpha) = (c - b) /
  (\gamma - \beta) > 0$. Let $L > 0$ be given. There exists a sequence
  $(L_k)_{k \geqslant - 1}$ defining the above localizers such that
  \[ \left\| \UU^{\varepsilon}_{>} f \right\|_{B^{\alpha,
     \varepsilon}_{\infty, \infty} (\rho^a)} \lesssim 2^{- (\beta - \alpha) L}
     \| f \|_{B^{\beta, \varepsilon}_{\infty, \infty} (\rho^b)}, \]
  \[ \left\| \UU^{\varepsilon}_{\leqslant} f \right\|_{B^{\gamma,
     \varepsilon}_{\infty, \infty} (\rho^c)} \lesssim 2^{(\gamma - \beta) L}
     \| f \|_{B^{\beta, \varepsilon}_{\infty, \infty} (\rho^b)}, \]
  where the proportionality constants do not depend on $\varepsilon \in
  \mathcal{A}$. Moreover, the sequence $(L_k)_{k \geqslant - 1}$ depends only
  on $L, \rho$ and the ratio $r$.
\end{lemma}



\begin{proof}
  Since $\alpha < \beta$ and $a < b$,  Lemma~\ref{lem:equiv} yields
  \[ \left\| \UU_{>}^{\varepsilon} f \right\|_{B^{\alpha,
     \varepsilon}_{\infty, \infty} (\rho^a)} \lesssim \sup_{- 1 \leqslant j
     \leqslant N - J} 2^{\alpha j} \sup_{m \in \mathbb{Z}^d} \rho^a (2^{- j -
     J} m) \left| \lambda_{j, m} \left( \UU^{\varepsilon}_{>} f \right)
     \right| \]
  \[ = \sup_{k \geqslant - 1} \sup_{m \sim 2^k, L_k < j \leqslant N - J}
     2^{(\alpha - \beta) j} \rho^{a - b} (2^{- j - J} m) 2^{\beta j} \rho^b
     (2^{- j - J} m) | \lambda_{j, m} (f) | \]
  \[ \lesssim \| f \|_{B^{\beta, \varepsilon}_{\infty, \infty} (\rho^b)}
     \sup_{k \geqslant - 1} \sup_{m \sim 2^k, L_k < j \leqslant N - J}
     2^{(\alpha - \beta) j} \rho^{a - b} (2^{- j - J} m) \]
  \[ \lesssim \| f \|_{B^{\beta, \varepsilon}_{\infty, \infty} (\rho^b)}
     \sup_{k \geqslant - 1} 2^{(\alpha - \beta) L_k} \rho^{a - b} (2^k), \]
  where we used the fact that $a < b$, $2^{- j} < 2^{- L_k}$ and that the
  weight is decreasing to get
  \[ \rho^{a - b} (2^{- j - J} m) \lesssim \rho^{a - b} (2^{- L_k - J} 2^k)
     \lesssim \rho^{a - b} (2^k) . \]
  Now we set $c_k = - \log_2 \rho (2^k)$ to obtain
  \begin{equation}
    \left\| \UU_{>}^{\varepsilon} f \right\|_{B^{\alpha, \varepsilon}_{\infty,
    \infty} (\rho^a)} \lesssim \| f \|_{B^{\beta, \varepsilon}_{\infty,
    \infty} (\rho^b)} \sup_{k \geqslant - 1} 2^{- (\beta - \alpha) L_k + (b -
    a) c_k} . \label{eq:26}
  \end{equation}
  
  
  On the other hand, since $\gamma > \beta$ and $c > b$ we have by the same
  arguments
  \[ \left\| \UU^{\varepsilon}_{\leqslant} f \right\|_{B^{\gamma,
     \varepsilon}_{\infty, \infty} (\rho^c)} \lesssim \sup_{- 1 \leqslant j
     \leqslant N - J} 2^{\gamma j} \sup_{m \in \mathbb{Z}^d} \rho^c (2^{- j -
     J} m) \left| \lambda_{j, m} \left( \UU_{\leqslant}^{\varepsilon} f
     \right) \right| \]
  \[ = \sup_{k \geqslant - 1} \sup_{m \sim 2^k, - 1 \leqslant j \leqslant L_k
     \wedge (N - J)} 2^{(\gamma - \beta) j} \rho^{c - b} (2^{- j - J} m)
     2^{\beta j} \rho^b (2^{- j - J} m) | \lambda_{j, m} (f) | \]
  \begin{equation}
    \lesssim \| f \|_{B^{\beta, \varepsilon}_{\infty, \infty} (\rho^b)}
    \sup_{k \geqslant - 1} 2^{(\gamma - \beta) L_k - (c - b) c_k} .
    \label{eq:27}
  \end{equation}
  We see that if the weight is decreasing at infinity, then $c_k
  \rightarrow \infty$. From {\eqref{eq:26}} we obtain the condition $- (\beta
  - \alpha) L_k + (b - a) c_k = - (\beta - \alpha) L$ hence we shall choose
  $L_k = L + (b - a) c_k / (\beta - \alpha)$. Similarly, {\eqref{eq:27}}
  yields $(\gamma - \beta) L_k - (c - b) c_k = (\gamma - \beta) L$ hence $L_k
  = L + (c - b) c_k / (\gamma - \beta)$. Balancing these two conditions gives
  $(b - a) / (\beta - \alpha) = (c - b) / (\gamma - \beta)$ and completes the
  proof.
  \end{proof}
  
  
  \subsection{Duality and commutators}
  \label{s:l2}
  
In this section we define various commutators and establish suitable bounds.  We denote by $C_{\varepsilon}$  the operator introduced in
Lemma 4.4 {\cite{MP17}}, which for smooth functions satisfies
\begin{equation}\label{eq:ce}
 C_{\varepsilon} (f, g, h) = h \circ (f \prec g) - f (h \circ g) .
 \end{equation}
We recall that if $p, p_1, p_2 \in [1, \infty]$ and $\alpha, \beta, \gamma \in
\mathbb{R}$ are such that $\frac{1}{p} = \frac{1}{p_1} + \frac{1}{p_2}$,
$\alpha + \beta + \gamma > 0$ and $\beta + \gamma \neq 0$, then the following
bound holds
\begin{equation}
  \| C_{\varepsilon} (f, g, h) \|_{B^{\beta + \gamma, \varepsilon}_{p, \infty}
  (\rho_1 \rho_2 \rho_3)} \lesssim \| f \|_{B^{\alpha, \varepsilon}_{p_1,
  \infty} (\rho_1)} \| g \|_{B_{\infty, \infty}^{\beta, \varepsilon}
  (\rho_2)_{}} \| h \|_{B_{p_2, \infty}^{\beta, \varepsilon} (\rho_3)_{}} .
  \label{eq:comm}
\end{equation}
As the next step, we show that $g \succ$ is an approximate adjoint of $g
\circ$ in a suitable sense, as first noted in~\cite{gubinelli_semilinear_2018}.
  
  \begin{lemma}
  \label{lem:dual1}Let $\varepsilon \in \mathcal{A}$. Let $\alpha, \beta,
  \gamma \in \mathbb{R}$ be such that $\alpha, \gamma > 0$, $\beta + \gamma <
  0$ and $\alpha + \beta + \gamma > 0$ and let $\rho_1, \rho_2, \rho_3$ be
  weights and let $\rho = \rho_1 \rho_2 \rho_3$. There exists a bounded
  trilinear operator
  \[ D_{\rho, \varepsilon} (f, g, h) : H^{\alpha, \varepsilon} (\rho_1) \times
     \CC^{\beta, \varepsilon} (\rho_2) \times H^{\gamma, \varepsilon} (\rho_3)
     \rightarrow \mathbb{R} \]
  such that
  \[ | D_{\rho, \varepsilon} (f, g, h) | \lesssim \| f \|_{H^{\alpha,
     \varepsilon} (\rho_1)} \| g \|_{\CC^{\beta, \varepsilon} (\rho_2)} \| h
     \|_{H^{\gamma, \varepsilon} (\rho_3)} \]
  where the proportionality constant is independent of $\varepsilon$, and for
  smooth functions we have
  \[ D_{\rho, \varepsilon} (f, g, h) = \langle \rho f, g \circ h
     \rangle_{\varepsilon} - \langle \rho (f \prec g), h \rangle_{\varepsilon}
     . \]
\end{lemma}


\begin{proof}
  We define
  \[ D_{\rho, \varepsilon} (f, g, h) \assign \langle \rho, C_{\varepsilon} (f,
     g, h) \rangle_{\varepsilon} - \langle \rho, (f \prec g) \succ h
     \rangle_{\varepsilon} - \langle \rho, (f \prec g) \prec h
     \rangle_{\varepsilon}, \]
  where $C_{\varepsilon}$ was defined above. Hence the desired formula holds
  for smooth functions. By {\eqref{eq:comm}} and the paraproduct estimates we
  have
  \[ \| C_{\varepsilon} (f, g, h) \|_{B^{\beta + \gamma - \delta,
     \varepsilon}_{1, 1} (\rho)} \lesssim \| C_{\varepsilon} (f, g, h)
     \|_{B^{\beta + \gamma, \varepsilon}_{1, \infty} (\rho)} \lesssim \| f
     \|_{B^{\alpha, \varepsilon}_{2, \infty} (\rho_1)} \| g \|_{B^{\beta,
     \varepsilon}_{\infty, \infty} (\rho_2)} \| h \|_{B^{\gamma,
     \varepsilon}_{2, \infty} (\rho_3)}, \]
  \[ \| (f \prec g) \succ h \|_{B^{\beta - \delta, \varepsilon}_{1, 1} (\rho)}
     \lesssim \| (f \prec g) \succ h \|_{B^{\beta, \varepsilon}_{1, \infty}
     (\rho)} \lesssim \| f \|_{B^{\alpha, \varepsilon}_{2, \infty} (\rho_1)}
     \| g \|_{B^{\beta, \varepsilon}_{\infty, \infty} (\rho_2)} \| h
     \|_{B^{\gamma, \varepsilon}_{2, \infty} (\rho_3)}, \]
  \[ \| (f \prec g) \prec h \|_{B^{\beta + \gamma - \delta, \varepsilon}_{1,
     1} (\rho)} \lesssim \| (f \prec g) \prec h \|_{B^{\beta + \gamma,
     \varepsilon}_{1, \infty} (\rho)} \lesssim \| f \|_{B^{\alpha,
     \varepsilon}_{2, \infty} (\rho_1)} \| g \|_{B^{\beta,
     \varepsilon}_{\infty, \infty} (\rho_2)} \| h \|_{B^{\gamma,
     \varepsilon}_{2, \infty} (\rho_3)}, \]
  and the right hand side is estimated by
  \[ \| f \|_{B^{\alpha, \varepsilon}_{2, \infty} (\rho_1)} \| g \|_{B^{\beta,
     \varepsilon}_{\infty, \infty} (\rho_2)} \| h \|_{B^{\gamma,
     \varepsilon}_{2, \infty} (\rho_3)} \lesssim \| f \|_{B^{\alpha,
     \varepsilon}_{2, 2} (\rho_1)} \| g \|_{B^{\beta, \varepsilon}_{\infty,
     \infty} (\rho_2)} \| h \|_{B^{\gamma, \varepsilon}_{2, 2} (\rho_3)} . \]
  Consequently,
  \[ | D_{\rho, \varepsilon} (f, g, h) | \lesssim \| 1 \|_{B^{- \beta +
     \delta, \varepsilon}_{\infty, \infty}} \| f \|_{B^{\alpha,
     \varepsilon}_{2, 2} (\rho_1)} \| g \|_{B^{\beta, \varepsilon}_{\infty,
     \infty} (\rho_2)} \| h \|_{B^{\gamma, \varepsilon}_{2, 2} (\rho_3)} \]
  which completes the proof.
  \end{proof}


Next, we show several commutator estimates. To this end, $\Delta_{\varepsilon}$ denotes the discrete
Laplacian on $\Lambda_{\varepsilon}$ and we define the corresponding elliptic
and parabolic operators by $\Q_{\varepsilon} \assign m^{2} -
\Delta_{\varepsilon}$ and $\LL_{\varepsilon} \assign \partial_t +
\Q_{\varepsilon}$, where $m^{2} > 0$.

\begin{lemma}
  \label{lem:comm1}Let $\varepsilon \in \mathcal{A}$. Let $\alpha, \beta,
  \gamma \in \mathbb{R}$ such that $\alpha\in (0,1)$, $\beta + \gamma + 2 < 0$ and $\alpha +
  \beta + \gamma + 2 > 0$. Let $\rho_1, \rho_2, \rho_3$ be space weights and
  let $\rho_4, \rho_5, \rho_6$ be space-time weights. Then there exist bounded
  trilinear operators
  \[ \tilde{C}_{\varepsilon} : H^{\alpha, \varepsilon} (\rho_1) \times
     \CC^{\beta, \varepsilon} (\rho_2) \times \CC^{\gamma + \delta,
     \varepsilon} (\rho_3) \rightarrow H^{\beta + \gamma + 2, \varepsilon}
     (\rho_1 \rho_2 \rho_3), \]
  \[ \bar{C}_{\varepsilon} : C_T \CC^{\alpha, \varepsilon} (\rho_4) \times C_T
     \CC^{\beta, \varepsilon} (\rho_5) \times C_T \CC^{\gamma + \delta,
     \varepsilon} (\rho_6) \rightarrow C_T \CC^{\beta + \gamma + 2,
     \varepsilon} (\rho_4 \rho_5 \rho_6) \]
  such that for every $\delta > 0$
  \[ \| \tilde{C}_{\varepsilon} (f, g, h) \|_{H^{\beta + \gamma + 2,
     \varepsilon} (\rho_1 \rho_2 \rho_3)} \lesssim \| f \|_{H^{\alpha,
     \varepsilon} (\rho_1)} \| g \|_{\CC^{\beta, \varepsilon} (\rho_2)} \| h
     \|_{\CC^{\gamma + \delta, \varepsilon} (\rho_3)}, \]
  \[ \| \bar{C}_{\varepsilon} (f, g, h) \|_{C_T \CC^{\beta + \gamma + 2,
     \varepsilon} (\rho_4 \rho_5 \rho_6)} \]
  \[ \lesssim \big( \| f \|_{C_T \CC^{\alpha, \varepsilon} (\rho_4)} + \| f
     \|_{C_T^{\alpha / 2} L^{\infty, \varepsilon} (\rho_4)} \big) \| g
     \|_{C_T \CC^{\beta, \varepsilon} (\rho_5)} \| h \|_{C_T \CC^{\gamma +
     \delta, \varepsilon} (\rho_6)}, \]
  where the proportionality constants are independent of $\varepsilon$, and
  for smooth functions we have
  \begin{equation}
    \tilde{C}_{\varepsilon} (f, g, h) = h \circ \Q_{\varepsilon}^{- 1} (f
    \prec g) - f \left( h \circ \Q_{\varepsilon}^{- 1} g \right), \label{eq:9}
  \end{equation}
  \[ \bar{C}_{\varepsilon} (f, g, h) = h \circ \LL_{\varepsilon}^{- 1} (f
     \prec g) - f \left( h \circ \LL_{\varepsilon}^{- 1} g \right) . \]
\end{lemma}








\begin{proof}
  First, we define
  \[ \tilde{C}_{\varepsilon} (f, g, h) \assign h \circ \left[
     \Q_{\varepsilon}^{- 1} (f \prec g) - f \prec \Q_{\varepsilon}^{- 1} g
     \right] + C_{\varepsilon} \left( f, \Q_{\varepsilon}^{- 1} g, h \right),
  \]
  where $C_{\varepsilon}$ was introduced above. Hence for smooth functions we
  obtain the desired formula {\eqref{eq:9}}. Moreover, by {\eqref{eq:comm}}
  the operator $C_{\varepsilon}$ can be estimated (uniformly in $\varepsilon$)
  for $\delta > 0$ as
  \[ \left\| C_{\varepsilon} \left( f, \Q_{\varepsilon}^{- 1} g, h \right)
     \right\|_{H^{\beta + \gamma + 2, \varepsilon} (\rho_1 \rho_2 \rho_3)}
     \lesssim \left\| C_{\varepsilon} \left( f, \Q_{\varepsilon}^{- 1} g, h
     \right) \right\|_{B_{2, \infty}^{\beta + \gamma + 2 + \delta,
     \varepsilon} (\rho_1 \rho_2 \rho_3)} \]
  \[ \lesssim \| f \|_{B_{2, \infty}^{\alpha, \varepsilon} (\rho_1)} \| g
     \|_{\CC^{\beta, \varepsilon} (\rho_2)} \| h \|_{\CC^{\gamma + \delta,
     \varepsilon} (\rho_3)} \lesssim \| f \|_{H^{\alpha, \varepsilon}
     (\rho_1)} \| g \|_{\CC^{\beta, \varepsilon} (\rho_2)} \| h
     \|_{\CC^{\gamma + \delta, \varepsilon} (\rho_3)} . \]
  For the first term in $\tilde{C}_{\varepsilon}$ we write
  \[ \Q_{\varepsilon}^{- 1} (f \prec g) - f \prec \Q_{\varepsilon}^{- 1} g =
     \Q_{\varepsilon}^{- 1} \left[ f \prec \Q_{\varepsilon}
     \Q_{\varepsilon}^{- 1} g - \Q_{\varepsilon} \left( f \prec
     \Q_{\varepsilon}^{- 1} g \right) \right] \]
  and as a consequence
  \[ \left\| h \circ \left[ \Q_{\varepsilon}^{- 1} (f \prec g) - f \prec
     \Q_{\varepsilon}^{- 1} g \right] \right\|_{H^{\alpha + \beta + \gamma +
     2, \varepsilon} (\rho_1 \rho_2 \rho_3)} \]
  \[ \lesssim \| h \|_{\CC^{\gamma + \delta, \varepsilon} (\rho_3)} \left\| f
     \prec \Q_{\varepsilon} \Q_{\varepsilon}^{- 1} g - \Q_{\varepsilon} \left(
     f \prec \Q_{\varepsilon}^{- 1} g \right) \right\|_{H^{\alpha + \beta -
     \delta, \varepsilon} (\rho_1 \rho_2)} . \]
  Finally, we observe that due to an argument similar to Lemma 4.9 
  {\cite{MP17}} we may control
  \[ \nabla_{\varepsilon} f \prec \nabla_{\varepsilon} g \assign
     \frac12\big(\Delta_{\varepsilon} (f \prec g) - \Delta_{\varepsilon} f \prec g - f
     \prec \Delta_{\varepsilon} g\big), \]
 hence
  \[ \left\| f \prec \Q_{\varepsilon} \Q_{\varepsilon}^{- 1} g -
     \Q_{\varepsilon} \left( f \prec \Q_{\varepsilon}^{- 1} g \right)
     \right\|_{H^{\alpha + \beta - \delta, \varepsilon} (\rho_1 \rho_2)} \]
  \[ \lesssim \left\| f \prec \Q_{\varepsilon} \Q_{\varepsilon}^{- 1} g -
     \Q_{\varepsilon} \left( f \prec \Q_{\varepsilon}^{- 1} g \right)
     \right\|_{B_{2, \infty}^{\alpha + \beta, \varepsilon} (\rho_1 \rho_2)}
     \lesssim \| f \|_{B_{2, \infty}^{\alpha, \varepsilon} (\rho_1)} \| g
     \|_{\CC^{\beta, \varepsilon} (\rho_2)} \]
  \[ \lesssim \| f \|_{H_{}^{\alpha, \varepsilon} (\rho_1)} \| g
     \|_{\CC^{\beta, \varepsilon} (\rho_2)} . \]
  We proceed similarly for the parabolic commutator $\bar{C}_{\varepsilon}$,
  but include additionally a modified paraproduct given by
  \[ f \precprec g \assign \sum_{1 \leqslant i, j \leqslant N - J, i < j - 1}
     \Delta^{\varepsilon}_i Q_i f \Delta^{\varepsilon}_j g, \]
  where
  \[ Q_i f (t) = \int_{\mathbb{R}} 2^{2 i} Q (2^{2 i} (t - s)) f ((s \vee 0)
     \wedge T) \mathd s \]
  for some smooth, nonnegative, compactly supported function $Q : \mathbb{R}
  \rightarrow \mathbb{R}$ that integrates to $1$. Namely, we define
  \[ \bar{C}_{\varepsilon} (f, g, h) \assign h \circ \left[
     \LL_{\varepsilon}^{- 1} (f \precprec g) - f \precprec
     \LL_{\varepsilon}^{- 1} g \right] + h \circ \left[ \LL_{\varepsilon}^{-
     1} (f \prec g - f \precprec g) \right] \]
  \[ + h \circ \left[ f \precprec \LL_{\varepsilon}^{- 1} g - f \prec
     \LL_{\varepsilon}^{- 1} g \right] + C_{\varepsilon} \left( f,
     \LL_{\varepsilon}^{- 1} g, h \right), \]
  and observe that for smooth functions
  \[ \bar{C}_{\varepsilon} (f, g, h) = h \circ \left[ \LL_{\varepsilon}^{- 1}
     (f \prec g) - f \prec \LL_{\varepsilon}^{- 1} g \right] + \left[ h \circ
     \left( f \prec \LL_{\varepsilon}^{- 1} g \right) - f \left( h \circ
     \LL_{\varepsilon}^{- 1} g \right) \right] \]
  \[ = h \circ \LL_{\varepsilon}^{- 1} (f \prec g) - f \left( h \circ
     \LL_{\varepsilon}^{- 1} g \right), \]
  and the desired bound follows from Lemma 4.9 in {\cite{MP17}} and
  {\eqref{eq:comm}}.
  \end{proof}



\subsection{Extension operators}

\label{s:ext}


In order to construct the Euclidean quantum field theory as a
limit of lattice approximations, we  need a suitable extension operator
that allows to extend distributions defined on the lattice
$\Lambda_{\varepsilon}$ to the full space $\mathbb{R}^d$.
To this end, we proceed as in Section 2.4, page 2072 in \cite{MP17}. Namely, let $\psi$ be a smooth and radially symmetric smear function satisfying the properties 1., 2., 3. on page 2072 in \cite{MP17} and let $\psi^{\varepsilon}(\cdot)=\psi(\varepsilon\cdot)$. We define
$$
\mathcal{E}^{\varepsilon}f:=\mathcal{F}_{\mathbb{R}^{d}}^{-1}\big(\psi^{\varepsilon}(\mathcal{F}_{\Lambda_{\varepsilon}}f)_{\rm ext}\big),\qquad f\in\mathcal{S}'(\Lambda_{\varepsilon}),
$$
where  $(\cdot)_{\rm ext}:\mathcal{S}'((\varepsilon^{-1}\mathbb{T})^{d})\to \mathcal{S}'(\mathbb{R}^{d})$ is the periodic extension operator defined by
$$
g_{\rm ext}(\varphi):=g\left(\sum_{k\in (\varepsilon^{-1}\mathbb{Z})^{d}}\varphi(\cdot-k)\right),\qquad \varphi\in\mathcal{S}(\mathbb{R}^{d}).
$$
With the definition of the Dirac comb distribution $f_{\rm dir}\in\mathcal{S}'(\mathbb{R}^{d})$ as in (10) in \cite{MP17}
$$
f_{\rm dir}=\varepsilon^{d}\sum_{k\in\Lambda_{\varepsilon}}f(k)\delta(\cdot - k),\qquad f\in \mathcal{S}'(\Lambda_{\varepsilon}),
$$
it was observed in (14) in \cite{MP17} that
$$
(\mathcal{F}_{\Lambda_{\varepsilon}}f)_{\rm ext}=\mathcal{F}_{\mathbb{R}^{d}}(f_{\rm dir}).
$$
Hence
\[ \mathcal{E}^{\varepsilon} f = \mathcal{F}_{\mathbb{R}^d}^{- 1} (\psi^{\varepsilon}
   (\mathcal{F}_{\Lambda_{\varepsilon}} f)_{\tmop{ext}}) = (\mathcal{F}_{\mathbb{R}^d}^{- 1}
   \psi^{\varepsilon}) \ast \mathcal{F}_{\mathbb{R}^d}^{- 1} \mathcal{F}_{\mathbb{R}^d}
   (f_{\tmop{dir}}) \backassign w^{\varepsilon} \ast f_{\tmop{dir}} =
   w^{\varepsilon} \ast_{\varepsilon} f, \]
where $w^{\varepsilon}(\cdot)=\mathcal{F}_{\mathbb{R}^d}^{- 1}
   \psi^{\varepsilon}(\cdot)=\varepsilon^{-d}\mathcal{F}_{\mathbb{R}^d}^{- 1}
   \psi(\varepsilon^{-1}\cdot)=:\varepsilon^{-d}w(\varepsilon^{-1}\cdot)\in\mathcal{S}(\mathbb{R}^{d})$. 
   With a slight abuse of notation we used the same notation  $\ast_{\varepsilon}$ as for the convolution on the lattice
$\Lambda_{\varepsilon}$ to denote the operation
$$
(w^{\varepsilon} \ast_{\varepsilon} f) (x):=\varepsilon^{d}\sum_{y\in\Lambda_{\varepsilon}}w^{\varepsilon}(x-y)f(y),\qquad x\in \mathbb{R}^{d},
$$
which defines a function on the full space $\mathbb{R}^{d}$.
Note that since $\psi$ is radially symmetric,  $w$ is radially symmetric as well.
   
 
%
% fix a smooth, compactly supported and radially symmetric
%nonnegative function $w \in C^{\infty}_c (\mathbb{R}^d)$ such that
%$\tmop{supp} w \subset B_{1 / 2}$ where $B_{1 / 2} \subset \mathbb{R}^d$ is
%the ball centered at $0$ with radius $1 / 2$ and $\int_{\mathbb{R}^d} w (x)
%\mathd x = 1$. Let $w^{\varepsilon} (\cdummy) \assign \varepsilon^{- d} w
%(\varepsilon^{- 1} \cdummy)$ and define the extension operator
%$\mathcal{E}^{\varepsilon}$ by
%\[ \mathcal{E}^{\varepsilon} f  \assign w^{\varepsilon} \ast_{\varepsilon} f,
%   \qquad f \in \mathcal{S}' (\Lambda_{\varepsilon}). \]



The following  result is Lemma 2.24 in \cite{MP17}.

\begin{lemma}
  \label{lem:ext}Let $\alpha \in \mathbb{R}$, $p, q \in [1, \infty]$ and let
  $\rho$ be a weight. Then the operators
  \[ \mathcal{E}^{\varepsilon} : B^{\alpha, \varepsilon}_{p, q} (\rho)
     \rightarrow B^{\alpha}_{p, q} (\rho) \]
  are bounded uniformly in $\varepsilon$.
\end{lemma}



  
  
\subsection{A Schauder estimate}

In this section we establish a suitable Schauder-type estimate needed in Section~\ref{s:chi-reg}.


\begin{lemma}
  \label{lem:Pt}Let $\rho$ be a weight and let $P^{\varepsilon}_t = e^{t
  (\Delta_{\varepsilon} - m^2)}$ denote the semigroup generated by
  $\Delta_{\varepsilon} - m^2$. Then there exists $c > 0$ uniform in
  $\varepsilon$ such that for all $- 1 \leqslant j \leqslant N - J$
  \[ \| P^{\varepsilon}_t \Delta^{\varepsilon}_j f \|_{L^{1, \varepsilon}
     (\rho)} \lesssim e^{- t (m^2 + c 2^{2 j})} \| \Delta^{\varepsilon}_j f
     \|_{L^{1, \varepsilon} (\rho)}, \]
  where the proportionality constant does not depend on $\varepsilon$ and $t
  \geqslant 0$.
\end{lemma}

\begin{proof}
  Recall that the discrete Laplacian $\Delta_{\varepsilon}$ acts in the
  Fourier space as
  \[ \mathcal{F} (e^{- t (\Delta_{\varepsilon} - m^2)} f) (k) = e^{- t
     l_{\varepsilon} (k)} \hat{f} (k), \]
  where
  \[ l_{\varepsilon} (k) = m^2 + 4 \sum_{j}\sin^2 (\varepsilon \pi k_{j}) /
     \varepsilon^2 . \]
  Consequently, for $- 1 \leqslant j \leqslant N - J$ we have using the fact
  that $\mathcal{F}^{- 1} (g h) =\mathcal{F}_{\mathbb{R}^d}^{- 1} (g)
  \ast_{\varepsilon} \mathcal{F}^{- 1} (h)$ (where $\mathcal{F}^{- 1}$ denotes
  the inverse Fourier transform on the lattice $\Lambda_{\varepsilon}$) we
  obtain
  
  
  \[ \Delta^{\varepsilon}_j [e^{t (m^2 - \Delta_{\varepsilon})} f] = [2^{j d}
     V_j (2^j \cdummy)] \ast_{\varepsilon} \Delta^{\varepsilon}_j f, \]
  where
  \[ V_j (x) \assign \int_{\mathbb{R}^d} e^{i 2 \pi x \cdummy \xi} e^{- t
     l_{\varepsilon} (2^j \xi)} \bar{\varphi} (\xi) \mathd \xi, \]
  where $\bar{\varphi}$ is obtained by a rescaling of $\bar{\varphi}_j =
  \sum_{- 1 \leqslant i < \infty ; i \sim j} \varphi_i$. Next, for $M \in
  \mathbb{N}$ we want to show that
  \begin{equation}
    | (1 + | 2 \pi x |^2)^M V_j (x) | \lesssim e^{- t (m^2 + c 2^{2 j})},
    \qquad x \in \mathbb{R}^d . \label{eq:v}
  \end{equation}
  Indeed, with this in hand we may apply Lemma~\ref{lem:young} to deduce the
  claim.
  
  In order to show {\eqref{eq:v}} we compute
  \[ (1 + | 2 \pi x |^2)^M V_j (x) = \int_{\mathbb{R}^d} [(1 - \Delta_{\xi})^M
     e^{i 2 \pi x \cdummy \xi}] e^{- t l_{\varepsilon} (2^j \xi)}
     \bar{\varphi} (\xi) \mathd \xi \]
  \[ = \int_{\mathbb{R}^d} e^{i 2 \pi x \cdummy \xi} (1 - \Delta_{\xi})^M
     [e^{- t l_{\varepsilon} (2^j \xi)} \bar{\varphi} (\xi)] \mathd \xi \]
  where for a multiindex $\alpha \in \mathbb{N}^d$
  \[ \partial_{\xi}^{\alpha} e^{- t l_{\varepsilon} (2^j \xi)} = e^{- t
     l_{\varepsilon} (2^j \xi)} \sum_{0 \leqslant | \beta | \leqslant | \alpha
     |} c_{\alpha, \beta} \partial_{\xi}^{\beta} l_{\varepsilon} (2^j \xi) \]
  therefore using the bounds from Lemma 3.5 in {\cite{MP17}} we obtain
  \[ | \partial_{\xi}^{\alpha} e^{- t l_{\varepsilon} (2^j \xi)} | \lesssim
     e^{- t m^2} e^{- 2 t c (2^j \xi)^2} \sum_{0 \leqslant | \beta | \leqslant
     | \alpha |} \varepsilon^{(| \beta | - 2) \vee 0} (1 + | 2^j \xi |^2)
     \lesssim e^{- t m^2} e^{- t c (2^j \xi)^2} . \]
  Therefore
  \[ | (1 + | 2 \pi x |^2)^M V_j (x) | \lesssim \int_{\mathbb{R}^d} e^{- t c
     (2^j \xi)^2} \bar{\varphi} (\xi) \mathd \xi \lesssim e^{- t m^2} e^{- t c
     2^{2 j}} \]
  and {\eqref{eq:v}} is proven.
\end{proof}

\begin{lemma}
  \label{lem:reg}Let $\alpha \in \mathbb{R}$ and let $\rho$ be a weight. Let
  $v$ solve
  \[ \LL_{\varepsilon} v = f, \qquad v (0) = v_0 . \]
  Then
  \[ \| v \|_{L_T^1 B_{1, 1}^{\alpha, \varepsilon} (\rho)} \lesssim \| v_0
     \|_{B^{\alpha - 2, \varepsilon}_{1, 1} (\rho)} + \| f \|_{L^1_T B_{1,
     1}^{\alpha - 2, \varepsilon} (\rho)}, \]
  where the proportionality constant does not depend on $T$ and $\varepsilon$.
\end{lemma}

\begin{proof}
  Applying the Littlewood--Paley projectors we obtain
  \[ \Delta^{\varepsilon}_j v (t) = P^{\varepsilon}_t \Delta^{\varepsilon}_j
     v_0 + \int_0^t P^{\varepsilon}_{t - s} \Delta^{\varepsilon}_j f (s)
     \mathd s. \]
  Hence according to Lemma~\ref{lem:Pt} there exists $c > 0$ such that for $-
  1 \leqslant j \leqslant N - J$ and uniformly in $T > 0$ and $\varepsilon$
  \[ \| v \|_{L_T^1 B_{1, 1}^{\alpha, \varepsilon} (\rho)} = \int_0^T \sum_{-
     1 \leqslant j \leqslant N - J} 2^{\alpha j} \| \Delta^{\varepsilon}_j v
     (t) \|_{L^{1, \varepsilon} (\rho)} \mathd t \leqslant \int_0^T \sum_{- 1
     \leqslant j \leqslant N - J} 2^{\alpha j} \| P^{\varepsilon}_t
     \Delta^{\varepsilon}_j v_0 \|_{L^{1, \varepsilon} (\rho)} \mathd t \]
  \[ + \int_0^T \sum_{- 1 \leqslant j \leqslant N - J} 2^{\alpha j} \int_0^t
     \| P^{\varepsilon}_{t - s} \Delta^{\varepsilon}_j f (s) \|_{L^{1,
     \varepsilon} (\rho)} \mathd s \mathd t \]
  \[ \leqslant \sum_{- 1 \leqslant j \leqslant N - J} 2^{\alpha j}
     \int_0^{\infty} e^{- t (m^2 + c 2^{2 j})} \mathd t \|
     \Delta^{\varepsilon}_j v_0 \|_{L^{1, \varepsilon} (\rho)} \]
  \[ + \sum_{- 1 \leqslant j \leqslant N - J} 2^{\alpha j} \int_0^T \left[
     \int_0^{\infty} e^{- (t - s) (m^2 + c 2^{2 j})} \mathd t \right] \|
     \Delta^{\varepsilon}_j f (s) \|_{L^{1, \varepsilon} (\rho)} \mathd s \]
  \[ \lesssim \sum_{- 1 \leqslant j \leqslant N - J} 2^{(\alpha - 2) j} \|
     \Delta_j v_0 \|_{L^{1, \varepsilon} (\rho)} + \sum_{- 1 \leqslant j
     \leqslant N - J} 2^{(\alpha - 2) j} \int_0^T \| \Delta^{\varepsilon}_j f
     (s) \|_{L^{1, \varepsilon} (\rho)} \mathd s \]
  \[ = \| v_0 \|_{B^{\alpha - 2, \varepsilon}_{1, 1} (\rho)} + \| f \|_{L^1_T
     B_{1, 1}^{\alpha - 2, \varepsilon} (\rho)} . \]
\end{proof}

  
  
\subsection{Regularity of $\chi_{M,\varepsilon}$}
\label{s:chi-reg}

Finally, we proceed with the proof of the proof of Proposition \ref{prop:reg}. 
\medskip

\noindent\textbf{Proof of Proposition \ref{prop:reg} }
  For notational simplicity we fix the parameter $M$ and omit the dependence
  of the various distributions on $M$ throughout the proof. In addition, the $\lambda$-dependent constants  are always bounded uniformly over  $\lambda\in [0,\lambda_{0}]$ for every  $\lambda_{0}>0$.
  
  In view of
  {\eqref{eq:chi1}} we obtain
  \[ \| \rho^{2 + \sigma} \chi_{\varepsilon} \|_{L_T^{\infty} L^{2,
     \varepsilon}} \leqslant \| \rho^2 \phi_{\varepsilon} \|_{L_T^{\infty}
     L^{2, \varepsilon}} + \| \rho^{2 + \sigma} ( 3 \lambda
     \ttwoone{X_{\varepsilon}} \succ \phi_{\varepsilon} )
     \|_{L_T^{\infty} L^{2, \varepsilon}} \leqslant C_{\lambda}\| \rho^2
     \phi_{\varepsilon} \|_{L_T^{\infty} L^{2, \varepsilon}} Q_{\rho}
     (\mathbb{X}_{\varepsilon}), \]
  where, by Theorem~\ref{th:energy-estimate},
  \begin{equation*}
  \begin{aligned}
  \| \rho^2 \phi_{\varepsilon} (t) \|_{L^{2, \varepsilon}}^2 & \leqslant 
     C_{t,\lambda} Q_{\rho} (\mathbb{X}_{\varepsilon})
     + \| \rho^2 \phi_{ \varepsilon} (0)\|_{L^{2, \varepsilon}}^2.
  \end{aligned}
  \end{equation*}
  Thus
  \begin{equation}
    \| \rho^{2 + \sigma} \chi_{\varepsilon} \|_{L_T^{\infty} L^{2,
    \varepsilon}} \leqslant C_{T,\lambda}Q_{\rho} (\mathbb{X}_{\varepsilon}) (1  +\|
    \rho^2 \phi_{\varepsilon} (0)\|_{L^{2, \varepsilon}}) . \label{eq:chi}
  \end{equation}
  Next, we intend to apply Lemma~\ref{lem:reg} to {\eqref{eq:chi11}} in the
  form
  \[ \| \rho^4 \chi_{\varepsilon} \|_{L^1_T B_{1, 1}^{1 + 3 \kappa,
     \varepsilon}} \lesssim \| \rho^4 \chi_{\varepsilon} (0) \|_{B_{1, 1}^{- 1
     + 3 \kappa, \varepsilon}} + \left\| \rho^4 \LL_{\varepsilon}
     \chi_{\varepsilon} \right\|_{L_T^1 B_{1, 1}^{- 1 + 3 \kappa,
     \varepsilon}}. \]
In view of
  the second term on the  right hand side of {\eqref{eq:chi11}} we shall
  therefore estimate $U_{\varepsilon}$ in $B_{1, 1}^{- 1 + 3 \kappa,
  \varepsilon} (\rho^{4 - \sigma})$ as the weight $\rho^{\sigma}$ will be lost
  to control $\ttwoone{X}$. Let us first show how to bound the terms that
  contain higher powers of $\phi$, all the other terms being straightforward.
  By paraproduct estimates Lemma~\ref{lem:mult} and Lemma~\ref{lem:15}, we
  obtain
  \[ \| \rho^{4 - \sigma} \lambda  X_{\varepsilon} \phi_{\varepsilon}^2 \|_{B_{1,
     1}^{- 1 + 3 \kappa, \varepsilon}} \lesssim \lambda  \| \rho^{\sigma}
     X_{\varepsilon} \|_{\CC^{- 1 / 2 - \kappa, \varepsilon}} \| \rho^{4 - 2
     \sigma} \phi_{\varepsilon}^2 \|^{}_{B_{1, 1}^{1 / 2 + 2 \kappa,
     \varepsilon}} \]
  \[ \lesssim \lambda  \| \rho^{\sigma} X_{\varepsilon} \|_{\CC^{- 1 / 2 - \kappa,
     \varepsilon}} \| \rho^{1 + \iota} \phi_{\varepsilon} \|_{L^{2,
     \varepsilon}} \| \rho^2 \phi_{\varepsilon} \|_{H^{1 / 2 + 3 \kappa,
     \varepsilon}} \leqslant \lambda  Q_{\rho} (\mathbb{X}_{\varepsilon}) \| \rho
     \phi_{\varepsilon} \|_{L^{4, \varepsilon}} \| \rho^2 \phi_{\varepsilon}
     \|_{H^{1 - 2 \kappa, \varepsilon}} \]
  while   
  \[ \| \rho^{4 - \sigma} 3\lambda  Y_{\varepsilon} \phi_{\varepsilon}^2 \|_{B_{1,
     1}^{- 1 + 3 \kappa, \varepsilon}} \lesssim \lambda  \| \rho^{\sigma}
     Y_{\varepsilon} \|_{\CC^{1 / 2 - \kappa, \varepsilon}} \| \rho^{4 - 2
     \sigma} \phi_{\varepsilon}^2 \|_{B^{\kappa, \varepsilon}_{1, 1}} \]
  \[ \lesssim \lambda \| \rho^{\sigma} Y_{\varepsilon} \|_{\CC^{1 / 2 - \kappa,
     \varepsilon}} \| \rho^{1 + \iota} \phi_{\varepsilon} \|_{L^{2,
     \varepsilon}} \| \rho^2 \phi_{\varepsilon} \|_{H^{2 \kappa, \varepsilon}}
     \leqslant \lambda^2 Q_{\rho} (\mathbb{X}_{\varepsilon}) \| \rho \phi_{\varepsilon}
     \|_{L^{4, \varepsilon}} \| \rho^2 \phi_{\varepsilon} \|_{H^{1 - 2 \kappa,
     \varepsilon}}, \]
  and by interpolation for $\theta = \frac{1 - 4 \kappa}{1 - 2 \kappa}$
  \[ \| \rho^{4 - \sigma} \lambda \phi_{\varepsilon}^3 \|_{B_{1, 1}^{- 1 + 3 \kappa,
     \varepsilon}} \lesssim \lambda \| \rho^{4 - \sigma} \phi_{\varepsilon}^3
     \|_{B_{1, 1}^{\kappa, \varepsilon}} \lesssim \lambda  \| \rho \phi_{\varepsilon}
     \|_{L^{4, \varepsilon}}^2 \| \rho^{2 - \sigma} \phi_{\varepsilon}
     \|_{H^{2 \kappa, \varepsilon}} \]
  \[ \lesssim \lambda  \| \rho \phi_{\varepsilon} \|_{L^{4, \varepsilon}}^2 \| \rho^{1
     + \iota} \phi_{\varepsilon} \|_{L^{2, \varepsilon}}^{\theta} \| \rho^2
     \phi_{\varepsilon} \|_{H^{1 - 2 \kappa, \varepsilon}}^{1 - \theta}
     \lesssim\lambda  \| \rho \phi_{\varepsilon} \|_{L^{4, \varepsilon}}^{2 + \theta}
     \| \rho^2 \phi_{\varepsilon} \|_{H^{1 - 2 \kappa, \varepsilon}}^{1 -
     \theta} . \]
  Consequently, we use the embeddings $B_{2, 2}^{\alpha + \kappa, \varepsilon}
  (\rho^{2 + \beta}) \subset B_{1, 1}^{\alpha, \varepsilon} (\rho^{4 -
  \sigma})$ and $B_{\infty, \infty}^{\alpha + \kappa, \varepsilon}
  (\rho^{\beta}) \subset B_{1, 1}^{\alpha, \varepsilon} (\rho^{4 - \sigma})$
  for $\alpha \in \mathbb{R}$ (provided the weight possesses enough
  integrability and $\beta, \sigma > 0$ are sufficiently small). We deduce
  \[ \begin{aligned}
       \| \rho^{4 - \sigma} U_{\varepsilon} \|_{B_{1, 1}^{- 1 + 3 \kappa,
       \varepsilon}} & \lesssim  \lambda^2  \| \ttwothreer{\rho^{\sigma}
       \tilde{X}_{\varepsilon}} \|_{\CC^{- \kappa, \varepsilon}} \| \rho^2
       \phi_{\varepsilon} \|_{H^{1 - 2 \kappa, \varepsilon}} + \lambda^2 | \log t | \|
       \rho^2 \phi_{\varepsilon} \|_{H^{1 - 2 \kappa, \varepsilon}}\\
       &  \quad + \lambda^2 \| \rho^{\sigma} \llbracket X_{\varepsilon}^2 \rrbracket
       \|_{\CC^{- 1 - \kappa, \varepsilon}} \| \rho^{\sigma}
       \ttwoone{X_{\varepsilon}} \|_{\CC^{1 - \kappa, \varepsilon}} \| \rho^2
       \phi_{\varepsilon} \|_{H^{1 - 2 \kappa, \varepsilon}}\\
       & \quad  + \lambda \| \rho^{\sigma} \llbracket X_{\varepsilon}^2 \rrbracket
       \|_{\CC^{- 1 - \kappa, \varepsilon}} \| \rho^{4 - 2 \sigma}
       \chi_{\varepsilon} \|_{B_{1, 1}^{1 + 2 \kappa, \varepsilon}} + \lambda^2 \|
       \rho^{\sigma} Z_{\varepsilon} \|_{\CC^{- 1 / 2 - \kappa,
       \varepsilon}}\\
       & \quad +\lambda \| \rho^{\sigma} \llbracket X_{\varepsilon}^2 \rrbracket
       \|_{\CC^{- 1 - \kappa, \varepsilon}} \left( \| \rho^{\sigma}
       Y_{\varepsilon} \|_{\CC^{1 / 2 - \kappa, \varepsilon}} + \| \rho^2
       \phi_{\varepsilon} \|_{H^{1 - 2 \kappa, \varepsilon}} \right)\\
       &  \quad + \lambda (1+\lambda \| \rho^{\sigma} \llbracket X_{\varepsilon}^2 \rrbracket
       \|_{\CC^{- 1 - \kappa, \varepsilon}} ) \| \rho^{\sigma} \llbracket X_{\varepsilon}^2 \rrbracket
       \|_{\CC^{- 1 - \kappa, \varepsilon}} \| \rho^{\sigma} Y_{\varepsilon}
       \|_{\CC^{1 / 2 - \kappa, \varepsilon}} \\
       & \quad +\lambda \| \rho^{\sigma}
       X_{\varepsilon} Y_{\varepsilon}^2 \|_{\CC^{- 1 / 2 - \kappa,
       \varepsilon}}  + \lambda\| \rho^{\sigma} X_{\varepsilon} Y_{\varepsilon} \|_{\CC^{- 1 /
       2 - \kappa, \varepsilon}} \| \rho^2 \phi_{\varepsilon} \|_{H^{1 - 2
       \kappa, \varepsilon}}\\
       & \quad  + \lambda\| \rho^{\sigma} X_{\varepsilon} \|_{\CC^{- 1 / 2 - \kappa,
       \varepsilon}} \| \rho \phi_{\varepsilon} \|_{L^{4, \varepsilon}} \|
       \rho^2 \phi_{\varepsilon} \|_{H^{1 - 2 \kappa, \varepsilon}} +\lambda \|
       \rho^{\sigma} Y_{\varepsilon} \|_{\CC^{1 / 2 - \kappa,
       \varepsilon}}^3\\
       &  \quad +\lambda \| \rho^{\sigma} Y_{\varepsilon} \|^2_{\CC^{1 / 2 - \kappa,
       \varepsilon}} \| \rho \phi_{\varepsilon} \|_{L^{4, \varepsilon}} +\lambda \|
       \rho^{\sigma} Y_{\varepsilon} \|_{\CC^{1 / 2 - \kappa, \varepsilon}} \|
       \rho \phi_{\varepsilon} \|_{L^{4, \varepsilon}} \| \rho^2
       \phi_{\varepsilon} \|_{H^{1 - 2 \kappa, \varepsilon}}\\
       & \quad  +\lambda \| \rho \phi_{\varepsilon} \|_{L^{4, \varepsilon}}^{2 + \theta}
       \| \rho^2 \phi_{\varepsilon} \|^{1 - \theta}_{H^{1 - 2 \kappa,
       \varepsilon}}
       \end{aligned}\]
       \[\begin{aligned}
       & \leqslant  | \log t | (\lambda^3 Q_{\rho} (\mathbb{X}_{\varepsilon}) +\lambda^2 \|
       \rho^2 \phi_{\varepsilon} \|_{H^{1 - 2 \kappa, \varepsilon}})  +Q_{\rho} (\mathbb{X}_{\varepsilon}) (\lambda^2 + \lambda^4) \\
       & \quad  +  (\lambda+\lambda^2) Q_{\rho} (\mathbb{X}_{\varepsilon}) (\| \rho^2
       \phi_{\varepsilon} \|_{H^{1 - 2 \kappa, \varepsilon}} + \| \rho^{4 - 2
       \sigma} \chi_{\varepsilon} \|_{B_{1, 1}^{1 + 2 \kappa, \varepsilon}} +
       \| \rho \phi_{\varepsilon} \|_{L^{4, \varepsilon}} \| \rho^2
       \phi_{\varepsilon} \|_{H^{1 - 2 \kappa, \varepsilon}})\\
       & \quad  + Q_{\rho} (\mathbb{X}_{\varepsilon}) (\lambda^3 \| \rho \phi_{\varepsilon}
       \|_{L^{4, \varepsilon}} + \lambda\| \rho \phi_{\varepsilon} \|_{L^{4,
       \varepsilon}}^{2 + \theta} \| \rho^2 \phi_{\varepsilon} \|^{1 -
       \theta}_{H^{1 - 2 \kappa, \varepsilon}}) .
     \end{aligned} \]
  Thus
  \[ \begin{aligned}
       \left\| \rho^4 \LL_{\varepsilon} \chi_{\varepsilon} \right\|_{B_{1,
       1}^{- 1 + 3 \kappa, \varepsilon}} & \lesssim  \| \rho^4
       U_{\varepsilon} \|_{B_{1, 1}^{- 1 + 3 \kappa, \varepsilon}}\\
       &  \quad +\lambda \| \ttwoone{\rho^{\sigma} X_{\varepsilon}} \|_{\CC^{1 - \kappa,
       \varepsilon}} (\lambda \| \rho^{\sigma} \llbracket X_{\varepsilon}^2 \rrbracket
       \|_{\CC^{- 1 - \kappa, \varepsilon}} \| \rho^{4 - 2 \sigma}
       \phi_{\varepsilon} \|_{L^{2, \varepsilon}} + \| \rho^{4 - \sigma}
       U_{\varepsilon} \|_{B_{1, 1}^{- 1 + 3 \kappa, \varepsilon}})\\
       &  \quad + \lambda\| \rho^{\sigma} \ttwoone{X_{\varepsilon}} \|_{\CC^{1 - \kappa,
       \varepsilon}} \| \rho^{4 - \sigma} \phi_{\varepsilon} \|_{H^{1 - 2
       \kappa, \varepsilon}}\end{aligned}
       \]
       \[\begin{aligned}
      \leqslant  &   C_{\lambda}| \log t | ( Q_{\rho} (\mathbb{X}_{\varepsilon}) + \|
       \rho^2 \phi_{\varepsilon} \|_{H^{1 - 2 \kappa, \varepsilon}})  +C_{\lambda}Q_{\rho} (\mathbb{X}_{\varepsilon}) \\
       &  + C_{\lambda} Q_{\rho} (\mathbb{X}_{\varepsilon}) (\| \rho^2
       \phi_{\varepsilon} \|_{H^{1 - 2 \kappa, \varepsilon}} + \| \rho^{4 - 2
       \sigma} \chi_{\varepsilon} \|_{B_{1, 1}^{1 + 2 \kappa, \varepsilon}} +
       \| \rho \phi_{\varepsilon} \|_{L^{4, \varepsilon}} \| \rho^2
       \phi_{\varepsilon} \|_{H^{1 - 2 \kappa, \varepsilon}})\\
       &  + C_{\lambda}Q_{\rho} (\mathbb{X}_{\varepsilon}) (\| \rho \phi_{\varepsilon}
       \|_{L^{4, \varepsilon}} + \| \rho \phi_{\varepsilon} \|_{L^{4,
       \varepsilon}}^{2 + \theta} \| \rho^2 \phi_{\varepsilon} \|^{1 -
       \theta}_{H^{1 - 2 \kappa, \varepsilon}}) \\
     \end{aligned} 
     \]     
Using repeatedly the Young inequality and also~\eqref{eq:17} we obtain
  \[ \begin{aligned}
     \left\| \rho^4 \LL_{\varepsilon} \chi_{\varepsilon} \right\|_{B_{1,
       1}^{- 1 + 3 \kappa, \varepsilon}} & \leqslant  C_{\lambda} (1+| \log t | +|\log t|^{2})Q_{\rho}
     (\mathbb{X}_{\varepsilon})+\lambda \| \rho \phi_{\varepsilon} \|_{L^{4, \varepsilon}}^4 +
     \| \rho^2 \phi_{\varepsilon} \|_{H^{1 - 2 \kappa, \varepsilon}}^2\\
     &  \quad + C_{\lambda }Q_{\rho} (\mathbb{X}_{\varepsilon}) \|
     \rho^{4 - 2 \sigma} \chi_{\varepsilon} \|_{B_{1, 1}^{1 + 2 \kappa,
     \varepsilon}}.
   \end{aligned} \]   
This bound, together with the energy estimate from Theorem~\ref{th:energy-estimate} imply
  \[ \left\| \rho^4 \LL_{\varepsilon} \chi_{\varepsilon} \right\|_{L_T^1 B_{1,
     1}^{- 1 + 3 \kappa, \varepsilon}} \leqslant C_{T,m^2,\lambda} Q_{\rho}
     (\mathbb{X}_{\varepsilon})  (1+ \| \rho^{4 - 2 \sigma} \chi_{\varepsilon}
     \|_{L^1_T B_{1, 1}^{1 + 2 \kappa, \varepsilon}}) . \]
  By interpolation, embedding and the bound {\eqref{eq:chi}} we obtain for
  $\theta = \frac{1 + 3 \kappa}{1 + 4 \kappa}$ (and under the condition that
  $\kappa, \sigma, \iota \in (0, 1)$ were chosen such that $\theta \leqslant
  \frac{2 - 3 \sigma - 2 \iota}{2 - \sigma - 2 \iota}$) that
  \[ \| \rho^{4 - 2 \sigma} \chi_{\varepsilon} \|_{L^1_T B_{1, 1}^{1 + 2
     \kappa, \varepsilon}} \lesssim \int_0^T \| \rho^{2 + \sigma + 2 \iota}
     \chi_{\varepsilon} (t) \|_{B^{- \kappa, \varepsilon}_{1, 1}}^{1 - \theta}
     \| \rho^4 \chi_{\varepsilon} (t) \|_{B^{1 + 3 \kappa, \varepsilon}_{1,
     1}}^{\theta} \mathd t \]
  \[ \lesssim \int_0^T \| \rho^{2 + \sigma} \chi_{\varepsilon} (t) \|_{L^{2,
     \varepsilon}}^{1 - \theta} \| \rho^4 \chi_{\varepsilon} (t) \|_{B^{1 + 3
     \kappa, \varepsilon}_{1, 1}}^{\theta} \mathd t \lesssim \| \rho^{2 +
     \sigma} \chi_{\varepsilon} (t) \|_{L^{\infty}_T L^{2, \varepsilon}}^{1 -
     \theta} \int_0^T \| \rho^4 \chi_{\varepsilon} (t) \|_{B^{1 + 3 \kappa,
     \varepsilon}_{1, 1}}^{\theta} \mathd t \]
  \[ \lesssim C_{T,\lambda} Q_{\rho} (\mathbb{X}_{\varepsilon}) (1 +\| \rho^2 \phi_{\varepsilon} (0)\|^{1 - \theta}_{L^{2, \varepsilon}}) \int_0^T \| \rho^4
     \chi_{\varepsilon} (t) \|_{B^{1 + 3 \kappa, \varepsilon}_{1, 1}}^{\theta}
     \mathd t .\]
  Consequently,
  \[ \left\| \rho^4 \LL_{\varepsilon} \chi_{\varepsilon} \right\|_{L_T^1 B_{1,
     1}^{- 1 + 3 \kappa, \varepsilon}} \leqslant C_{T,m^2,\lambda} Q_{\rho}
     (\mathbb{X}_{\varepsilon}) \]
     \[ \qquad +C_{T,\lambda} Q_{\rho} (\mathbb{X}_{\varepsilon}) ( 1+\|
     \rho^2 \phi_{\varepsilon} (0)\|^{1 - \theta}_{L^{2, \varepsilon}})
     \int_0^T \| \rho^4 \chi_{\varepsilon} (t) \|_{B^{1 + 3 \kappa,
     \varepsilon}_{1, 1}}^{\theta} \mathd t \]
  \[ \leqslant C_{T,m^{2},\lambda,\delta}  Q_{\rho} (\mathbb{X}_{\varepsilon}) (1 +\| \rho^2 \phi_{
     \varepsilon} (0)\|_{L^{2, \varepsilon}}) + \delta \| \rho^4
     \chi_{\varepsilon} \|_{L^1_T B_{1, 1}^{1 + 3 \kappa, \varepsilon}}, \]
  which finally leads to
  \[ 
  \begin{aligned}
  \| \rho^4 \chi_{\varepsilon} \|_{L^1_T B_{1, 1}^{1 + 3 \kappa,
     \varepsilon}} & \lesssim \| \rho^4 \chi_{\varepsilon} (0) \|_{B_{1, 1}^{- 1
     + 3 \kappa, \varepsilon}}  +C_{T,m^{2},\lambda}  Q_{\rho} (\mathbb{X}_{\varepsilon}) (1 +\| \rho^2 \phi_{
     \varepsilon} (0)\|_{L^{2, \varepsilon}})
  \end{aligned}
    \]
  by Lemma~\ref{lem:reg} and since $\chi_{\varepsilon} (0) = \phi_{\varepsilon}
  (0)$ and $L^{2, \varepsilon} (\rho^2) \subset B_{1, 1}^{- 1 + 3 \kappa,
  \varepsilon} (\rho^4)$, the claim follows.
\hspace*{\fill}$\Box$\medskip





%\bibliographystyle{alpha}
%\bibliography{weak-solutions}

\newcommand{\etalchar}[1]{$^{#1}$}
\begin{thebibliography}{BCG{\etalchar{+}}78}

\bibitem[Abd07]{MR2350436}
Abdelmalek Abdesselam.
\newblock A complete renormalization group trajectory between two fixed points.
\newblock {\em Comm. Math. Phys.}, 276(3):727--772, 2007.

\bibitem[Abd18]{MR3874867}
Abdelmalek Abdesselam.
\newblock Towards three-dimensional conformal probability.
\newblock {\em p-Adic Numbers Ultrametric Anal. Appl.}, 10(4):233--252, 2018.

\bibitem[AK17]{albeverio_invariant_2017}
S.~Albeverio and S.~Kusuoka.
\newblock The invariant measure and the flow associated to the
  {$\Phi^4_3$}-quantum field model.
\newblock {\em arXiv:1711.07108}, November 2017.

\bibitem[ALZ06]{albeverio_remark_2006}
S.~Albeverio, S.~Liang, and B.~Zegarlinski.
\newblock Remark on the integration by parts formula for the $\phi^4_3$-quantum
  field model.
\newblock {\em Infinite Dimensional Analysis, Quantum Probability and Related
  Topics}, 9(1):149--154, 2006.

\bibitem[Amm00]{A00}
H.~Amman.
\newblock Compact embeddings of vector-valued sobolev and besov spaces.
\newblock {\em Glasnik Matemati\v{c}ki}, 35(55):161--177, 2000.

\bibitem[AR91]{albeverio_stochastic_1991}
S.~Albeverio and M.~R\"ockner.
\newblock Stochastic differential equations in infinite dimensions: solutions
  via {Dirichlet} forms.
\newblock {\em Probability Theory and Related Fields}, 89(3):347--386, 1991.

\bibitem[AY02]{albeverio_HC1_2002}
S.~Albeverio and M.~W. Yoshida.
\newblock {$H-C^1$} maps and elliptic {SPDEs} with polynomial and exponential
  perturbations of {Nelson}'s {Euclidean} free field.
\newblock {\em Journal of Functional Analysis}, 196(2):265--322, 2002.

\bibitem[AY09]{albeverio_hida_2009}
S.~Albeverio and M.~W. Yoshida.
\newblock Hida distribution construction of non-{Gaussian} reflection positive
  generalized random fields.
\newblock {\em Infinite Dimensional Analysis, Quantum Probability and Related
  Topics}, 12(1):21--49, 2009.

\bibitem[Ba{\l}83]{MR733476}
T.~Ba{\l}aban.
\newblock Ultraviolet stability in field theory. {T}he {$\varphi _{3}^{4}$}
  model.
\newblock In {\em Scaling and self-similarity in physics ({B}ures-sur-{Y}vette,
  1981/1982)}, volume~7 of {\em Progr. Phys.}, pages 297--319. Birkh\"{a}user
  Boston, Boston, MA, 1983.

\bibitem[Bat99]{battle_wavelets_1999}
G.~Battle.
\newblock {\em Wavelets and {Renormalization}}.
\newblock World Scientific, 1999.

\bibitem[BB20]{bailleul_high_2016}
I.~Bailleul and F.~Bernicot.
\newblock High order paracontrolled calculus.
\newblock {\em Forum of Mathematics, Sigma}, 2020.
\newblock to appear.

\bibitem[BCD11]{BCD}
H.~Bahouri, J.-Y. Chemin, and R.~Danchin.
\newblock {\em Fourier {Analysis} and {Nonlinear} {Partial} {Differential}
  {Equations}}.
\newblock Springer, January 2011.

\bibitem[BCG{\etalchar{+}}78]{benfatto_probabilistic_1978}
G.~Benfatto, M.~Cassandro, G.~Gallavotti, F.~Nicol\`o, E.~Olivieri,
  E.~Presutti, and E.~Scacciatelli.
\newblock Some probabilistic techniques in field theory.
\newblock {\em Communications in Mathematical Physics}, 59(2):143--166, 1978.

\bibitem[BCM88]{borkar_stochastic_1988}
V.~S. Borkar, R.~T. Chari, and S.~K. Mitter.
\newblock Stochastic quantization of field theory in finite and infinite
  volume.
\newblock {\em Journal of Functional Analysis}, 81(1):184--206, November 1988.

\bibitem[BDH95]{brydges_short_1995}
D.~Brydges, J.~Dimock, and T.~R. Hurd.
\newblock The short distance behavior of $\phi^4_3$.
\newblock {\em Communications in Mathematical Physics}, 172(1):143--186, 1995.

\bibitem[BDH98]{brydges_non_gaussian_1998}
D.~Brydges, J.~Dimock, and T.~R. Hurd.
\newblock A {Non}-{Gaussian} {Fixed} {Point} for $\phi^4$ in $4?\varepsilon$
  {Dimensions}.
\newblock {\em Communications in Mathematical Physics}, 198(1):111--156,
  November 1998.

\bibitem[Beh19]{behan_bootstrapping_2019}
Connor Behan.
\newblock Bootstrapping the long-range {Ising} model in three dimensions.
\newblock {\em Journal of Physics A: Mathematical and Theoretical},
  52(7):075401, January 2019.

\bibitem[BFS83]{MR723546}
D.~C. Brydges, J.~Fr\"{o}hlich, and A.~D. Sokal.
\newblock A new proof of the existence and nontriviality of the continuum
  {$\varphi ^{4}_{2}$} and {$\varphi ^{4}_{3}$} quantum field theories.
\newblock {\em Comm. Math. Phys.}, 91(2):141--186, 1983.

\bibitem[BG18]{barashkov_variational_2018}
N.~Barashkov and M.~Gubinelli.
\newblock Variational approach to {Euclidean} {QFT}.
\newblock {\em arXiv:1805.10814}, May 2018.
\newblock arXiv: 1805.10814.

\bibitem[Bis09]{kotecky_reflection_2009}
M.~Biskup.
\newblock Reflection {Positivity} and {Phase} {Transitions} in {Lattice} {Spin}
  {Models}.
\newblock In R.~Koteck\'y, editor, {\em Methods of {Contemporary}
  {Mathematical} {Statistical} {Physics}}, volume 1970, pages 1--86. Springer
  Berlin Heidelberg, Berlin, Heidelberg, 2009.

\bibitem[BMS03]{MR2004988}
D.~C. Brydges, P.~K. Mitter, and B.~Scoppola.
\newblock Critical {$(\Phi^4)_{3,\epsilon}$}.
\newblock {\em Comm. Math. Phys.}, 240(1-2):281--327, 2003.

\bibitem[Bon81]{bony_calcul_1981}
J.-M. Bony.
\newblock Calcul symbolique et propagation des singularit\'es pour les
  \'equations aux d\'eriv\'ees partielles non lin\'eaires.
\newblock {\em Annales scientifiques de l'\'Ecole normale sup\'erieure},
  14(2):209--246, 1981.

\bibitem[BOP15]{benyi_probabilistic_2015}
\'A. B\'enyi, T.~Oh, and O.~Pocovnicu.
\newblock On the probabilistic {Cauchy} theory of the cubic nonlinear
  {Schr\"odinger} equation on {$R^d$} , $d\ge 3$.
\newblock {\em Transactions of the American Mathematical Society, Series B},
  2(1):1--50, 2015.

\bibitem[Bou94]{bourgain_periodic_1994}
J.~Bourgain.
\newblock Periodic nonlinear {Schr\"odinger} equation and invariant measures.
\newblock {\em Communications in Mathematical Physics}, 166(1):1--26, 1994.

\bibitem[Bou96]{bourgain_invariant_1996}
J.~Bourgain.
\newblock Invariant measures for the {2D-}defocusing nonlinear {Schr\"odinger}
  equation.
\newblock {\em Communications in Mathematical Physics}, 176(2):421--445, 1996.

\bibitem[BSZ92]{baez_introduction_1992}
J.~C. Baez, I.~E. Segal, and Z.-F. Zhou.
\newblock {\em Introduction to algebraic and constructive quantum field
  theory}.
\newblock Princeton {Series} in {Physics}. Princeton University Press,
  Princeton, NJ, 1992.

\bibitem[BT08a]{burq_random_2008_1}
N.~Burq and N.~Tzvetkov.
\newblock Random data {Cauchy} theory for supercritical wave equations. {I}.
  {Local} theory.
\newblock {\em Inventiones Mathematicae}, 173(3):449--475, 2008.

\bibitem[BT08b]{burq_random_2008}
N.~Burq and N.~Tzvetkov.
\newblock Random data {Cauchy} theory for supercritical wave equations. {II}.
  {A} global existence result.
\newblock {\em Inventiones Mathematicae}, 173(3):477--496, 2008.

\bibitem[CC18]{CC}
R.~Catellier and K.~Chouk.
\newblock Paracontrolled distributions and the 3-dimensional stochastic
  quantization equation.
\newblock {\em to appear in The Annals of Probability}, 2018.

\bibitem[Cha14]{chatterjee_invariant_2014}
S.~Chatterjee.
\newblock Invariant {Measures} and the {Soliton} {Resolution} {Conjecture}.
\newblock {\em Communications on Pure and Applied Mathematics},
  67(11):1737--1842, November 2014.

\bibitem[CK12]{chatterjee_probabilistic_2012}
S.~Chatterjee and K.~Kirkpatrick.
\newblock Probabilistic {Methods} for {Discrete} {Nonlinear} {Schr\"odinger}
  {Equations}.
\newblock {\em Communications on Pure and Applied Mathematics}, 65(5):727--757,
  May 2012.

\bibitem[CO12]{colliander_almost_2012}
J.~Colliander and T.~Oh.
\newblock Almost sure well-posedness of the cubic nonlinear {Schr\"odinger}
  equation below ${L}^2(\mathbb{T})$.
\newblock {\em Duke Mathematical Journal}, 161(3):367--414, February 2012.

\bibitem[Dim13a]{dimock_renormalization_2013_1}
J.~Dimock.
\newblock The renormalization group according to {Balaban}, {I}. {Small}
  fields.
\newblock {\em Reviews in Mathematical Physics}, 25(07):1330010, July 2013.

\bibitem[Dim13b]{dimock_renormalization_2013_2}
J.~Dimock.
\newblock The renormalization group according to {Balaban}. {II}. {Large}
  fields.
\newblock {\em Journal of Mathematical Physics}, 54(9):092301, September 2013.

\bibitem[Dim14]{dimock_renormalization_2014_3}
J.~Dimock.
\newblock The renormalization group according to {Balaban} {III}.
  {Convergence}.
\newblock {\em Annales Henri Poincar\'e}, 15(11):2133--2175, November 2014.

\bibitem[DPD03]{da_prato_strong_2003}
G.~Da~Prato and A.~Debussche.
\newblock Strong solutions to the stochastic quantization equations.
\newblock {\em The Annals of Probability}, 31(4):1900--1916, 2003.

\bibitem[EE79]{eckmann_time_ordered_1979}
J.~P. Eckmann and H.~Epstein.
\newblock Time-ordered products and {Schwinger} functions.
\newblock {\em Communications in Mathematical Physics}, 64(2):95--130, June
  1979.

\bibitem[FFS92]{fernandez_random_1992}
R.~Fern\'andez, J.~Fr\"ohlich, and A.~D. Sokal.
\newblock {\em Random {Walks}, {Critical} {Phenomena}, and {Triviality} in
  {Quantum} {Field} {Theory}}.
\newblock Springer Berlin Heidelberg, Berlin, Heidelberg, 1992.

\bibitem[FH14]{friz_course_2014}
P.~K. Friz and M.~Hairer.
\newblock {\em A {Course} on {Rough} {Paths}: {With} an {Introduction} to
  {Regularity} {Structures}}.
\newblock Springer, August 2014.

\bibitem[FO76]{feldman_wightman_1976}
J.~S. Feldman and K.~Osterwalder.
\newblock The {Wightman} axioms and the mass gap for weakly coupled $\phi^4_3$
  quantum field theories.
\newblock {\em Annals of Physics}, 97(1):80--135, 1976.

\bibitem[FR77]{feldman77}
J.~S. Feldman and R.~R\c{a}czka.
\newblock The relativistic field equation of the $\lambda\phi^4_3$ quantum
  field theory.
\newblock {\em Annals of Physics}, 108(1):212--229, September 1977.

\bibitem[FV17]{friedli2017statistical}
S.~Friedli and Y.~Velenik.
\newblock {\em Statistical mechanics of lattice systems: a concrete
  mathematical introduction}.
\newblock Cambridge University Press, 2017.

\bibitem[GH18]{GH18}
M.~{Gubinelli} and M.~{Hofmanov{\'a}}.
\newblock {Global solutions to elliptic and parabolic $\Phi^4$ models in
  Euclidean space}.
\newblock {\em ArXiv e-prints}, April 2018.

\bibitem[GIP15]{GIP}
M.~Gubinelli, P.~Imkeller, and N.~Perkowski.
\newblock Paracontrolled distributions and singular {PDEs}.
\newblock {\em Forum of Mathematics. Pi}, 3:e6, 75, 2015.

\bibitem[GJ73]{glimm_positivity_1973}
J.~Glimm and A.~Jaffe.
\newblock Positivity of the $\phi^4_3$ {Hamiltonian}.
\newblock {\em Fortschritte der Physik. Progress of Physics}, 21:327--376,
  1973.

\bibitem[GJ87]{MR887102}
J.~Glimm and A.~Jaffe.
\newblock {\em Quantum physics. A functional integral point of view}.
\newblock Springer-Verlag, New York, second edition, 1987.

\bibitem[GK86]{gawpolhk_edzki_asymptotic_1986}
K.~Gaw\k{e}dzki and A.~Kupiainen.
\newblock Asymptotic freedom beyond perturbation theory.
\newblock In {\em Ph\'enom\`enes critiques, syst\`emes al\'eatoires, th\'eories
  de jauge, {Part} {I}, {II} ({Les} {Houches}, 1984)}, pages 185--292.
  North-Holland, Amsterdam, 1986.

\bibitem[Gli68]{glimm_boson_1968}
J.~Glimm.
\newblock Boson fields with the $:\phi^4:$ interaction in three dimensions.
\newblock {\em Communications in Mathematical Physics}, 10:1--47, 1968.

\bibitem[GP17]{gubinelli_kpz_2017}
M.~Gubinelli and N.~Perkowski.
\newblock {KPZ} {Reloaded}.
\newblock {\em Communications in Mathematical Physics}, 349(1):165--269,
  January 2017.

\bibitem[Gub04]{gubinelli_controlling_2004}
M.~Gubinelli.
\newblock Controlling rough paths.
\newblock {\em Journal of Functional Analysis}, 216(1):86--140, 2004.

\bibitem[GUZ19]{gubinelli_semilinear_2018}
M.~Gubinelli, B.~Ugurcan, and I.~Zachhuber.
\newblock Semilinear evolution equations for the {Anderson} {Hamiltonian} in
  two and three dimensions.
\newblock {\em Stochastics and Partial Differential Equations: Analysis and
  Computations}, May 2019.

\bibitem[Hai14]{hairer_theory_2014}
M.~Hairer.
\newblock A theory of regularity structures.
\newblock {\em Inventiones mathematicae}, 198(2):269--504, March 2014.

\bibitem[Hai15]{hairer_regularity_2015}
M.~Hairer.
\newblock Regularity structures and the dynamical $\phi^4_3$ model.
\newblock {\em arXiv:1508.05261}, August 2015.

\bibitem[HI18]{hairer_tightness_2018}
M.~Hairer and M.~Iberti.
\newblock Tightness of the {Ising}-{Kac} model on the two-dimensional torus.
\newblock {\em Journal of Statistical Physics}, 171(4):632--655, 2018.

\bibitem[HM18a]{hairer_discretisations_2018}
M.~Hairer and K.~Matetski.
\newblock Discretisations of rough stochastic {PDEs}.
\newblock {\em The Annals of Probability}, 46(3):1651--1709, May 2018.

\bibitem[HM18b]{Hairer:2018:10.1214/17-AIHP840}
M.~Hairer and J.~Mattingly.
\newblock The strong feller property for singular stochastic pdes.
\newblock {\em Annales de l'Institut Henri Poincare Probabilites et
  Statistiques}, 54:1314--1340, 2018.

\bibitem[Iwa87]{iwata_infinite_1987}
K.~Iwata.
\newblock An infinite dimensional stochastic differential equation with state
  space {C}({R}).
\newblock {\em Probability Theory and Related Fields}, 74(1):141--159, March
  1987.

\bibitem[Jaf00]{jaffe_constructive_2000}
A.~Jaffe.
\newblock Constructive quantum field theory.
\newblock In {\em Mathematical {Physics} 2000}, pages 111--127. May 2000.

\bibitem[Jaf08]{MR2391806}
A.~Jaffe.
\newblock Quantum theory and relativity.
\newblock In {\em Group representations, ergodic theory, and mathematical
  physics: a tribute to {G}eorge {W}. {M}ackey}, volume 449 of {\em Contemp.
  Math.}, pages 209--245. Amer. Math. Soc., Providence, RI, 2008.

\bibitem[Jaf15]{MR3392505}
A.~Jaffe.
\newblock Stochastic quantization, reflection positivity, and quantum fields.
\newblock {\em J. Stat. Phys.}, 161(1):1--15, 2015.

\bibitem[Jaf18]{jaffe_reflection_2018}
A.~Jaffe.
\newblock Reflection {Positivity} {Then} and {Now}.
\newblock {\em arXiv:1802.07880 [hep-th, physics:math-ph]}, February 2018.
\newblock arXiv: 1802.07880.

\bibitem[JLM85]{jona-lasinio_stochastic_1985}
G.~Jona-Lasinio and P.~K. Mitter.
\newblock On the stochastic quantization of field theory.
\newblock {\em Communications in Mathematical Physics (1965-1997)},
  101(3):409--436, 1985.

\bibitem[JT18]{jorgensen_reflection_2018}
P.~Jorgensen and F.~Tian.
\newblock Reflection positivity, duality, and spectral theory.
\newblock {\em Journal of Applied Mathematics and Computing}, April 2018.

\bibitem[Kha11]{khasminskii2011stochastic}
R.~Khasminskii.
\newblock {\em Stochastic stability of differential equations}, volume~66.
\newblock Springer Science \& Business Media, 2011.

\bibitem[Kup16]{kupiainen_renormalization_2016}
A.~Kupiainen.
\newblock Renormalization {Group} and {Stochastic} {PDEs}.
\newblock {\em Annales Henri Poincar{\'e}}, 17(3):497--535, March 2016.

\bibitem[LCL07]{lyons_differential_2007}
T.~J. Lyons, M.~J. Caruana, and T.~L\'evy.
\newblock {\em Differential {Equations} {Driven} by {Rough} {Paths}: {Ecole}
  d'{Et\'e} de {Probabilit\'es} de {Saint}-{Flour} {XXXIV}-2004}.
\newblock Springer, 1 edition, June 2007.

\bibitem[LQ02]{lyons_system_2002}
T.~Lyons and Z.~Qian.
\newblock {\em System {Control} and {Rough} {Paths}}.
\newblock Oxford University Press, 2002.

\bibitem[LRS88]{lebowitz_statistical_1988}
J.~L. Lebowitz, H.~A. Rose, and E.~R. Speer.
\newblock Statistical mechanics of the nonlinear {Schr\"odinger} equation.
\newblock {\em Journal of Statistical Physics}, 50(3-4):657--687, 1988.

\bibitem[LRS89]{lebowitz_statistical_1989}
J.~L. Lebowitz, H.~A. Rose, and E.~R. Speer.
\newblock Statistical mechanics of the nonlinear {Schr\"odinger} equation.
  {II}. {Mean} field approximation.
\newblock {\em Journal of Statistical Physics}, 54(1-2):17--56, 1989.

\bibitem[Lyo98]{lyons_differential_1998}
T.~Lyons.
\newblock Differential equations driven by rough signals.
\newblock {\em Revista Matem\'atica Iberoamericana}, pages 215--310, 1998.

\bibitem[McK95a]{mckean_1995_err}
H.~P. McKean.
\newblock Erratum: Statistical mechanics of nonlinear wave equations. iv. cubic
  schr\"odinger.
\newblock {\em Comm. Math. Phys.}, 173(3):675, 1995.

\bibitem[McK95b]{mckean_1995}
H.~P. McKean.
\newblock Statistical mechanics of nonlinear wave equations. iv. cubic
  schr\"odinger.
\newblock {\em Comm. Math. Phys.}, 168(3):479--491, 1995.

\bibitem[Mey81]{meyer_remarques_1981}
Y.~Meyer.
\newblock Remarques sur un th\'eor\`eme de {J}.-{M}. {Bony}.
\newblock In {\em Rendiconti del {Circolo} {Matematico} di {Palermo}. {Serie}
  {II}}, pages 1--20, 1981.

\bibitem[MP17]{MP17}
J.~{Martin} and N.~{Perkowski}.
\newblock {Paracontrolled distributions on Bravais lattices and weak
  universality of the 2d parabolic Anderson model}.
\newblock {\em Ann. Inst. H. Poincar\'e - Probabilit\'es et Statistiques}, (55)(4):2058--2110, 2019.

\bibitem[MS76]{magnen_infinite_1976}
J.~Magnen and R.~S\'en\'eor.
\newblock The infinite volume limit of the $\phi^4_3$ model.
\newblock {\em Ann. Inst. H. Poincar\'e Sect. A (N.S.)}, 24(2):95--159, 1976.

\bibitem[MW17a]{MWcomedown}
J.-C. Mourrat and H.~Weber.
\newblock The dynamic {$\Phi^4_3$} model comes down from infinity.
\newblock {\em Comm. Math. Phys.}, 356(3):673--753, 2017.

\bibitem[MW17b]{MW17}
J.-C. Mourrat and H.~Weber.
\newblock Global well-posedness of the dynamic {$\Phi^4$} model in the plane.
\newblock {\em The Annals of Probability}, 45(4):2398--2476, July 2017.

\bibitem[MW18]{moinat_space_time_2018}
A.~Moinat and H.~Weber.
\newblock Space-time localisation for the dynamic $\phi^4_3$ model.
\newblock {\em arXiv:1811.05764}, November 2018.
\newblock arXiv: 1811.05764.

\bibitem[MWX16]{mourrat_construction_2016}
J.-C. Mourrat, H.~Weber, and W.~Xu.
\newblock Construction of $\phi^4_3$ diagrams for pedestrians.
\newblock {\em arXiv:1610.08897}, October 2016.
\newblock arXiv: 1610.08897.

\bibitem[Nel66]{nelson1966}
E.~Nelson.
\newblock Derivation of the {S}chr\"odinger equation from {N}ewtonian
  mechanics.
\newblock {\em Phys. Rev.}, (150):1079--1085, 1966.

\bibitem[Nel67]{MR0214150}
E.~Nelson.
\newblock {\em Dynamical theories of {B}rownian motion}.
\newblock Princeton University Press, Princeton, N.J., 1967.

\bibitem[NO18]{neeb_reflection_2018}
K.-H. Neeb and G.~Olafsson.
\newblock Reflection {Positivity}---{A} {Representation} {Theoretic}
  {Perspective}.
\newblock {\em arXiv:1802.09037}, February 2018.
\newblock arXiv: 1802.09037.

\bibitem[NPS13]{nahmod_almost_2013}
A.~Nahmod, N.~Pavlovi{\'c}, and G.~Staffilani.
\newblock Almost {Sure} {Existence} of {Global} {Weak} {Solutions} for
  {Supercritical} {Navier}--{Stokes} {Equations}.
\newblock {\em SIAM Journal on Mathematical Analysis}, 45(6):3431--3452,
  January 2013.

\bibitem[OS73]{osterwalder_axioms_1973}
K.~Osterwalder and R.~Schrader.
\newblock Axioms for {Euclidean} {Green}'s functions.
\newblock {\em Communications in Mathematical Physics}, 31(2):83--112, June
  1973.

\bibitem[OS75]{osterwalder_axioms_1975}
K.~Osterwalder and R.~Schrader.
\newblock Axioms for {Euclidean} {Green}'s functions {II}.
\newblock {\em Communications in Mathematical Physics}, 42(3):281--305, October
  1975.

\bibitem[Par77]{park_convergence_1977}
Y.~M. Park.
\newblock Convergence of lattice approximations and infinite volume limit in
  the $(\lambda \phi^4-\sigma \phi^2 -\tau \phi )_3$ field theory.
\newblock {\em Journal of Mathematical Physics}, 18(3):354--366, 1977.

\bibitem[PRV19a]{MR3942977}
D.~Poland, S.~Rychkov, and A.~Vichi.
\newblock The conformal bootstrap: theory, numerical techniques, and
  applications.
\newblock {\em Rev. Modern Phys.}, 91(1):015002, 74, 2019.

\bibitem[PRV19b]{poland_conformal_2019}
David Poland, Slava Rychkov, and Alessandro Vichi.
\newblock The conformal bootstrap: {Theory}, numerical techniques, and
  applications.
\newblock {\em Reviews of Modern Physics}, 91(1):015002, January 2019.

\bibitem[PW81]{parisi_perturbation_1981}
G.~Parisi and Y.~S. Wu.
\newblock Perturbation theory without gauge fixing.
\newblock {\em Scientia Sinica. Zhongguo Kexue}, 24(4):483--496, 1981.

\bibitem[Riv91]{rivasseau_perturbative_1991}
V.~Rivasseau.
\newblock {\em From {Perturbative} to {Constructive} {Renormalization}}.
\newblock Princeton University Press, Princeton, N.J, 2 edition edition, May
  1991.

\bibitem[Sim74]{simon_po2_1974}
B.~Simon.
\newblock {\em $P(\phi)_2$ {Euclidean} ({Quantum}) {Field} {Theory}}.
\newblock Princeton University Press, Princeton, N.J, April 1974.

\bibitem[Sla18]{slade_critical_2018}
Gordon Slade.
\newblock Critical {Exponents} for {Long}-{Range} $o(n)$ {Models} {Below} the
  {Upper} {Critical} {Dimension}.
\newblock {\em Communications in Mathematical Physics}, 358(1):343--436,
  February 2018.

\bibitem[SS76]{seiler_nelsons_1976}
E.~Seiler and B.~Simon.
\newblock Nelson's symmetry and all that in the ${Yukawa}_2$ and $(\phi^4)_3$
  field theories.
\newblock {\em Annals of Physics}, 97(2):470--518, April 1976.

\bibitem[Sum12]{summers_perspective_2012}
S.~J. Summers.
\newblock A {Perspective} on {Constructive} {Quantum} {Field} {Theory}.
\newblock {\em arXiv:1203.3991 [math-ph]}, March 2012.
\newblock arXiv: 1203.3991.

\bibitem[Sym64]{Symanzik1964}
K.~Symanzik.
\newblock A modified model of {E}uclidean quantum field theory.
\newblock Courant Institute of Mathematical Sciences, Report IMM-NYU 327, 1964.

\bibitem[Tri06]{T06}
H.~Triebel.
\newblock {\em Theory of {Function} {Spaces} {III}}.
\newblock Springer, August 2006.

\bibitem[Tzv16]{tzvetkov_random_2016}
N.~Tzvetkov.
\newblock Random data wave equations.
\newblock 2016.

\bibitem[VW73]{velo_constructive_1973}
G.~Velo and A.~Wightman, editors.
\newblock {\em Constructive quantum field theory}.
\newblock Springer-Verlag, Berlin-New York, 1973.

\bibitem[Wat89]{watanabe_block_1989}
H.~Watanabe.
\newblock Block spin approach to $\phi^4_3$ field theory.
\newblock {\em Journal of Statistical Physics}, 54(1-2):171--190, 1989.

\bibitem[Wig76]{MR0436800}
A.~S. Wightman.
\newblock Hilbert's sixth problem: mathematical treatment of the axioms of
  physics.
\newblock pages 147--240, 1976.

\bibitem[Zab89]{Zab89}
J. Zabczyk.
\newblock Symmetric solutions of semilinear stochastic equations.
\newblock {\em In Stochastic Partial Differential Equations and Applications II}, 237?256. Springer, 1989.

\bibitem[ZZ18]{ZZ18}
R.~Zhu and X.~Zhu.
\newblock Lattice approximation to the dynamical $\phi^4_3$ model.
\newblock {\em The Annals of Probability}, 46(1):397--455, 2018.

\end{thebibliography}

\end{document}
